% ------------------------------------------------------------------------
% ------------------------------------------------------------------------
% ICMC: Modelo de Trabalho Acadêmico (tese de doutorado, dissertação de
% mestrado e trabalhos monográficos em geral) em conformidade com 
% ABNT NBR 14724:2011: Informação e documentação - Trabalhos acadêmicos -
% Apresentação
% ------------------------------------------------------------------------
% ------------------------------------------------------------------------

% Opções: 
%   Qualificação          = qualificacao 
%   Curso                 = doutorado/mestrado
%   Situação do trabalho  = pre-defesa/pos-defesa (exceto para qualificação)
%   Versão para impressão = impressao
\documentclass[doutorado, pos-defesa]{packages/icmc}

% ---------------------------------------------------------------------------
% Pacotes Opcionais
% ---------------------------------------------------------------------------
\usepackage{rotating}           % Usado para rotacionar o texto
\usepackage[all,knot,arc,import,poly]{xy}   % Pacote para desenhos gráficos
% Este pacote pode conflitar com outros pacotes gráficos como o ``pictex''
% Então é necessário usar apenas um dos pacotes conflitantes
\newcommand{\VerbL}{0.52\textwidth}
\newcommand{\LatL}{0.42\textwidth}
% ---------------------------------------------------------------------------


% ---
% Informações de dados para CAPA e FOLHA DE ROSTO
% ---
% Tanto na capa quanto nas folhas de rosto apenas a primeira letra da primeira palavra (ou nomes próprios) devem estar em letra maiúscula, todas as demais devem ser em letra minúscula.
\tituloPT{Modelo de teses e dissertações em LaTeX do ICMC}
\tituloEN{Model of theses and dissertations in LaTeX of the ICMC}
\autor[Antonelli, H. L.]{Humberto Lidio Antonelli}
\genero{M} % Gênero do autor (M = Masculino / F = Feminino)
\orientador[Orientadora]{Profa. Dra.}{Renata Pontin de Mattos Fortes}
%\coorientador{Prof. Dr.}{Fulano de Tal}
\curso{CCMC}
\data{01}{12}{2020} % Data do depósito
\idioma{PT} % Idioma principal do documento (PT = português / EN = inglês)
% ---


% ---
% RESUMOS
% ---

% Resumo em PORTUGUÊS
% conter no máximo 500 palavras
% conter no mínimo 1 e no máximo 5 palavras-chave
\textoresumo[brazil]{
    Este trabalho é um breve modelo  para a escrita de monografias de qualificação, dissertações e teses utilizando o ambiente \LaTeX, de acordo com as normas exigidas pelo Instituto de Ciências Matemáticas e de Computação (ICMC), da Universidade de São Paulo (USP). Para a confecção deste modelo foi utilizado a última versão (1.9.7) do pacote de classes \textit{abnTeX2} que segue as normas da Associação Brasileira de Normas Técnicas. A elaboração de uma monografia, dissertação ou tese pode ser feita sobrescrevendo o conteúdo deste modelo. 
    }{Modelo, Monografia de qualificação, Dissertação, Tese, Latex}


% resumo em INGLÊS
% conter no máximo 500 palavras
% conter no mínimo 1 e no máximo 5 palavras-chave
\textoresumo[english]{
    This paper is a brief model for writing qualification monographs, dissertations and thesis using \LaTeX environment, in accordance with the standards required by the Institute of Mathematics and Computer Sciences (ICMC), University of São Paulo (USP). For making this model, the latest version (1.9.7) \textit{abnTeX2} classes package was used. This package follow the rules of the Brazilian Association of Technical Standards. A drafting a monograph, dissertation or thesis can be done by overwriting the contents of this model.
    }{Template, Qualification monograph, Dissertation, Thesis, Latex}


% ----------------------------------------------------------
% ELEMENTOS PRÉ-TEXTUAIS
% ----------------------------------------------------------

% Inserir a ficha catalográfica
\incluifichacatalografica{tex/pre-textual/ficha-catalografica.pdf}

% DEDICATÓRIA / AGRADECIMENTO / EPÍGRAFE
\textodedicatoria*{tex/pre-textual/dedicatoria}
\textoagradecimentos*{tex/pre-textual/agradecimentos}
\textoepigrafe*{tex/pre-textual/epigrafe}

% Inclui a lista de figuras
\incluilistadefiguras

% Inclui a lista de tabelas
\incluilistadetabelas

% Inclui a lista de quadros
\incluilistadequadros

% Inclui a lista de algoritmos
\incluilistadealgoritmos

% Inclui a lista de códigos
\incluilistadecodigos

% Inclui a lista de siglas e abreviaturas
\incluilistadesiglas

% Inclui a lista de símbolos
\incluilistadesimbolos

% ----
% Início do documento
% ----
\begin{document}
% ----------------------------------------------------------
% ELEMENTOS TEXTUAIS
% ----------------------------------------------------------
\textual

\chapter{Introdução}
\label{chapter:introducao}
Superf\'icies m\'inimas constituem hoje um dos objetos de estudo
mais importantes em Geometria Diferencial. De particular interesse, 
s\~ao as superf\'icies m\'inimas em variedades de curvatura constante, 
como o espa\c co Euclidiano $\R^3$, o espa\c co hiperb\'olico $\Hy^3$
e a esfera $\Sp^3$. O caso de superf\'icies m\'inimas em $\R^3$ \'e um
assunto cl\'assico que tem despertado a aten\c c\~ao de v\'arias 
gera\c c\~oes de ge\^ometras, desde o problema inicial proposto por 
Lagrange at\'e os dias atuais. Nesta disserta\c c\~ao daremos
\^enfase ao estudo de superf\'icies m\'inimas na esfera $\Sp^3$, 
identificando-a com a esfera unit\'aria em $\R^4$, i.e.,
\[
\Sp^3=\{x\in\R^4:x_1^2+x_2^2+x_3^2+x_4^2=1\}.
\]



Ao contr\'ario do que ocorre em $\R^3$, de que n\~ao existem 
superf\'icies m\'inimas fechadas, existem exemplos interessantes
desse fen\^omeno na esfera $\Sp^3$. Um exemplo simples de 
superf\'icie m\'inima fechada em $\Sp^3$ \'e o {\em equador}
\[
M=\{x\in\Sp^3\subset\R^4:x_4=0\}.
\]
Neste caso, as curvaturas principais s\~ao ambas iguais a zero. 
Al\'em disso, o equador tem curvatura Gaussiana
constante igual a $1$. Assim, munido da m\'etrica induzida, $M$ \'e
isom\'etrica \`a esfera usual $\Sp^2$.

Outro exemplo de superf\'icie m\'inima em $\Sp^3$ \'e o {\em toro de
	Clifford}, definido por
\[
M=\left\{x\in\Sp^3:x_1^2+x_2^2=x_3^2+x_4^2=1/2\right\}.
\]
Neste caso, as curvaturas principais s\~ao iguais a $1$ e $-1$, 
resultando em $H=0$. Al\'em disso, a curvatura Gaussiana \'e 
identicamente nula e $M$, munido da m\'etrica induzida, \'e 
isom\'etrica ao toro flat 
$\Sp^1(\frac{1}{\sqrt2})\times\Sp^1(\frac{1}{\sqrt2})$.

Um problema cl\'assico nessa \'area \'e construir exemplos
de superf\'icies m\'inimas mergulhadas, i.e., superf\'icies sem 
auto-interse\c c\~oes. Durante um longo tempo, o equador e 
o toro de Clifford foram os \'unicos exemplos conhecidos de 
superf\'icies m\'inimas mergulhadas em $\Sp^3$. No entanto, 
no final da d\'ecada de 1960, Lawson \cite{Lawson1970} 
descobriu uma fam\'ilia infinita de superf\'icies m\'inimas 
mergulhadas em $\Sp^3$ de genus relativamente grande.

\begin{teorema}\label{teo:lawson}
	\cite{Lawson1970}. Dado um par de inteiros positivos $n$ e $k$, existe uma superf\'icie
	m\'inima mergulhada em $\Sp^3$ de genus $nk$. Em particular, 
	existe pelo menos uma superf\'icie m\'inima mergulhada em $\Sp^3$
	de qualquer genus $g$.
\end{teorema}

Um problema natural, relacionado \`a exist\^encia de tais superf\'icies,
\'e a quest\~ao da unicidade. Em 1966, Almgren \cite{Almgren1966} 
provou que, a menos de isometrias de $\Sp^3$, o equador \'e a \'unica
superf\'icie m\'inima imersa em $\Sp^3$ de genus $0$.
Em 1970, Blaine Lawson \cite{Lawson1970a} conjecturou uma 
propriedade similar de unicidade para toros m\'inimos na esfera
$\Sp^3$. Mais precisamente,

\begin{conjectura}\label{teo:Lawson}
	\cite{Lawson1970a}. A menos de isometrias de $\Sp^3$, o toro de Clifford \'e a \'unica 
	superf\'icie m\'inima mergulhada em $\Sp^3$ de genus $1$.
\end{conjectura}

A conjectura de Lawson \'e falsa se permitirmos que a superf\'icie
tenha auto-interse\c c\~oes (cf. \cite{Lawson1969}). Em 2012, 
Simon Brendle deu uma resposta positiva para esta conjectura. 
A prova de Brendle, apresentada em \cite{Brendle2013a}, envolve
uma aplica\c c\~ao do princ\'ipio do m\'aximo a uma fun\c c\~ao 
que depende em um par de pontos. Essa t\'ecnica foi introduzida
por Huisken \cite{Huisken1998} em seu trabalho que aborda o 
fluxo de curvas mergulhadas no plano.

Nesta disserta\c c\~ao de mestrado apresentaremos a 
demonstra\c c\~ao da conjectura de Lawson seguindo
o trabalho original de Brendle \cite{Brendle2013a}, onde
a prova \'e apresentada. O texto est\'a dividido em dois
cap\'itulos, que passaremos a descrever.

No Cap\'itulo 2 apresentamos alguns fatos b\'asicos da
teoria de superf\'icies m\'inimas no espa\c co Euclidiano
$\R^3$ e na esfera $\Sp^3$. Iniciamos o cap\'itulo 
relembrando as equa\c c\~oes fundamentais de uma 
imers\~ao isom\'etrica, dando destaque para a equa\c c\~ao
de Gauss quando o espa\c co ambiente \'e uma forma 
espacial tridimensional. Nas se\c c\~oes seguintes 
apresentamos alguns resultados b\'asicos da teoria de
superf\'icies m\'inimas em $\R^3$, dando \^enfase para
a representa\c c\~ao de Weierstrass. Finalizamos o
cap\'itulo com no\c c\~oes b\'asicas de superf\'icies 
m\'inimas em $\Sp^3$, onde damos enfoque no toro de
Clifford.

O Cap\'itulo 3 ser\'a dedicado, integralmente, \`a exposi\c c\~ao
da prova da conjectura de Lawson obtida por S. Brende
\cite{Brendle2013a}.


\chapter{Instalando o abnTeX2}
\label{chapter:instalando-abntex}
\input{tex/instalando-abntex}

\chapter{Orientações gerais}
\label{chapter:orientacoes-gerais}
\input{tex/orientacoes-gerais}

\chapter{Configuração dos elementos pré-textuais}
\label{chapter:config-pre-textual}
\input{tex/config-pre-textual}

\chapter{Corpos flutuantes}
\label{chapter:corpos-flutuantes}
\input{tex/corpos-flutuantes}

\chapter{Listas}
\label{chapter:listas}
\input{tex/listas}

\chapter{Ferramentas úteis}
\label{chapter:ferramentas-uteis}
\input{tex/ferramentas-uteis}

\chapter{Citações e referências}
\label{chapter:citacoes}
\input{tex/citacoes}


% ---
% Finaliza a parte no bookmark do PDF, para que se inicie o bookmark na raiz
% ---
\bookmarksetup{startatroot}% 
% ---

% ----------------------------------------------------------
% ELEMENTOS PÓS-TEXTUAIS
% ----------------------------------------------------------
\postextual

% ----------------------------------------------------------
% Referências bibliográficas
% ----------------------------------------------------------
\bibliography{references}

% ---------------------------------------------------------------------
% GLOSSÁRIO
% ---------------------------------------------------------------------

% Arquivo que contém as definições que vão aparecer no glossário
\input{tex/glossario}
% Comando para incluir todas as definições do arquivo glossario.tex
\glsaddall
% Impressão do glossário
\printglossaries

% ----------------------------------------------------------
% Apêndices
% ----------------------------------------------------------

% ---
% Inicia os apêndices
% ---
\begin{apendicesenv}

    \chapter{Documento básico usando a classe \textit{icmc}}
    \label{chapter:documento-basico}
    \input{tex/appendix/documento-basico}
    
    \chapter{Configuração do programa JabRef}
    \label{chapter:configuracao-jabref}
    \input{tex/appendix/configuracao-jabref}

\end{apendicesenv}
% ---


% ----------------------------------------------------------
% Anexos
% ----------------------------------------------------------

% ---
% Inicia os anexos
% ---
\begin{anexosenv}

    \chapter{Páginas interessantes na Internet} 
    \label{chapter:paginas-interessantes}
    \input{tex/annex/paginas-interessantes}

\end{anexosenv}
% ---

\end{document}