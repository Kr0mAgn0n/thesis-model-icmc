% ------------------------------------------------------------------------
% ------------------------------------------------------------------------
% ICMC: Modelo de Trabalho Acadêmico (tese de doutorado, dissertação de
% mestrado e trabalhos monográficos em geral) em conformidade com 
% ABNT NBR 14724:2011: Informação e documentação - Trabalhos acadêmicos -
% Apresentação
% ------------------------------------------------------------------------
% ------------------------------------------------------------------------

% Opções: 
%   Qualificação          = qualificacao 
%   Curso                 = doutorado/mestrado
%   Situação do trabalho  = pre-defesa/pos-defesa (exceto para qualificação)
%   Versão para impressão = impressao
\documentclass[mestrado, pre-defesa]{packages/icmc}

% ---------------------------------------------------------------------------
% Pacotes Opcionais
% ---------------------------------------------------------------------------
\usepackage{rotating}           % Usado para rotacionar o texto
\usepackage[all,knot,arc,import,poly]{xy}   % Pacote para desenhos gráficos
% Este pacote pode conflitar com outros pacotes gráficos como o ``pictex''
% Então é necessário usar apenas um dos pacotes conflitantes

\usepackage{amsmath}
\usepackage{amsfonts}

\newcommand{\VerbL}{0.52\textwidth}
\newcommand{\LatL}{0.42\textwidth}
% ---------------------------------------------------------------------------


% ---
% Informações de dados para CAPA e FOLHA DE ROSTO
% ---
% Tanto na capa quanto nas folhas de rosto apenas a primeira letra da primeira palavra (ou nomes próprios) devem estar em letra maiúscula, todas as demais devem ser em letra minúscula.
\tituloPT{Prova da conjetura de Lawson}
\tituloEN{Lawson conjecture proof}
\autor[Lamas Espinoza, M. A.]{Mario Alexis Lamas Espinoza}
\genero{M} % Gênero do autor (M = Masculino / F = Feminino)
\orientador[Orientador]{Prof. Dr.}{Fernando Manfio}
%\coorientador{Prof. Dr.}{Fulano de Tal}
\curso{MAT}
\data{16}{10}{2019} % Data do depósito
\idioma{PT} % Idioma principal do documento (PT = português / EN = inglês)
% ---


% ---
% RESUMOS
% ---

% Resumo em PORTUGUÊS
% conter no máximo 500 palavras
% conter no mínimo 1 e no máximo 5 palavras-chave
\textoresumo[brazil]{
    Este trabalho é um breve modelo  para a escrita de monografias de qualificação, dissertações e teses utilizando o ambiente \LaTeX, de acordo com as normas exigidas pelo Instituto de Ciências Matemáticas e de Computação (ICMC), da Universidade de São Paulo (USP). Para a confecção deste modelo foi utilizado a última versão (1.9.6) do pacote de classes \textit{abnTeX2} que segue as normas da Associação Brasileira de Normas Técnicas. A elaboração de uma monografia, dissertação ou tese pode ser feita sobrescrevendo o conteúdo deste modelo. 
    }{Modelo, Monografia de qualificação, Dissertação, Tese, Latex}


% resumo em INGLÊS
% conter no máximo 500 palavras
% conter no mínimo 1 e no máximo 5 palavras-chave
\textoresumo[english]{
    This paper is a brief model for writing qualification monographs, dissertations and thesis using \LaTeX environment, in accordance with the standards required by the Institute of Mathematics and Computer Sciences (ICMC), University of São Paulo (USP). For making this model, the latest version (1.9.6) \textit{abnTeX2} classes package was used. This package follow the rules of the Brazilian Association of Technical Standards. A drafting a monograph, dissertation or thesis can be done by overwriting the contents of this model.
    }{Template, Qualification monograph, Dissertation, Thesis, Latex}


% ----------------------------------------------------------
% ELEMENTOS PRÉ-TEXTUAIS
% ----------------------------------------------------------

% Inserir a ficha catalográfica
\incluifichacatalografica{tex/pre-textual/ficha-catalografica.pdf}

% DEDICATÓRIA / AGRADECIMENTO / EPÍGRAFE
\textodedicatoria*{tex/pre-textual/dedicatoria}
\textoagradecimentos*{tex/pre-textual/agradecimentos}
\textoepigrafe*{tex/pre-textual/epigrafe}

% Inclui a lista de figuras
\incluilistadefiguras

% Inclui a lista de tabelas
%\incluilistadetabelas

% Inclui a lista de quadros
%\incluilistadequadros

% Inclui a lista de algoritmos
%\incluilistadealgoritmos

% Inclui a lista de códigos
%\incluilistadecodigos

% Inclui a lista de siglas e abreviaturas
%\incluilistadesiglas

% Inclui a lista de símbolos
\incluilistadesimbolos

% ----
% Início do documento
% ----
\begin{document}
% ----------------------------------------------------------
% ELEMENTOS TEXTUAIS
% ----------------------------------------------------------
\textual

\newcommand{\comando}[1]{\textbf{$\backslash$#1}}

\newcommand{\vectorfieldsspace}[1]{
	\mathfrak{X}(#1)
}

\newcommand{\normalvectorfieldsspace}[1]{
	\mathfrak{N}(#1)
}

\newcommand{\smoothfunctions}[1]{
	C^\infty(#1)
}

\newcommand{\innerproduct}[2]{
	\langle #1,#2 \rangle
}

\newcommand{\realnumbers}{
	\mathbb{R}
}

\newcommand{\xbarra}{
	\overline{x}	
}

\newcommand{\xb}{
	\overline{x}	
}

\newcommand{\ybarra}{
	\overline{y}	
}

\newcommand{\yb}{
	\overline{y}	
}

\newcommand{\R}{\mathbb R}
\newcommand{\Sp}{\mathbb{S}}


\newcommand{\pdiff}[1]{
	\frac{\partial}{\partial #1}
}

\newcommand{\partialdiff}[2]{
	\left( #1 \right)_{#2}
}

\newcommand{\partialdifffrac}[2]{\frac{\partial #1}{\partial #2}}

\newcommand{\npartialdifffrac}[3]{\frac{\partial^#3 #1}{\partial #2^#3}}

\newcommand{\conection}[2]{
	\nabla_{#1} \left( #2 \right)
}

\newcommand{\christoffel}[2]{
	\Gamma_{#1}^{#2}
}

\newcommand{\vectorfields}[1]{
	\mathfrak{X}(#1)
}

\newcommand{\liebrackets}[2]{
	\left[ #1, #2 \right]
}

\newcommand{\smoothfunction}[1]{
	C^{\infty}(#1)
}

\newcommand{\curvaturetensor}[3]{
	\mathcal{R} \left( #1,#2 \right) #3
}

\newcommand{\norm}[1]{
	\left\| #1 \right\|
}


\newcommand{\complexnumbers}{\mathbb{C}}

\DeclareMathOperator\supp{supp}

\chapter{Introdução}
\label{chapter:introducao}
Superf\'icies m\'inimas constituem hoje um dos objetos de estudo
mais importantes em Geometria Diferencial. De particular interesse, 
s\~ao as superf\'icies m\'inimas em variedades de curvatura constante, 
como o espa\c co Euclidiano $\R^3$, o espa\c co hiperb\'olico $\Hy^3$
e a esfera $\Sp^3$. O caso de superf\'icies m\'inimas em $\R^3$ \'e um
assunto cl\'assico que tem despertado a aten\c c\~ao de v\'arias 
gera\c c\~oes de ge\^ometras, desde o problema inicial proposto por 
Lagrange at\'e os dias atuais. Nesta disserta\c c\~ao daremos
\^enfase ao estudo de superf\'icies m\'inimas na esfera $\Sp^3$, 
identificando-a com a esfera unit\'aria em $\R^4$, i.e.,
\[
\Sp^3=\{x\in\R^4:x_1^2+x_2^2+x_3^2+x_4^2=1\}.
\]



Ao contr\'ario do que ocorre em $\R^3$, de que n\~ao existem 
superf\'icies m\'inimas fechadas, existem exemplos interessantes
desse fen\^omeno na esfera $\Sp^3$. Um exemplo simples de 
superf\'icie m\'inima fechada em $\Sp^3$ \'e o {\em equador}
\[
M=\{x\in\Sp^3\subset\R^4:x_4=0\}.
\]
Neste caso, as curvaturas principais s\~ao ambas iguais a zero. 
Al\'em disso, o equador tem curvatura Gaussiana
constante igual a $1$. Assim, munido da m\'etrica induzida, $M$ \'e
isom\'etrica \`a esfera usual $\Sp^2$.

Outro exemplo de superf\'icie m\'inima em $\Sp^3$ \'e o {\em toro de
	Clifford}, definido por
\[
M=\left\{x\in\Sp^3:x_1^2+x_2^2=x_3^2+x_4^2=1/2\right\}.
\]
Neste caso, as curvaturas principais s\~ao iguais a $1$ e $-1$, 
resultando em $H=0$. Al\'em disso, a curvatura Gaussiana \'e 
identicamente nula e $M$, munido da m\'etrica induzida, \'e 
isom\'etrica ao toro flat 
$\Sp^1(\frac{1}{\sqrt2})\times\Sp^1(\frac{1}{\sqrt2})$.

Um problema cl\'assico nessa \'area \'e construir exemplos
de superf\'icies m\'inimas mergulhadas, i.e., superf\'icies sem 
auto-interse\c c\~oes. Durante um longo tempo, o equador e 
o toro de Clifford foram os \'unicos exemplos conhecidos de 
superf\'icies m\'inimas mergulhadas em $\Sp^3$. No entanto, 
no final da d\'ecada de 1960, Lawson \cite{Lawson1970} 
descobriu uma fam\'ilia infinita de superf\'icies m\'inimas 
mergulhadas em $\Sp^3$ de genus relativamente grande.

\begin{teorema}\label{teo:lawson}
	\cite{Lawson1970}. Dado um par de inteiros positivos $n$ e $k$, existe uma superf\'icie
	m\'inima mergulhada em $\Sp^3$ de genus $nk$. Em particular, 
	existe pelo menos uma superf\'icie m\'inima mergulhada em $\Sp^3$
	de qualquer genus $g$.
\end{teorema}

Um problema natural, relacionado \`a exist\^encia de tais superf\'icies,
\'e a quest\~ao da unicidade. Em 1966, Almgren \cite{Almgren1966} 
provou que, a menos de isometrias de $\Sp^3$, o equador \'e a \'unica
superf\'icie m\'inima imersa em $\Sp^3$ de genus $0$.
Em 1970, Blaine Lawson \cite{Lawson1970a} conjecturou uma 
propriedade similar de unicidade para toros m\'inimos na esfera
$\Sp^3$. Mais precisamente,

\begin{conjectura}\label{teo:Lawson}
	\cite{Lawson1970a}. A menos de isometrias de $\Sp^3$, o toro de Clifford \'e a \'unica 
	superf\'icie m\'inima mergulhada em $\Sp^3$ de genus $1$.
\end{conjectura}

A conjectura de Lawson \'e falsa se permitirmos que a superf\'icie
tenha auto-interse\c c\~oes (cf. \cite{Lawson1969}). Em 2012, 
Simon Brendle deu uma resposta positiva para esta conjectura. 
A prova de Brendle, apresentada em \cite{Brendle2013a}, envolve
uma aplica\c c\~ao do princ\'ipio do m\'aximo a uma fun\c c\~ao 
que depende em um par de pontos. Essa t\'ecnica foi introduzida
por Huisken \cite{Huisken1998} em seu trabalho que aborda o 
fluxo de curvas mergulhadas no plano.

Nesta disserta\c c\~ao de mestrado apresentaremos a 
demonstra\c c\~ao da conjectura de Lawson seguindo
o trabalho original de Brendle \cite{Brendle2013a}, onde
a prova \'e apresentada. O texto est\'a dividido em dois
cap\'itulos, que passaremos a descrever.

No Cap\'itulo 2 apresentamos alguns fatos b\'asicos da
teoria de superf\'icies m\'inimas no espa\c co Euclidiano
$\R^3$ e na esfera $\Sp^3$. Iniciamos o cap\'itulo 
relembrando as equa\c c\~oes fundamentais de uma 
imers\~ao isom\'etrica, dando destaque para a equa\c c\~ao
de Gauss quando o espa\c co ambiente \'e uma forma 
espacial tridimensional. Nas se\c c\~oes seguintes 
apresentamos alguns resultados b\'asicos da teoria de
superf\'icies m\'inimas em $\R^3$, dando \^enfase para
a representa\c c\~ao de Weierstrass. Finalizamos o
cap\'itulo com no\c c\~oes b\'asicas de superf\'icies 
m\'inimas em $\Sp^3$, onde damos enfoque no toro de
Clifford.

O Cap\'itulo 3 ser\'a dedicado, integralmente, \`a exposi\c c\~ao
da prova da conjectura de Lawson obtida por S. Brende
\cite{Brendle2013a}.


\chapter{Variedades Riemannianas}
\label{chapter:variedades-riemannianas}

\cite{Lee2012}
\cite{Lee1997}
\cite{Lee1997a}
\cite{Lee1997b}



\section{Quê é uma variedade Riemanniana?}

\begin{definicao}
	Seja $M$ uma variedade diferenciável. Uma \emph{métrica Riemanniana em $M$} é um campo 2-tensorial covariante, simétrico e diferenciável que é definido positivo em cada ponto.	
\end{definicao}

\begin{definicao}
	Uma \emph{variedade Riemanniana} é um par $(M,g)$, onde $M$ é uma variedade diferenciável e $g$ é uma métrica Riemanniana em $M$.
\end{definicao}

\begin{observacao}
	Para qualquer carta $(x_1, \ldots, x_n)$, uma métrica Riemanniana pode ser escrita como
	\begin{equation*}
		g = \sum_{i,j=1}^n g_{ij} dx_i \otimes dx_j
	\end{equation*}
	onde $(g_{ij})$ é uma matriz definida positiva simétrica de funções diferenciáveis.
\end{observacao}

\begin{proposicao}
Toda variedade diferenciável admite uma métrica Riemanniana.
\end{proposicao}

\begin{proposicao}
	Supor que $(M,g)$ é uma variedade Riemanniana, e $(X_j)$ é uma estrutura local diferenciável para $M$ sobre um subconjunto aberto $U \subset M$. Então, existe uma estrutura ortonormal diferenciável $\{ (E_j)\}$ sobre $U$ tal que
	\begin{equation*}
		\text{span} \{ X_{1|p}, \ldots, X_{n|p} \} = \text{span} \{ E_{1|p}, \ldots, E_{n|p} \}
	\end{equation*}
	para cada $ j=1, \ldots, n $ e $p \in U$.
\end{proposicao}

\begin{corolario}
	Seja $(M,g)$ uma variedade Riemanniana. Para cada $p \in M$, existe uma estrutura ortonormal diferenciável em uma vizinhança de $p$.
\end{corolario}

\section{Métrica pullback}

\begin{definicao}\label{metrica_pullback}
	Supor que $M,N$ são variedades diferenciáveis, $g$ é uma métrica Riemanniana em $N$, e $F: M \rightarrow N$ é diferenciável. O \emph{pullback} $F^* g$ é um campo 2-tensorial diferenciável em $M$. Se é definido positivo, é uma métrica Riemanniana em $M$, chamada de \emph{métrica pullback} determinada pelo $F$.
\end{definicao}

\begin{proposicao}
	Supor que $F: M \rightarrow N$ é uma função diferenciável e $g$ é a métrica Riemanniana em $N$. Então, $F^* g$ é uma métrica Riemanniana em $M$ se e somente se $F$ é uma imersão diferenciável.
\end{proposicao}

\begin{definicao}
	Se $(M,g)$ e $(\tilde{M}, \tilde{g})$ são variedades Riemannianas, a função diferenciável $F: M \rightarrow \tilde{M}$ é chamada de \emph{isometria} se é um difeomorfismo que satisfaz $F^* \tilde{g} = g$.
\end{definicao}

\begin{observacao}
	Mais geralmente, $F$ é chamada de \emph{isometria local} se para cada ponto $p \in M$ existe uma vizinhança $U$ tal que $F_{|U}$ é uma isometria de $U$ com um subconjunto aberto de $\tilde{M}$.
\end{observacao}

\begin{observacao}
	Se existe uma isometria entre $(M,g)$ e $(\tilde{M}, \tilde{g})$, dizemos que ambos são variedades Riemannianas isométricas e se existe uma isometria local entre elas, então são chamadas de variedades Riemannianas localmente isométricas.
\end{observacao}

\begin{definicao}
	Uma $n$-variedade Riemanniana é chamada de \emph{variedade Riemanniana plana}, e $g$ é uma \emph{métrica plana}, se $(M,g)$ é localmente isométrica a $(\mathbb{R}^n,\overline{g})$.
\end{definicao}

\begin{observacao}
	Existem métricas Riemannianas que não são planas.
\end{observacao}

\section{Subvariedades Riemannianas}

\begin{observacao}
	Se $(M,g)$ é uma variedade Riemanniana, toda subvariedade $S \subset M$, imersa o mergulhada, automaticamente herda a métrica pullback $i^* g$, onde $i: S \rightarrow M$ é a função inclusão.
\end{observacao}

\begin{definicao}\label{metrica_induzida}
	A métrica pullback sobre a subvariedade de uma variedade Riemanniana é chamada de \emph{métrica induzida}.
\end{definicao}

\begin{observacao}
	Pela definição \ref{metrica_pullback}, seja $v,w \in T_p S$ tal que
	\begin{equation*}
		(i^* g)(v,w) = g(di_p v, di_p w) = g(v,w). 
	\end{equation*}
	Portanto, $i^* g$ é a restrição de $g$ a um par de vetores tangentes a $S$. 
\end{observacao}

\begin{definicao}
	Seja $(M,g)$ uma variedade Riemanniana e $S \subset M$ uma subvariedade imersa o mergulhada. $S$, com a métrica induzida, é chamada de \emph{subvariedade Riemanniana} de $M$.
\end{definicao}

\section{Fibrado Normal}

\begin{definicao}
	Supor que $(M,g)$ é uma $n$-variedade Riemanniana, e $S \subset M$ é uma $k$-subvariedade Riemanniana. Para qualquer $p \in S$, dizemos que um vetor $v$ é \emph{normal a S}  se é ortogonal a todo vetor em $T_p S$ com respeito ao produto interno $\langle , \rangle_g$.
\end{definicao}

\begin{definicao}
	O \emph{espaço normal a S em} $p$ é o subespaço $N_p S \subset T_p M$ que consiste de todos os vetores que são normais a $S$ em $p$.
\end{definicao}

\begin{definicao}
	O \emph{fibrado normal de} $S$ é o subconjunto $NS \subset TM$ que consiste da união de todos os espaços normais aos ponto de $S$.
\end{definicao}

\begin{definicao}
	A projeção $\pi_{NS}: NS \rightarrow S$ é a restrição a $NS$ da projeção $\pi: TM \rightarrow M$.
\end{definicao}

\section{Conexões afins}

\begin{definicao}
	Uma \emph{conexão afim} numa variedade diferenciável $M$ é uma função
	\begin{equation*}
		\nabla: \mathfrak{X}(M) \times \mathfrak{X}(M) \rightarrow \mathfrak{X}(M)
	\end{equation*}
	tal que para $X,Y \in \mathfrak{X}(M)$, satisfaz as propriedades
	\begin{enumerate}
		\item $\nabla_X Y$ é linear sobre $C^\infty (M)$ em $X$, i.e.,
		\begin{equation*}
			\nabla_{f X_1 + g X_2} Y = f \nabla_{X_1} Y + g \nabla_{X_2} Y
		\end{equation*}
		para $f,g \in C^{\infty} (M)$.
		
		\item $\nabla_X Y$ é linear sobre $\mathbb{R}$ em $Y$, i.e.,
		\begin{equation*}
			\nabla_X (a Y_1 + b Y_2) = a \nabla_X Y_1 + b \nabla_X Y_2
		\end{equation*}
		para $a,b \in \mathbb{R}$.
		
		\item $\nabla$ satisfaz a seguente regra:
		\begin{equation*}
			\nabla_X (f Y) = f \nabla_X Y + (X f) Y
		\end{equation*}
		para $f \in C^{\infty}(M)$.
	\end{enumerate}
\end{definicao}

\begin{observacao}
	$\nabla_X Y$ é também chamado de \emph{derivada covariante de} $Y$ na direção de $X$.
\end{observacao}

\begin{lema}\label{boa_definicao_conexao_1}
	Se $\nabla$ é uma conexão afim de uma variedade diferenciável $M$, $X,Y \in \mathfrak{X}(M)$, e $p \in M$, então $\nabla_X Y_{|p}$ depende só dos valores de $X$ e $Y$ numa vizinhança pequena de $p$, i.e., se $X = \tilde{X}$ e $Y = \tilde{Y}$ numa vizinhança pequena de $p$, então $\nabla_X Y_{|p} = \nabla_{\tilde{X}} \tilde{Y}_{|p}$.
\end{lema}

\begin{lema}\label{boa_definicao_conexao_2}
	No lema \ref{boa_definicao_conexao_1}, se pode adicionar que $\nabla_X Y_{|p}$ depende só dos valores de $Y$ numa vizinhança pequena de $p$ e do valor de $X$ em $p$.
\end{lema}

\begin{observacao}
	Pelo lema \ref{boa_definicao_conexao_2}, podemos escrever $\nabla_{X_p} Y$ em lugar de $\nabla_X Y_{|p}$, i.e., podemos dizer que é a derivada direcional de $Y$ em $p$ com direção $X_p$.
\end{observacao}

\begin{observacao}\label{obs_simbolos_christoffel}
	Seja $M$ uma variedade diferenciável, e $(E_i)$ uma estrutura local de $TM$ num conjunto aberto $U \subset M$. Para qualquer escolha de índices $i$ e $j$, podemos expandir $\nabla_{E_i} E_j$ em termos da mesma estrutura
	\begin{equation*}
		\nabla_{E_i} E_j = \sum_k \Gamma^k_{ij} E_k
	\end{equation*}
	Isto define $n^3$ funções $\Gamma^k_{ij}$ em $U$.
\end{observacao}

\begin{definicao} 
	As funções definidas na observação \ref{obs_simbolos_christoffel} , são chamadas de \emph{símbolos de Christoffel} de $\nabla$ com respeito à estrutura dada.
\end{definicao}

\begin{lema}
	Seja $nabla$ uma conexão afim, e seja $X,Y \in \mathfrak{X}(M)$ tal que tal que são expressados em termos uma estrutura local por $X = \sum_i X^i E_i$ e $Y = \sum_j Y^j E_j$. Então
	\begin{equation*}
		\nabla_X Y = \sum_{i,k,k} \left( X Y^k + X^i Y^j \Gamma^k_{ij} \right) E_k
	\end{equation*}
\end{lema}

\section{Quantas conexões afins existem?}

\begin{definicao}
	Em $\mathbb{R}^n$, definamos a \emph{conexão Euclideana} $\overline{\nabla}$ como
	\begin{equation*}
		\overline{\nabla}_X \left( \sum_j Y^j \partial_j \right) = \sum_j \left( X Y^j \right) \partial_j
	\end{equation*}
	onde $X,Y \in \mathfrak{X}(\mathbb{R}^n)$, e $Y = \sum_j Y^j \partial_j$.
\end{definicao}

\begin{lema}
	Supor que $M$ é uma variedade diferenciável coberta por só uma carta. Existe uma bijeção entre o conjunto das conexões afins em $M$ e o conjunto das escolhas das $n^3$ funções diferenciáveis $\{ \Gamma^k_{ij} \}$ (símbolos de Christoffel) em $M$, dada por
	\begin{equation*}
		\nabla_X Y = \sum_{i,j,k} \left( X^i \partial_i Y^k + X^i Y^j \Gamma^k_{ij} \right) \partial_k
	\end{equation*}
\end{lema}

\begin{proposicao}
	Toda variedade diferenciável admite uma conexão afim.
\end{proposicao}


\section{A conexão Riemanniana}

\begin{definicao}
	Seja $M \subset \mathbb{R}^n$ uma subvariedade diferenciável. Uma função
	\begin{equation*}
		\nabla^\top: \mathfrak{X}(M) \times \mathfrak{X}(M) \rightarrow \mathfrak{X}(M)
	\end{equation*}
	dada por
	\begin{equation*}
		\nabla^\top_X Y = \pi^\top \left( \overline{\nabla}_X Y \right)
	\end{equation*}
	onde $X,Y \in \mathfrak{X}(M)$ podem ser estendidas como campos em $\mathbb{R}^n$, $\overline{\nabla}$ é a conexão Euclideana em $\mathbb{R}^n$, e para qualquer ponto $p \in M$, $\pi^\top: T_p \mathbb{R}^n \rightarrow T_p M$ é a projeção ortogonal.
\end{definicao}

\begin{lema}
	O operador $\nabla^\top$ está bem definido, e é uma conexão em $M$.
\end{lema}

\begin{teorema}[John Nash]
	Qualquer métrica Riemanniana em uma variedade pode-se considerar como a métrica induzida em algum espaço Euclideano.
\end{teorema}

\begin{definicao}
	Seja $g$ uma métrica Riemanniana em uma variedade diferenciável $M$. Uma conexão afim $\nabla$ é chamada de \emph{compatível com} $g$ se satisfaz a regra:
	\begin{equation*}
		\nabla_X \langle Y,Z \rangle = \langle \nabla_X Y, Z \rangle + \langle Y, \nabla_X Z \rangle
	\end{equation*}
	onde $X,Y,Z \in \mathfrak{X}(M)$.
\end{definicao}

\begin{definicao}
	O \emph{tensor de torção da conexão} é um campo $ \binom{2}{1} $-tensorial $\tau: \mathfrak{X}(M) \times \mathfrak{X}(M) \rightarrow \mathfrak{X}(M)$ definido por
	\begin{equation*}
		\tau(X,Y) = \nabla_X Y - \nabla_Y X - [X,Y].
	\end{equation*}
\end{definicao}

\begin{definicao}
	Uma conexão afim é chamada de \emph{simétrica} se a torção é identicamente nula, i.e.,
	\begin{equation*}
		\nabla_X Y - \nabla_Y X = [X,Y].
	\end{equation*}
\end{definicao}

\section{Quantas conexões afins existem?}

\begin{teorema}
	Seja $(M,g)$ uma variedade Riemanniana. Existe uma única conexão afim $\nabla$ em $M$ que é compatível com $g$ e simétrica.
	Ista conexão é chamada de \emph{conexão Riemanniana} o de \emph{conexão de Levi-Civita} de $g$.
\end{teorema}

\begin{proposicao}
	Supor que $F: (M,g) \rightarrow (\tilde{M}, \tilde{g})$ é uma isometria, então $F$ leva a conexão Riemanniana $\nabla$ de $g$ à conexão Riemanniana de $\tilde{\nabla}$ de $\tilde{g}$ com a regra:
	\begin{equation*}
		F_* (\nabla_X Y) = \tilde{\nabla}_{F_* X} (F_* Y)
	\end{equation*}
\end{proposicao}

\begin{observacao}
	Sejam $(M,g)$ e $(\tilde{M}, \tilde{g})$ variedades Riemannianas, $\nabla$ e $\tilde{\nabla}$ as conexões Levi-Civita de $M$ e $\tilde{M}$ respetivamente, e $F: (M,g) \rightarrow (\tilde{M},\tilde{g})$ uma função diferenciável. A função
	\begin{equation*}
		F^* \tilde{\nabla}: \mathfrak{X}(M) \times \mathfrak{X}(M) \rightarrow \mathfrak{X}(M)
	\end{equation*}
	definida por
	\begin{equation*}
		\left( F^* \tilde{\nabla} \right)_X Y = F^{-1}_* \left( \tilde{\nabla}_{F_* X} (F_* Y) \right)
	\end{equation*}
	onde $X,Y \in \mathfrak{X}(M)$ descreve uma conexão em $M$ chamada de \emph{conexão pullback} que é simétrica e compatível com $g$. Portanto, pela unicidade da conexão Levi-Civita, $F^* \tilde{\nabla} = \nabla$.
\end{observacao}

\section{Quê é curvatura?}

\begin{definicao}
	Se $M$ é uma variedade Riemanniana, o \emph{endomorfismo de curvatura} é a função $R: \vectorfieldsspace{M} \times \vectorfieldsspace{M} \times \vectorfieldsspace{M} \rightarrow \vectorfieldsspace{M} $ definido por
	\begin{equation*}
		R(X,Y)Z = \nabla_X \nabla_Y Z - \nabla_Y \nabla_X Z - \nabla_{[X,Y]}Z.
	\end{equation*}
\end{definicao}

\begin{proposicao}
	O endomorfismo de curvatura é um campo tensorial $\binom{3}{1}$.
\end{proposicao}

\begin{definicao}
	O \emph{tensor de curvatura} é o campo covariante 4-tensorial $R_m = R^b$, definido por
	\begin{equation*}
		R_m (X,Y,Z,W) = \langle R(X,Y)Z,W \rangle
	\end{equation*}
\end{definicao}

\begin{lema}
	O endomorfismo de curvatura e o tensor de curvatura são invariantes isométricos
\end{lema}

\section{Quê mede o tensor de curvatura?}

\begin{teorema}
	Uma variedade Riemanniana é plana se e somente se seu tensor de curvatura é identicamente nulo.
\end{teorema}


\section{Propriedades do tensor de curvatura}

\begin{proposicao}
	O tensor de curvatura tem as seguentes simetrias para qualquer campo vetorial $X,Y,Z,W$:
	\begin{enumerate}
		\item $R_m(W,X,Y,Z) = -R_m(X,W,Y,Z)$.
		\item $R_m(W,X,Y,Z) = -R_m(W,X,Z,Y)$.
		\item $R_m(W,X,Y,Z) = -R_m(Y,Z,W,X)$.
		\item $R_m(W,X,Y,Z) + R_m(X,Y,W,Z) + R_m(Y,W,X,Z) = 0$.
	\end{enumerate}
\end{proposicao}


\section{A equação de Gauss}

\begin{definicao}
	A \emph{segunda forma fundamental} é a função $\alpha: \vectorfieldsspace{M} \times \vectorfieldsspace{M} \rightarrow \vectorfieldsspace{M}^\perp$ dada por
	\begin{equation*}
		\alpha(X,Y) = (\tilde{\nabla}_X Y)^\perp
	\end{equation*} 
	onde $X,Y$ são estendidos a $\tilde{M}$.
\end{definicao}

\begin{lema}
	A segunda forma fundamental é
	\begin{enumerate}
		\item independente das extensões $X$ e $Y$;
		\item bilinear sobre $\smoothfunctions{M}$; e
		\item simétrico em $X$ e $Y$.
	\end{enumerate}
\end{lema}

\begin{teorema}[A Formula de Gauss]
	Se $X,Y \in \vectorfieldsspace{M}$ são estendidos arbitrariamente a campos vetoriais de $\tilde{M}$, a seguente formula cumpre-se em $M$:
	\begin{equation*}
		\tilde{\nabla}_X Y = \nabla_X Y + \alpha(X,Y).
	\end{equation*}
\end{teorema}

\begin{lema}[A Equação de Weingraten]
	Supor que $X,Y \in \vectorfieldsspace{M}$ e $N \in \vectorfieldsspace{M}^\perp$. Quando $X,Y,N$ são estendidos a $\tilde{M}$, cumpre-se a seguente equação:
	\begin{equation*}
		\innerproduct{\tilde{\nabla}_X N}{Y} = -\innerproduct{N}{\alpha(X,Y)}.
	\end{equation*}
\end{lema}

\begin{teorema}[A Equação de Gauss]
	Para qualquer $X,Y,Z,W \in \vectorfieldsspace{M}$, cumpre-se o seguente:
	\begin{equation*}
		\tilde{R}_m(X,Y,Z,W) = R_m(X,Y,Z,W) - \innerproduct{\alpha(X,W)}{\alpha(Y,Z)} + \innerproduct{\alpha(X,Z)}{\alpha(Y,W)}.
	\end{equation*}
\end{teorema}

\section{Curvaturas seccionais}

\begin{definicao}
	Seja $M$ uma $n$-variedade Riemanniana e $p \in M$. Se $\Pi$ é um subespaço de $T_p M$ de dimensão 2, e $V \subset T_p M$ é uma vizinhança de zero  na qual $\exp_p$ é um difeomorfismo, então $S_{\Pi} = \exp_p(\Pi \cap V)$ é uma subvariedade de $M$ de dimensão 2 que contem a  $p$ chamada \emph{seção plana} determinada por $\Pi$.
\end{definicao}

\begin{definicao}
	A \emph{curvatura seccional} de $M$ associada $\Pi$, denotada por $K(\Pi)$, é a curvatura Gaussiana da superfície $S_{\Pi}$ em $p$ com a métrica induzida. Se $x,y$ é uma base para $\Pi$, é denotada também por $K(x,y)$.
\end{definicao}

\begin{proposicao}
	Se $\{ x,y \}$ é uma base para $\Pi \subset T_p M$, então
	\begin{equation*}
		K(x,y) = \frac{R_m(x,y,y,x)}{|x|^2 |y|^2 - \innerproduct{x}{y}^2}.
	\end{equation*}
\end{proposicao}

\begin{lema}
	Supor que $\mathfrak{R}_1$ e $\mathfrak{R}_2$ são 4-tensores covariantes em um espaço vetorial $V$ com um produto interno, e ambos tem as simetrias do tensor de curvatura (ref). Se para cada par de vetores linearmente independentes $x,y \in V$,
	\begin{equation*}
		\frac{\mathfrak{R}_1(x,y,y,x)}{|x|^2 |y|^2 - \innerproduct{x}{y}^2} = \frac{\mathfrak{R}_2(x,y,y,x)}{|x|^2 |y|^2 - \innerproduct{x}{y}^2},
	\end{equation*}
	então $\mathfrak{R}_1 = \mathfrak{R}_2$.
\end{lema}

\begin{lema}
	Supor que $(M,g)$ é $n$-variedade Riemanniana com curvatura seccional constante $C$. O endomorfismo de curvatura e o tensor de curvatura estão dados pelas formulas
	\begin{align*}
		R(X,Y)Z &= C(\innerproduct{Y}{Z}X - \innerproduct{X}{Z}Y),\\
		R_m(X,Y,Z,W) = C(\innerproduct{X}{W} \innerproduct{X}{Z} - \innerproduct{X}{Z} \innerproduct{Y}{W}).
	\end{align*}
\end{lema}

\chapter{Superfícies mínimas}
\label{chapter:superficies-minimas}
\section{Introdução}

\begin{definicao}
	Uma superfície regular $M$ em $\realnumbers^3$ é chamada \emph{superfície mínimas} se $H(p)=0$ para qualquer $p \in M$.
\end{definicao}

\begin{observacao}
	Se $H \equiv 0$, então $K_1 + K_2 \equiv 0$. Logo $K_1 = -K_2$
\end{observacao}

\begin{exemplo}
	Um plano em $\realnumbers^3$ é trivialmente mínima, pois $K_1=K_2=0$.
\end{exemplo}

A motivação histórica do estudo das superfícies mínimas foi dada por Lagrange o ano 1760 como seguinte problema:

Dado uma curva fechada $\gamma$ em $\realnumbers^3$, sem autointerseções, determinar a superfície de área mínima, e que tem $\gamma$ como fronteira.

Seja $M$ uma superfície regular orientada em $\realnumbers^3$, e considere uma função $f \in \smoothfunctionsspace{M}$.

\begin{definicao}
	Uma \emph{variação normal} de $M$, relativa à função $f$, é uma família de superfícies $M_t$, com $t \in (-\epsilon,\epsilon)$, dadas por:
	\begin{equation*}
	p_t = p + t f(p) N(p),
	\end{equation*}
	
	onde $N$ é o campo unitário normal a $M$, na orientação de $M$.
	
\end{definicao}

Para $\epsilon > 0$ suficientemente pequeno, cada conjunto $M_t$ também e uma superfície regular chamada uma \emph{superfície de variação}.

Note que para $t=0$, $M_0=M$. Se $f \equiv 1$, $M_t$ é uma superfície \emph{paralela} a $M$ a uma distancia $t$.

**gráfico**

Dados uma variação normal $M_t$ de $M$ relativa a uma função suave $f: M \rightarrow \realnumbers$, com $t \in (-\epsilon,\epsilon)$, e $D \subset M$ um domínio limitado, considere
\begin{equation*}
D_t = \{ p_t \in M_t: p \in D \}
\end{equation*}

para cada $t \in (-\epsilon,\epsilon)$, $D_t$ é um domínio correspondente em $M_t$. Definimos em cada $t$
\begin{equation*}
A(t) = \text{Area}(D_t)
\end{equation*}

\begin{teorema}
	\begin{equation*}
	A'(0) = -2 \int_D Hf dA
	\end{equation*}
\end{teorema}

A expressão acima chama-se a \emph{formula da primeira variação da área}.

\begin{proof}
	contenidos...
\end{proof}

\section{Superfícies mínimas em $\realnumbers^3$}

\begin{proposicao}\label{caracteristica_das_superficies_minimas}
	Uma superfície $M$ em $\realnumbers^3$ é mínima se e somente se $A'(0) = 0$.
\end{proposicao}

\begin{demonstracao}
	contenidos...
\end{demonstracao}

\begin{observacao}
	Suponha que exista uma solução $M$ para o problema de Lagrange, e considere uma variação normal $M_t$ de $M$, com $t \in (-\epsilon,\epsilon)$, dada por uma função suave $f: M \rightarrow \realnumbers$ tal que $f_{|\partial M} = 0$. Como a área de $M$ é mínima temos, em particular, que
	\begin{equation*}
	A(t) \geq A(0)
	\end{equation*}
	
	para qualquer $t \in (-\epsilon,\epsilon)$. Portanto, $A'(0)=0$, para toda variação normal $M_t$ de $M$ com $f_{|\partial M}=0$.
	
	Isso mostra, em virtude da proposição \ref{caracteristica_das_superficies_minimas}, que as superfícies de área minima são superfícies minimas no sentido da nossa definição. A reciproca é falsa!
\end{observacao}

\begin{proposicao}
	Não existe superfície minima compacta em $\realnumbers^3$.
\end{proposicao}

\begin{demonstracao}
	Se $M$ é minima, então
	\begin{equation*}
	H = \frac{1}{2} (k_1 + k_2) = 0.
	\end{equation*}
	
	Logo $k_1 = -k_2$ e se tem que $K = k_1 k_2 \leq 0$.
	
	Se $M$ for compacta, existe $p \in M$ tal que $K(p) > 0$.
\end{demonstracao}

Dada suma superfície regular $M$ em $\realnumbers^3$, considere uma carta local isoterma $(U,\varphi)$ para $M$, i.e., 
\begin{equation*}
E = G = \lambda^2 \text{ e } F=0,
\end{equation*}

onde $\lambda: U \rightarrow \realnumbers$ é uma função diferenciável com $\lambda > 0$. Note que, nas coordenadas isotermas $\varphi \sim (x,y)$, a curvatura media se expressa como
\begin{align*}
H &= \frac{eG - 2fF + gE}{2(EG - F^2)}\\
&= \frac{e + g}{2 \lambda^2}
\end{align*}

\begin{definicao}
	Dado uma função diferenciável $f: U \subset \realnumbers^2 \rightarrow \realnumbers$, o \emph{Laplaciano} de $f$, denotado por $\Delta f$, é definido por
	\begin{equation*}
	\Delta f = \frac{\partial^2 f}{\partial x^2} + \frac{\partial^2 f}{\partial y^2}.
	\end{equation*}
	
	Dizemos que $f$ é \emph{harmônica} se $\Delta f = 0$.
	
	Se $(U, \varphi)$ é uma carta local para $M$ como $\varphi = (\varphi_1, \varphi_2, \varphi_3)$, definimos
	\begin{equation*}
	\Delta \varphi = (\Delta \varphi_1, \Delta \varphi_2, \Delta \varphi_3).
	\end{equation*}
\end{definicao}

\begin{proposicao}
	Se $(U, \varphi)$ é uma carta local isoterma em $M$, então 
	\begin{equation*}
	\Delta \varphi = 2 \lambda^2 H N
	\end{equation*}
\end{proposicao}

\begin{demonstracao}
	Como $\varphi$ é isoterma, com $\varphi \sim (u, v)$, temos
	\begin{gather*}
	\innerproduct{\varphi_u}{\varphi_v} = \lambda^2 = \innerproduct{\varphi_v}{\varphi_u} \\
	\text{ e } \innerproduct{\varphi_u}{\varphi_v} = 0
	\end{gather*}
	
	Derivando, obtemos:
	\begin{align*}
	\innerproduct{\varphi_{uu}}{\varphi_u} &= \innerproduct{\varphi_{vu}}{\varphi_v}\\
	&= - \innerproduct{\varphi_u}{\varphi_{vv}}
	\end{align*}
	
	Disso decorre que
	\begin{equation}\label{eq1}
	\innerproduct{\varphi_{uu} + \varphi_{vv}}{\varphi_u} = 0
	\end{equation}
	
	Analogamente, obtemos:
	\begin{equation}\label{eq2}
	\innerproduct{\varphi_{uu} + \varphi_{vv}}{\varphi_v} = 0
	\end{equation}
	
	De \ref{eq1} e \ref{eq2} concluímos que $\varphi_{uu} + \varphi_{vv}$ é paralela a $N$. Alem disso, como
	\begin{equation}
	H = \frac{e+g}{2 \lambda^2}
	\end{equation}
	
	obtemos:
	\begin{align*}
	2 \lambda^2 H &= e + g = \innerproduct{\varphi_{uu}}{N} + \innerproduct{\varphi_{vv}}{N}\\
	&= \innerproduct{\varphi_{uu} + \varphi_{vv}}{N}.
	\end{align*}
	
	Isso mostra que
	\begin{equation}
	\Delta \varphi = 2 \lambda^2 H N
	\end{equation}
\end{demonstracao}

\begin{corolario}
	Uma superfície $M$ em $\realnumbers^3$ é minima se e somente se toda carta local isoterma é harmônica.
\end{corolario}

\begin{exemplo}
	O \emph{catenoide} é a superfície em $\realnumbers^3$ gerada pela rotação da catenária 
	\begin{equation*}
	y = a \cosh \left( \frac{z}{a} \right)
	\end{equation*}
	
	em torno ao eixo $z$.
	
	(Gráfico)
	
	Assim, o catenoide pode ser parametrizado por
	\begin{equation*}
	\varphi(u,v) = \left( a \cosh v \cos u, a \cosh v \sin u, av \right)
	\end{equation*}
	
	onde $u \in (0, 2 \pi)$ e $v \in \realnumbers$. Para tal $\varphi$, obtemos
	\begin{gather*}
	E = G = a^2 \cosh^2 v,\\
	F = 0,\\
	\varphi_{uu} + \varphi_{vv} = 0.
	\end{gather*}
	
	Portanto o catenoide e uma superfície minima.
\end{exemplo}

\begin{exemplo}
	Considere uma hélice dada por
	\begin{equation*}
	\alpha(u) = \left( \cos u, \sin u, au \right).
	\end{equation*}
	
	Por cada ponto da hélice, trace uma reta paralela ao plano $XY$ que intercepta o eixo $Z$.
	
	(Gráfico)
	
	A superfície gerada por tais retas é o \emph{helicoide} e pode ser parametrizada por
	\begin{equation*}
	\varphi(u,v) = \left( v \cos u, v \sin u, au \right)
	\end{equation*}
	
	com $u \in (0, 2 \pi)$ e $v \in \realnumbers$. Temos
	\begin{gather*}
	E = G = a^2 \cosh^2 v\\
	F = 0\\
	\varphi_{uu} + \varphi_{vv} = 0.
	\end{gather*}
	
	Portanto o helicoide é superfície minima.
\end{exemplo}

\begin{teorema}
	Alem do plano,
	\begin{enumerate}
		\item O catenoide é a única superfície rotacional minima.
		\item O helicoide é a única superfície regrada minima.
	\end{enumerate}
\end{teorema}

\begin{exemplo}
	Dado uma função diferenciável $f: U \rightarrow \realnumbers$, definida num aberto $U \subset \realnumbers^2$, considere o gráfico $\text{Gr}(f)$ de $f$, parametrizado por
	\begin{equation*}
	\varphi(x,y) = (x,y,f(x,y)), (x,y) \in U.
	\end{equation*}
	
	Temos
	\begin{align*}
	\varphi_x &= (1,0,f_x)\\
	\varphi_y &= (0,1,f_y).
	\end{align*}
	
	Assim
	\begin{align*}
	E &= \innerproduct{\varphi_x}{\varphi_x} = 1 + f_x^2\\
	F &= \innerproduct{\varphi_x}{\varphi_y} = f_x f_y\\
	G &= \innerproduct{\varphi_y}{\varphi_y} = 1 + f_y^2.
	\end{align*}
	
	Un campo $n$, normal a $\text{Gr}(f)$, é dado por
	\begin{align*}
	n = \varphi_x \times \varphi_y &= \det \left[ \begin{matrix}
	i & j & k\\
	1 & 0 & f_x\\
	0 & 1 & f_y
	\end{matrix} \right]\\
	&= (-f_x, -f_y, 1).
	\end{align*}
	
	Normalizando, temos
	\begin{equation*}
	N = \frac{n}{\norm{n}} = \frac{1}{\sqrt{1 + f_x^2 + f_y^2}}(-f_x, -f_y, 1).
	\end{equation*}
	
	Como
	\begin{align*}
	\varphi_{xx} &= (0, 0, f_{xx})\\
	\varphi_{xy} &= (0, 0, f_{xy})\\
	\varphi_{yy} &= (0, 0, f_{yy})
	\end{align*}
	
	obtemos
	\begin{align*}
	e &= \innerproduct{\varphi_{xx}}{N} = \frac{f_{xx}}{\sqrt{1 + f_x^2 + f_y^2}}\\
	f &= \innerproduct{\varphi_{xy}}{N} = \frac{f_{xy}}{\sqrt{1 + f_x^2 + f_y^2}}\\
	g &= \innerproduct{\varphi_{yy}}{N} = \frac{f_{yy}}{\sqrt{1 + f_x^2 + f_y^2}}.
	\end{align*}
	
	Assim, como
	\begin{equation*}
	H = \frac{eG - 2fF + gE}{2(EG - F^2)}
	\end{equation*}
	
	segue que se $H \equiv 0$, temos
	\begin{equation}\label{edp_superficies_minimas}
	(1 + f_y^2) f_{xx}  - 2 f_x f_y f_{xy} + (1+f_x^2) f_{yy} = 0
	\end{equation}
	
	que é uma EDP de 2da ordem.
	
	Um exemplo trivial da equação \ref{edp_superficies_minimas} é a função linear
	\begin{equation*}
	f(x,y) = ax + by + c,
	\end{equation*}
	
	como $a, b, c \in \realnumbers$.
\end{exemplo}

\begin{exemplo}[Superfície de Scherk]
	Suponha que
	\begin{equation*}
	f(x,y) = g(x) + h(y).
	\end{equation*}
	
	Neste caso, a equação \ref{edp_superficies_minimas} pode ser escrita como
	\begin{equation*}
	(1 + (h')^2(y)) g''(x) + (1 + (g')^2(x)) h''(y) = 0,
	\end{equation*}
	
	ou seja
	\begin{equation*}
	\frac{g''(x)}{1 + (g')^2(x)} + \frac{h''(y)}{1 + (h')^2(y)} = 0.
	\end{equation*}
	
	Isso implica
	\begin{equation*}
	\frac{g''(x)}{1 + (g')^2(x)} = - \frac{h''(y)}{1 + (h')^2(y)} = \text{constante}.
	\end{equation*}
	
	Integrando, obtemos (a menos de constantes) que
	\begin{align*}
	g(x) &= \ln (\cos x)\\
	h(y) &= -\ln (\cos y).
	\end{align*}
	
	A menos de dilatações e translações uma parte da superfície pode ser representada pelo gráfico da função
	\begin{equation*}
	\ln \left( \frac{\cos x}{\cos y} \right), 0 < x,y < \frac{\pi}{2}
	\end{equation*}
\end{exemplo}

\subsection{A representação de Weierstrass}

Considere o plano complexo $\complexnumbers$ identificado com $\realnumbers^2$
\begin{equation*}
(x,y) \in \realnumbers^2 \mapsto x + iy \in \complexnumbers.
\end{equation*}

Uma função complexa $f: U \subset \complexnumbers \rightarrow \complexnumbers$ pode ser escrita na forma
\begin{equation*}
f(u,v) = f_1(u,v) + i f_2(u,v)
\end{equation*}

onde $f_1, f_2: U \rightarrow \realnumbers$ são funções reais, denotadas por
\begin{align*}
f_1 &= \Re(f)\\
f_2 &= \Im(f)
\end{align*}

tal que $\Re(f)$ é parte real da função $f$ e $\Im(f)$ é a parte imaginaria da função $f$.

\begin{definicao}
	Uma função $f: U \subset \complexnumbers \rightarrow \complexnumbers$, definida no aberto $U$, é dita \emph{holomorfa} se $f_1, f_2$ possuem derivadas parciais continuas e satisfazem as equações de Cauchy-Riemann
	\begin{align*}
	\partialdifffrac{f_1}{u} &= \partialdifffrac{f_2}{v}\\
	\partialdifffrac{f_1}{v} &= - \partialdifffrac{f_2}{u}
	\end{align*}
\end{definicao}

\begin{definicao}
	Uma carta local $(U, \varphi)$ em $M$ é dita \emph{mínima} se $H(p) = 0, \forall p \in \varphi(U)$.
\end{definicao}

\begin{corolario}\label{equiv_isoterma_harmonica}
	Seja $(U, \varphi)$ uma carta local isoterma de uma superfície $M \subset \realnumbers^3$. Então $(U, \varphi)$ é mínima se e somente se $\varphi$ é harmônica, i.e., $\varphi_{uu} + \varphi_{vv} = 0$.
\end{corolario}

Dadas uma superfície $M \subset \realnumbers^3$ e uma carta local $(U, \varphi)$ em $M$, com
\begin{equation*}
\varphi(u,v) = (x_1(u,v), x_2(u,v), x_3(u,v)),
\end{equation*}

considere as funções complexas $f_j: U \subset \complexnumbers \rightarrow \complexnumbers, 1 \leq j \leq 3,$ dadas por
\begin{equation}\label{carta_isoterma_cauchy-riemann}
f_j = \partialdifffrac{x_j}{u} - i \partialdifffrac{x_j}{v}, 1 \leq j \leq 3
\end{equation}

\begin{lema}
	Seja $(U, \varphi)$ uma carta local isoterma em $M$. Então, $\varphi$ é mínima se e somente se cada $f_j$, definida em \ref{carta_isoterma_cauchy-riemann}, é holomorfa.
\end{lema}

\begin{demonstracao}
	Pelo corolário \ref{equiv_isoterma_harmonica}, temos que $\varphi$ é mínima se e somente se $\varphi$ é harmônica, i.e., $\varphi_{uu} + \varphi_{vv} = 0$. Isso significa que
	\begin{equation*}
	\npartialdifffrac{x_j}{u}{2} + \npartialdifffrac{x_j}{v}{2} = 0, 1 \leq j \leq 3.
	\end{equation*}
	
	Queremos provar que
	\begin{align*}
	\pdiff{u} \Re(f_j) &= \pdiff{v} \Im(f_j),\\
	\pdiff{v} \Re(f_J) &= - \pdiff{u} \Im(f_j)
	\end{align*}
	
	Assim
	\begin{multline*}
	\pdiff{u} \Re(f_J) = \pdiff{u} \partialdifffrac{x_j}{u} = \npartialdifffrac{x_j}{u}{2} = - \npartialdifffrac{x_j}{v}{2} = \pdiff{v} \left( - \partialdifffrac{x_j}{v} \right) = \pdiff{v} \Im(f_j)
	\end{multline*}
	
	Isso prova a primeira equação de Cauchy-Riemann. Por outro lado, como a superfície é regular, vale
	\begin{equation*}
	\varphi_{uv} = \varphi_{vu},
	\end{equation*}
	
	ou seja
	\begin{equation*}
	\frac{\partial^2 x_j}{\partial u \partial v} = \frac{\partial^2 x_j}{\partial v \partial u}.
	\end{equation*}
	
	Assim
	\begin{align*}
	\pdiff{v} \Re(f_j) &= \pdiff{v} \partialdifffrac{x_j}{u} = \pdiff{u} \partialdifffrac{x_j}{v}\\
	&= \pdiff{u} \left( - \Im(f_j) \right)\\
	&= - \pdiff{u} \Im(f_j),
	\end{align*}
	
	que é a segunda equação de Cauchy-Riemann.
\end{demonstracao}

\begin{lema}\label{lema_fj_2}
	Sejam $M \subset \realnumbers^3$ uma superfície mínima e $(U, \varphi)$ uma carta local isoterma. Então, as funções holomorfas $f_j$, definidas em \ref{carta_isoterma_cauchy-riemann}, satisfazem
	\begin{gather}\label{sum_fj_2}
	f_1^2 + f_2^2 + f_3^2 = 0\\ \label{sum_norm_fj_2}
	|f_1|^2 + |f_2|^2 + |f_3|^2 \neq 0
	\end{gather}
	
	Reciprocamente, sejam $f_1, f_2, f_3$ funções holomorfas, definidas num aberto simplesmente conexo $U \subset \complexnumbers$, satisfazendo \ref{sum_fj_2} e \ref{sum_norm_fj_2}. Então, tais funções dão origem a uma carta local isoterma mínima $(U, \varphi)$.
\end{lema}

\begin{demonstracao}
	Seja $(U, \varphi)$ a carta local isoterma em $M$. Então,
	\begin{align*}
	f_1^2 + f_2^2 + f_3^2 &= \sum_{j=1}^{3} \left[ \left( \partialdifffrac{x_j}{u} \right)^2 - \left( \partialdifffrac{x_j}{v} \right)^2 - 2i \partialdifffrac{x_j}{u} \partialdifffrac{x_j}{v} \right]\\
	&= E - G - 2iF = 0,
	\end{align*}
	
	pois $E=G$ e $F=0$. A equação \ref{sum_norm_fj_2} segue da regularidade de $\varphi$, pois $\varphi_u \neq 0$ e $\varphi_v \neq 0$.
	
	Reciprocamente, defina
	\begin{equation}\label{carta_isoterma_eq_integral}
	x_j(u,v) = \int_{\xi_0}^{\xi} f_j(z) dz, 1 \leq j \leq 3
	\end{equation}
	
	com $\xi = (u,v) \in U$, para algum $\xi_0 \in U$ fixado. Note que cada $x_j$ está bem definida pois $U$ é simplesmente conexo e $f_j$ é holomorfa, o que nos dá uma função holomorfa definida em $U$, para o qual podemos aplicar as equações de Cauchy-Riemann, obtendo:
	\begin{align*}
	\frac{d}{d \xi} \int_{\xi_0}^{\xi} f_j &= \frac{d}{d \xi} \left[ \Re \int_{\xi_0}^{\xi} f_j + i \Im \int_{\xi_0}^{\xi} f_j \right]\\
	&= \pdiff{u} \Re \int_{\xi_0}^{\xi} f_j + i \pdiff{u} \Im \int_{\xi_0}^{\xi} f_j\\
	&= \pdiff{u} \Re \int_{\xi_0}^{\xi} f_j - i \pdiff{v} \Re \int_{\xi_0}^{\xi} f_j
	\end{align*}
	
	de modo que a equação \ref{carta_isoterma_cauchy-riemann} é válida. Considere a aplicação $\varphi: U \rightarrow \realnumbers^3$, cujas funções coordenadas
	\begin{equation*}
	\varphi = (x_1,x_2,x_3)
	\end{equation*}
	
	são dadas como em \ref{carta_isoterma_eq_integral}. De \ref{sum_fj_2} e \ref{sum_norm_fj_2} segue que $(U,\varphi)$ é uma carta local isoterma. Além disso, as funções $f_j$ serem holomorfas implicam que as funções coordenadas $x_j$ são harmônicas, logo, pelo corolário, $\varphi$ é mínima.
\end{demonstracao}

\begin{observacao}
	As funções $x_j$ definidas em \ref{carta_isoterma_eq_integral} estão definidas a menos de uma constante aditiva, de modo que a superfície está definida a menos de uma translação. Assim, o estudo local de superfícies mínimas em $\realnumbers^3$ reduz-se a resolver as equações \ref{sum_fj_2} e \ref*{sum_norm_fj_2} para uma terna de funções holomorfas.
\end{observacao}

\begin{teorema}
	Sejam $f: U \subset \complexnumbers \rightarrow \complexnumbers$ uma função holomorfa e $g: U \rightarrow \complexnumbers$ uma função meromorfa tais que $fg^2$ seja holomorfa. Assuma que se $\xi \in U$ é um polo de ordem $n$ para $g$ então $\xi$ é um zero para $f$ de ordem $2n$, e que estes sejam os únicos zeros de $f$.
	
	Então, a aplicação
	\begin{equation}\label{carta_minima_duas_funcoes}
	\varphi(z) = \frac{1}{2} f(z) \left( (1-g(z)^2), i (1+g(z)^2), 2g(z) \right)
	\end{equation}
	
	satisfaz as condições do Lema \ref{lema_fj_2}. Além disso, para toda tal $\varphi$, existem funções holomorfa $f$ e meromorfa $g$ tais que vale \ref{carta_minima_duas_funcoes}.
\end{teorema}

\begin{demonstracao}
	Se $\varphi$ satisfaz \ref{carta_minima_duas_funcoes}, temos
	\begin{align*}
	f_1^2 + f_2^2 + f_3^2 &= \frac{1}{4} f(z)^2 (1 - g(z)^2)^2 - \frac{1}{4} f(z)^2 (1 + g(z)^2)^2 + f(z)^2 g(z)^2\\
	&= 0
	\end{align*}
	
	Afirmamos que $\varphi(z) \neq 0, \forall z \in U$. De fato, a hipótese sobre os zeros de $f$ e os polos de $g$ implicam que $f(z) g(z)^2 \neq 0$. Assim, para qualquer $z$ fixado, a primeira e a segunda coordenada de $\varphi$ não podem ser ambas nulas.
	
	Assim, podemos assumir que $\varphi$ é holomorfa satisfazendo
	\begin{equation*}
	\varphi_1^2 + \varphi_2^2 + \varphi_3^2 \not\equiv 0,
	\end{equation*}
	
	$\varphi$ nunca é zero, e considere
	\begin{align*}
	f(z) &= \varphi_1(z) - i \varphi_2(z)\\
	g(z) &= \frac{\varphi_3(z)}{\varphi_1(z) - i \varphi_2(z)}
	\end{align*}
	
	$f$ é uma função holomorfa e $g$ é o quociente de funções holomorfas. Se o denominador de $g$ é identicamente nulo, façamos
	\begin{equation*}
	g(z) = \frac{\varphi_3(z)}{\varphi_1(z) + i \varphi_2(z)}
	\end{equation*}
	
	e proceder de forma similar.
	
	Assim, sendo o denominador de $g$ não nulo, tem-se que $g$ é meromorfa. Assim, a relação
	\begin{equation*}
	\varphi_1^2 + \varphi_2^2 + \varphi_3^2 = 0
	\end{equation*}
	
	implica
	\begin{equation*}
	(\varphi_1 + i \varphi_2)(\varphi_1 - i \varphi_2) = -\varphi_3^2
	\end{equation*}
	
	que, em termos de $f$ e $g$ torna-se
	\begin{align*}
	\varphi_1 + i \varphi_2 &= \frac{-\varphi_3^2}{\varphi_1 - i \varphi_2}\\
	&= \frac{-\varphi_3^2}{(\varphi_1 - i \varphi_2)^2} (\varphi_1 - i \varphi_2)\\
	&= -fg^2
	\end{align*}
	
	Esta última equação, juntamente com as condições sobre $f$ e $g$, nos dão $\varphi$ como em \ref{carta_minima_duas_funcoes}.
\end{demonstracao}

\begin{definicao}
	Sejam $U \subset \complexnumbers$ um aberto simplesmente conexo e $\gamma \subset U$ uma curva de um ponto fixado $z_0 \in U$ a um ponto arbitrário $z \in U$, $z = u + iv$.
	
	Sejam $f,g$ como no teorema anterior. Então,
	\begin{equation*}
	\varphi(u,v) = (x_1(u,v), x_2(u,v), x_3(u,v)),
	\end{equation*}
	
	onde
	\begin{align*}
	x_1 &= \Re \int_{\gamma} \frac{1}{2} f(z) (1 - g(z)^2) dz\\
	x_2 &= \Re \int_{\gamma} \frac{1}{2} f(z) (1 + g(z)^2) dz\\
	x_3 &= \Re \int_{\gamma} f(z) g(z) dz
	\end{align*}
	
	é uma carta local mínima, chamada \emph{a representação de Weierstrass} da teoria local de superfícies mínimas.
\end{definicao}

\begin{exemplo}[Catenoide]
	O catenoide pode ser representado pelas funções holomorfas $f, g: \complexnumbers \rightarrow \complexnumbers$ dadas por
	\begin{align*}
	f(z) &= e^{-z},\\
	g(z) &= e^z.
	\end{align*}
	
	Substituindo tais funções na formula da representação de Weierstrass e integrando de $z_0 = 0$ a um ponto arbitrário $z = u + iv$, obtemos
	\begin{align*}
	\varphi(u,v) &= x_0 + \Re \int_{z_0}^{z} \frac{f(\xi)}{2} (1 - g(\xi)^2, i (1 + g(\xi)^2), 2 g(\xi)) d\xi\\
	&= x_0 + \Re \int_{z_0}^{z} \frac{e^{-\xi}}{2} (1 - e^{2\xi}, i (1 + e^{2\xi}), 2e^{\xi}) d\xi\\
	&= \Re \int_{0}^{z} \frac{1}{2} (e^{-\xi} - e^{\xi}, i (e^{-\xi} + e^{\xi}), 1) d\xi\\
	&= \Re \left[ \frac{1}{2} \left(-e^{-z} - e^z, -\frac{1}{2i} (-e^{-z} + e^z), z \right) \right] \\
	&= \Re \left( -\cosh z, i \sinh z, z \right) \\
	&= \left( -\cosh u \cos v, -\cosh u \sin v, u \right)
	\end{align*} 
\end{exemplo}

\begin{exemplo}[Superfície de Enneper]
	A superfície de Enneper pode ser representada pelas funções holomorfas $f,g: \complexnumbers \rightarrow \complexnumbers$ dadas por
	\begin{align*}
	f(z) &= 1, \\
	g(z) &= z.
	\end{align*}
	
	Assim, a representação de Weierstrass torna-se
	\begin{align*}
	\varphi(u,v) &= \Re \left( \frac{1}{2} \int_{0}^{z} \left( 1 - \xi^2, i (1 + \xi^2), 2\xi \right) \right) d\xi \\
	&= \frac{1}{2} \Re \left( z - \frac{z^3}{3}, iz + \frac{iz^3}{3}, z^2 \right) \\
	&= \frac{1}{2} \left( u - \frac{u^3}{3} + uv^2, -v + \frac{v^3}{3} - u^2v, u^2 - v^2 \right)
	\end{align*}
	
	gráfico
\end{exemplo}

\begin{exemplo}[Superfície de Scherk]
	A superfície de Scherk, definida pela equação
	\begin{equation*}
	e^z = \frac{\cos y}{\cos x},
	\end{equation*}
	
	pode ser representada pelas funções holomorfas $f: \complexnumbers \setminus \{\pm 1, \pm i \} \rightarrow \complexnumbers$ e $g: \complexnumbers \rightarrow \complexnumbers$ dadas por
	\begin{align*}
	f(z) &= \frac{2}{1 - z^4}, \\
	g(z) &= z.
	\end{align*}
	
	Note que
	\begin{align*}
	f (1 - g^2) &= \frac{2}{1 + z^2} = \frac{i}{z + i} - \frac{i}{z - i}, \\
	i f (1 + g^2) &= \frac{2i}{1 - z^2} = \frac{i}{z + 1} - \frac{i}{z - 1}, \\
	2fg &= \frac{4z}{1 - z^4} = \frac{2z}{z^2 + 1} - \frac{2z}{z^2 - 1}.
	\end{align*}
	
	Assim, substituindo na representação de Weierstrass e integrando, obtemos
	\begin{equation*}
	\varphi(z) = \left( -\arg \frac{z + i}{z - i}, -\arg \frac{z + i}{z - i}, \log \left\| \frac{z^2 + 1}{z^2 - 1} \right\| \right).
	\end{equation*}
	
	Usando as identidades
	\begin{align*}
	\frac{z + i}{z - i} &= \frac{|z|^2 - 1}{|z^2 - i|^2} + i \frac{z + \overline{z}}{|z - i|^2}, \\
	\frac{z + 1}{z - 1} &= \frac{|z|^2 - 1}{|z - 1|^2} + i \frac{\overline{z} - z}{|z - 1|^2},
	\end{align*}
	
	podemos encontrar as expressões para $\cos x$ e $\cos y$. Temos
	\begin{align*}
	\cos x &= \cos \left( -\arg \frac{z + i}{z - i} \right) \\
	&= \cos \left( \arg \frac{z + i}{z - i} \right) \\
	&= \cos \left( \arg \left( \frac{|z - i|}{|z + i|} \frac{z + i}{z - i} \right) \right) \\
	&= \Re \left( \frac{|z - i|}{|z + i|} \frac{z + i}{z - i} \right) \\
	&= \frac{|z - i|}{|z + i|} \Re \left( \frac{z + i}{z - i} \right) \\
	&= \frac{|z - i|}{|z + i|} \frac{|z|^2 - 1}{|z - i|^2} = \frac{|z|^2 - 1}{|z^2 + 1|}.
	\end{align*}
	
	Analogamente, temos
	\begin{align*}
	\cos y &= \cos \left( -\arg \frac{z + 1}{z - 1} \right) \\
	&= \frac{|z - 1|}{|z + 1|} \frac{|z|^2 - i}{|z - 1|^2} \\
	&= \frac{|z|^2 - i}{|z^2 - 1|}
	\end{align*}
	
	Isso implica que
	\begin{equation*}
	\frac{\cos y}{\cos x} = \frac{z^2 + 1}{|z^2 - 1|} = e^z
	\end{equation*}
	
	Vejamos uma aplicação da representação de Weierstrass. Dado uma superfície mínima $M \subset \realnumbers^3$, seja $(U, \varphi)$ uma carta local isoterma.
	
	Isso significa que
	\begin{align*}
	E = G &= \lambda^2, \\
	F &= 0
	\end{align*}
	
	onde
	\begin{align*}
	\lambda^2 &= \frac{1}{2} \sum_{j=1}^{3} |f_j|^2 \\
	&= \frac{1}{4} |f|^2 |1 + g|^2 + \frac{1}{4} |f|^2 |1 + g|^2 + |fg|^2\\
	&= \left( \frac{|f| (| + |g|^2)}{2} \right)^2
	\end{align*} 
	
	Além disso, temos
	\begin{align*}
	\varphi_u \times \varphi_v &= \left( \Im (f_2 \overline{f}_3), \Im (f_3 \overline{f}_1), \Im (f_1 \overline{f}_2) \right) \\
	&= \frac{|f|^2 (1 + |g|^2)}{4} \left( 2 \Re(g), 2 \Im(g), |g|^2 - | \right), \\
	\| \varphi_u \times \varphi_v \| &= \sqrt{EG - F^2} = \lambda^2,
	\end{align*}
	
	de modo que
	\begin{equation*}
	N = \left( \frac{2 \Re(g)}{|g|^2 + 1}, \frac{2 \Im(g)}{|g|^2 + 1}, \frac{|g|^2 - 1}{|g|^2 + 1} \right)
	\end{equation*}
	
	Lembremos que a projeção estereográfica
	\begin{equation*}
	\pi: S^2 \setminus \{ (0,0,1) \} \rightarrow \complexnumbers
	\end{equation*}
	
	é a aplicação
	\begin{equation*}
	\pi(x_1, x_2, x_3) = \frac{x_1 + ix_2}{1 - x_3},
	\end{equation*}
	
	e uma inversa é dada por
	\begin{equation*}
	\pi^{-1}(z) = \left( \frac{2 \Re(z)}{|z|^2 + 1}, \frac{2 \Im(z)}{|z|^2 + 1}, \frac{|z|^2 - 1}{|z|^2 + 1} \right)
	\end{equation*}
	
	Portanto, temos que
	\begin{equation*}
	N = \pi^{-1} \circ g
	\end{equation*}
	
	Podemos resumir isso no seguinte resultado.
\end{exemplo}

\begin{proposicao}
	Sejam $M \subset \realnumbers^3$ uma superfície mínima e $(U, \varphi)$ uma carta local isoterma. Então, um dos campos unitários $N$, normal a $M$ é a inversa da projeção estereográfica da função $g$ dada pela representação de Weierstrass.
\end{proposicao}

\begin{corolario}
	Seja $M \subset \realnumbers^3$ uma superfície mínima definida no plano todo. Então, ou $M$ é um plano ou a imagem da aplicação de Gauss omite pelo menos dois pontos.
\end{corolario}

\begin{demonstracao}
	Se $M$ não está contida num plano, podemos construir a função $g$ que é meromorfa no plano todo $\complexnumbers$. Pelo teorema de Picard, ela atinge todos seus valores com, pelo menos, duas excepções, ou $g$ é constante. A equação (referencia) mostra que o mesmo se aplica a $N$ e, no último caso, $M$ está contida em um plano.
\end{demonstracao}

\begin{teorema}[Existência local de parâmetros isotermos]
	Seja $M \subset \realnumbers^3$ uma superfície mínima. Então, todo $p \in M$ pertence a uma vizinhança coordenada isoterma.
\end{teorema}

\begin{demonstracao}
	Seja $U \subset M$ uma vizinhança coordenada de $p$ que é o gráfico de uma função diferenciável que, podemos assumir, ser da forma $z = h(x,y), (x,y) \in U$.
	
	Lembrando que a equação para gráficos mínimos e
	\begin{equation*}
	(1 + h_y^2) h_{xx} - 2 h_x h_y h_{xy} + (1 + h_x^2) h_{yy} = 0,
	\end{equation*}
	
	obtemos a equação
	\begin{equation*}
	\pdiff{x} \frac{1 + h_y^2}{W} = \pdiff{y} \frac{h_x h_y}{W}
	\end{equation*}
	
	em $U$, onde $W = \sqrt{1 + h_x^2 + h_y^2}$. Escolhendo $U$ simplesmente conexo, isso implica que existe uma função diferenciável $\phi: U \rightarrow \realnumbers$, com
	\begin{align*}
	\partialdifffrac{\phi}{x} &= \frac{h_x h_y}{W} \\
	\partialdifffrac{\phi}{y} &= \frac{1 + h_y^2}{W}
	\end{align*}
	
	Introduza novas coordenadas
	\begin{align*}
	\overline{x} &= x \\
	\overline{y} &= \phi(x,y)
	\end{align*}
	
	Um calculo simples mostra que
	\begin{align*}
	\partialdifffrac{x}{\overline{x}} &= 1, \\
	\partialdifffrac{x}{\overline{y}} &= 0, \\
	\partialdifffrac{y}{\overline{x}} &= -\frac{h_x h_y}{1 + h_y^2}, \\
	\partialdifffrac{y}{\overline{y}} &= \frac{W}{1 + h_y^2},
	\end{align*}
	
	e os coeficientes da Segunda Forma Fundamental, em relação a $(\overline{x}, \overline{y})$ são
	\begin{align*}
	E = G &= \frac{W^2}{1 + h_y^2} \\
	F &= 0
	\end{align*}
	
	como queríamos.
\end{demonstracao}

\section{Superfícies mínimas em $S^3$}

\cite{Brendle2013}

\begin{definicao}
	A esfera unitária em $\realnumbers^4$, $S^3$, é o conjunto definido por:
	\begin{equation*}
	S^3 = \left\{ x \in \realnumbers^4: \| x \| = 1 \right\}
	\end{equation*}
\end{definicao}

\begin{definicao}
	Uma superfície em $S^3$ chama-se de \emph{superfície mínima} se a curvatura media é identicamente nula.
\end{definicao}

\begin{teorema}\label{propriedades_sup_min_S3}
	Seja $\Sigma$ uma superfície em $S^3$. As seguentes afirmações são equivalentes:
	\begin{enumerate}
		\item $\Sigma$ é uma superfície mínima
		\item $\Sigma$ é um ponto critico do funcional de área
		\item Se $\Sigma$ for descrito pelas funções coordenadas $x_i: \realnumbers \rightarrow \realnumbers$ onde $i=1,2,3,4$, então tem-se:
		\begin{equation*}
		\Delta_{\Sigma} x_i + 2 x_i = 0
		\end{equation*}
		para $i=1,2,3,4$.
	\end{enumerate}
\end{teorema}

%\begin{observacao}
%	No existem superfície mínimas compactas em $\realnumbers^3$.
%\end{observacao}

\begin{definicao}
	O \emph{equador} é um subconjunto de $S^3$ definido por:
	\begin{equation}
	M = \left\{ x \in S^3: x_4 = 0 \right\}
	\end{equation}
\end{definicao}

\begin{proposicao}
	As curvaturas principais do equador são nulas.
\end{proposicao}

\begin{demonstracao}
	É claro que o campo de vetores normais unitário é dado por $(0,0,0,1)$. Portanto a diferencial do campo é nulo e as curvaturas são nulas.
\end{demonstracao}

\begin{corolario}
	O equador é uma superfície mínima em $S^3$.
\end{corolario}

\begin{demonstracao}
	Como a equação da curvatura media em $S^3$ está dada pela soma das curvaturas principais, então, pela proposição anterior, a curvatura media é nula. Portanto é uma superfície mínima.
\end{demonstracao}

\begin{definicao}
	O \emph{toro de Clifford} é o subconjunto de $S^3$ definido por:
	\begin{equation*}
	M = \left\{ x \in S^3: x_1^2 + x_2^2 = x_3^2 + x_4^2 = \frac{1}{2} \right\}.
	\end{equation*}
\end{definicao}

\begin{proposicao}
	As curvaturas principais do toro de Clifford são 1 e -1.
\end{proposicao}

\begin{proposicao}
	O toro de Clifford é uma superfície mínima plana.
\end{proposicao}

\subsection{Quantas superfícies mínimas mergulhadas em $S^3$ existem?}

\begin{observacao}
	Por muito tempo o equador e o toro de Clifford foram os únicos exemplos de superfícies mínimas mergulhadas em $S^3$.
\end{observacao}

\begin{teorema}[Lawson]
	Existem ao menos uma superfície mínima mergulhada em $S^3$ de gênero $g$, onde $g$ é um numero natural.
\end{teorema}


\subsection{Unicidade das superfícies de gênero 0 e 1}

\begin{teorema}[Almgren]
	O equador é a única superfície mínima imersa de gênero 0 em $S^3$, salvo movimentos rígidos
\end{teorema}

\begin{observacao}[Conjetura de Lawson]
	O toro de Clifford é uma única superfície mínima mergulhada de gênero 1 em $S^3$, salvo movimentos rígidos
\end{observacao}

\section{O toro de Clifford}

Seja
$x: \R^2 \rightarrow S^3$
definido por
\begin{equation*}
	x(u,v) = \frac{1}{\sqrt{2}} \left(\cos(\sqrt{2} u), \sin(\sqrt{2} u), \cos(\sqrt{2} v), \sin(\sqrt{2} v)\right).
\end{equation*}
e $\Sigma$ a imagem de $x$. Calculando a diferencial de $x$ tem-se
\begin{equation*}
	dx_{u,v} = \left(-\sin(\sqrt{2} u) du, \cos(\sqrt{2} u) du, -\sin(\sqrt{2} v) dv, \cos(\sqrt{2} v) dv\right).
\end{equation*}
Observa-se que $dx_{u,v}$ é injetiva porque
\begin{align*}
	dx_{u,v} \frac{\partial}{\partial u} &= \left(-\sin(\sqrt{2} u), \cos(\sqrt{2} u), 0, 0\right)\\
	dx_{u,v} \frac{\partial}{\partial v} &= \left(0, 0, -\sin(\sqrt{2} v), \cos(\sqrt{2} v)\right)
\end{align*}
são linearmente independentes. Ver que
$T_{x(u,v)} \Sigma = \text{span} \left\{dx_{u,v} \frac{\partial}{\partial u}, dx_{u,v} \frac{\partial}{\partial v}\right\}$
e é subespaço vetorial de
$T_{x(u,v)} S^3$.
Como $x(u,v)$ é ortogonal a $T_{x(u,v)} \Sigma$ poderia-se considerar
\begin{equation*}
	T_{x(u,v)} S^3 = \text{span} \left\{dx_{u,v} \frac{\partial}{\partial u}, dx_{u,v} \frac{\partial}{\partial v}, x(u,v)\right\}
\end{equation*}
mas vamos mostrar que isso não é possível.
Seja 
$p \in S^3$,
$ v \in T_p S^3 $
e
$\lambda: (-\epsilon, \epsilon) \rightarrow S^3$
um caminho tal que
$\lambda(0) = p$
e
$\lambda'(0) = v$.
Então
\begin{equation*}
	\innerproduct{\lambda(t)}{\lambda(t)} = 1.
\end{equation*}
Derivando com respeito a $t$ tem-se
\begin{equation*}
	\innerproduct{\lambda'(t)}{\lambda(t)} = 0.
\end{equation*}
Logo, avaliando em $ t=0 $, obtém-se
$ \innerproduct{v}{p} = 0 $.
Portanto
$ v \in \text{span} \{p\}^\perp $.
Se $x(u,v) \in T_{x(u,v)} S^3$ teríamos que $x(u,v) \in \text{span} \left\{x(u,v)\right\}^\perp$ o que é um absurdo.
Para obter uma base para $T_{x(u,v)} S^3$ definamos
\begin{equation*}
	\eta(u,v) = \frac{1}{\sqrt{2}} \left(-\cos(\sqrt{2} u), -\sin(\sqrt{2} u), \cos(\sqrt{2} v), \sin(\sqrt{2} v)\right)
\end{equation*}
onde pode-se ver que
$\eta \in \text{span} \{x(u,v)\}^\perp$ e
$\eta$ é ortogonal a $T_{x(u,v)} \Sigma$.
Portanto tem-se que
$T_{x(u,v)} S^3 = \text{span} \left\{dx_{u,v} \frac{\partial}{\partial u}, dx_{u,v} \frac{\partial}{\partial v}, \eta\right\}$.
Calculando a diferencial de $\eta$ obtemos
\begin{equation*}
	d\eta_{u,v} = \left(\sin(\sqrt{2} u) du, -\cos(\sqrt{2} u) du, -\sin(\sqrt{2} v) dv, \cos(\sqrt{2} v) dv\right).
\end{equation*}
Calculando as entradas da matriz da segunda forma fundamental
\begin{align*}
	h_{11} = \innerproduct{d\eta_{u,v} \frac{\partial}{\partial u}}{dx_{u,v} \frac{\partial}{\partial u}} &= -1,\\
	h_{22} = \innerproduct{d\eta_{u,v} \frac{\partial}{\partial v}}{dx_{u,v} \frac{\partial}{\partial v}} &= 1,\\
	h_{12} = \innerproduct{d\eta_{u,v} \frac{\partial}{\partial u}}{dx_{u,v} \frac{\partial}{\partial v}} &= 0,
\end{align*}
construímos a matriz
\begin{equation*}
	A^{\eta} = \left[\begin{matrix}
	-1 & 0\\
	0 & 1
	\end{matrix}\right].
\end{equation*}
Observando a matriz é claro que a curvatura média 
$H = 0$,
a curvatura extrínseca 
$K_{\text{ext}} = -1$ e, pela equação de Gauss,
$K_{\text{int}} = 0$.

\chapter{A Conjectura de Lawson}
\label{chapter:a-conjectura-de-lawson}
Neste capítulo apresentaremos a demonstração da conjetura de
Lawson obtida por Simon Brendle \cite{Brendle2013a}.


\section{Introdu\c c\~ao}

Em 1970, Blaine Lawson \cite{Lawson1970a} conjecturou que o toro de 
Clifford é a única superfície mínima, compacta e mergulhada em $\Sp^3$,
com genus $1$, e tal conjectura foi provada somente quatro décadas depois por Simon
Brendle \cite{Brendle2013a} em 2013. A hipótese de ser mergulhada é
fundamental. De fato, em \cite{Lawson1970} Lawson construiu uma 
família (infinita) de imersões mínimas de toros em $\Sp^3$.

A prova da conjetura de Lawson em \cite{Brendle2013a} envolve uma 
aplicação do princípio do máximo para uma função que depende de 
duas vari\'aveis. Esta técnica foi inicialmente desenvolvida por Huisken
\cite{Huisken1998} no estudo do fluxo de comprimento de curvas para 
curvas mergulhadas no plano e, posteriormente, por Andrews 
\cite{Andrews2012}.

Antes de apresentarmos os argumentos usados em \cite{Brendle2013a},
na prova da conjectura de Lawson, faremos algumas considera\c c\~oes
iniciais. Seguiremos aqui o artigo \cite{Andrews2012}, onde o autor
apresenta a no\c c\~ao de n\~ao-colapsante para hipersuperf\'icies
Euclidianas mergulhadas.

Considere uma superf\'icie $\Sigma$ em $\R^3$, cuja curvatura
m\'edia $H$ \'e positiva em todos os pontos, e limitando uma 
regi\~ao aberta $\Omega$ em $\R^3$.

\begin{definicao}
A superf\'icie $\Sigma$ \'e dita ser {\em $\delta$-n\~ao-colapsante}
se, para todo ponto $x\in\Sigma$, existe uma bola aberta $B$ de
raio $\delta/H(x)$, contida em $\Omega$, com $x\in\partial B$. 
\end{definicao}

Dado uma superf\'icie $F:\Sigma\to\R^3$, defina uma fun\c c\~ao
$Z:\Sigma\times\Sigma\to\R$ pondo
\[
Z(x,y) = \frac{H(x)}{2}\|F(y)-F(x)\|^2 + \delta\langle F(y)-F(x), \nu(x)\rangle,
\]
onde $\nu$ \'e um campo unit\'ario, normal a $\Sigma$.

\begin{proposicao}[\cite{Andrews2012}]\label{prop:Int}
A superf\'icie $\Sigma$ \'e $\delta$-n\~ao-colapsante se, e somente
se, $Z(x,y)\geq0$, para quaisquer $x,y\in\Sigma$.
\end{proposicao}
\begin{demonstracao}
Sem perda de generalidade, escolha o campo normal $\nu$ que
aponta para fora de $\Sigma$. Assim, uma bola em $\Omega$ de
raio $\delta/H(x)$, com $F(x)$ sendo um ponto da fronteira, deve
ter centro no ponto
\[
p(x) = F(x) - (\delta/H(x))\nu(x).
\]
A afirma\c c\~ao de que esta bola est\'a contida em $\Omega$
\'e equivalente ao fato de que nenhum ponto de $\Sigma$ tem
dist\^ancia menor do que $\delta/H(x)$ ao ponto $p$. Ou seja,
\[
0\leq\|F(y)-p(x)\|^2-\left(\frac{\delta}{H(x)}\right)^2 = 
2\cdot\frac{Z(x,y)}{H(x)},
\]
para quaisquer $x,y\in\Sigma$. Como $H>0$ em todos os pontos
de $\Sigma$, isso \'e equivalente ao fato de que $Z$ seja
n\~ao-negativa em todos os pontos. A rec\'iproca \'e imediata.
\end{demonstracao}

Considere agora uma superf\'icie m\'inima e mergulhada 
$F:\Sigma\to\Sp^3$ e $ \Phi: \Sigma\to\R$ uma função positiva. Defina
uma função $Z:\Sigma\times\Sigma\to\R$ pondo
\begin{equation}\label{def_de_Z}
Z(x,y) = \Phi(x) (1 - \langle F(x),F(y) \rangle ) + \langle \nu(x), F(y) \rangle,
\end{equation}
onde $\nu$ \'e um campo unitário, normal a $\Sigma$. Observando que
uma bola geod\'esica na esfera $\Sp^3$ \'e simplesmente a interseção
de uma bola de $\R^4$ com $\Sp^3$ podemos provar, de forma
an\'aloga \`a Proposi\c c\~ao \ref{prop:Int}, que a fun\c c\~ao $Z(x,y)$,
definida em \eqref{def_de_Z}, \'e n\~ao-negativa se, e somente se,
a superf\'icie $\Sigma$ \'e n\~ao-colapsante.




\section{Alguns resultados t\'ecnicos}

Nesta se\c c\~ao apresentaremos alguns resultados preliminares
que ser\~ao usados na demonstra\c c\~ao da conjectura de
Lawson. 

\vspace{.2cm}

Dados uma superfície mínima e mergulhada $F:\Sigma\to\Sp^3$ 
e uma função positiva $\Phi: \Sigma\to\R$, considere a função 
$Z:\Sigma\times\Sigma\to\R$ dada por
\begin{eqnarray}\label{eq:funcaoZ}
Z(x,y) = \Phi(x) (1 - \langle F(x), F(y) \rangle) + \langle \nu(x), F(y) \rangle,
\end{eqnarray}
onde $\nu$ \'e um campo unitário, normal a $\Sigma$. Considere dois 
pontos $\xb, \yb \in \Sigma$, com $\xb \neq \yb$, tais que $Z(\xb, \yb) = 0$ e $dZ(\xb, \yb) = 0$. Sejam $(x_1,x_2), (y_1,y_2)$ sistemas de coordenadas geodésicas em torno dos pontos $\xb, \yb$, respectivamente. No ponto $(\xb,\yb)$, temos:
\begin{eqnarray}\label{diff_Z_x}
\begin{aligned}
0 = \frac{\partial Z}{\partial x_i} (\xb, \yb) = &
\frac{\partial \Phi}{\partial x_i}(\xb) (1 - \langle F(\xb, F(\yb) \rangle) -  
\Phi(\xb) \left\langle \frac{\partial F}{\partial x_i}(\xb), F(\yb) \right\rangle \\ & + \sum_{k=1}^{2} h_{ik}(\xb) \left\langle \frac{\partial F}{\partial x_k}(\xb), F(\yb) \right\rangle
\end{aligned}
\end{eqnarray}
e
\begin{equation}\label{diff_Z_y}
0 = \frac{\partial Z}{\partial y_i} (\xb, \yb) = - \Phi(\xb) \left\langle F(\xb), \frac{\partial F}{\partial y_i} (\yb) \right\rangle + \left\langle \nu(\xb), \frac{\partial F}{\partial y_i}(\yb) \right\rangle
\end{equation}
onde $h_{ij}(\xb)$ denota a $(ij)$-ésima coordenada da matriz da segunda forma fundamental de $F$ no ponto $\xb$, i.e.,
\begin{eqnarray}\label{eq:partialNu}
\frac{\partial\nu}{\partial x_i}(\xb) = 
\sum_{k=1}^2h_{ik}\frac{\partial F}{\partial x_k}(\xb).
\end{eqnarray}

\vspace{.2cm}

Sem perda de generalidade, podemos supor que a segunda forma
fundamental de $F$ est\'a diagonalizada no ponto $\xb$, i.e.,
\[
h_{11}(\xb)=\lambda_1, \quad h_{12}(\xb) = 0 \quad\mbox{e}\quad
h_{22}(\xb) = \lambda_2.
\]
Denotemos por $w_i$ a reflex\~ao do vetor
\[
\frac{\partial F}{\partial x_i}(\xb)
\]
em rela\c c\~ao ao plano ortogonal ao vetor $F(\xb) - F(\yb)$, i.e.,
\begin{eqnarray}\label{eq:reflexao}
w_i = \frac{\partial F}{\partial x_i}(\xb) - 
2 \left\langle \frac{\partial F}{\partial x_i}(\xb), \frac{F(\xb) - 
F(\yb)}{|F(\xb) - F(\yb)|} \right\rangle \frac{F(\xb) - 
F(\yb)}{|F(\xb) - F(\yb)|}.
\end{eqnarray}
Escolhendo um sistema de coordenadas apropriado $(y_1,y_2)$,
podemos supor que
\begin{equation*}
\left\langle w_1, \frac{\partial F}{\partial y_1}(\yb) \right\rangle \geq 0, 
\quad 
\left\langle w_1, \frac{\partial F}{\partial y_2}(\yb) \right\rangle = 0 
\quad \text{e} \quad 
\left\langle w_2, \frac{\partial F}{\partial y_2}(\yb) \right\rangle \geq 0.
\end{equation*}

\begin{lema}
Os vetores $F(\yb)$ e $ \Phi(\xb) F(\xb) - \nu(\xb) $ são linearmente independentes.
\end{lema}
\begin{demonstracao}
Usando a identidade
\[
\langle \Phi(\xb) F(\xb) - \nu(\xb), F(\yb) \rangle = 
\Phi(\xb) - Z(\xb,\yb) = \Phi(\xb),
\]
obtemos
\begin{eqnarray*}
| \Phi(\xb) F(\xb) - \nu(\xb) |^2 | F(\yb) |^2 &-& 
\langle \Phi(\xb) F(\xb) - \nu(\xb), F(\yb) \rangle^2 \\
&=& | \Phi(\xb) F(\xb) - \nu(\xb) |^2 - \Phi(\xb)^2 = 1 \neq 0.
\end{eqnarray*}
Disso decorre que a desigualdade de Cauchy-Schwarz \'e estrita,
logo os vetores $F(\yb)$ e $\Phi(\xb)F(\xb)-\nu(\xb)$
s\~ao linearmente independentes.
\end{demonstracao}

Em rela\c c\~ao \`a reflex\~ao dada em \eqref{eq:reflexao}, obtemos o
seguinte:

\begin{lema} \label{lem:w_1 w_2}
Valem as seguintes igualdades:
\begin{equation*}
w_1 = \frac{\partial F}{\partial y_1}(\yb) 
\quad \text{e} \quad 
w_2 = \frac{\partial F}{\partial y_2}(\yb).
\end{equation*}
\end{lema}
\begin{demonstracao}
Usando a express\~ao de $w_i$, dada em \eqref{eq:reflexao}, obtemos:
\begin{eqnarray*}
\langle w_i, F(\yb) \rangle &=& \left\langle \frac{\partial F}{\partial x_i}(\xb), F(\yb) \right\rangle + 2 \left\langle \frac{\partial F}{\partial x_i}, F(\yb) \right\rangle \frac{\langle F(\xb) - F(\yb),F(\yb) \rangle}{| F(\xb) - F(\yb) |^2} \\
&=& \left\langle \frac{\partial F}{\partial x_i}(\xb), F(\yb) \right\rangle 
+ 2 \left\langle \frac{\partial F}{\partial x_i}, F(\yb) \right\rangle \frac{\langle F(\xb),F(\yb) \rangle - 1}{2 - 2 \langle F(\xb), F(\yb) \rangle} \\
&=& 0
\end{eqnarray*}
e
\begin{eqnarray*}
\langle w_i, \Phi(\xb) F(\xb) - \nu(\xb) \rangle &=&
2 \left\langle \frac{\partial F}{\partial x_i}(\xb), F(\yb) \right\rangle \frac{\langle F(\xb) - F(\yb), \Phi(\xb) F(\xb) - \nu(\xb) \rangle}{| F(\xb) - F(\yb) |^2} \\
& = & 2 \left\langle \frac{\partial F}{\partial x_i}(\xb), F(\yb) \right\rangle \frac{Z(\xb,\yb)}{| F(\xb) - F(\yb) |^2} \\
& = & 0.
\end{eqnarray*}
Por outro lado, os vetores
\begin{equation*}
\frac{\partial F}{\partial y_1}(\yb) \quad \text{e} \quad \frac{\partial F}
{\partial y_2}(\yb)
\end{equation*}
satisfazem
\begin{equation*}
\left\langle \frac{\partial F}{\partial y_i}(\yb), F(\yb)\right\rangle = 0
\end{equation*}
e
\begin{equation*}
\left\langle \frac{\partial F}{\partial y_i}(\yb), \Phi(\xb) F(\xb) - 
\nu(\xb) \right\rangle = -\frac{\partial Z}{\partial y_i}(\xb,\yb) 
= 0.
\end{equation*}
Como os vetores $F(\yb)$ e $ \Phi(\xb) F(\xb) - \nu(\xb)$
são linearmente independentes, conclu\'imos que o plano gerado por
\begin{equation*}
\frac{\partial F}{\partial y_1}(\yb) \quad \text{e} \quad 
\frac{\partial F}{\partial y_2}(\yb)
\end{equation*}
é o mesmo plano gerado por $w_1$ e $w_2$. Al\'em disso, os vetores
$w_1$ e $w_2$ s\~ao ortonormais. Como
\begin{equation*}
\left\langle w_1, \frac{\partial F}{\partial y_2}(\yb) \right\rangle = 0,
\end{equation*}
conclu\'imos que
\begin{equation*}
w_1 = \pm \frac{\partial F}{\partial y_1}(\yb) 
\quad \text{e} \quad 
w_2 = \pm \frac{\partial F}{\partial y_2}(\yb).
\end{equation*}
Finalmente, como
\begin{equation*}
\left\langle w_1, \frac{\partial F}{\partial y_1}(\yb) \right\rangle \geq 0 
\quad \text{e} \quad 
\left\langle w_2, \frac{\partial F}{\partial y_2}(\yb) \right\rangle \geq 0
\end{equation*}	
obtemos que
\begin{equation*}
w_1 = \frac{\partial F}{\partial y_1}(\yb) \quad \text{e} \quad 
w_2 = \frac{\partial F}{\partial y_2}(\yb),
\end{equation*}
como quer\'iamos.
\end{demonstracao}

No resultado seguinte iremos considerar as derivadas de segunda
ordem da fun\c c\~ao $Z$ no ponto $(\xb,\yb)$.

\begin{proposicao}
A derivada segunda da fun\c c\~ao $Z$, dada em \eqref{eq:funcaoZ},
satisfaz:
\begin{eqnarray*}\label{2_diff_Z_x}
\sum_{i=1}^{2} \frac{\partial^2 Z}{\partial x_i^2} (\xb,\yb) &=& 
\left( \Delta_{\Sigma} \Phi(\xb) - \frac{| \nabla \Phi(\xb) |^2}{\Phi(\xb)} 
+ (|A(\xb)|^2 - 2) \Phi(\xb) \right) (1 - \left\langle F(\xb), F(\yb) \right\rangle) \\ 
&&+ 2 \Phi(\xb) - \frac{2 \Phi(\xb)^2 - |A(\xb)|^2}{2 \Phi(\xb)
 (1 - \langle F(\xb), F(\yb) \rangle)} \sum_{i=1}^2 \left\langle 
 \frac{\partial F}{\partial x_i}(\xb), F(\yb) \right\rangle^2.
\end{eqnarray*}
\end{proposicao}
\begin{demonstracao}
Derivando a equação \eqref{diff_Z_x} em rela\c c\~ao a $x_i$,
e somando, obtemos:
\begin{eqnarray}\label{Z_seg_dev_x}
\begin{aligned}
\sum_{i=1}^2\frac{\partial^2 Z}{\partial x_i^2}(\xb,\yb) = &
\sum_{i=1}^2\frac{\partial^2 \Phi}{\partial x_i^2}(\xb)
(1 - \langle F(\xb), F(\yb) \rangle) -
2\sum_{i=1}^2\frac{\partial \Phi}{\partial x_i}(\xb) 
\left\langle \frac{\partial F}{\partial x_i}(\xb), F(\yb) \right\rangle \\
& - \sum_{i=1}^2\Phi(\xb) \left\langle\frac{\partial^2 F}{\partial x_i^2}(\xb),F(\yb)
\right\rangle + \sum_{i,k=1}^2\frac{\partial h_{ik}}{\partial x_i}(\xb)
\left\langle \frac{\partial F}{\partial x_k}(\xb), F(\yb)\right\rangle \\ 
&+ \sum_{i,k=1}^2h_{ik}(\xb)\left\langle
\frac{\partial^2 F}{\partial x_i \partial x_k}(\xb), F(\yb)\right\rangle.
\end{aligned}
\end{eqnarray}
Iremos, inicialmente, reescrever a equa\c c\~ao \eqref{Z_seg_dev_x}.
Note que, da equa\c c\~ao de Codazzi \eqref{eq-codazzi}, tem-se
\begin{eqnarray}\label{eq:Prop4.5-1}
\frac{\partial h_{11}}{\partial x_2}(\xb) = \frac{\partial h_{21}}{\partial x_1}(\xb)
\quad\mbox{e}\quad
\frac{\partial h_{12}}{\partial x_2}(\xb) = \frac{\partial h_{22}}{\partial x_1}(\xb),
\end{eqnarray}
e do fato de $\Sigma$ ser m\'inima, obtemos
\begin{eqnarray}\label{eq:Prop4.5-2}
\sum_{i=1}^2\frac{\partial h_{ii}}{\partial x_k}(\xb) = 0.
\end{eqnarray}
De \eqref{eq:Prop4.5-1} e \eqref{eq:Prop4.5-2}, obtemos
\begin{eqnarray}\label{eq:Prop4.5-3}
\sum_{i=1}^2\frac{\partial h_{ik}}{\partial x_i}(\xb) = 0.
\end{eqnarray}
Analisemos o termo 
$\displaystyle\sum_{i=1}^2 \frac{\partial^2 F}{\partial x_i^2}(\xb)$. 
Pelo teorema \ref{propriedades_sup_min_S3}, tem-se
\begin{equation*}
\sum_{i=1}^2 \frac{\partial^2 F_j}{\partial x_i^2}(\xb) + 2F_j(\xb) = 0,
\end{equation*}
para $j=1,\ldots,4$, logo,
\[
\sum_{i=1}^2 \frac{\partial^2 F}{\partial x_i^2}(\xb) = -2 F(\xb).
\]
Analisemos agora o termo 
\[
\sum_{i=1}^2 h_{ik}(\xb) \left\langle \frac{\partial^2 F}
{\partial x_i \partial x_k}(\xb), F(\yb) \right\rangle.
\]
Derivamos \eqref{eq:partialNu} em rela\c c\~ao a $x_i$, somando
e usando \eqref{eq:Prop4.5-3}, obtemos:
\begin{eqnarray}\label{eq:Prop4.5-4}
\sum_{i=1}^2 \frac{\partial^2 \nu}{\partial x_i^2}(\xb) = 
\sum_{i,k=1}^2 h_{ik}(\xb)\frac{\partial^2 F}{\partial x_i \partial x_k}(\xb).
\end{eqnarray}
Note que, como
\[
\left\langle \sum_{i=1}^2 \frac{\partial^2 \nu}{\partial x_i^2}(\xb), 
\frac{\partial F}{\partial x_j}(\xb) \right\rangle = 0,
\]
para $j=1,2$, segue que o vetor
$\displaystyle\sum_{i=1}^2 \frac{\partial^2 \nu}{\partial x_i^2}(\xb)$ 
não tem componentes no plano tangente 
$T_{\xb}\Sigma\subset T_{\xb}\Sp^3$. 
Por outro lado, derivando
\[
\left\langle \frac{\partial \nu}{\partial x_i}(\xb), \nu(\xb) \right\rangle = 0
\]
em rela\c c\~ao a $x_i$, obtemos:
\begin{equation}\label{eq:Prop4.5-5}
\left\langle \frac{\partial^2 \nu}{\partial x_i^2}(\xb), \nu(\xb)\right\rangle 
+ \left\langle \frac{\partial \nu}{\partial x_i}(\xb), 
\frac{\partial \nu}{\partial x_i}(\xb) \right\rangle = 0.
\end{equation}	
Substituindo \eqref{eq:partialNu} em \eqref{eq:Prop4.5-5}, notando que
$\left\langle \frac{\partial F}{\partial x_k}(\xb), 
\frac{\partial F}{\partial x_j}(\xb) \right\rangle = \delta_{kj}$ e somando,
a equa\c c\~ao \eqref{eq:Prop4.5-5} torna-se
\begin{equation*}
\left\langle \sum_{i=1}^2 \frac{\partial^2 \nu}{\partial x_i^2}(\xb), \nu(\xb)
\right\rangle + \sum_{i=1}^2 (h_{ik}(\xb))^2 = 0.
\end{equation*}	
Como $\displaystyle\sum_{i=1}^2 (h_{ik}(\xb))^2 = |A(\xb)|^2$ e
lembrando que 
$\displaystyle\sum_{i=1}^2 \frac{\partial^2 \nu}{\partial x_i^2}(\xb)$ 
só tem componente na direção $\nu$, obtemos
\begin{eqnarray} \label{eq:Prop4.5-6}	
\sum_{i=1}^2 \frac{\partial^2 \nu}{\partial x_i^2}(\xb) = - | A(\xb) |^2 \nu(\xb).
\end{eqnarray}
Usando \eqref{eq:Prop4.5-4} e \eqref{eq:Prop4.5-6}, temos que
\begin{eqnarray} \label{eq:Prop4.5-7}
\sum_{i=1}^2 h_{ik}(\xb) \frac{\partial^2 F}{\partial x_i \partial x_k}(\xb) 
= - | A(\xb) |^2 \nu(\xb).
\end{eqnarray}	
Assim, usando \eqref{eq:Prop4.5-3} e \eqref{eq:Prop4.5-7}, podemos
escrever a equa\c c\~ao \eqref{Z_seg_dev_x} como:
\begin{eqnarray}
\begin{aligned} \label{lap_Z_x}
\sum_{i=1}^2 \frac{\partial^2 Z}{\partial x_i^2}(\xb,\yb) & =   
\sum_{i=1}^2 \frac{\partial^2 \Phi}{\partial x_i^2}(\xb)
(1 - \langle F(\xb), F(\yb) \rangle)    
+ 2  \Phi(\xb) \left\langle F(\xb), F(\yb) \right\rangle \\
& -2 \sum_{i=1}^2 \frac{\partial \Phi}{\partial x_i}(\xb) 
\left\langle \frac{\partial F}{\partial x_i}(\xb), F(\yb) \right\rangle
- | A(\xb) |^2 \left\langle \nu(\xb), F(\yb) \right\rangle.
\end{aligned}
\end{eqnarray}
Como
\[
\left\langle \nu(\xb), F(\yb) \right\rangle = Z(\xb,\yb)
-\phi(\xb)(1-\langle F(\xb),F(\yb)\rangle)
\]
podemos reescrever \eqref{lap_Z_x} como sendo
\begin{eqnarray*}
\begin{aligned}
\sum_{i=1}^2 \frac{\partial^2 Z}{\partial x_i^2}(\xb,\yb) = &
\left(\Delta_{\Sigma} \Phi(\xb) + 
(|A(\xb)|^2 - 2)\Phi(\xb)\right)(1 - \langle F(\xb), F(\yb) \rangle) \\
&+ 2 \Phi(\xb) \left\langle F(\xb), F(\yb) \right\rangle
- 2 \sum_{i=1}^2 \frac{\partial \Phi}{\partial x_i}(\xb) 
\left\langle \frac{\partial F}{\partial x_i}(\xb), F(\yb) \right\rangle \\
&- |A(\xb)|^2Z(\xb,\yb).
\end{aligned}
\end{eqnarray*}	
Somando e subtraindo
\[
\frac{|\nabla \Phi(\xb)|^2}{\Phi(\xb)} (1 - \langle F(\xb), F(\yb) \rangle) + \frac{\Phi(\xb)}{1 - \langle F(\xb),F(\yb) \rangle} \sum_{i=1}^2 \left\langle \frac{\partial F}{\partial x_i} (\xb), F(\yb) \right\rangle^2,
\]
obtemos:
\begin{eqnarray*}
\begin{aligned}
\sum_{i=1}^2 \frac{\partial^2 Z}{\partial x_i^2}(\xb,\yb) &=
\left(\Delta_{\Sigma} \Phi(\xb) - \frac{|\nabla \Phi(\xb)|^2}{\Phi(\xb)} + (|A(\xb)|^2 - 2)\Phi(\xb)\right)(1 - \langle F(\xb), F(\yb) \rangle) \\
&+ 2 \Phi(\xb) + \frac{|\nabla \Phi(\xb)|^2}{\Phi(\xb)} (1 - \langle F(\xb), F(\yb) \rangle) - 2 \sum_{i=1}^2 \frac{\partial \Phi}{\partial x_i}(\xb) \left\langle \frac{\partial F}{\partial x_i}(\xb), F(\yb) \right\rangle \\
&+ \frac{\Phi(\xb)}{1 - \langle F(\xb),F(\yb) \rangle} \sum_{i=1}^2 \left\langle \frac{\partial F}{\partial x_i} (\xb), F(\yb) \right\rangle^2 \\
& - \frac{\Phi(\xb)}{1 - \langle F(\xb),F(\yb) \rangle} \sum_{i=1}^2
\left\langle \frac{\partial F}{\partial x_i} (\xb), F(\yb) \right\rangle^2.
\end{aligned}
\end{eqnarray*}
Fatorando a express\~ao acima adequadamente, podemos escrever
\begin{eqnarray*}
\begin{aligned}
\sum_{i=1}^2 \frac{\partial^2 Z}{\partial x_i^2}(\xb,\yb) &= \left(\Delta_{\Sigma} \Phi(\xb) - \frac{|\nabla \Phi(\xb)|^2}{\Phi(\xb)} + (|A(\xb)|^2 - 2)\Phi(\xb)\right)(1 - \langle F(\xb), F(\yb) \rangle)+ 2 \Phi(\xb)	\\
& - \frac{\Phi(\xb)}{1 - \langle F(\xb),F(\yb) \rangle} \sum_{i=1}^2 \left\langle \frac{\partial F}{\partial x_i} (\xb), F(\yb) \right\rangle^2 \\
& + \frac{1}{\Phi(\xb)(1 - \langle F(\xb), F(\yb) \rangle)}m(\xb,\yb),
\end{aligned}
\end{eqnarray*}	
onde
\begin{eqnarray*}
\begin{aligned}
m(\xb,\yb) &= - 2 \sum_{i=1}^2 \frac{\partial \Phi}{\partial x_i}(\xb) 
(1 - \langle F(\xb), F(\yb) \rangle) \Phi(\xb) \left\langle \frac{\partial F}
{\partial x_i}(\xb), F(\yb) \right\rangle \\
&+ |\nabla \Phi(\xb)|^2 (1 - \langle F(\xb), F(\yb) \rangle)^2 
 + \Phi(\xb)^2 \sum_{i=1}^2 \left\langle \frac{\partial F}
 {\partial x_i} (\xb), F(\yb) \right\rangle^2.
\end{aligned}
\end{eqnarray*}	
Observe que
\[
m(\xb,\yb) = \sum_{i=1}^2 \left(  \frac{\partial \Phi}{\partial x_i}(\xb)
(1 - \langle F(\xb),F(\yb) \rangle) - \Phi(\xb) \left\langle 
\frac{\partial F}{\partial x_i}(\xb), F(\yb) \right\rangle \right)^2.
\]
Pela equação \eqref{diff_Z_x}, e lembrando que 
$\frac{\partial Z}{\partial x_i}(\xb,\yb)=0$, podemos escrever
\begin{eqnarray} \label{eq:Prop4.5-8}
m(\xb,\yb) = \sum_{i=1}^2 \left( h_{ik}(\xb) 
\left\langle \frac{\partial F}{\partial x_k}(\xb), F(\yb) \right\rangle \right)^2.
\end{eqnarray}	
Expandindo o termo quadr\'atico em \eqref{eq:Prop4.5-8}, obtemos:
\begin{eqnarray*}
\begin{aligned}
(h_{i1}(\xb))^2 \left\langle \frac{\partial F}{\partial x_1}(\xb), F(\yb)\right\rangle^2 
&+ 
(h_{i2}(\xb))^2 \left\langle \frac{\partial F}{\partial x_2}(\xb),F(\yb)\right\rangle^2 \\
&+ 
2 h_{i1}(\xb) h_{i2}(\xb) \left\langle \frac{\partial F}
{\partial x_1}(\xb), F(\yb) \right\rangle \left\langle 
\frac{\partial F}{\partial x_2}(\xb), F(\yb) \right\rangle.
\end{aligned}
\end{eqnarray*}	
Como o determinante de $A$ \'e $-\lambda^2$, tem-se
\begin{equation*}
h_{11}(\xb) h_{22}(\xb) - h_{12}(\xb) h_{21}(\xb) = h_{11}(\xb) h_{22}(\xb) - (h_{12}(\xb))^2 = -\lambda^2,
\end{equation*}	
pois $h_{12}(\xb) = h_{21}(\xb)$. Al\'em disso, como 
$h_{11}(\xb) + h_{22}(\xb) = 0$, tem-se
\[
\sum_{i=1}^2 (h_{i1}(\xb))^2 = \lambda^2 \quad\mbox{e}\quad
\sum_{i=1}^2 (h_{i2}(\xb))^2 = \lambda^2.
\]
Em rela\c c\~ao ao termo $\displaystyle\sum_{i=1}^2 h_{i1}(\xb) h_{i2}(\xb)$,
tem-se
\begin{equation*}
h_{11}(\xb) h_{12}(\xb) + h_{21}(\xb) h_{22}(\xb) = h_{12}(\xb) (h_{11}(\xb) + h_{22}(\xb)) = 0.
\end{equation*}	
Portanto, podemos reescrever \eqref{eq:Prop4.5-8} como sendo
\begin{equation*}
m(\xb,\yb) = \sum_{i=1}^2 \lambda^2 \left\langle 
\frac{\partial F}{\partial x_i}(\xb), F(\yb) \right\rangle^2.
\end{equation*}	
Lembrando agora que $\lambda^2 = \frac{1}{2} |A(\xb)|$, obtemos
a equação desejada.
\end{demonstracao}


\begin{proposicao}
Em rela\c c\~ao \`as derivadas mistas, vale a seguinte rela\c c\~ao:
\begin{equation}\label{diff_Z_x_y}
\frac{\partial^2 Z}{\partial x_i \partial y_i}(\xb,\yb) = \lambda_i - \Phi(\xb).
\end{equation}
\end{proposicao}
\begin{demonstracao}
Derivando em rela\c c\~ao  a $x_i$ a equação \eqref{diff_Z_y}, obtemos:
\begin{eqnarray*}
\begin{aligned}
\frac{\partial^2 Z}{\partial x_i \partial y_i}(\xb,\yb) =  & 
-\frac{\partial \Phi}{\partial x_i}(\xb) \left\langle F(\xb), 
\frac{\partial F}{\partial y_i}(\yb) \right\rangle - \Phi(\xb)
\left\langle \frac{\partial F}{\partial x_i}(\xb), 
\frac{\partial F}{\partial y_i}(\yb) \right\rangle \\
& + \left\langle \frac{\partial \nu}{\partial x_i}(\xb), 
\frac{\partial F}{\partial y_i}(\yb) \right\rangle.
\end{aligned}
\end{eqnarray*}	
Lembrando que 
\[
\frac{\partial \nu}{\partial x_i}(\xb) = 
\lambda_i(\xb) \frac{\partial F}{\partial x_i}(\xb),
\]
temos
\begin{eqnarray} \label{eq:Prop2.3_1}
\frac{\partial^2 Z}{\partial x_i \partial y_i}(\xb,\yb) = -\frac{\partial \Phi}{\partial x_i}(\xb) \left\langle F(\xb), \frac{\partial F}{\partial y_i}(\yb) \right\rangle + (\lambda_i(\xb) - \Phi(\xb)) \left\langle \frac{\partial F}{\partial x_i}(\xb), \frac{\partial F}{\partial y_i}(\yb) \right\rangle.
\end{eqnarray}	
Da equa\c c\~ao \eqref{diff_Z_x}, tem-se:
\begin{eqnarray} \label{eq:Prop2.3_2}
-\frac{\partial \Phi}{\partial x_i}(\xb) = \frac{1}
{1 - \langle F(\xb),F(\yb) \rangle} (\lambda_i(\xb) - \Phi(\xb)) 
\left\langle \frac{\partial F}{\partial x_i}(\xb), F(\yb) \right\rangle.
\end{eqnarray}
Substituindo \eqref{eq:Prop2.3_2} em \eqref{eq:Prop2.3_1}, 
obtemos:
\begin{eqnarray*}
\begin{aligned}
\frac{\partial^2 Z}{\partial x_i \partial y_i}(\xb,\yb) =&  \frac{1}{1 - \langle F(\xb),F(\yb) \rangle} (\lambda_i(\xb) - \Phi(\xb)) \left\langle \frac{\partial F}{\partial x_i}(\xb), F(\yb) \right\rangle  \left\langle F(\xb), \frac{\partial F}{\partial y_i}(\yb) \right\rangle \\
&+ 
(\lambda_i(\xb) - \Phi(\xb)) \left\langle \frac{\partial F}{\partial x_i}(\xb), \frac{\partial F}{\partial y_i}(\yb) \right\rangle.
\end{aligned}
\end{eqnarray*}	
Como 
\[
|F(\xb) - F(\yb)|^2 = \langle F(\xb) - F(\yb), F(\xb) - F(\yb) \rangle = 
2 - 2 \langle F(\xb), F(\yb) \rangle,
\]
a equa\c c\~ao anterior torna-se
\begin{eqnarray} \label{eq:Prop2.3_3}
\begin{aligned}
\frac{\partial^2 Z}{\partial x_i \partial y_i}(\xb,\yb) = & 
-2 (\lambda_i(\xb) - \Phi(\xb)) \left\langle \frac{\partial F}{\partial x_i}(\xb), 
\frac{-F(\yb)}{|F(\xb) - F(\yb)|} \right\rangle \left\langle \frac{F(\xb)}{|F(\xb) - 
F(\yb)|}, \frac{\partial F}{\partial y_i}(\yb) \right\rangle \\
& + 
(\lambda_i - \Phi(\xb)) \left\langle \frac{\partial F}{\partial x_i}(\xb), 
\frac{\partial F}{\partial y_i}(\yb) \right\rangle.
\end{aligned}
\end{eqnarray}	
Al\'em disso, como
\[
\left\langle \frac{\partial F}{\partial x_i}(\xb), F(\xb) \right\rangle = 
\left\langle \frac{\partial F}{\partial y_i}(\yb), F(\yb) \right\rangle = 0,
\]
a equação \eqref{eq:Prop2.3_3} se escreve como sendo
\begin{eqnarray*}
%\begin{aligned}
\frac{\partial^2 Z}{\partial x_i \partial y_i}(\xb,\yb) & = &
-2 (\lambda_i(\xb) - \Phi(\xb)) \left\langle \frac{\partial F}{\partial x_i}(\xb), \frac{F(\xb) -F(\yb)}{|F(\xb) - F(\yb)|} \right\rangle \left\langle 
\frac{F(\xb) - F(\yb)}{|F(\xb) - F(\yb)|}, \frac{\partial F}{\partial y_i}(\yb) 
\right\rangle \\
&& + 
(\lambda_i - \Phi(\xb)) \left\langle \frac{\partial F}{\partial x_i}(\xb), 
\frac{\partial F}{\partial y_i}(\yb) \right\rangle \\
&= &
(\lambda_i - \Phi(\xb)) \left\langle w_i,\frac{\partial F}{\partial y_i}(\yb)
\right\rangle \\
&= & \lambda_i - \Phi(\xb),
%\end{aligned}
\end{eqnarray*}	
como quer\'iamos.
\end{demonstracao}


\begin{proposicao}
\begin{equation}\label{2_diff_Z_y}
\sum_{i=1}^2 \frac{\partial^2 Z}{\partial y_i^2}(\xb,\yb) = 2 \Phi(\xb).
\end{equation}
\end{proposicao}
\begin{demonstracao}
Derivando a equa\c c\~ao \eqref{diff_Z_y} em rela\c c\~ao a $y_i$,
e somando, obtemos:
\begin{eqnarray} \label{eq:Prop4.7-1}
\sum_{i=1}^2 \frac{\partial^2 Z}{\partial y_i^2}(\xb,\yb) = - \Phi(\xb) \left\langle F(\xb), \sum_{i=1}^2 \frac{\partial^2 F}{\partial y_i^2}(\yb) \right\rangle + \left\langle \nu(\xb), \sum_{i=1}^2 \frac{\partial^2 F}{\partial y_i^2}(\yb) \right\rangle.
\end{eqnarray}	
Como
\[
\sum_{i=1}^2 \frac{\partial^2 F}{\partial y_i}(\yb) = -2 F(\yb)
\]
e
\[
0 = Z(\xb,\yb) = \Phi(\xb)(1 - \langle F(\xb), F(\yb) \rangle) + 
\langle \nu(\xb), F(\yb) \rangle
\] 
a express\~ao em \eqref{eq:Prop4.7-1} torna-se
\begin{eqnarray*}
\sum_{i=1}^2 \frac{\partial^2 Z}{\partial y_i^2}(\xb,\yb) &=&
2 \Phi(\xb) \langle F(\xb), F(\yb) \rangle - 2 \langle \nu(\xb), F(\yb) 
\rangle \\
&=&
2 \Phi(\xb),
\end{eqnarray*}
como quer\'iamos.
\end{demonstracao}


\begin{proposicao} \label{prop:somaDerSeg}
Vale a seguinte igualdade:
\begin{eqnarray} \label{suma_2das_derivadas}
\begin{aligned}
\sum_{i=1}^2 \frac{\partial^2 Z}{\partial x_i^2}(\xb,\yb) + & 
2 \sum_{i=1}^2 \frac{\partial^2 Z}{\partial x_i \partial y_i}(\xb,\yb) + \sum_{i=1}^2 \frac{\partial^2 Z}{\partial y_i^2}(\xb,\yb) \\
&= 
- \frac{2 \Phi(\xb)^2- |A(\xb)|^2}{2 \Phi(\xb)
(1 - \langle F(\xb),F(\yb) \rangle)} \sum_{i=1}^2
\left\langle \frac{\partial F}{\partial x_i}(\xb), F(\yb) \right\rangle^2 \\
&+
\left( \Delta_{\Sigma} \Phi(\xb) - \frac{| \nabla \Phi(\xb) |^2}
{\Phi(\xb)} + ( | A(\xb) |^2 - 2 ) \Phi(\xb) \right)
\left(1 - \langle F(\xb),F(\yb) \rangle\right).
\end{aligned}
\end{eqnarray}
\end{proposicao}
\begin{demonstracao}
A equa\c c\~ao \eqref{suma_2das_derivadas} segue das equa\c c\~oes \eqref{2_diff_Z_x}, \eqref{diff_Z_x_y} e \eqref{2_diff_Z_y}.
\end{demonstracao}



\section{Prova do Teorema \ref{teo:Lawson}}

Usando os resultados preliminares apresentados nas se\c c\~oes
anteriores, apresentaremos nesta se\c c\~ao a prova do teorema
principal dessa disserta\c c\~ao. 

\vspace{.2cm}

Inicialmente, iremos determinar uma identidade tipo-Simons para
a fun\c c\~ao
\begin{eqnarray}\label{eq:PsiA(x)}
\Psi(x)=\frac{1}{\sqrt2}|A(x)|,
\end{eqnarray}
onde $A$ denota o operador de forma da superf\'icie $F:\Sigma\to\Sp^3$.

\begin{proposicao}\label{edp_principal}
Seja $F:\Sigma\to\Sp^3$ um toro m\'inimo e mergulhado em $\Sp^3$.
Então, a função $\Psi$ dada em \eqref{eq:PsiA(x)} \'e estritamente 
positiva e satisfaz a EDP
\[
\Delta_\Sigma \Psi - \frac{|\nabla \Psi|^2}{\Psi} + (|A|^2 - 2) \Psi = 0.
\]
\end{proposicao}
\begin{demonstracao}
Segue do trabalho de Lawson que um toro m\'inimo na esfera $\Sp^3$
n\~ao tem pontos umb\'ilicos (cf. \cite[Proposition 1.5]{Lawson1970}).
Disso decorre que a função $|A|$ \'e estritamente positiva. Usando a
identidade de Simons \cite[Theorem 5.3.1]{Simons1968}, obtemos
\begin{eqnarray} \label{eq:prop43-1}
\Delta_{\Sigma} h_{ik} + (|A|^2 - 2)h_{ik} = 0,
\end{eqnarray}	
visto que $(h_{ik})$ \'e a matriz que representa $A$. Multiplicando a
equa\c c\~ao \eqref{eq:prop43-1} por $2 h_{ik}$, tem-se
\begin{eqnarray} \label{eq:prop43-2}
2 \Delta_{\Sigma} h_{ik} h_{ik} + 2 (|A|^2 - 2)h_{ik}^2 = 0.
\end{eqnarray}	
Somando e subtraindo a quantidade
\[
2 \sum_{j=1}^2 \left( \frac{\partial h_{ik}}{\partial x_j} \right)^2
\]
na equa\c c\~ao \eqref{eq:prop43-2}, obtemos
\begin{eqnarray} \label{eq:prop43-3}
2 \sum_{j=1}^2 \frac{\partial^2 h_{ik}}{\partial x_j^2} h_{ik} + 2 \sum_{j=1}^2 \left( \frac{\partial h_{ik}}{\partial x_j} \right)^2 - 2 \sum_{j=1}^2 \left( \frac{\partial h_{ik}}{\partial x_j} \right)^2 + 2 (|A|^2 - 2)h_{ik}^2 = 0.
\end{eqnarray}	
Como 
\[
2 \sum_{j=1}^2 \frac{\partial^2 h_{ik}}{\partial x_j^2} h_{ik} + 2 \sum_{j=1}^2 \left( \frac{\partial h_{ik}}{\partial x_j} \right)^2 = 2 \sum_{j=1}^2 \frac{\partial }{\partial x_j} \left( \frac{\partial h_{ik}}{\partial x_j} h_{ik} \right) = \sum_{j=1}^2 \frac{\partial^2 h_{ik}^2}{\partial x_j^2},
\]
podemos reescrever a equa\c c\~ao \eqref{eq:prop43-3} como sendo
\begin{eqnarray} \label{eq:prop43-4}
\sum_{j=1}^2 \frac{\partial^2 h_{ik}^2}{\partial x_j^2} - 2 \sum_{j=1}^2 \left( \frac{\partial h_{ik}}{\partial x_j} \right)^2 + 2 (|A|^2 - 2)h_{ik}^2 = 0.
\end{eqnarray}
Na equa\c c\~ao \eqref{eq:prop43-4}, somando em rela\c c\~ao a $i$
e $k$, obtemos:
\begin{equation} \label{eq:prop43-5}
\Delta_{\Sigma} (|A|^2) - 2 | \nabla A |^2 + 2 (|A|^2 - 2) |A|^2 = 0.
\end{equation}
Note agora que
\begin{eqnarray}\label{eq:prop43-6}
\begin{aligned}
\Delta_{\Sigma} (|A|^2) &= \sum_{j=1}^2 \frac{\partial^2 |A|^2}{\partial x_j^2}
= \sum_{j=1}^2 \frac{\partial}{\partial x_j} 
\left( \frac{\partial |A|^2}{\partial x_j} \right) =
2\sum_{j=1}^2\frac{\partial}{\partial x_j}\left(|A|
\frac{\partial|A|}{\partial x_j}\right) \\
&= 
2\sum_{j=1}^2\left(\frac{\partial|A|}{\partial x_j}\frac{\partial|A|}{\partial x_j}
+|A|\frac{\partial^2 |A|}{\partial x_j^2} \right) =
2|\nabla|A||^2+2|A|\sum_{j=1}^2\frac{\partial^2 |A|}{\partial x_j^2} \\
&= 2|\nabla|A||^2+2|A|\Delta_\Sigma|A|.
\end{aligned}
\end{eqnarray}	
Assim, substituindo \eqref{eq:prop43-6} em \eqref{eq:prop43-5},
obtemos
\begin{equation}\label{edp_sff}
\Delta_{\Sigma} (|A|) + \frac{|\nabla |A||^2}{|A|} - \frac{|\nabla A|^2}{|A|} + 
(|A|^2 - 2) |A| = 0.
\end{equation}	
Provemos agora que
\begin{eqnarray} \label{eq:prop43-14}
|\nabla A|^2 = 2|\nabla|A||^2.
\end{eqnarray}
De fato, elevando ao quadrado a equa\c c\~ao \eqref{eq:Prop4.5-3}, 
obtemos:
\begin{eqnarray} \label{eq:prop43-7}
\begin{aligned}
\left( \frac{\partial h_{11}}{\partial x_1} \right)^2 + \left( \frac{\partial h_{21}}{\partial x_2} \right)^2  &= - 2 \frac{\partial h_{11}}{\partial x_1} \frac{\partial h_{21}}{\partial x_2},  \\
\left( \frac{\partial h_{12}}{\partial x_1} \right)^2 + \left( \frac{\partial h_{22}}{\partial x_2} \right)^2 &= - 2 \frac{\partial h_{12}}{\partial x_1} \frac{\partial h_{22}}{\partial x_2}.
\end{aligned}
\end{eqnarray}	
Cada uma das equa\c c\~oes em \eqref{eq:prop43-7} corresponde
a $k=1$ e $k=2$, respectivamente. Disso decorre que
\begin{eqnarray} \label{eq:prop43-8}
\begin{aligned}
\left( \frac{\partial h_{11}}{\partial x_1} \right)^2 + \left( \frac{\partial h_{21}}{\partial x_2} \right)^2  &=  2  \left( \frac{\partial h_{21}}{\partial x_2} \right)^2,  \\
\left( \frac{\partial h_{12}}{\partial x_1} \right)^2 + \left( \frac{\partial h_{22}}{\partial x_2} \right)^2 &=  2 \left( \frac{\partial h_{12}}{\partial x_1} \right)^2, \\
\left( \frac{\partial h_{11}}{\partial x_2} \right)^2 + \left( \frac{\partial h_{21}}{\partial x_1} \right)^2  &= 2 \left( \frac{\partial h_{21}}{\partial x_1} \right)^2, \\
\left( \frac{\partial h_{12}}{\partial x_2} \right)^2 + \left( \frac{\partial h_{22}}{\partial x_1} \right)^2 &=  2 \left( \frac{\partial h_{12}}{\partial x_2} \right)^2  .
\end{aligned}
\end{eqnarray}	
Somando as equações em \eqref{eq:prop43-8}, e lembrando que
$h_{12} = h_{21}$, obtemos:
\begin{equation} \label{eq:prop43-13}
| \nabla A |^2 = 4 \left( \frac{\partial h_{12}}{\partial x_1} \right)^2 + 4\left( \frac{\partial h_{12}}{\partial x_2} \right)^2.
\end{equation}	
Note que 
\begin{eqnarray} \label{eq:prop43-15}
|A| = \sqrt{2 h_{11}^2 + 2 h_{12}^2}
\end{eqnarray}
e
\begin{eqnarray} \label{eq:prop43-11}
| \nabla |A| |^2 = \left( \frac{\partial |A|}{\partial x_1} \right)^2 +
\left( \frac{\partial |A|}{\partial x_2} \right)^2.
\end{eqnarray}
Derivando $|A|$, dada em \eqref{eq:prop43-15}, em rela\c c\~ao a 
$x_1$ e $x_2$ obtemos, respectivamente, que
\begin{eqnarray} \label{eq:prop43-9}
\frac{\partial |A|}{\partial x_1} = \frac{2 h_{11} \frac{\partial h_{11}}{\partial x_1} + 2 h_{12} \frac{\partial h_{12}}{\partial x_1}}{|A|}
\quad\mbox{e}\quad
\frac{\partial |A|}{\partial x_2} = \frac{2 h_{11} \frac{\partial h_{11}}{\partial x_2} + 2 h_{12} \frac{\partial h_{12}}{\partial x_2}}{|A|}.
\end{eqnarray}	
Elevando ao quadrado as duas equa\c c\~oes em \eqref{eq:prop43-9},
obtemos
\begin{eqnarray} \label{eq:prop43-10}
\begin{aligned}
\left( \frac{\partial |A|}{\partial x_1} \right)^2 &= \frac{4 h_{11}^2 \left( \frac{\partial h_{11}}{\partial x_1} \right)^2 + 4 h_{12}^2 \left( \frac{\partial h_{12}}{\partial x_1} \right)^2 + 8 h_{11} h_{12} \frac{\partial h_{11}}{\partial x_1} \frac{\partial h_{12}}{\partial x_1}}{|A|^2} \\
\left( \frac{\partial |A|}{\partial x_2} \right)^2 &= \frac{4 h_{11}^2 \left( \frac{\partial h_{11}}{\partial x_2} \right)^2 + 4 h_{12}^2 \left( \frac{\partial h_{12}}{\partial x_2} \right)^2 + 8 h_{11} h_{12} \frac{\partial h_{11}}{\partial x_2} \frac{\partial h_{12}}{\partial x_2}}{|A|^2}
\end{aligned}
\end{eqnarray}	
Somando as duas equa\c c\~oes em \eqref{eq:prop43-10}, e usando
\eqref{eq:prop43-11} e \eqref{eq:prop43-9}, obtemos
\begin{eqnarray} \label{eq:prop43-12}
| \nabla |A| |^2 = 2 \left( \frac{\partial h_{12}}{\partial x_1} \right)^2 + 2 \left( \frac{\partial h_{12}}{\partial x_2} \right)^2.
\end{eqnarray}	
Das equa\c c\~oes \eqref{eq:prop43-13} e \eqref{eq:prop43-12}
obtém-se \eqref{eq:prop43-14}. Finalmente, substituindo a
equa\c c\~ao \eqref{eq:prop43-14} em \eqref{edp_sff}, obtemos
\begin{equation*}
\Delta_\Sigma \Psi - \frac{|\nabla \Psi|^2}{\Psi} + (|A|^2 - 2) \Psi = 0,
\end{equation*}
finalizando a demonstra\c c\~ao.
\end{demonstracao}


\begin{proposicao} \label{prop:torus1}
Seja $F:\Sigma\to\Sp^3$ um toro m\'inimo e mergulhado na esfera
$\Sp^3$. Se 
\begin{equation} \label{eq:Prop432-1}
\sup_{\substack{x,y \in \Sigma\\ x \neq y}} \frac{| \langle  \nu(x), F(y) \rangle |}
{\Psi(x) (1 - \langle F(x), F(y) \rangle)} \leq 1,
\end{equation}
ent\~ao $F$ \'e congruente ao toro de Clifford.
\end{proposicao}
\begin{demonstracao}
Segue da hip\'otese \eqref{eq:Prop432-1} que
\begin{equation} \label{eq:Prop432-2}
\Psi(x) (1 - \langle F(x), F(y) \rangle) + \langle \nu(x), F(y) \rangle \geq 0,
\end{equation}
para quaisquer $x,y\in\Sigma $. Por uma quest\~ao de simplicidade,
identificaremos a superf\'icie $\Sigma$ com sua imagem atrav\'es 
do mergulho $F$, i.e., $F(x) = x$, para todo $x \in \Sigma$. 
Fixado um ponto arbitr\'ario $\xb\in\Sigma$, podemos encontrar uma
base ortonormal $\{e_1,e_2 \}$ de $T_{\xb}\Sigma$ tal que
\begin{equation*}
h(e_1,e_1) = \Psi(\xb), \quad h(e_1,e_2)=0 \quad \text{e} \quad 
h(e_2,e_2) = -\Psi(\xb),
\end{equation*}
onde $h$ denota a segunda forma fundamental de $F$ no ponto $\xb$.
Dado uma geod\'esica $\gamma$ na superf\'icie $\Sigma$, com 
$\gamma(0)=p$ e $\gamma'(0)=e_1$, definimos uma função $f:\R\to\R$
pondo
\begin{equation*}
f(t) = Z(\xb,\gamma(t)) = 
\Psi(\xb) (1 - \langle \xb, \gamma(t) \rangle) + \langle \nu(\xb), 
\gamma(t) \rangle.
\end{equation*}
Note que, em virtude de \eqref{eq:Prop432-2}, tem-se $f(t)\geq0$,
para todo $t\in\R$.
Calculando a derivada de primeira ordem de $f$, obtemos:
\[
f'(t) = -\langle \Psi(\xb) \xb - \nu(\xb), \gamma'(t) \rangle + 
\langle \nu(\xb), \gamma'(t) \rangle
\]
Como $\gamma$ \'e geodésica, $\gamma'(t)$ \'e o transporte paralelo
de $\gamma'(0)$, logo
\[
\langle\nu(\xb),\gamma'(t)\rangle = \langle\nu(\xb),
\gamma'(0)\rangle = 0.
\]
Assim, $f'(t)$ pode ser escrito como
\begin{equation*}
f'(t) = -\langle \Psi(\xb) \xb - \nu(\xb), \gamma'(t) \rangle.
\end{equation*}
Calculando a derivada segunda de $f$, obtemos:
\begin{equation*}
f''(t) = -\langle \Psi(\xb) \xb - \nu(\xb), \gamma''(t) \rangle.
\end{equation*}
Derivando a igualdade
\[
\langle \nu(\gamma(t)), \gamma'(t) \rangle = 0,
\]
tem-se
\[
\langle D \nu(\gamma(t)) \gamma'(t), \gamma'(t) \rangle + 
\langle \nu(\gamma(t)), \gamma''(t) \rangle = 0,
\]
onde $D$ denota a derivada usual em $\Sp^3$. Como $\gamma''(t)$ 
não tem componentes em $T_{\gamma(t)} \Sigma$, tem-se
\begin{equation*}
-\gamma''(t) = \gamma(t) + h(\gamma'(t), \gamma'(t)) \nu(\xb).
\end{equation*}
Assim, $f''(t)$ se expressa como
\begin{equation*}
f''(t) = \langle \Psi(\xb) \xb - \nu(\xb), \gamma(t) + h(\gamma'(t), 
\gamma'(t)) \nu(\xb) \rangle.
\end{equation*}
Calculando a derivada de ordem $3$ de $f$, obtemos:
\begin{eqnarray}\label{3ra_der_f}
\begin{aligned}
f'''(t) =& \langle \Psi(\xb)\xb - \nu(\xb), \gamma'(t) \rangle + h(\gamma'(t), 
\gamma'(t)) \langle \Psi(\xb)\xb - \nu(\xb), D_{\gamma'(t)} \nu(\gamma(t)) 
\rangle \\
&+ \left( D_{\gamma'(t)}^{\Sigma} h \right) (\gamma'(t), \gamma'(t)) 
\langle \Psi(\xb)\xb - \nu(\xb), \nu(\gamma(t)) \rangle.
\end{aligned}
\end{eqnarray}
Observe que $f(0)=0$. Temos também $f'(0)=0$, pois 
$\Psi(\xb)\xb-\nu(\xb)$ \'e perpendicular a $T_{\xb}\Sigma$. 
Al\'em disso, temos $f''(0)=0$ pois
\[
f''(0) = \Psi(\xb) - h(e_1,e_1)=0.
\]
Provemos agora que $f'''(0)=0$. Usando a expansão de Taylor em $f$,
tem-se:
\begin{equation*}
f(t) = f(0) + f'(0)t + f''(0)t^2 + f'''(0)t^3 + r_3(t),
\end{equation*}  
onde
\[
\lim_{t \rightarrow 0} \frac{r_3(t)}{t^3}=0.
\]
Como $f(t) \geq 0$ e $f(0)=f'(0)=f''(0)=0$ tem-se
\begin{equation} \label{eq:taylor1}
0 \leq f'''(0) t^3 + r_3(t).
\end{equation}
Dividindo por $t^3$ em \eqref{eq:taylor1}, e fazendo $t\to0$, obtemos
\[
0 \leq f'''(0).
\]
Por outro lado, usando $-t$ ao inv\'es de $t$ e aplicando a mesma ideia,
obtemos
\[
f'''(0) \leq 0,
\]
logo $f'''(0)=0$. Fazendo agora $t=0$ na equa\c c\~ao \eqref{3ra_der_f},
tem-se 
\begin{equation} \label{eq:taylor2}
(D_{e_1}^{\Sigma} h) (e_1,e_1) = 0,
\end{equation}
pois $e_1, D_{e_1} \nu(\xb) \in T_p \Sigma$ e $\Psi(\xb)p - \nu(\xb)$ e 
$\nu(\xb)$ são paralelos. Trocando o referencial $\{ e_1, e_2, \nu \}$
por $\{ e_2, e_1, -\nu \}$, e argumentando de maneira an\'aloga, obtemos
\begin{equation} \label{eq:taylor3}
(D_{e_2}^{\Sigma} h) (e_2,e_2)=0.
\end{equation}
Como $h_{11}+h_{22}=0$, segue de \eqref{eq:taylor2} e \eqref{eq:taylor3}
que
\[
(D_{e_2}^{\Sigma} h) (e_1,e_1) = (D_{e_1}^{\Sigma} h) (e_2,e_1) = 0
\]
e
\[
(D_{e_1}^{\Sigma} h) (e_2,e_2) = (D_{e_2}^{\Sigma} h) (e_1,e_2) = 0,
\]
implicando que 
\[
(D_{e_1}^{\Sigma} h) (e_2,e_2) = 0
\]
e
\[
(D_{e_2}^{\Sigma} h) (e_1,e_1) = 0.
\]
Aplicando as equa\c c\~oes de Codazzi nas identidades acima, 
obtemos que $\nabla h = 0$. Disso decorre, em particular, que a
curvatura intrínseca $K$ de $\Sigma$ é constante e, assim, a 
m\'etrica induzida em $\Sigma$ por $F$ \'e flat. Logo, pelo trabalho
de Lawson (cf. \cite[Corollary 3]{Lawson1969}), tem-se que $K\equiv0$ 
ou $K\equiv1$. Como $\Sigma$ não tem pontos umbílicos, segue que
$K\equiv0$ e, portanto, $\Sigma$ é um subconjunto aberto do toro de 
Clifford. A compacidade de $\Sigma$ implica que $F$ é congruente
ao toro de Clifford.
\end{demonstracao}

Iremos a partir de agora finalizar a prova do Teorema 
\ref{teo:Lawson}. Dado um toro m\'inimo e mergulhado 
$F:\Sigma\to\Sp^3$, considere
a express\~ao
\begin{equation} \label{eq:aleph-1}
\aleph = \sup_{\substack{\xb,\yb\in\Sigma\\ \xb\neq\yb}} 
\frac{|\langle\nu(\xb),F(\yb)\rangle|}{\Psi(\xb)(1-\langle F(\xb),F(\yb)\rangle)},
\end{equation}
onde $\Psi$ \'e dada em \eqref{eq:PsiA(x)}. Se $\aleph\leq1$, segue
da Proposi\c c\~ao \ref{prop:torus1} que $F$ \'e congruente ao toro de
Clifford. Assim, resta considerar o caso em que $\aleph > 1$. Trocando
a dire\c c\~ao normal $\nu$ por $-\nu$, caso necess\'ario, podemos
escrever
\begin{equation} \label{eq:aleph-2}
\aleph = \sup_{\substack{\xb,\yb\in\Sigma\\ \xb\neq\yb}} 
\frac{-\langle\nu(\xb),F(\yb)\rangle}{\Psi(\xb)(1-\langle F(\xb),F(\yb)\rangle)}.
\end{equation}
Assim, em virtude de \eqref{eq:aleph-2}, a fun\c c\~ao $Z$, dada por
\begin{equation*}
Z(\xb,\yb) = \aleph\Psi(\xb)(1-\langle F(\xb),F(\yb)\rangle)+
\langle\nu(\xb),F(\yb) \rangle,
\end{equation*}
satisfaz
\begin{equation} \label{eq:aleph-3}
Z(\xb,\yb)\geq0,
\end{equation}
para quaisquer $\xb,\yb\in\Sigma$. Considere agora o conjunto
\begin{equation} \label{eq:ConjOmega}
\Omega = \{\xb\in\Sigma:\mbox{existe} \ \yb\in\Sigma\setminus\{\xb\}
\text{ tal que } Z(\xb,\yb) = 0 \}
\end{equation}


\begin{lema}
O conjunto $\Omega$, dado em \eqref{eq:ConjOmega}, \'e n\~ao-vazio.
\end{lema}

\begin{demonstracao}
	Sejam $x,y \in \Sigma$. Como
	$\Psi(x) (1 - \innerproduct{F(x)}{F(y)})$ é contínua e
	$\Sigma$ é compacto então existe $M > 0$ tal que
	\begin{equation*}
	\Psi(x) (1 - \innerproduct{F(x)}{F(y)}) \leq M.
	\end{equation*}
	Pela definição de $\aleph$, tem-se
	que para qualquer $n \in \N$ existem $x_n,y_n \in \Sigma$, onde $x_n \neq y_n$, tal que
	\begin{equation*}
	\frac{- \innerproduct{\nu(x_n)}{F(y_n)}}{\Psi(x_n) (1 - \innerproduct{F(x_n)}{F(y_n)})} > \aleph - \frac{1}{nM}.
	\end{equation*}
	Desenvolvendo a desigualdade anterior, podemos escrevê-la como:
	\begin{equation}\label{eq:omega-nao-e-vazio-eq-a}
	\frac{1}{n} \geq \frac{\Psi(x_n) (1 - \innerproduct{F(x_n)}{F(y_n)})}{nM} > Z(x_n,y_n) \geq 0.
	\end{equation}
	Como tais $(x_n)$ e $(y_n)$ são sequências limitadas, então existem subsequências convergentes $(x_{n_k})$ e $(y_{n_k})$ tais que
	$x_{n_k} \rightarrow \xb$ e
	$y_{n_k} \rightarrow \yb$,
	onde $\xb, \yb \in \Sigma$.
	Usando as subsequências já mencionadas em \eqref{eq:omega-nao-e-vazio-eq-a} e tomando limite, tem-se que
	\begin{equation*}
	Z(\xb,\yb) = 0.
	\end{equation*}
	Se $\xb = \yb$, então
	\begin{equation*}
	Z(\xb,\yb) = Z(\xb, \xb) = \innerproduct{\nu(\xb)}{F(\xb)} = 0.
	\end{equation*}
	Isso implica que 
	$F(\xb) \in T_{\xb} \Sigma$,
	então $F(\xb)$ pode-se expressar como combinação linear dos $\frac{\partial F}{\partial x_i}(\xb)$, logo
	\begin{equation}\label{eq:omega-nao-e-vazio-eq-b}
	F(\xb) = a^i \frac{\partial F}{\partial x_i}(\xb).
	\end{equation}
	Lembrando que
	$\innerproduct{F(\xb)}{\frac{\partial F}{\partial x_j}(\xb)} = 0$
	e tomando o produto interno em \eqref{eq:omega-nao-e-vazio-eq-b}, obtêm-se
	\begin{equation*}
	0 = \innerproduct{F(\xb)}{\frac{\partial F}{\partial x_j}(\xb)} = a^i \delta_{ij}.
	\end{equation*}
	Então
	$a^i = 0$,
	o que implica que
	$F(\xb) = 0$, uma contradição.
	Portanto
	$\xb \neq \yb$ e
	$\xb \in \Omega$.
\end{demonstracao}

O pr\'oximo passo agora \'e provar que o conjunto $\Omega$,
dado em \eqref{eq:ConjOmega}, \'e aberto. Para isso, faremos
uso de dois resultados. O primeiro deles \'e o {\em princ\'ipio
do m\'aximo estrito de Bony} para equa\c c\~oes el\'ipticas 
n\~ao-degeneradas (cf. \cite[Corollary 9.7]{Brendle2010}).

\begin{teorema} \label{teo:bony}
Dados um aberto $U\subset\R^n$ e campos vetoriais 
$X_1,\ldots,X_m\in\mathfrak{X}(U)$, considere uma fun\c c\~ao
diferenci\'avel n\~ao-negativa $\varphi:U\to\R$ satisfazendo
\begin{equation}\label{eq:bony}
\sum_{j=1}^{m} (D^2 \varphi)(X_j,X_j) \leq -L \inf_{|\xi| \leq 1} 
(D^2 \varphi)(\xi,\xi) + L |d \varphi| + L \varphi,
\end{equation}
onde $L$ é uma constante positiva. Seja $F=\{x\in U:\varphi(x)=0\}$
o conjunto dos zeros de $\varphi$ e suponha que $\gamma:[0,1]\to U$
seja uma curva diferenciável tal que $\gamma(0)\in F$ e 
$\gamma'(s) = \displaystyle\sum_{j=1}^{m} f_j(s) X_j(\gamma(s))$, para certas
funções diferenciáveis $f_1,\ldots,f_m:[0,1]\to\R$. Então 
$\gamma(s)\in F$, para todo $s\in[0,1]$.
Este teorema também é válido para um conjunto aberto $U$ de uma variedade Riemanniana em vez de $\R^n$. Para demonstrar isso, divide-se o caminho $\gamma$ em segmentos pequenos onde cada segmento está numa carta coordenada. Logo se aplica o teorema a cada segmento.
\end{teorema}

O seguinte resultado \'e uma estimativa obtida fazendo-se uma
adapta\c c\~ao na demonstra\c c\~ao da Proposi\c c\~ao
\ref{prop:somaDerSeg}.

Considere dois pontos $\xb,\yb\in\Sigma$, com $\xb\neq\yb$. Seja
$(x_1,x_2)$ um sistema de coordenadas normais geod\'esicas em
torno de $\xb$ tal que
\[
h_{11}(\xb)=\lambda_1, \quad \lambda_{12}(\xb)=0
\quad\mbox{e}\quad h_{22}(\xb)=\lambda_2. 
\]
Al\'em disso, seja $(y_1,y_2)$ um sistema de coordenadas normais 
geod\'esicas em torno de $\yb$ tal que
\begin{equation*}
\left\langle w_1, \frac{\partial F}{\partial y_1}(\yb) \right\rangle \geq 0, 
\quad 
\left\langle w_1, \frac{\partial F}{\partial y_2}(\yb) \right\rangle = 0 
\quad \text{e} \quad 
\left\langle w_2, \frac{\partial F}{\partial y_2}(\yb) \right\rangle \geq 0,
\end{equation*}
onde $w_1$ e $w_2$ s\~ao dados em \eqref{eq:reflexao}. Adaptando
a demonstra\c c\~ao do Lema \ref{lem:w_1 w_2}, podemos concluir
que
\begin{equation}\label{eq:Gamma-ineq}
	\sum_{i=1}^2\left| w_i-\frac{\partial F}{\partial y_i}(\yb)\right| \leq
	\Lambda(\xb,\yb)\left(Z(\xb,\yb)+\sum_{i=1}^2
	\left|\frac{\partial Z}{\partial y_i}(\xb,\yb)\right|\right),
\end{equation}
onde $\Lambda$ \'e uma fun\c c\~ao cont\'inua sobre o conjunto
\[
\{(\xb,\yb)\in\Sigma\times\Sigma:\xb\neq\yb\}
\]
que, eventualmente, deixa de ser limitada numa vizinhan\c ca
da diagonal de $\Sigma\times\Sigma$. 

\begin{lema}
Dados dois pontos $\xb,\yb\in\Sigma$, com $\xb\neq\yb$, tem-se:
\begin{equation} \label{eq:lema2Omegaopen}
\begin{aligned}
\sum_{i=1}^2 \frac{\partial^2 Z}{\partial x_i^2}(\xb,\yb) \ + \ & 
2 \sum_{i=1}^2 \frac{\partial^2 Z}{\partial x_i \partial y_i}(\xb,\yb) \ + \ 
\sum_{i=1}^2 \frac{\partial^2 Z}{\partial y_i^2}(\xb,\yb)  \\
& \leq
- \frac{\aleph^2 - 1}{\aleph} \frac{\Psi(\xb)}{1 - \langle F(\xb), F(\yb) \rangle} \sum_{i=1}^2 \left\langle \frac{\partial F}{\partial x_i}(\xb), F(\yb) \right\rangle^2 \\
& + 
\tilde{\Lambda}(\xb,\yb) \left( Z(\xb,\yb) + \sum_{i=1}^2 \left| \frac{\partial Z}{\partial x_i}(\xb,\yb) \right| + \sum_{i=1}^2 \left| \frac{\partial Z}{\partial y_i}(\xb,\yb) \right| \right)
\end{aligned}
\end{equation}
onde $\tilde{\Lambda}(\xb,\yb)$ \'e uma função cont\'inua no conjunto
$\{(\xb,\yb)\in\Sigma\times\Sigma:\xb\neq\yb \}$, a qual pode ser não
limitada numa vizinhan\c ca da diagonal de $\Sigma\times\Sigma$.
\end{lema}
\begin{demonstracao}
Em virtude das Proposi\c c\~oes \ref{prop:somaDerSeg} e 
\ref{edp_principal}, podemos escrever
\begin{equation*}
\begin{aligned}
&\sum_{i=1}^2\frac{\partial^2 Z}{\partial x_i^2}(\xb,\yb) \ + \ 
2 \sum_{i=1}^2 \frac{\partial^2 Z}{\partial x_i \partial y_i}(\xb,\yb) \ + \ 
\sum_{i=1}^2 \frac{\partial^2 Z}{\partial y_i^2}(\xb,\yb)  \\
& =
- \frac{\aleph^2 - 1}{\aleph} \frac{\Psi(\xb)}{1 - \langle F(\xb), F(\yb) \rangle} \sum_{i=1}^2 \left\langle \frac{\partial F}{\partial x_i}(\xb), F(\yb) \right\rangle^2 
+ 4\aleph\Psi(\xb) - (|A(\xb)|^2+2)Z(\xb,\yb) \\ 
& 
+2\sum_{i=1}^2(\lambda_i-\aleph\Psi(\xb))\left\langle w_i,
\frac{\partial F}{\partial y_i}(\yb)\right\rangle - 
\frac{2}{1 - \langle F(\xb), F(\yb) \rangle}\sum_{i=1}^2
\frac{\partial Z}{\partial x_i}(\xb,\yb)\left\langle F(\xb),
\frac{\partial F}{\partial y_i}(\yb)\right\rangle \\
& +
\frac{1}{\aleph\Psi(\xb)(1 - \langle F(\xb), F(\yb) \rangle)}\sum_{i=1}^2
\left[\left(\frac{\partial Z}{\partial x_i}(\xb,\yb)\right)^2
-2\lambda_i\left\langle\frac{\partial F}{\partial x_i}(\xb),F(\yb)\right\rangle
\frac{\partial Z}{\partial x_i}(\xb,\yb)\right].
\end{aligned}
\end{equation*}
Como 
\[
Z(\xb,\yb)\geq0 \quad\mbox{e}\quad
|F(\xb)| = \left|\frac{\partial F}{\partial y_i}(\yb) \right|=1,
\]
podemos escrever
\begin{equation*}
\begin{aligned}
&\sum_{i=1}^2\frac{\partial^2 Z}{\partial x_i^2}(\xb,\yb) \ + \ 
2 \sum_{i=1}^2 \frac{\partial^2 Z}{\partial x_i \partial y_i}(\xb,\yb) \ + \ 
\sum_{i=1}^2 \frac{\partial^2 Z}{\partial y_i^2}(\xb,\yb)  \\
& =
- \frac{\aleph^2 - 1}{\aleph} \frac{\Psi(\xb)}{1 - \langle F(\xb), F(\yb) \rangle} \sum_{i=1}^2 \left\langle \frac{\partial F}{\partial x_i}(\xb), F(\yb) \right\rangle^2
+ 4\aleph\Psi(\xb) \\ 
& 
+2\sum_{i=1}^2(\lambda_i-\aleph\Psi(\xb))\left\langle w_i,
\frac{\partial F}{\partial y_i}(\yb)\right\rangle +
\frac{2}{1 - \langle F(\xb), F(\yb) \rangle}\sum_{i=1}^2
\left|\frac{\partial Z}{\partial x_i}(\xb,\yb)\right| \\
& +
\frac{1}{\aleph\Psi(\xb)(1 - \langle F(\xb), F(\yb) \rangle)}\sum_{i=1}^2
\left[\left(\frac{\partial Z}{\partial x_i}(\xb,\yb)\right)^2
-2\lambda_i\left\langle\frac{\partial F}{\partial x_i}(\xb),F(\yb)\right\rangle
\frac{\partial Z}{\partial x_i}(\xb,\yb)\right].
\end{aligned}
\end{equation*}
Estimemos, inicialmente, o termo
\[
\frac{\partial Z}{\partial x_i}(\xb,\yb)\left[
\frac{\partial Z}{\partial x_i}(\xb,\yb)
-2\lambda_i\left\langle\frac{\partial F}{\partial x_i}(\xb),F(\yb)\right\rangle
\right].
\]
Como
\[
\frac{\partial Z}{\partial x_i}(\xb,\yb) = \aleph\frac{\partial\Psi}{\partial x_i}(\xb)
(1 - \langle F(\xb), F(\yb) \rangle) +
(\lambda_i-\aleph\Psi(\xb))\left\langle\frac{\partial F}{\partial x_i}(\xb),F(\yb)
\right\rangle,
\]
temos
\begin{equation*}
\begin{aligned}
&
\frac{\partial Z}{\partial x_i}(\xb,\yb)\left[
\frac{\partial Z}{\partial x_i}(\xb,\yb)
-2\lambda_i\left\langle\frac{\partial F}{\partial x_i}(\xb),F(\yb)\right\rangle
\right] \\
&= 
\frac{\partial Z}{\partial x_i}(\xb,\yb)\left[
\aleph\frac{\partial\Psi}{\partial x_i}(\xb)
(1 - \langle F(\xb), F(\yb) \rangle) +
(\lambda_i-\aleph\Psi(\xb))\left\langle\frac{\partial F}{\partial x_i}(\xb),F(\yb)
\right\rangle\right] \\
&\leq
\left|\frac{\partial Z}{\partial x_i}(\xb,\yb)\right|\aleph M
(1 - \langle F(\xb), F(\yb) \rangle) +
\left|\frac{\partial Z}{\partial x_i}(\xb,\yb)\right| 2\aleph\Psi(\xb),
\end{aligned}
\end{equation*}
onde
\[
M = \sup_{\substack{\xb\in\Sigma\\ i=1,2}}
\left|\frac{\partial\Psi}{\partial x_i}(\xb)\right|
\]
e notando que $\lambda_i\leq\aleph\Psi(\xb)$, com $i=1,2$. Assim,
\begin{equation*}
\begin{aligned}
&\sum_{i=1}^2\frac{\partial^2 Z}{\partial x_i^2}(\xb,\yb) \ + \ 
2 \sum_{i=1}^2 \frac{\partial^2 Z}{\partial x_i \partial y_i}(\xb,\yb) \ + \ 
\sum_{i=1}^2 \frac{\partial^2 Z}{\partial y_i^2}(\xb,\yb)  \\
& \leq
- \frac{\aleph^2 - 1}{\aleph} \frac{\Psi(\xb)}{1 - \langle F(\xb), F(\yb) \rangle} \sum_{i=1}^2 \left\langle \frac{\partial F}{\partial x_i}(\xb), F(\yb) \right\rangle^2
+ 4\aleph\Psi(\xb) \\ 
& +
2\sum_{i=1}^2(\lambda_i-\aleph\Psi(\xb))\left\langle w_i,
\frac{\partial F}{\partial y_i}(\yb)\right\rangle +
\left(\frac{4}{1 - \langle F(\xb), F(\yb) \rangle} + \frac{M}{\Psi(\xb)}\right)
\sum_{i=1}^2\left|\frac{\partial Z}{\partial x_i}(\xb,\yb)\right|.
\end{aligned}
\end{equation*}
Estimemos agora o termo
\[
2\sum_{i=1}^2(\lambda_i-\aleph\Psi(\xb))\left\langle w_i,
\frac{\partial F}{\partial y_i}(\yb)\right\rangle.
\]
Desenvolvendo, obtemos:
\begin{equation*}
\begin{aligned}
\left| w_i-\frac{\partial F}{\partial y_i}(\yb)\right|^2 &=
|w_i|^2 + \left|\frac{\partial F}{\partial y_i}(\yb)\right|^2 -
2\left\langle w_i,\frac{\partial F}{\partial y_i}(\yb)\right\rangle \\
&=
2 - 2\left\langle w_i,\frac{\partial F}{\partial y_i}(\yb)\right\rangle.
\end{aligned}
\end{equation*}
Assim, usando a desigualdade \eqref{eq:Gamma-ineq}, obtemos:
\begin{equation*}
\begin{aligned}
2&\sum_{i=1}^2(\lambda_i-\aleph\Psi(\xb))\left\langle w_i,
\frac{\partial F}{\partial y_i}(\yb)\right\rangle \\
&=
\sum_{i=1}^2\lambda_i\left(2-\left| w_i-\frac{\partial F}{\partial y_i}(\yb)
\right|^2\right) +
\aleph\Psi(\xb)\sum_{i=1}^2\left(
\left| w_i-\frac{\partial F}{\partial y_i}(\yb)\right|^2-2\right) \\
&\leq 
\Psi(\xb)\sum_{i=1}^2\left| w_i-\frac{\partial F}{\partial y_i}(\yb)\right|^2 +
\aleph\Psi(\xb)\sum_{i=1}^2
\left| w_i-\frac{\partial F}{\partial y_i}(\yb)\right|^2 
-4\aleph\Psi(\xb) \\
&\leq
2\Psi(\xb)(\aleph+1)\sum_{i=1}^2
\left| w_i-\frac{\partial F}{\partial y_i}(\yb)\right| -4\aleph\Psi(\xb) \\
& \leq
2\Psi(\xb)(\aleph+1)\Lambda(\xb,\yb)\left(Z(\xb,\yb)
+\sum_{i=1}^2\left|\frac{\partial Z}{\partial x_i}(\xb,\yb)\right|\right)
-4\aleph\Psi(\xb).
\end{aligned}
\end{equation*}
Portanto,
\begin{equation*}
\begin{aligned}
&\sum_{i=1}^2\frac{\partial^2 Z}{\partial x_i^2}(\xb,\yb) \ + \ 
2 \sum_{i=1}^2 \frac{\partial^2 Z}{\partial x_i \partial y_i}(\xb,\yb) \ + \ 
\sum_{i=1}^2 \frac{\partial^2 Z}{\partial y_i^2}(\xb,\yb)  \\
& \leq
- \frac{\aleph^2 - 1}{\aleph} \frac{\Psi(\xb)}{1 - \langle F(\xb), F(\yb) \rangle} \sum_{i=1}^2 \left\langle \frac{\partial F}{\partial x_i}(\xb), F(\yb) \right\rangle^2 \\
&+
2\Psi(\xb)(\aleph+1)\Lambda(\xb,\yb)\left(Z(\xb,\yb)
+\sum_{i=1}^2\left|\frac{\partial Z}{\partial x_i}(\xb,\yb)\right|\right) \\
&+\left(\frac{4}{1 - \langle F(\xb), F(\yb) \rangle} + \frac{M}{\Psi(\xb)}\right)
\sum_{i=1}^2\left|\frac{\partial Z}{\partial x_i}(\xb,\yb)\right|,
\end{aligned}
\end{equation*}
e, assim, conclu\'imos que
\begin{equation*}
\begin{aligned}
&\sum_{i=1}^2\frac{\partial^2 Z}{\partial x_i^2}(\xb,\yb) \ + \ 
2 \sum_{i=1}^2 \frac{\partial^2 Z}{\partial x_i \partial y_i}(\xb,\yb) \ + \ 
\sum_{i=1}^2 \frac{\partial^2 Z}{\partial y_i^2}(\xb,\yb)  \\
& \leq
- \frac{\aleph^2 - 1}{\aleph} \frac{\Psi(\xb)}{1 - \langle F(\xb), F(\yb) \rangle} \sum_{i=1}^2 \left\langle \frac{\partial F}{\partial x_i}(\xb), F(\yb) \right\rangle^2 \\
&+
\left(2\Psi(\xb)(\aleph+1)\Lambda(\xb,\yb) +
\frac{4}{1 - \langle F(\xb), F(\yb) \rangle} + \frac{M}{\Psi(\xb)}\right) \\
&\cdot\left( Z(\xb,\yb) + \sum_{i=1}^2 \left| \frac{\partial Z}{\partial x_i}(\xb,\yb) \right| + \sum_{i=1}^2 \left| \frac{\partial Z}{\partial y_i}(\xb,\yb) \right| \right).
\end{aligned}
\end{equation*}
Basta considerar ent\~ao
\[
\tilde\Lambda(\xb,\yb) =
\left(2\Psi(\xb)(\aleph+1)\Lambda(\xb,\yb) +
\frac{4}{1 - \langle F(\xb), F(\yb) \rangle} + \frac{M}{\Psi(\xb)}\right),
\]
o que prova \eqref{eq:lema2Omegaopen}.
\end{demonstracao}


\begin{lema}
O conjunto $\Omega$, dado em \eqref{eq:ConjOmega}, \'e aberto.
\end{lema}

\begin{demonstracao}
Dado $x \in \Omega$,	definimos o conjunto
\[ Y_x = \{ y \in \Sigma: Z(x,y)=0 \}. \]
Com a notação do Teorema \ref{teo:bony},  identifiquemos a função $\varphi$ com $Z$, e o conjunto $F$ estará definido por
\[F = \bigcup_{x \in \Omega} \{x\} \times Y_x.  \]
Em cada ponto $(x,y) \in F$, $Z$ atinge o seu mínimo. Logo, a derivada segunda da função $Z$ naqueles pontos serão operadores bilineares definidos positivos. Pela continuidade da derivada segunda, para cada ponto $(x,y) \in F$ existe uma vizinhança $V_{(x,y)}$ tal que a derivada segunda avaliada nos pontos da vizinhança segue sendo um operador bilinear definido positivo. 
Considere o aberto
\[ U = \bigcup_{(x,y) \in F} V_{(x,y)} \]
e dois campos vetoriais $X_1$ e $X_2$ definidos por:
\[ X_1 = \left[ \begin{matrix}
1\\
0\\
1\\
0
\end{matrix} \right] \quad \text{e} \quad X_2 = \left[ \begin{matrix}
0\\
1\\
0\\
1
\end{matrix} \right]. \]
Calculando o valor da expressão em \eqref{eq:bony}, obtemos:
\[ \sum_{i=1}^{2} D^2 Z (X_i,X_i) = \sum_{i=1}^{2} \frac{\partial^2 Z}{\partial x_i^2} + 2 \sum_{i=1}^{2} \frac{\partial^2 Z}{\partial x_i \partial y_i} + \sum_{i=1}^{2} \frac{\partial^2 Z}{\partial y_i^2}. \] 
No aberto $U$, a derivada segunda de $Z$ é definida positiva, i.e., $0 < D^2 Z(\xi,\xi)$, para qualquer campo vetorial $\xi$ em $U$. Logo, obtemos que
\[ 0 \leq \inf_{|\xi| \leq 1} D^2 Z(\xi,\xi) \leq D^2 Z (0,0) = 0. \]
Portanto $\inf_{|\xi| \leq 1} D^2 Z(\xi,\xi) = 0$.
Como estamos no caso $\aleph > 1$, tem-se
\[ - \frac{\aleph^2 -1}{\aleph} \frac{\Psi(x)}{1 - \innerproduct{F(x)}{F(y)}} \sum_{i=1}^{2} \innerproduct{\frac{\partial F}{\partial x_i}(x)}{F(y)}^2 \leq 0. \]
Podemos construir $U$ de tal forma que esteja contido no complemento da vizinhança da diagonal onde a função $\overline{\Lambda}$ é ilimitada, logo $\overline{\Lambda}$ é limitada em $U$.
Estamos, assim, nas condições do Teorema \ref{teo:bony} e, portanto, podemos chegar à conclusão que para uma geodésica $\gamma: [0,1] \rightarrow U$ tal que $\gamma(0) \in U$ e $\gamma'(s) = f_1(s) X_1(\gamma(s)) + f_2(s) X_2(\gamma(s))$ para quaisquer funções $f_1,f_2: [0,1] \rightarrow \R$ tem-se que $\gamma(s) \in F$.
Aplicando a projeção sobre a primeira variável no conjunto $F$ obtemos o conjunto $\Omega$ que, pela escolha de $X_1$ e $X_2$, conclui-se que é aberto porque é uma bola geodésica. 
\end{demonstracao}


Considere agora dois pontos $\xb,\yb\in\Sigma$, com $\xb\neq\yb$,
tais que
\[
Z(\xb,\yb) = \frac{\partial Z}{\partial x_i}(\xb,\yb) 
= \frac{\partial Z}{\partial y_i}(\xb,\yb) = 0.
\]
Usando a Proposi\c c\~ao \ref{edp_principal}, a equação
\eqref{suma_2das_derivadas} pode ser escrita como
\begin{equation} \label{suma_2da_der_aleph}
\begin{aligned}
\sum_{i=1}^2 \frac{\partial^2 Z}{\partial x_i^2}(\xb,\yb) + & 
2 \sum_{i=1}^2 \frac{\partial^2 Z}{\partial x_i \partial y_i}(\xb,\yb) + 
\sum_{i=1}^2 \frac{\partial^2 Z}{\partial y_i^2}(\xb,\yb)  \\
& = - 
\frac{\aleph^2 - 1}{\aleph} \frac{\Psi(\xb)}{1 - \langle F(\xb), 
F(\yb) \rangle} \sum_{i=1}^2 \left\langle \frac{\partial F}
{\partial x_i}(\xb), F(\yb) \right\rangle^2.
\end{aligned}
\end{equation}



\begin{proposicao} \label{gradiente_nulo}
Para todo ponto $\xb\in\Omega$, tem-se
\[
\nabla \Psi(\xb)=0.
\] 
\end{proposicao}
\begin{demonstracao}
Fixemos um ponto arbitr\'ario $\xb\in\Omega$. Pela defini\c c\~ao
de $\Omega$, existe um ponto $\yb\in\Sigma\setminus\{x\}$ tal que 
$Z(\xb,\yb)=0$. Como a fun\c c\~ao $Z$ atinge seu m\'inimo
global no ponto $(\xb,\yb)$, a igualdade em 
\eqref{suma_2da_der_aleph} torna-se
\begin{equation*} 
\begin{aligned}
0 & \leq
\sum_{i=1}^2 \frac{\partial^2 Z}{\partial x_i^2}(\xb,\yb) + 
2 \sum_{i=1}^2 \frac{\partial^2 Z}{\partial x_i \partial y_i}(\xb,\yb) + 
\sum_{i=1}^2 \frac{\partial^2 Z}{\partial y_i^2}(\xb,\yb)  \\
& = - 
\frac{\aleph^2 - 1}{\aleph} \frac{\Psi(\xb)}{1 - \langle F(\xb), 
F(\yb) \rangle} \sum_{i=1}^2 \left\langle \frac{\partial F}
{\partial x_i}(\xb), F(\yb) \right\rangle^2 \leq 0.
\end{aligned}
\end{equation*}
Como estamos supondo $\aleph>1$, conclu\'imos que
\[
\left\langle \frac{\partial F}{\partial x_i}(\xb), F(\yb) \right\rangle = 0,
\]
para cada $i=1,2$. Usando agora a equação \eqref{diff_Z_x},
conclu\'imos que 
\[
0 = \frac{\partial Z}{\partial x_i}(\xb,\yb) = 
\aleph \frac{\partial \Psi}{\partial x_i}(\xb)
(1 - \langle F(\xb),F(\yb) \rangle),
\]
provando que $\nabla\Psi(\xb)=0$, para todo $\xb\in\Omega$, como
quer\'iamos.
\end{demonstracao}

Finalizemos agora a prova do Teorema \ref{teo:Lawson}. Como
$\Omega$ \'e aberto, segue da Proposi\c c\~ao \ref{gradiente_nulo}
que
\[
\Delta_\Sigma\Psi(\xb)=0,
\]
para todo ponto $\xb\in\Omega$. Assim, a Proposi\c c\~ao 
\ref{edp_principal} implica que $\Psi(\xb)=1$, para todo
$\xb\in\Omega$. Usando o teorema de extens\~ao \'unica
para solu\c c\~oes de equa\c c\~oes diferencias parciais
el\'ipticas (cf., por exemplo, \cite{Aronszajn1957}), conclu\'imos
que $\Psi(\xb)=1$, para todo ponto $\xb\in\Sigma$. Como
consequ\^encia, a curvatura Gaussiana de $\Sigma$ \'e
identicamente nula. Como anteriormente, segue do trabalho
de Lawson \cite{Lawson1969} que $F$ \'e congruente ao
toro de Clifford, e isso finaliza a demonstra\c c\~ao do
Teorema \ref{teo:Lawson}.

%
%\chapter{Orientações gerais}
%\label{chapter:orientacoes-gerais}
%\input{tex/orientacoes-gerais}
%
%\chapter{Configuração dos elementos pré-textuais}
%\label{chapter:config-pre-textual}
%\input{tex/config-pre-textual}
%
%\chapter{Corpos flutuantes}
%\label{chapter:corpos-flutuantes}
%\input{tex/corpos-flutuantes}
%
%\chapter{Listas}
%\label{chapter:listas}
%\input{tex/listas}
%
%\chapter{Ferramentas úteis}
%\label{chapter:ferramentas-uteis}
%\input{tex/ferramentas-uteis}
%
%\chapter{Citações e referências}
%\label{chapter:citacoes}
%\input{tex/citacoes}


% ---
% Finaliza a parte no bookmark do PDF, para que se inicie o bookmark na raiz
% ---
\bookmarksetup{startatroot}% 
% ---

% ----------------------------------------------------------
% ELEMENTOS PÓS-TEXTUAIS
% ----------------------------------------------------------
\postextual

% ----------------------------------------------------------
% Referências bibliográficas
% ----------------------------------------------------------
\bibliography{references}

% ---------------------------------------------------------------------
% GLOSSÁRIO
% ---------------------------------------------------------------------

% Arquivo que contém as definições que vão aparecer no glossário
\input{tex/glossario}
% Comando para incluir todas as definições do arquivo glossario.tex
\glsaddall
% Impressão do glossário
\printglossaries

% ----------------------------------------------------------
% Apêndices
% ----------------------------------------------------------

% ---
% Inicia os apêndices
% ---
%\begin{apendicesenv}
%
%    \chapter{Documento básico usando a classe \textit{icmc}}
%    \label{chapter:documento-basico}
%    \input{tex/appendix/documento-basico}
%    
%    \chapter{Configuração do programa JabRef}
%    \label{chapter:configuracao-jabref}
%    \input{tex/appendix/configuracao-jabref}
%
%\end{apendicesenv}
% ---


% ----------------------------------------------------------
% Anexos
% ----------------------------------------------------------

% ---
% Inicia os anexos
% ---
%\begin{anexosenv}
%
%    \chapter{Páginas interessantes na Internet} 
%    \label{chapter:paginas-interessantes}
%    \input{tex/annex/paginas-interessantes}
%
%\end{anexosenv}
% ---

\end{document}
