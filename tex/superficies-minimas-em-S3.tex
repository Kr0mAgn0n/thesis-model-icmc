\cite{Brendle2013}

\begin{definicao}
	A esfera unitária em $\realnumbers^4$, $S^3$, é o conjunto definido por:
	\begin{equation*}
		S^3 = \left\{ x \in \realnumbers^4: \| x \| = 1 \right\}
	\end{equation*}
\end{definicao}

\begin{definicao}
	Uma superfície em $S^3$ chama-se de \emph{superfície mínima} se a curvatura media é identicamente nula.
\end{definicao}

\begin{teorema}\label{propriedades_sup_min_S3}
	Seja $\Sigma$ uma superfície em $S^3$. As seguentes afirmações são equivalentes:
	\begin{enumerate}
		\item $\Sigma$ é uma superfície mínima
		\item $\Sigma$ é um ponto critico do funcional de área
		\item Se $\Sigma$ for descrito pelas funções coordenadas $x_i: \realnumbers \rightarrow \realnumbers$ onde $i=1,2,3,4$, então tem-se:
		\begin{equation*}
			\Delta_{\Sigma} x_i + 2 x_i = 0
		\end{equation*}
		para $i=1,2,3,4$.
	\end{enumerate}
\end{teorema}

%\begin{observacao}
%	No existem superfície mínimas compactas em $\realnumbers^3$.
%\end{observacao}

\begin{definicao}
	O \emph{equador} é um subconjunto de $S^3$ definido por:
	\begin{equation}
		M = \left\{ x \in S^3: x_4 = 0 \right\}
	\end{equation}
\end{definicao}

\begin{proposicao}
	As curvaturas principais do equador são nulas.
\end{proposicao}

\begin{demonstracao}
	É claro que o campo de vetores normais unitário é dado por $(0,0,0,1)$. Portanto a diferencial do campo é nulo e as curvaturas são nulas.
\end{demonstracao}

\begin{corolario}
	O equador é uma superfície mínima em $S^3$.
\end{corolario}

\begin{demonstracao}
	Como a equação da curvatura media em $S^3$ está dada pela soma das curvaturas principais, então, pela proposição anterior, a curvatura media é nula. Portanto é uma superfície mínima.
\end{demonstracao}

\begin{definicao}
	O \emph{toro de Clifford} é o subconjunto de $S^3$ definido por:
	\begin{equation*}
		M = \left\{ x \in S^3: x_1^2 + x_2^2 = x_3^2 + x_4^2 = \frac{1}{2} \right\}.
	\end{equation*}
\end{definicao}

\begin{proposicao}
	As curvaturas principais do toro de Clifford são 1 e -1.
\end{proposicao}

\begin{proposicao}
	O toro de Clifford é uma superfície mínima plana.
\end{proposicao}

\section{Quantas superfícies mínimas mergulhadas em $S^3$ existem?}

\begin{observacao}
	Por muito tempo o equador e o toro de Clifford foram os únicos exemplos de superfícies mínimas mergulhadas em $S^3$.
\end{observacao}

\begin{teorema}[Lawson]
	Existem ao menos uma superfície mínima mergulhada em $S^3$ de gênero $g$, onde $g$ é um numero natural.
\end{teorema}


\section{Unicidade das superfícies de gênero 0 e 1}

\begin{teorema}[Almgren]
	O equador é a única superfície mínima imersa de gênero 0 em $S^3$, salvo movimentos rígidos
\end{teorema}

\begin{observacao}[Conjetura de Lawson]
	O toro de Clifford é uma única superfície mínima mergulhada de gênero 1 em $S^3$, salvo movimentos rígidos
\end{observacao}