Superf\'icies m\'inimas constituem hoje um dos objetos de estudo
mais importantes em Geometria Diferencial. De particular interesse, 
s\~ao as superf\'icies m\'inimas em variedades de curvatura constante, 
como o espa\c co Euclidiano $\R^3$, o espa\c co hiperb\'olico $\Hy^3$
e a esfera $\Sp^3$. O caso de superf\'icies m\'inimas em $\R^3$ \'e um
assunto cl\'assico que tem despertado a aten\c c\~ao de v\'arias 
gera\c c\~oes de ge\^ometras, desde o problema inicial proposto por 
Lagrange at\'e os dias atuais. Nesta disserta\c c\~ao daremos
\^enfase ao estudo de superf\'icies m\'inimas na esfera $\Sp^3$, 
identificando-a com a esfera unit\'aria em $\R^4$, i.e.,
\[
\Sp^3=\{x\in\R^4:x_1^2+x_2^2+x_3^2+x_4^2=1\}.
\]



Ao contr\'ario do que ocorre em $\R^3$, de que n\~ao existem 
superf\'icies m\'inimas fechadas, existem exemplos interessantes
desse fen\^omeno na esfera $\Sp^3$. Um exemplo simples de 
superf\'icie m\'inima fechada em $\Sp^3$ \'e o {\em equador}
\[
M=\{x\in\Sp^3\subset\R^4:x_4=0\}.
\]
Neste caso, as curvaturas principais s\~ao ambas iguais a zero. 
Al\'em disso, o equador tem curvatura Gaussiana
constante igual a $1$. Assim, munido da m\'etrica induzida, $M$ \'e
isom\'etrica \`a esfera usual $\Sp^2$.

Outro exemplo de superf\'icie m\'inima em $\Sp^3$ \'e o {\em toro de
	Clifford}, definido por
\[
M=\left\{x\in\Sp^3:x_1^2+x_2^2=x_3^2+x_4^2=1/2\right\}.
\]
Neste caso, as curvaturas principais s\~ao iguais a $1$ e $-1$, 
resultando em $H=0$. Al\'em disso, a curvatura Gaussiana \'e 
identicamente nula e $M$, munido da m\'etrica induzida, \'e 
isom\'etrica ao toro flat 
$\Sp^1(\frac{1}{\sqrt2})\times\Sp^1(\frac{1}{\sqrt2})$.

Um problema cl\'assico nessa \'area \'e construir exemplos
de superf\'icies m\'inimas mergulhadas, i.e., superf\'icies sem 
auto-interse\c c\~oes. Durante um longo tempo, o equador e 
o toro de Clifford foram os \'unicos exemplos conhecidos de 
superf\'icies m\'inimas mergulhadas em $\Sp^3$. No entanto, 
no final da d\'ecada de 1960, Lawson \cite{Lawson1970} 
descobriu uma fam\'ilia infinita de superf\'icies m\'inimas 
mergulhadas em $\Sp^3$ de genus relativamente grande.

\begin{teorema}\label{teo:lawson}
	\cite{Lawson1970}. Dado um par de inteiros positivos $n$ e $k$, existe uma superf\'icie
	m\'inima mergulhada em $\Sp^3$ de genus $nk$. Em particular, 
	existe pelo menos uma superf\'icie m\'inima mergulhada em $\Sp^3$
	de qualquer genus $g$.
\end{teorema}

Um problema natural, relacionado \`a exist\^encia de tais superf\'icies,
\'e a quest\~ao da unicidade. Em 1966, Almgren \cite{Almgren1966} 
provou que, a menos de isometrias de $\Sp^3$, o equador \'e a \'unica
superf\'icie m\'inima imersa em $\Sp^3$ de genus $0$.
Em 1970, Blaine Lawson \cite{Lawson1970a} conjecturou uma 
propriedade similar de unicidade para toros m\'inimos na esfera
$\Sp^3$. Mais precisamente,

\begin{conjectura}\label{teo:Lawson}
	\cite{Lawson1970a}. A menos de isometrias de $\Sp^3$, o toro de Clifford \'e a \'unica 
	superf\'icie m\'inima mergulhada em $\Sp^3$ de genus $1$.
\end{conjectura}

A conjectura de Lawson \'e falsa se permitirmos que a superf\'icie
tenha auto-interse\c c\~oes (cf. \cite{Lawson1969}). Em 2012, 
Simon Brendle deu uma resposta positiva para esta conjectura. 
A prova de Brendle, apresentada em \cite{Brendle2013a}, envolve
uma aplica\c c\~ao do princ\'ipio do m\'aximo a uma fun\c c\~ao 
que depende em um par de pontos. Essa t\'ecnica foi introduzida
por Huisken \cite{Huisken1998} em seu trabalho que aborda o 
fluxo de curvas mergulhadas no plano.

Nesta disserta\c c\~ao de mestrado apresentaremos a 
demonstra\c c\~ao da conjectura de Lawson seguindo
o trabalho original de Brendle \cite{Brendle2013a}, onde
a prova \'e apresentada. O texto est\'a dividido em dois
cap\'itulos, que passaremos a descrever.

No Cap\'itulo 2 apresentamos alguns fatos b\'asicos da
teoria de superf\'icies m\'inimas no espa\c co Euclidiano
$\R^3$ e na esfera $\Sp^3$. Iniciamos o cap\'itulo 
relembrando as equa\c c\~oes fundamentais de uma 
imers\~ao isom\'etrica, dando destaque para a equa\c c\~ao
de Gauss quando o espa\c co ambiente \'e uma forma 
espacial tridimensional. Nas se\c c\~oes seguintes 
apresentamos alguns resultados b\'asicos da teoria de
superf\'icies m\'inimas em $\R^3$, dando \^enfase para
a representa\c c\~ao de Weierstrass. Finalizamos o
cap\'itulo com no\c c\~oes b\'asicas de superf\'icies 
m\'inimas em $\Sp^3$, onde damos enfoque no toro de
Clifford.

O Cap\'itulo 3 ser\'a dedicado, integralmente, \`a exposi\c c\~ao
da prova da conjectura de Lawson obtida por S. Brende
\cite{Brendle2013a}.
