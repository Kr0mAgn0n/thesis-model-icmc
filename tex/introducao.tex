%Esta dissertação apresentará a prova da conjectura de Lawson procurando desenvolver algumas resultados que são mostrados mas que não são provados. A conjectura foi publicada por H. B. Lawson em \cite{Lawson1970a}.
%O enunciado da conjectura de Lawson é o seguinte:
%\begin{teorema}[Conjectura de Lawson]\label{teo:Lawson}
%	Seja
%	$M$ uma 2-variedade diferenciável de gênero 1 e
%	$F:M \rightarrow S^3$ um mergulho mínimo.
%	Então a superfície mergulhada é isométrica ao toro de Clifford.
%\end{teorema}
%A prova da conjectura foi realizada por Simon Brendle em \cite{Brendle2013a}.
%Antes de desenvolver a prova da conjectura vai-se apresentar resultados da teoria de variedades riemannianas e superfícies mínimas.
%
%Ao longo da dissertação vai-se usar a notação de Einstein. 
%Para exemplificar, seja $V$ um espaço vetorial,
%$\{w_i\}$ uma base de $V$ e
%$v \in V$.
%Isto que dizer que a combinação linear
%\begin{equation*}
%	v = \sum_{i=1}^n a^i w_i
%\end{equation*}
%vai-se escrever
%\begin{equation*}
%	v = a^i w_i,
%\end{equation*}
%entendendo-se que existe uma suma em essa expressão.
%Quando houver confusão em uma expressão, vai-se notificar no texto.
%
%No capitulo 2 exporá-se a teoria de variedade riemannianas. O principal conceito a desenvolver-se aqui é a curvatura, em particular, a curvatura em hipersuperfícies, i.e., subvariedades riemannianas de codimensão 1.
%
%No capitulo 3 exporá-se a teoria de superfícies mínimas. Este capitulo está dividido em dois partes. Na primeira parte desenvolverá-se a teoria de superfícies mínimas em $\R^3$ e na segunda parte desenvolverá-se a teoria de superfícies mínimas em $S^3$. Na primeira parte apresentará-se principalmente o fato que em $\R^3$ não existem superfícies mínimas compactas e que existe uma solução ao problema de construir de superfícies mínimas em $\R^3$ mediante a teoria de representação de Wierstrass, que usa a teoria de superfícies de Riemann e Analise Complexa. Na segunda parte apresentará-se que, a diferença do caso de $\R^3$, em $S^3$ existem superfícies mínimas compactas e, como primeiros exemplos, exporá-se ao equador e ao toro de Clifford. 
%Também usará-se os resultados do capitulo 2 para expor propriedades das curvaturas das superfícies em $S^3$.
%No caso do equador, mencionará-se que F. Almgrem Jr. em \cite{Almgren1966} mostrou a unicidade (salvo isometrias) do equador como superfície mínima compacta imersa de gênero 0 em $S^3$. 
%Também, mencionará-se que H. B Lawson em \cite{Lawson1970} mostrou que, para todo inteiro positivo, existe ao menos uma superfície mínima compacta mergulhada em $S^3$ e, se o inteiro não é primo, então existem ao menos dois de essas superfícies.
%Por último, apresentará-se uma seção sobre o toro de Clifford onde calculará-se propriedades tais como: curvaturas principais, curvatura Gaussiana, media e, com a última, provar que o toro de Clifford é uma superfície mínima compacta em $S^3$. 
%%Na conjectura de Lawson, o fato de considerar que o toro seja mergulhado e não imerso é importante.
%%H. B. Lawson em \cite{Lawson1970} construiu uma família infinita de toros imersos em $S^3$.
%
%Por último, no capitulo 4, desenvolverá-se a prova da conjectura de Lawson.
%Na introdução do capitulo escreverá-se sobre a técnica usada na demostração: uma aplicação do principio do máximo numa função de duas variáveis.
%Esta técnica foi introduzida por G. Huisken em \cite{Huisken1998}.
%No que segue do capitulo, desenvolverá-se detalhadamente a prova da conjectura em \cite{Brendle2013a}.

Superf\'icies m\'inimas constituem hoje um dos objetos de estudo
mais importantes em Geometria Diferencial. De particular interesse, 
s\~ao as superf\'icies m\'inimas em variedades de curvatura constante, 
como o espa\c co Euclidiano $\R^3$, o espa\c co hiperb\'olico $\Hy^3$
e a esfera $\Sp^3$. O caso de superf\'icies m\'inimas em $\R^3$ \'e um
assunto cl\'assico que tem despertado a aten\c c\~ao de v\'arias 
gera\c c\~oes de ge\^ometras, desde o problema inicial proposto por 
Lagrange at\'e os dias atuais. Nesta disserta\c c\~ao daremos
\^enfase ao estudo de superf\'icies m\'inimas na esfera $\Sp^3$, 
identificando-a com a esfera unit\'aria em $\R^4$, i.e.,
\[
\Sp^3=\{x\in\R^4:x_1^2+x_2^2+x_3^2+x_4^2=1\}.
\]



Ao contr\'ario do que ocorre em $\R^3$, de que n\~ao existem 
superf\'icies m\'inimas fechadas, existem exemplos interessantes
desse fen\^omeno na esfera $\Sp^3$. Um exemplo simples de 
superf\'icie m\'inima fechada em $\Sp^3$ \'e o {\em equador}
\[
M=\{x\in\Sp^3\subset\R^4:x_4=0\}.
\]
Neste caso, as curvaturas principais s\~ao ambas iguais a zero. 
Al\'em disso, o equador tem curvatura Gaussiana
constante igual a $1$. Assim, munido da m\'etrica induzida, $M$ \'e
isom\'etrica \`a esfera usual $\Sp^2$.

Outro exemplo de superf\'icie m\'inima em $\Sp^3$ \'e o {\em toro de
	Clifford}, definido por
\[
M=\left\{x\in\Sp^3:x_1^2+x_2^2=x_3^2+x_4^2=1/2\right\}.
\]
Neste caso, as curvaturas principais s\~ao iguais a $1$ e $-1$, 
resultando em $H=0$. Al\'em disso, a curvatura Gaussiana \'e 
identicamente nula e $M$, munido da m\'etrica induzida, \'e 
isom\'etrica ao toro flat 
$\Sp^1(\frac{1}{\sqrt2})\times\Sp^1(\frac{1}{\sqrt2})$.

Um problema cl\'assico nessa \'area \'e construir exemplos
de superf\'icies m\'inimas mergulhadas, i.e., superf\'icies sem 
auto-interse\c c\~oes. Durante um longo tempo, o equador e 
o toro de Clifford foram os \'unicos exemplos conhecidos de 
superf\'icies m\'inimas mergulhadas em $\Sp^3$. No entanto, 
no final da d\'ecada de 1960, Lawson \cite{Lawson1970} 
descobriu uma fam\'ilia infinita de superf\'icies m\'inimas 
mergulhadas em $\Sp^3$ de genus relativamente grande.

\begin{teorema}[Lawson \cite{Lawson1970}] \label{teo:lawson}
	Dado um par de inteiros positivos $n$ e $k$, existe uma superf\'icie
	m\'inima mergulhada em $\Sp^3$ de genus $nk$. Em particular, 
	existe pelo menos uma superf\'icie m\'inima mergulhada em $\Sp^3$
	de qualquer genus $g$.
\end{teorema}

Um problema natural, relacionado \`a exist\^encia de tais superf\'icies,
\'e a quest\~ao da unicidade. Em 1966, Almgren \cite{Almgren1966} 
provou que, a menos de isometrias de $\Sp^3$, o equador \'e a \'unica
superf\'icie m\'inima imersa em $\Sp^3$ de genus $0$.
Em 1970, Blaine Lawson \cite{Lawson1970a} conjecturou uma 
propriedade similar de unicidade para toros m\'inimos na esfera
$\Sp^3$. Mais precisamente,

\begin{conjectura}[Lawson]\label{teo:Lawson}
	A menos de isometrias de $\Sp^3$, o toro de Clifford \'e a \'unica 
	superf\'icie m\'inima mergulhada em $\Sp^3$ de genus $1$.
\end{conjectura}

A conjectura de Lawson \'e falsa se permitirmos que a superf\'icie
tenha auto-interse\c c\~oes (cf. \cite{Lawson1969}). Em 2012, 
Simon Brendle deu uma resposta positiva para esta conjectura. 
A prova de Brendle, apresentada em \cite{Brendle2013a}, envolve
uma aplica\c c\~ao do princ\'ipio do m\'aximo a uma fun\c c\~ao 
que depende em um par de pontos. Essa t\'ecnica foi introduzida
por Huisken \cite{Huisken1998} em seu trabalho que aborda o 
fluxo de curvas mergulhadas no plano.

Nesta disserta\c c\~ao de mestrado apresentaremos a 
demonstra\c c\~ao da conjectura de Lawson seguindo
o trabalho original de Brendle \cite{Brendle2013a}, onde
a prova \'e apresentada. O texto est\'a dividido em dois
cap\'itulos, que passaremos a descrever.

No Cap\'itulo 2 apresentamos alguns fatos b\'asicos da
teoria de superf\'icies m\'inimas no espa\c co Euclidiano
$\R^3$ e na esfera $\Sp^3$. Iniciamos o cap\'itulo 
relembrando as equa\c c\~oes fundamentais de uma 
imers\~ao isom\'etrica, dando destaque para a equa\c c\~ao
de Gauss quando o espa\c co ambiente \'e uma forma 
espacial tridimensional. Nas se\c c\~oes seguintes 
apresentamos alguns resultados b\'asicos da teoria de
superf\'icies m\'inimas em $\R^3$, dando \^enfase para
a representa\c c\~ao de Weierstrass. Finalizamos o
cap\'itulo com no\c c\~oes b\'asicas de superf\'icies 
m\'inimas em $\Sp^3$, onde damos enfoque no toro de
Clifford.

O Cap\'itulo 3 ser\'a dedicado, integralmente, \`a exposi\c c\~ao
da prova da conjectura de Lawson obtida por S. Brende
\cite{Brendle2013a}.
