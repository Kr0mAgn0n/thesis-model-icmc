Esta dissertação apresentará a prova da conjectura de Lawson procurando desenvolver algumas resultados que são mostrados mas que não são provados. A conjectura foi publicada por H. B. Lawson em \cite{Lawson1970a}.
O enunciado da conjectura de Lawson é o seguinte:
\begin{teorema}[Conjectura de Lawson]\label{teo:Lawson}
	Seja
	$M$ uma 2-variedade diferenciável de gênero 1 e
	$F:M \rightarrow S^3$ um mergulho mínimo.
	Então a superfície mergulhada é isométrica ao toro de Clifford.
\end{teorema}
A prova da conjectura foi realizada por Simon Brendle em \cite{Brendle2013a}.
Antes de desenvolver a prova da conjectura vai-se apresentar resultados da teoria de variedades riemannianas e superfícies mínimas.

Ao longo da dissertação vai-se usar a notação de Einstein. 
Para exemplificar, seja $V$ um espaço vetorial,
$\{w_i\}$ uma base de $V$ e
$v \in V$.
Isto que dizer que a combinação linear
\begin{equation*}
	v = \sum_{i=1}^n a^i w_i
\end{equation*}
vai-se escrever
\begin{equation*}
	v = a^i w_i,
\end{equation*}
entendendo-se que existe uma suma em essa expressão.
Quando houver confusão em uma expressão, vai-se notificar no texto.

No capitulo 2 exporá-se a teoria de variedade riemannianas. O principal conceito a desenvolver-se aqui é a curvatura, em particular, a curvatura em hipersuperfícies, i.e., subvariedades riemannianas de codimensão 1.

No capitulo 3 exporá-se a teoria de superfícies mínimas. Este capitulo está dividido em dois partes. Na primeira parte desenvolverá-se a teoria de superfícies mínimas em $\R^3$ e na segunda parte desenvolverá-se a teoria de superfícies mínimas em $S^3$. Na primeira parte apresentará-se principalmente o fato que em $\R^3$ não existem superfícies mínimas compactas e que existe uma solução ao problema de construir de superfícies mínimas em $\R^3$ mediante a teoria de representação de Wierstrass, que usa a teoria de superfícies de Riemann e Analise Complexa. Na segunda parte apresentará-se que, a diferença do caso de $\R^3$, em $S^3$ existem superfícies mínimas compactas e, como primeiros exemplos, exporá-se ao equador e ao toro de Clifford. 
Também usará-se os resultados do capitulo 2 para expor propriedades das curvaturas das superfícies em $S^3$.
No caso do equador, mencionará-se que F. Almgrem Jr. em \cite{Almgren1966} mostrou a unicidade (salvo isometrias) do equador como superfície mínima compacta imersa de gênero 0 em $S^3$. 
Também, mencionará-se que H. B Lawson em \cite{Lawson1970} mostrou que, para todo inteiro positivo, existe ao menos uma superfície mínima compacta mergulhada em $S^3$ e, se o inteiro não é primo, então existem ao menos dois de essas superfícies.
Por último, apresentará-se uma seção sobre o toro de Clifford onde calculará-se propriedades tais como: curvaturas principais, curvatura Gaussiana, media e, com a última, provar que o toro de Clifford é uma superfície mínima compacta em $S^3$. 
%Na conjectura de Lawson, o fato de considerar que o toro seja mergulhado e não imerso é importante.
%H. B. Lawson em \cite{Lawson1970} construiu uma família infinita de toros imersos em $S^3$.

Por último, no capitulo 4, desenvolverá-se a prova da conjectura de Lawson.
Na introdução do capitulo escreverá-se sobre a técnica usada na demostração: uma aplicação do principio do máximo numa função de duas variáveis.
Esta técnica foi introduzida por G. Huisken em \cite{Huisken1998}.
No que segue do capitulo, desenvolverá-se detalhadamente a prova da conjectura em \cite{Brendle2013a}.