%\cite{Brendle2013}
%\cite{Brendle2013a}

%\begin{proposicao}\label{nao_tem_pontos_umbilicos}
%	Um superfície mínima imersa em $S^3$ de gênero 1 não tem pontos umbílicos, i.e., a segunda forma fundamental é não nula em cada ponto da superfície
%\end{proposicao}

%\begin{proposicao}
%	Supor que $F: M \rightarrow S^3$ é um toro mínimo mergulhado em $S^3$. Então a norma da segunda forma fundamental satisfaz a equação em derivadas parciais
%	\begin{equation*}
%		\Delta_M (|A|) - \frac{| \nabla |A| |^2}{|A|} + (|A|^2 - 2) |A| = 0.
%	\end{equation*}
%	onde $A$ é a segunda forma fundamental.
%\end{proposicao}

Neste capítulo apresentaremos a demonstração da conjetura de Lawson obtida por Simon Brendle \cite{Brendle2013a}.

\section{Introdução}

Em 1970, Blaine Lawson \cite{Lawson1970a} conjeturou que o toro de Clifford é a única superfície mínima, compacta e mergulhada em $S^3$, com genus 1, e foi provada somente quatro décadas depois por Simon Brendle \cite{Brendle2013a} em 2013. A hipótese de ser mergulhada é fundamental. De fato, em \cite{Lawson1970} Lawson construiu uma família (infinita) de imersões mínimas de toros em $S^3$.

A prova da conjetura de Lawson em \cite{Brendle2013a} envolve uma aplicação do princípio do máximo a uma função que depende de uma par de pontos. Esta técnica foi inicialmente desenvolvida por Huisken \cite{Huisken1998} no estudo de fluxo de comprimento de curvas para curvas mergulhadas no plano e, posteriormente, por Andrews \cite{Andrews2012}.

Descreveremos a seguir os argumentos usados em \cite{Brendle2013a}. Seguindo as ideias e notações de \cite{Andrews2012}, considere uma superfície mínima e mergulhada $ F: \Sigma \rightarrow S^3 $ e $ \Phi: \Sigma \rightarrow \realnumbers $  uma função positiva. Definia uma função $ Z: \Sigma \times \Sigma \rightarrow \realnumbers $ pondo
\begin{equation}\label{def_de_Z}
	Z(x,y) = \Phi(x) (1 - \langle F(x),F(y) \rangle ) + \langle \nu(x), F(y) \rangle,
\end{equation}
onde $ \nu $ é um campo unitário , normal a $ \Sigma $.

A função $Z$ tem a seguinte interpretação geométrica:

\begin{proposicao}
	A função $ Z(x,y) $, definida em \ref{def_de_Z}, é não-negativa se, e somente se, para todo ponto $ x \in \Sigma $, existe uma bola geodésica $B$ contida em $\Omega$, com curvatura média do bordo igual a $ \Phi(x) $ e $ F(x) \in \partial B $.
\end{proposicao}

\begin{demonstracao}
	Observe, inicialmente, que uma bola geodésica em $S^3$ é simplesmente a interseção de uma bola de $\realnumbers^4$ com $S^3$. Em particular, a bola em $S^3$ contida em $\Omega$, com curvatura média do bordo igual a $\Phi$ e que é tangente a $ F(\Sigma) $ no ponto $F(x)$ é $B=B(p)$, onde 
	\begin{equation*}
		p = F(x) - \Phi^{-1}(x) \nu(x),
	\end{equation*}
	e $\nu$ é o campo unitário, normal a $F(\Sigma)$ no ponto $F(x) \in S^3$, e que aponta para fora de $\Omega$.
	
	Afirmar que uma bola geodésica de $S^3$ está inteiramente contida em $\Omega$ é equivalente a afirmar que, para cada ponto $y \in \Sigma$, tem-se
	\begin{equation*}
		| F(y) - p | \geq \Phi^{-2}(x),
	\end{equation*}
	ou seja,
	\begin{equation*}
		| F(y) - ( F(x) - \Phi^{-1}(x) \nu(x) ) |^2 - \Phi^2(x) \geq 0.
	\end{equation*}
	
	Desenvolvendo o lado esquerdo da desigualdade acima e multiplicando por $ \Phi(x)/2 $, obtemos:
	\begin{equation*}
		\frac{\Phi(x)}{2} | F(y) - F(x) |^2 + \langle F(y) - F(x), \nu(x) \rangle \geq 0.
	\end{equation*}
	
	Como $F(x), F(y) \in S^3$, tem-se
	\begin{equation*}
		|F(x)|^2 = |F(y)|^2 = 1 \quad \text{e} \quad \langle F(x), \nu(x) \rangle = 0.
	\end{equation*}
	
	Asim,
	\begin{equation*}
		\Phi(x) (1 - \langle F(x), F(y) \rangle) + \langle \nu(x), F(y) \rangle \geq 0,
	\end{equation*} 
	como queríamos.
\end{demonstracao}


\section{Alguns resultados técnicos}

Como na seção anterior, sejam $F: \Sigma \rightarrow S^3$ uma superfície mínima e mergulhada, $\Phi: \Sigma \rightarrow \realnumbers$ uma função positiva e considere a função
\begin{equation*}
	Z(x,y) = \Phi(x) (1 - \langle F(x), F(y) \rangle) + \langle \nu(x), F(y) \rangle.
\end{equation*}

Considere dois pontos $\xbarra, \ybarra \in \Sigma$, com $\xbarra \neq \ybarra$, tais que $Z(\xbarra, \ybarra) = 0$ e $dZ(\xbarra, \ybarra) = 0$. Sejam $(x_1,x_2), (y_1,y_2)$ sistemas de coordenadas geodésicas em torno dos pontos $\xbarra, \ybarra$, respectivamente. No ponto $(\xbarra,\ybarra)$, temos:
\begin{multline}\label{diff_Z_x}
0 = \frac{\partial Z}{\partial x_i} (\xbarra, \ybarra) =  \frac{\partial \Phi}{\partial x_i}(\xbarra) (1 - \langle F(\xbarra, \ybarra) \rangle) -  \Phi(\xbarra) \left\langle \frac{\partial F}{\partial x_i}(\xbarra), F(\ybarra) \right\rangle\\
+ \sum_{k=1}^{2} h_{ik}(\xbarra) \left\langle \frac{\partial F}{\partial x_k}(\xbarra), F(\ybarra) \right\rangle
\end{multline}
e
\begin{equation}\label{diff_Z_y}
0 = \frac{\partial Z}{\partial y_i} (\xbarra, \ybarra) = - \Phi(\xbarra) \left\langle F(\xbarra), \frac{\partial F}{\partial y_i} (\ybarra) \right\rangle + \left\langle \nu(\xbarra), \frac{\partial F}{\partial y_i}(\ybarra) \right\rangle
\end{equation}
onde $h_{ik}(\xbarra)$ denota a $(ik)$-ésima coordenada da matriz da segunda forma fundamental de $F$ no ponto $\xbarra$.


%\begin{definicao}
%	A função $Z: \Sigma \times \Sigma \rightarrow \realnumbers$ é definida por
%	\begin{equation*}
%		Z(x,y) =  \Phi(x) (1 - \langle F(x), F(y) \rangle) + \langle \nu(x), F(y) \rangle.
%	\end{equation*}
%	onde $F: \Sigma \rightarrow S^3$ é uma imersão mínima de gênero 1 em $S^3$, e $\nu(x) \in T_{F(x)} S^3$ é um campo vectorial normal unitário.
%\end{definicao}

%\begin{observacao}
%	 A partir de agora vamos supor que para $p \in \Sigma$ existe $q \in \Sigma$ tal que $Z(p,q) = \frac{\partial Z}{\partial x_i}(p,q) = 0$ para $i = 1,2$. Ista suposição é feita para ser usada nos cálculos posteriores.
%\end{observacao}

%\begin{proposicao}\label{primeira_derivada_x_Z}
%	As primeiras derivadas da função $Z$ com respeito a $x_1$ e $x_2$ estão dadas por:
%	\begin{equation}\label{diff_Z_x}
%	\frac{\partial Z}{\partial x_i} (p, q) =  \frac{\partial \Phi}{\partial x_i}(p) (1 - \langle F(p, q) \rangle) -  \Phi(p) \left\langle \frac{\partial F}{\partial x_i}(p), F(q) \right\rangle + h_i^k(p) \left\langle \frac{\partial F}{\partial x_k}(p), F(q) \right\rangle
%	\end{equation}
%\end{proposicao}


\begin{observacao}
	Derivando com respeito a $x_1$ e $x_2$:
	\begin{equation*}
		\frac{\partial Z}{\partial x_i} (p, q) =  \frac{\partial \Phi}{\partial x_i}(p) (1 - \langle F(p, q) \rangle) -  \Phi(p) \left\langle \frac{\partial F}{\partial x_i}(p), F(q) \right\rangle + \left\langle \frac{\partial \nu}{\partial x_i}(p), F(q) \right\rangle
	\end{equation*}
	
	Em quanto ao termo $ \left\langle \frac{\partial \nu}{\partial x_i}(p), F(q) \right\rangle $, olhar que $\langle \nu(x), \nu(x) \rangle=1, \forall x \in \Sigma$. Então, quando se derivar se tem:
	\begin{equation*}
		\left\langle \frac{\partial \nu}{\partial x_i}(x), \nu(x) \right\rangle = 0, \forall x \in \Sigma.
	\end{equation*}
	
	Isto quer dizer que $\frac{\partial \nu}{\partial x_i}(p) \in T_{p} \Sigma$. Olhar que estamos identificando $p$ com $F(p)$. Nós sabemos que $\{ \frac{\partial F}{\partial x_i}(p) \}$ gera $T_{p} \Sigma$, então, usando a notação de Einstein, podemos expressar $\frac{\partial \nu}{\partial x_i}(p)$ como combinação linear dos vetores anteriores.
	\begin{equation}\label{der_nu_x}
		\frac{\partial \nu}{\partial x_i} (p) = h_i^k \frac{\partial F}{\partial x_k}
	\end{equation}
	
	Lembrar que a notação de Einstein está expressando uma suma com respeito ao índice $k$ que toma os valores 1 e 2.
	
	Portanto ao final temos a expressão requerida.
\end{observacao}


\begin{definicao}
	Denote-se por $w_i$ a reflexão do vetor
	\begin{equation*}
		\frac{\partial F}{\partial x_i}(\xbarra)
	\end{equation*}
	com respeito ao plano ortogonal a $F(\xbarra) - F(\ybarra)$, i.e.
	\begin{equation*}
		w_i = \frac{\partial F}{\partial x_i}(\xbarra) - 2 \left\langle \frac{\partial F}{\partial x_i}(\xbarra), \frac{F(\xbarra) - F(\ybarra)}{|F(\xbarra) - F(\ybarra)|} \right\rangle \frac{F(\xbarra) - F(\ybarra)}{|F(\xbarra) - F(\ybarra)|}.
	\end{equation*}
	
	
\end{definicao}

Escolhendo um sistema de coordenadas $(y_1,y_2)$ apropriado pode-se ter
\begin{equation*}
\left\langle w_1, \frac{\partial F}{\partial y_1}(\ybarra) \right\rangle \geq 0, \quad \left\langle w_1, \frac{\partial F}{\partial y_2}(\ybarra) \right\rangle = 0 \quad \text{e} \quad \left\langle w_2, \frac{\partial F}{\partial y_2}(\ybarra) \right\rangle \geq 0.
\end{equation*}

\begin{lema}
	Os vetores $F(\ybarra)$ e $ \Phi(\xbarra) F(\xbarra) - \nu(\xbarra) $ são linearmente independentes.
\end{lema}

\begin{demonstracao}
	Pela definição de $Z$ tem-se
	\begin{equation*}
		\langle \Phi(\xbarra) F(\xbarra) - \nu(\xbarra), F(\ybarra) \rangle = \Phi(\xbarra) - Z(\xbarra,\ybarra) = \Phi(\xbarra).
	\end{equation*}
	
	Portanto
	\begin{equation*}
		| \Phi(\xbarra) F(\xbarra) - \nu(\xbarra) |^2 | F(\ybarra) |^2 - \langle \Phi(\xbarra) F(\xbarra) - \nu(\xbarra), F(\ybarra) \rangle^2 = | \Phi(\xbarra) F(\xbarra) - \nu(\xbarra) |^2 - \Phi(\xbarra)^2 = 1 \neq 0
	\end{equation*}
\end{demonstracao}

\begin{lema}
	Temos\begin{equation*}
		w_1 = \frac{\partial F}{\partial y_1}(\ybarra) \quad \text{e} \quad w_2 = \frac{\partial F}{\partial y_2}(\ybarra)
	\end{equation*}
\end{lema}

\begin{demonstracao}
	Tem-se
	\begin{align*}
		\langle w_i, F(\ybarra) \rangle &= \left\langle \frac{\partial F}{\partial x_i}(\xbarra), F(\ybarra) \right\rangle + 2 \left\langle \frac{\partial F}{\partial x_i}, F(\ybarra) \right\rangle \frac{\langle F(\xbarra) - F(\ybarra),F(\ybarra) \rangle}{| F(\xbarra) - F(\ybarra) |^2}\\
		&= \left\langle \frac{\partial F}{\partial x_i}(\xbarra), F(\ybarra) \right\rangle + 2 \left\langle \frac{\partial F}{\partial x_i}, F(\ybarra) \right\rangle \frac{\langle F(\xbarra),F(\ybarra) \rangle - 1}{2 - 2 \langle F(\xbarra), F(\ybarra) \rangle}\\
		&= 0
	\end{align*}
	e
	\begin{align*}
		\langle w_i, \Phi(\xbarra) F(\xbarra) - \nu(\xbarra) \rangle &= 2 \left\langle \frac{\partial F}{\partial x_i}(\xbarra), F(\ybarra) \right\rangle \frac{\langle F(\xbarra) - F(\ybarra), \Phi(\xbarra) F(\xbarra) - \nu(\xbarra) \rangle}{| F(\xbarra) - F(\ybarra) |^2}\\
		&= 2 \left\langle \frac{\partial F}{\partial x_i}(\xbarra), F(\ybarra) \right\rangle \frac{Z(\xbarra,\ybarra)}{| F(\xbarra) - F(\ybarra) |^2} = 0
	\end{align*}
	
	Por outro lado, os vetores
	\begin{equation*}
		\frac{\partial F}{\partial y_1}(\ybarra) \quad \text{e} \quad \frac{\partial F}{\partial y_2}(\ybarra)
	\end{equation*}
	satisfazem
	\begin{equation*}
		\left\langle \frac{\partial F}{\partial y_i}(\ybarra), F(\ybarra) \right\rangle = 0
	\end{equation*}
	e
	\begin{equation*}
		\left\langle \frac{\partial F}{\partial y_i}(\ybarra), \Phi(\xbarra) F(\xbarra) - \nu(\xbarra) \right\rangle = -\frac{\partial Z}{\partial y_i}(\xbarra,\ybarra) = 0
	\end{equation*}
	
	Com estos resultados se conclui que o plano gerado por
	\begin{equation*}
		\frac{\partial F}{\partial y_1}(\ybarra) \quad \text{e} \quad \frac{\partial F}{\partial y_2}(\ybarra)
	\end{equation*}
	é o mesmo plano gerado por $w_1$ e $w_2$.
	
	Pela definição de $w_i$ tem-se que $w_1$ e $w_2$ são ortonormais como $\frac{\partial F}{\partial y_1}(\ybarra)$ e $\frac{\partial F}{\partial y_2}(\ybarra)$. Adicionalmente tem-se a condição
	\begin{equation*}
		\left\langle w_1, \frac{\partial F}{\partial y_2}(\ybarra) \right\rangle = 0,
	\end{equation*}
	portanto
	\begin{equation*}
		w_1 = \pm \frac{\partial F}{\partial y_1}(\ybarra) \quad \text{e} \quad w_2 = \pm \frac{\partial F}{\partial y_2}(\ybarra)
	\end{equation*}
	
	Finalmente, tem-se a condição 
	\begin{equation*}
		\left\langle w_1, \frac{\partial F}{\partial y_1}(\ybarra) \right\rangle \geq 0 \quad \text{e} \quad \left\langle w_2, \frac{\partial F}{\partial y_2}(\ybarra) \right\rangle \geq 0
	\end{equation*}	
	com o que se conclui
	\begin{equation*}
	w_1 = \frac{\partial F}{\partial y_1}(\ybarra) \quad \text{e} \quad w_2 = \frac{\partial F}{\partial y_2}(\ybarra)
	\end{equation*}
\end{demonstracao}
%\begin{proposicao}
%	As primeiras derivadas da função $Z$ com respeito a $y_1$ e $y_2$ estão dadas por:
%	\begin{equation}\label{diff_Z_y}
%		\frac{\partial Z}{\partial y_i} (p, q) = - \Phi(p) \left\langle F(p), \frac{\partial F}{\partial y_i} (q) \right\rangle + \left\langle \nu(p), \frac{\partial F}{\partial y_i}(q) \right\rangle
%	\end{equation}
%\end{proposicao}

%\begin{demonstracao}
%	Cálculo simples.
%\end{demonstracao}



\begin{proposicao}
	O Laplaciano de $Z$ com respeito a $x$ satisfaz a desigualdade
	\begin{multline}\label{2_diff_Z_x}
		\sum_{i=1}^{2} \frac{\partial^2 Z}{\partial x_i^2} (p,q) = \left( \Delta_{\Sigma} \Phi(p) - \frac{| \nabla \Phi(p) |^2}{\Phi(p)} + (|A(p)|^2 - 2) \Phi(p) \right) (1 - \left\langle F(p), F(q) \right\rangle) + 2 \Phi(p)\\
		 - \frac{2 \Phi(p)^2 - |A(p)|^2}{2 \Phi(p) (1 - \langle F(p), F(q) \rangle)} \sum_i \left\langle \frac{\partial F}{\partial x_i}(p), F(q) \right\rangle^2
	\end{multline}
\end{proposicao}

\begin{demonstracao}
	Vamos derivar com respeito a $x_i$ à equação \ref{diff_Z_x} e depois vamos somar com respeito ao índice $i$.
	\begin{multline}\label{Z_seg_dev_x}
		\sum_i \frac{\partial^2 Z}{\partial x_i^2}(x,y) = \sum_i \frac{\partial^2 \Phi}{\partial x_i^2}(x)(1 - \langle F(x), F(y) \rangle) -2  \sum_i \frac{\partial \Phi}{\partial x_i} \left\langle \frac{\partial F}{\partial x_i}(x), F(y) \right\rangle\\
		 -  \Phi(x) \left\langle \sum_i \frac{\partial^2 F}{\partial x_i^2}(x), F(y) \right\rangle + \sum_i \frac{\partial h_i^k}{\partial x_i}(x) \left\langle \frac{\partial F}{\partial x_k}, F(y) \right\rangle + \sum_i h_i^k(x) \left\langle \frac{\partial^2 F}{\partial x_i \partial x_k}, F(y) \right\rangle
	\end{multline}
	
	Lembrar que o índice $k$ representa uma somatória. Vamos analisar o termo $\sum_i \frac{\partial^2 F}{\partial x_i^2}(x)$. Pelo teorema \ref{propriedades_sup_min_S3} se tem
	\begin{equation*}
		\sum_i \frac{\partial^2 F}{\partial x_i^2}(x) + 2 F(x) = 0.
	\end{equation*}
	
	Então $\sum_i \frac{\partial^2 F}{\partial x_i^2}(x) = -2 F(x) $.
	
	Agora vamos analisar o termo $\sum_i \frac{\partial h_i^k}{\partial x_i}(x)$. Pela equação de Codazzi se tem
	\begin{equation}\label{codazzi_eq}
		\begin{split}
		-\frac{\partial h_1^1}{\partial x_1} = \frac{\partial h_2^2}{\partial x_1} = \frac{\partial h_1^2}{\partial x_2} = \frac{\partial h_2^1}{\partial x_2}\\
		-\frac{\partial h_2^2}{\partial x_2} = \frac{\partial h_1^1}{\partial x_2} = \frac{\partial h_2^1}{\partial x_1} = \frac{\partial h_1^2}{\partial x_1}.
		\end{split}		
	\end{equation}
	
	Olhar que 
	\begin{equation*}
		A(x) = \left[\begin{matrix}
		h_1^1(x) & h_1^2(x)\\
		h_2^1(x) & h_2^2(x)
		\end{matrix}\right],
	\end{equation*}
	
	a superfície é minima e o fato as matrizes semelhantes tem a mesma traça, então temos $h_1^1(x) + h_2^2 = 0$. Portanto $\sum_i \frac{\partial h_i^k}{\partial x_i} = 0$.
	
	Analisando o termo $\sum_i h_i^k(x) \left\langle \frac{\partial^2 F}{\partial x_i \partial x_k}, F(y) \right\rangle$, derivamos \ref{der_nu_x} com respeito a $x_i$ e somamos com o índice $i$
	\begin{equation*}
		\sum_i \frac{\partial^2 \nu}{\partial x_i^2}(x) = \sum_i \frac{\partial h_i^k}{\partial x_i}(x) \frac{\partial F}{\partial x_k}(x) + \sum_i h_i^k \frac{\partial^2 F}{\partial x_i \partial x_k}(x)
	\end{equation*}
	
	Lembrando $\sum_i \frac{\partial h_i^k}{x_i}(x) = 0$, e expandindo a somatória  temos:
	\begin{equation*}
		\sum_i \frac{\partial^2 \nu}{\partial x_i^2}(x) =  h_1^1 \frac{\partial^2 F}{\partial x_1^2}(x) + h_1^2 \frac{\partial^2 F}{\partial x_1 \partial x_2}(x) + h_2^1 \frac{\partial^2 F}{\partial x_2 \partial x_1}(x) + h_2^2 \frac{\partial^2 F}{\partial x_2^2}(x)
	\end{equation*}
	
	Fazendo produto interno com $\frac{\partial F}{\partial x_1}(x)$:
	\begin{multline*}
		\left\langle \sum_i \frac{\partial^2 \nu}{\partial x_i^2}(x), \frac{\partial F}{\partial x_1}(x) \right\rangle =\\
		\left\langle h_1^1 \frac{\partial^2 F}{\partial x_1^2}(x) + h_1^2 \frac{\partial^2 F}{\partial x_1 \partial x_2}(x) + h_2^1 \frac{\partial^2 F}{\partial x_2 \partial x_1}(x) + h_2^2 \frac{\partial^2 F}{\partial x_2^2}(x), \frac{\partial F}{\partial x_1}(x) \right\rangle.
	\end{multline*}
	
	Temos que $\left\langle \frac{\partial F}{\partial x_1}, \frac{\partial F}{\partial x_1} = 1 \right\rangle$. Derivando com respeito a $x_1$ se tem $\left\langle \frac{\partial^2 F}{\partial x_1^2}, \frac{\partial F}{\partial x_1} \right\rangle = 0$. Igualmente derivando com respeito a $x_2$ e lembrando que as derivadas mistas comutam temos que $\left\langle \frac{\partial^2 F}{\partial x_1 \partial x_2}(x), \frac{\partial F}{\partial x_1} \right\rangle = \left\langle \frac{\partial^2 F}{\partial x_2 \partial x_1}(x), \frac{\partial F}{\partial x_1} \right\rangle = 0$. Analisando $\left\langle \frac{\partial^2 F}{\partial x_2^2}, \frac{\partial F}{\partial x_1} \right\rangle$, lembrando que $\left\langle \frac{\partial F}{\partial x_1}, \frac{\partial F}{\partial x_2} \right\rangle = 0$ derivamos com respeito a $x_2$:
	\begin{equation*}
		\left\langle \frac{\partial^2 F}{\partial x_2^2}, \frac{\partial F}{\partial x_1} \right\rangle + \left\langle \frac{\partial F}{\partial x_2}, \frac{\partial^2 F}{\partial x_2 \partial x_1} \right\rangle = 0
	\end{equation*}
	
	Sabemos que $\left\langle \frac{\partial F}{\partial x_2}, \frac{\partial^2 F}{\partial x_2 \partial x_1} \right\rangle = 0$. Portanto, $\left\langle \frac{\partial^2 F}{\partial x_2^2}, \frac{\partial F}{\partial x_1} \right\rangle = 0$.
	
	Logo
	\begin{equation*}
		\left\langle \sum_i \frac{\partial^2 \nu}{\partial x_i^2}(x), \frac{\partial F}{\partial x_1}(x) \right\rangle = 0.
	\end{equation*}
	
	De maneira similar:
	\begin{equation*}
		\left\langle \sum_i \frac{\partial^2 \nu}{\partial x_i^2}(x), \frac{\partial F}{\partial x_2}(x) \right\rangle = 0.
	\end{equation*}
	
	Isto quer dizer que $\sum_i \frac{\partial^2 \nu}{\partial x_i^2}(p)$ não tem componentes no plano tangente $T_p \Sigma$ dentro de $T_p S^3$.
	
	Por outro lado, lembrado que $\left\langle \frac{\partial \nu}{\partial x_i}(x), \nu(x) \right\rangle = 0$, derivamos com respeito a $x_1$:
	\begin{equation*}
		\left\langle \frac{\partial^2 \nu}{\partial x_i^2}(x), \nu(x) \right\rangle + \left\langle \frac{\partial \nu}{\partial x_i}(x), \frac{\partial \nu}{\partial x_i}(x) \right\rangle = 0
	\end{equation*}
	
	Como $\frac{\partial \nu}{\partial x_i}(x) = h_i^k(x) \frac{\partial F}{\partial x_k}$, então
	\begin{equation*}
		\left\langle \frac{\partial^2 \nu}{\partial x_i^2}(x), \nu(x) \right\rangle + \left\langle h_i^k(x) \frac{\partial F}{\partial x_k}, h_i^j(x) \frac{\partial F}{\partial x_j}  \right\rangle = 0.
	\end{equation*}
	
	Como $\left\langle \frac{\partial F}{\partial x_k}, \frac{\partial F}{\partial x_j} \right\rangle = \delta_{kj}$, então
	\begin{equation*}
		\left\langle \frac{\partial^2 \nu}{\partial x_i^2}(x), \nu(x) \right\rangle + (h_i^k)^2 = 0.
	\end{equation*}
	
	Logo, somando com respeito a $i$
	\begin{equation*}
		\left\langle \sum_i \frac{\partial^2 \nu}{\partial x_i^2}(x), \nu(x) \right\rangle + \sum_i (h_i^k)^2 = 0.
	\end{equation*}
	
	Olhar que $\sum_i (h_i^k(x))^2$ é $| A(x) |^2$ pela norma de Frobenius. Portanto, lembrando que $\sum_i \frac{\partial^2 \nu}{\partial x_i^2}(x)$ só tem componentes na direção $\nu$, temos
	\begin{equation*}
		\sum_i \frac{\partial^2 \nu}{\partial x_i^2}(p) = - | A(p) |^2 \nu(p)
	\end{equation*}
	
	Portanto, temos:
	\begin{equation*}
		\sum_i h_i^k \frac{\partial^2 F}{\partial x_i \partial x_k}(p) = \sum_i \frac{\partial^2 \nu}{\partial x_i^2}(p) = - | A(p) |^2 \nu(p)
	\end{equation*}
	
	Ao final a equação \ref{Z_seg_dev_x} se escreve:
	\begin{multline}\label{lap_Z_x}
		\sum_i \frac{\partial^2 Z}{\partial x_i^2}(p,q) =  \sum_i \frac{\partial^2 \Phi}{\partial x_i^2}(p)(1 - \langle F(p), F(q) \rangle) -2  \sum_i \frac{\partial \Phi}{\partial x_i} \left\langle \frac{\partial F}{\partial x_i}(p), F(q) \right\rangle\\
		+ 2  \Phi(p) \left\langle F(p), F(q) \right\rangle - | A(p) |^2 \left\langle \nu(p), F(q) \right\rangle
	\end{multline}
	
	Multiplicando por $|A(p)|^2$ a $Z(p,q)$
	\begin{equation*}
		0 = |A(p)|^2 Z(p,q) = |A(p)|^2 \Phi(p) (1 - \langle F(p), F(q) \rangle) + |A(p)|^2 \langle \nu(p), F(q) \rangle
	\end{equation*}
	
	temos $-|A(p)|^2 \langle \nu(p), F(q) \rangle = |A(p)|^2 \Phi(p) (1 - \langle F(p), F(q) \rangle)$. Usando isto em \ref{lap_Z_x}:
	\begin{multline*}
	\sum_i \frac{\partial^2 Z}{\partial x_i^2}(p,q) =  \sum_i \frac{\partial^2 \Phi}{\partial x_i^2}(p)(1 - \langle F(p), F(q) \rangle) -2  \sum_i \frac{\partial \Phi}{\partial x_i} \left\langle \frac{\partial F}{\partial x_i}(p), F(q) \right\rangle\\
	+ 2  \Phi(p) \left\langle F(p), F(q) \right\rangle + |A(p)|^2 \Phi(p) (1 - \langle F(p), F(q) \rangle)
	\end{multline*}
	
	Reordenando, se tem
	\begin{multline*}
		\sum_i \frac{\partial^2 Z}{\partial x_i^2}(p,q) = (\Delta_{\Sigma} \Phi(p) + (|A(p)|^2 - 2)\Phi(p))(1 - \langle F(p), F(q) \rangle) + 2 \Phi(p)\\
		- 2 \sum_i \frac{\partial \Phi}{\partial x_i}(p) \left\langle \frac{\partial F}{\partial x_i}(p), F(q) \right\rangle
	\end{multline*}
	
	Somando e restando $\frac{|\nabla \Phi(p)|^2}{\Phi(p)} (1 - \langle F(p), F(q) \rangle) + \frac{\Phi(p)}{1 - \langle F(p),F(q) \rangle} \sum_i \left\langle \frac{\partial F}{\partial x_i} (p), F(q) \right\rangle^2$ temos:
	\begin{multline*}
		\sum_i \frac{\partial^2 Z}{\partial x_i^2}(p,q) = \left(\Delta_{\Sigma} \Phi(p) - \frac{|\nabla \Phi(p)|^2}{\Phi(p)} + (|A(p)|^2 - 2)\Phi(p)\right)(1 - \langle F(p), F(q) \rangle)\\
		+ 2 \Phi(p) + \frac{|\nabla \Phi(p)|^2}{\Phi(p)} (1 - \langle F(p), F(q) \rangle) - 2 \sum_i \frac{\partial \Phi}{\partial x_i}(p) \left\langle \frac{\partial F}{\partial x_i}(p), F(q) \right\rangle\\
		+ \frac{\Phi(p)}{1 - \langle F(p),F(q) \rangle} \sum_i \left\langle \frac{\partial F}{\partial x_i} (p), F(q) \right\rangle^2 - \frac{\Phi(p)}{1 - \langle F(p),F(q) \rangle} \sum_i \left\langle \frac{\partial F}{\partial x_i} (p), F(q) \right\rangle^2
	\end{multline*}
	
	Fatorizando adequadamente
	\begin{multline*}
		\sum_i \frac{\partial^2 Z}{\partial x_i^2}(p,q) = \left(\Delta_{\Sigma} \Phi(p) - \frac{|\nabla \Phi(p)|^2}{\Phi(p)} + (|A(p)|^2 - 2)\Phi(p)\right)(1 - \langle F(p), F(q) \rangle)+ 2 \Phi(p)	\\
		+ \frac{1}{\Phi(p)(1 - \langle F(p), F(q) \rangle)}	m(p,q) - \frac{\Phi(p)}{1 - \langle F(p),F(q) \rangle} \sum_i \left\langle \frac{\partial F}{\partial x_i} (p), F(q) \right\rangle^2
	\end{multline*}
	
	onde
	\begin{multline*}
		m(p,q) = |\nabla \Phi(p)|^2 (1 - \langle F(p), F(q) \rangle)^2 \\
		- 2 \sum_i \frac{\partial \Phi}{\partial x_i}(p) (1 - \langle F(p), F(q) \rangle) \Phi(p) \left\langle \frac{\partial F}{\partial x_i}(p), F(q) \right\rangle + \Phi(p)^2 \sum_i \left\langle \frac{\partial F}{\partial x_i} (p), F(q) \right\rangle^2
	\end{multline*}
	
	Podemos olhar que
	\begin{equation*}
		m(p,q) = \sum_i \left(  \frac{\partial \Phi}{\partial x_i}(p)(1 - \langle F(p),F(q) \rangle) - \Phi(p) \left\langle \frac{\partial F}{\partial x_i}(p), F(q) \right\rangle \right)^2
	\end{equation*}
	
	Pela equação \ref{diff_Z_x} e lembrando que $\frac{\partial Z}{\partial x_i}(p,q)=0$ temos
	\begin{equation*}
		m(p,q) = \sum_i \left( h_i^k(p) \left\langle \frac{\partial F}{x_k}(p), F(q) \right\rangle \right)^2
	\end{equation*}
	
	Expandindo o termo quadrático temos
	\begin{multline*}
		(h_i^1(p))^2 \left\langle \frac{\partial F}{x_1}(p), F(q) \right\rangle^2 + (h_i^2(p))^2 \left\langle \frac{\partial F}{x_2}(p), F(q) \right\rangle^2\\
		+ 2 h_i^1(p) h_i^2(p) \left\langle \frac{\partial F}{x_1}(p), F(q) \right\rangle \left\langle \frac{\partial F}{x_2}(p), F(q) \right\rangle
	\end{multline*}
	
	Lembremos que o determinante de $A$ é $-\lambda^2$. Então:
	\begin{equation*}
		h_1^1(p) h_2^2(p) - h_1^2(p) h_2^1(p) = h_1^1(p) h_2^2(p) - (h_1^2(p))^2 = -\lambda^2
	\end{equation*}
	
	porque $h_1^2(p) = h_2^1(p)$. Como $h_1^1(p) + h_2^2(p) = 0$:
	\begin{align*}
		\sum_i (h_i^1(p))^2 &= \lambda^2,\\
		\sum_i (h_i^2(p))^2 &= \lambda^2
	\end{align*}
	
	Em quanto ao termo $\sum_i h_i^1(p) h_i^2(p)$, se tem
	\begin{equation*}
		h_1^1(p) h_1^2(p) + h_2^1(p) h_2^2(p) = h_1^2(p) (h_1^1(p) + h_2^2(p)) = 0.
	\end{equation*}
	
	Portanto,
	\begin{equation*}
		m(p,q) = \sum_i \lambda^2 \left\langle \frac{\partial F}{\partial x_i}(p), F(q) \right\rangle^2
	\end{equation*}
	
	Lembrando que $\lambda^2 = \frac{1}{2} |A(p)|$, ao final se tem a equação desejada.
\end{demonstracao}

%\begin{lema}
%	O Laplaciano de $Z$ com respeito a $y$ satisfaz a desigualdade
%	\begin{equation*}
%		\sum_{i=1}^{2} \frac{\partial^2 Z}{\partial y_i^2} (p,q) \leq 2  \Phi(p) + 2 |Z(p,q)|.
%	\end{equation*}
%\end{lema}

\begin{proposicao}
	Temos
	\begin{equation}\label{diff_Z_x_y}
		\frac{\partial^2 Z}{\partial x_i \partial y_i}(p,q) = \lambda_i - \Phi(p)
	\end{equation}
\end{proposicao}

\begin{demonstracao}
	Derivando com respeito a $x_i$ a equação \ref{diff_Z_y}:
	\begin{equation*}
		\frac{\partial^2 Z}{\partial x_i \partial y_i}(p,q) = -\frac{\partial \Phi}{\partial x_i}(p) \left\langle F(p), \frac{\partial F}{\partial y_i}(q) \right\rangle - \Phi(p) \left\langle \frac{\partial F}{\partial x_i}(p), \frac{\partial F}{\partial y_i}(q) \right\rangle + \left\langle \frac{\partial \nu}{\partial x_i}(p), \frac{\partial F}{\partial y_i}(q) \right\rangle
	\end{equation*}
	
	Lembrando que $\frac{\partial \nu}{\partial x_i}(p) = \lambda_i(p) \frac{\partial F}{\partial x_i}(p)$, temos
	\begin{equation*}
		\frac{\partial^2 Z}{\partial x_i \partial y_i}(p,q) = -\frac{\partial \Phi}{\partial x_i}(p) \left\langle F(p), \frac{\partial F}{\partial y_i}(q) \right\rangle + (\lambda_i(p) - \Phi(p)) \left\langle \frac{\partial F}{\partial x_i}(p), \frac{\partial F}{\partial y_i}(q) \right\rangle
	\end{equation*}
	
	De \ref{diff_Z_x}, temos
	\begin{equation*}
		-\frac{\partial \Phi}{\partial x_i}(p) = \frac{1}{1 - \langle F(p),F(q) \rangle} (\lambda_i(p) - \Phi(p)) \left\langle \frac{\partial F}{\partial x_i}(p), F(q) \right\rangle
	\end{equation*}
	
	Substituindo na equação anterior:
	\begin{multline*}
		\frac{\partial^2 Z}{\partial x_i \partial y_i}(p,q) =  \frac{1}{1 - \langle F(p),F(q) \rangle} (\lambda_i(p) - \Phi(p)) \left\langle \frac{\partial F}{\partial x_i}(p), F(q) \right\rangle  \left\langle F(p), \frac{\partial F}{\partial y_i}(q) \right\rangle +\\
		(\lambda_i(p) - \Phi(p)) \left\langle \frac{\partial F}{\partial x_i}(p), \frac{\partial F}{\partial y_i}(q) \right\rangle
	\end{multline*}
	
	Nós sabemos que $|F(p) - F(q)|^2 = \langle F(p) - F(q), F(p) - F(q) \rangle = 2 - 2 \langle F(p), F(q) \rangle$, então
	\begin{multline*}
		\frac{\partial^2 Z}{\partial x_i \partial y_i}(p,q) = -2 (\lambda_i(p) - \Phi(p)) \left\langle \frac{\partial F}{\partial x_i}(p), \frac{-F(q)}{|F(p) - F(q)|} \right\rangle \left\langle \frac{F(p)}{|F(p) - F(q)|}, \frac{\partial F}{\partial y_i}(q) \right\rangle\\
		+ (\lambda_i - \Phi(p)) \left\langle \frac{\partial F}{\partial x_i}(p), \frac{\partial F}{\partial y_i}(q) \right\rangle
	\end{multline*}
	
	Lembrar que $\left\langle \frac{\partial F}{\partial x_i}(p), F(p) \right\rangle = \left\langle \frac{\partial F}{\partial y_i}(q), F(q) \right\rangle = 0$. Com isso a equação anterior se escreve
	\begin{multline*}
	\frac{\partial^2 Z}{\partial x_i \partial y_i}(p,q) = -2 (\lambda_i(p) - \Phi(p)) \left\langle \frac{\partial F}{\partial x_i}(p), \frac{F(p) -F(q)}{|F(p) - F(q)|} \right\rangle \left\langle \frac{F(p) - F(q)}{|F(p) - F(q)|}, \frac{\partial F}{\partial y_i}(q) \right\rangle\\
	+ (\lambda_i - \Phi(p)) \left\langle \frac{\partial F}{\partial x_i}(p), \frac{\partial F}{\partial y_i}(q) \right\rangle
	\end{multline*}
	
	Associando
	\begin{equation*}
	\frac{\partial^2 Z}{\partial x_i \partial y_i}(p,q) = (\lambda_i(p) - \Phi(p))  \left\langle w_i , \frac{\partial F}{\partial y_i}(q) \right\rangle
	\end{equation*}
	
	onde $w_i = \frac{\partial F}{\partial x_i}(p)  -2 \left\langle \frac{\partial F}{\partial x_i}(p), \frac{F(p) -F(q)}{|F(p) - F(q)|} \right\rangle \frac{F(p) - F(q)}{|F(p) - F(q)|}$.
\end{demonstracao}

%\begin{lema}
%	\begin{equation*}
%		\begin{split}
%		\sum_{i=1}^{2} \frac{\partial^2 Z_{\alpha}}{\partial x_i \partial y_i} (p,q) &\leq -2 \alpha \Phi(p) + \Lambda_4 (|F(p) - F(q)|)\\
%		& \left( |Z_{\alpha}(p,q)| + \sum_{i=1}^{2} \left| \frac{\partial Z_{\alpha}}{\partial x_i} (p,q) \right| + \sum_{i=1}^{2} \left| \frac{\partial Z_{\alpha}}{\partial y_i} (p,q) \right| \right),
%		\end{split}		
%	\end{equation*}
%	onde $\Lambda_4: (0,\infty) \rightarrow (0,\infty)$ é uma função continua.
%\end{lema}

%\begin{proposicao}
%	\begin{multline*}
%		\sum_{i=1}^{2} \frac{\partial^2 Z_{\alpha}}{\partial x_i^2} (p,q) + 2 \sum_{i=1}^{2} \frac{\partial^2 Z_{\alpha}}{\partial x_i \partial y_i} (p,q) + \sum_{i=1}^{2} \frac{\partial^2 Z_{\alpha}}{\partial y_i^2} (p,q) \leq \\
%		- \frac{\alpha^2 - 1}{\alpha} \frac{\Phi(p)}{1 - \langle F(p), F(q) \rangle} \sum_{i=1}^{2} \left\langle \frac{\partial F}{\partial x_i} (p), F(q) \right\rangle^2 \\
%		+ \Lambda_5(|F(p) - F(q)|) \left( |Z_{\alpha}(p,q)| + \sum_{i=1}^{2} \left| \frac{\partial Z_{\alpha}}{\partial x_i} (p,q) \right| + \sum_{i=1}^{2} \left| \frac{\partial Z_{\alpha}}{\partial y_i} (p,q) \right| \right),
%	\end{multline*}
%	onde $\Lambda_5: (0,\infty) \rightarrow (0,\infty)$ é uma função continua.
%\end{proposicao}

\begin{proposicao}
	\begin{equation}\label{2_diff_Z_y}
	\sum_i \frac{\partial^2 Z}{\partial y_i^2}(p,q) = 2 \Phi(p).
	\end{equation}
\end{proposicao}

\begin{demonstracao}
	Derivando \ref{diff_Z_y} em $p$ e $q$ e somando  temos
	\begin{equation*}
		\sum_i \frac{\partial^2 Z}{\partial y_i^2}(p,q) = - \Phi(p) \left\langle F(p), \sum_i \frac{\partial^2 F}{\partial y_i^2}(q) \right\rangle + \left\langle \nu(p), \sum_i \frac{\partial^2 F}{\partial y_i^2}(q) \right\rangle
	\end{equation*}
	
	Como $\sum_i \frac{\partial^2 F}{\partial y_i}(q) = -2 F(q)$, então
	\begin{equation*}
		\sum_i \frac{\partial^2 Z}{\partial y_i^2}(p,q) = 2 \Phi(p) \langle F(p), F(q) \rangle - 2 \langle \nu(p), F(q) \rangle
	\end{equation*}
	
	Como $0 = Z(p,q) = \Phi(p)(1 - \langle F(p), F(q) \rangle) + \langle \nu(p), F(q) \rangle$, então
	\begin{equation*}
		\sum_i \frac{\partial^2 Z}{\partial y_i^2}(p,q) = 2 \Phi(p).
	\end{equation*}
\end{demonstracao}


\begin{proposicao}
	\begin{multline}\label{suma_2das_derivadas}
		\sum_i \frac{\partial^2 Z}{\partial x_i^2}(\xbarra,\ybarra) + 2 \sum_i \frac{\partial^2 Z}{\partial x_i \partial y_i}(\xbarra,\ybarra) + \sum_i \frac{\partial^2 Z}{\partial y_i^2}(\xbarra,\ybarra) =\\
		\left( \Delta_{\Sigma} \Phi(\xbarra) - \frac{| \nabla \Phi(\xbarra) |^2}{\Phi(\xbarra)} + ( | A(\xbarra) |^2 - 2 ) \Phi(\xbarra) \right) (1 - \langle F(\xbarra,\ybarra) \rangle)\\
		- \frac{2 \Phi(\xbarra)^2- |A(\xbarra)|^2}{2 \Phi(\xbarra)(1 - \langle F(\xbarra),F(\ybarra) \rangle)} \sum_i \left\langle \frac{\partial F}{\partial x_i}(\xbarra), F(\ybarra) \right\rangle
	\end{multline}
\end{proposicao}

\begin{demonstracao}
	Somando adequadamente \eqref{2_diff_Z_x}, \eqref{2_diff_Z_y} e \eqref{diff_Z_x_y} tem-se a igualdade.
\end{demonstracao}

\begin{teorema}\label{Simon's_identity}
	Seja $\Sigma$ uma superfície mínima imersa em $S^3$. Então
	\begin{equation*}
		\Delta_{\Sigma} A = 2 A - |A|^2 A
	\end{equation*} 
\end{teorema}

\begin{demonstracao}
	Olhar \cite{Simons1968} Teorema 5.3.1.
\end{demonstracao}

\begin{teorema}\label{nao_existem_pontos_umbilicos}
	Uma superfície minima imersa em $S^3$ de gênero 1 não tem pontos umbílicos, então, a segunda forma fundamental não é zero em nenhum ponto da superfície.
\end{teorema}

\begin{demonstracao}
	Olhar \cite{Brendle2013} Proposição 3.3.
\end{demonstracao}

\begin{definicao}
	A função $\Psi: M \rightarrow \realnumbers$ está definida por
	\begin{equation*}
	\Psi(x) = \frac{1}{\sqrt{2}} |A(x)|.
	\end{equation*}
	onde $A$ é a segunda forma fundamental.
\end{definicao}

\begin{proposicao}\label{edp_principal}
	Supor que $F: \Sigma \rightarrow S^3$ é um toro mínimo mergulhado em $S^3$. Então a função $\Psi = \frac{|A|}{\sqrt{2}}$ é estritamente positiva e satisfaz a E.D.P
	\begin{equation*}
		\Delta_\Sigma \Psi - \frac{|\nabla \Psi|^2}{\Psi} + (|A|^2 - 2) \Psi = 0
	\end{equation*}
\end{proposicao}

\begin{demonstracao}
	A função $|A|$ é estritamente positiva pelo teorema \ref{nao_existem_pontos_umbilicos} . Usando o  teorema \ref{Simon's_identity} sabendo que $A = [h_{ik}]$, a equação pode se escrever
	\begin{equation*}
		\Delta_{\Sigma} h_{ik} + (|A|^2 - 2)h_{ik} = 0
	\end{equation*}
	
	Multiplicando por $2 h_{ik}$
	\begin{equation*}
		2 \Delta_{\Sigma} h_{ik} h_{ik} + 2 (|A|^2 - 2)h_{ik}^2 = 0
	\end{equation*}
	
	Somando e restando $ 2 \sum_j \left( \frac{\partial h_{ik}}{\partial x_j} \right)^2 $
	\begin{equation*}
		2 \sum_j \frac{\partial^2 h_{ik}}{\partial x_j^2} h_{ik} + 2 \sum_j \left( \frac{\partial h_{ik}}{\partial x_j} \right)^2 - 2 \sum_j \left( \frac{\partial h_{ik}}{\partial x_j} \right)^2 + 2 (|A|^2 - 2)h_{ik}^2 = 0
	\end{equation*}
	
	Como $ 2 \sum_j \frac{\partial^2 h_{ik}}{\partial x_j^2} h_{ik} + 2 \sum_j \left( \frac{\partial h_{ik}}{\partial x_j} \right)^2 = 2 \sum_j \frac{\partial }{\partial x_j} \left( \frac{\partial h_{ik}}{\partial x_j} h_{ik} \right) = \sum_j \frac{\partial^2 h_{ik}^2}{\partial x_j^2} $, então
	\begin{equation*}
		\sum_j \frac{\partial^2 h_{ik}^2}{\partial x_j^2} - 2 \sum_j \left( \frac{\partial h_{ik}}{\partial x_j} \right)^2 + 2 (|A|^2 - 2)h_{ik}^2 = 0
	\end{equation*}
	
	Somando com respeito aos índices $i$ e $k$
	\begin{equation*}
	\Delta_{\Sigma} (|A|^2) - 2 | \nabla A |^2 + 2 (|A|^2 - 2) |A|^2 = 0
	\end{equation*}
	
	Olhar que $ \Delta_{\Sigma} (|A|^2) = \sum_j \frac{\partial^2 |A|^2}{\partial x_j^2} = \sum_j \frac{\partial}{\partial x_j} \left( \frac{\partial |A|^2}{\partial x_j} \right) = 2 \sum_j \frac{\partial^2 |A|}{\partial x_j} + 2 \sum_j \left( \frac{\partial |A|}{\partial x_j} \right)^2 $. Portanto
	\begin{equation}\label{edp_sff}
		\Delta_{\Sigma} (|A|) + \frac{|\nabla |A||^2}{|A|} - \frac{|\nabla A|^2}{|A|} + (|A|^2 - 2) |A| = 0
	\end{equation}
	
	Temos que $ | \nabla A |^2 = \sum_j \left( \frac{\partial h_{ik}}{\partial x_j} \right)^2 $ onde se está somando sobre os índices $i$ e $k$. Pela equação \ref{codazzi_eq} temos
	\begin{align*}
		\frac{\partial h_{11}}{\partial x_1} + \frac{\partial h_{21}}{\partial x_2} &= 0\\
		\frac{\partial h_{12}}{\partial x_1} + \frac{\partial h_{22}}{\partial x_2} &= 0
	\end{align*}
	
	Elevando ao quadrado temos
	\begin{align*}
		\left( \frac{\partial h_{11}}{\partial x_1} \right)^2 + \left( \frac{\partial h_{21}}{\partial x_2} \right)^2  &= - 2 \frac{\partial h_{11}}{\partial x_1} \frac{\partial h_{21}}{\partial x_2}  \\
		\left( \frac{\partial h_{12}}{\partial x_1} \right)^2 + \left( \frac{\partial h_{22}}{\partial x_2} \right)^2 &= - 2 \frac{\partial h_{12}}{\partial x_1} \frac{\partial h_{22}}{\partial x_2}
	\end{align*}
	
	Usando a equação \ref{codazzi_eq} outra vez temos
	\begin{align*}
		\left( \frac{\partial h_{11}}{\partial x_1} \right)^2 + \left( \frac{\partial h_{21}}{\partial x_2} \right)^2  &=  2  \left( \frac{\partial h_{21}}{\partial x_2} \right)^2  \\
		\left( \frac{\partial h_{12}}{\partial x_1} \right)^2 + \left( \frac{\partial h_{22}}{\partial x_2} \right)^2 &=  2 \left( \frac{\partial h_{12}}{\partial x_1} \right)^2\\
		\left( \frac{\partial h_{11}}{\partial x_2} \right)^2 + \left( \frac{\partial h_{21}}{\partial x_1} \right)^2  &= 2 \left( \frac{\partial h_{21}}{\partial x_1} \right)^2\\
		\left( \frac{\partial h_{12}}{\partial x_2} \right)^2 + \left( \frac{\partial h_{22}}{\partial x_1} \right)^2 &=  2 \left( \frac{\partial h_{12}}{\partial x_2} \right)^2  
	\end{align*}
	
	Somando e lembrando que $h_{12} = h_{21}$ temos
	\begin{equation*}
		| \nabla A |^2 = 4 \left( \frac{\partial h_{12}}{\partial x_1} \right) + 4 \left( \frac{\partial h_{12}}{\partial x_2} \right)
	\end{equation*}
	
	Observar que $ |A| = \sqrt{2 h_{11}^2 + 2 h_{12}^2} $ e que $ | \nabla |A| |^2 = \left( \frac{\partial |A|}{\partial x_1} \right)^2 + \left( \frac{\partial |A|}{\partial x_2} \right)^2 $. Derivando $ |A| $ com respeito a $x_1$ temos
	\begin{equation*}
		\frac{\partial |A|}{\partial x_1} = \frac{2 h_{11} \frac{\partial h_{11}}{\partial x_1} + 2 h_{12} \frac{\partial h_{12}}{\partial x_1}}{|A|}
	\end{equation*}
	
	De forma similar, derivando $ |A| $ com respeito a $ x_2 $ temos
	\begin{equation*}
		\frac{\partial |A|}{\partial x_2} = \frac{2 h_{11} \frac{\partial h_{11}}{\partial x_2} + 2 h_{12} \frac{\partial h_{12}}{\partial x_2}}{|A|}
	\end{equation*}
	
	Obtendo o quadrado de ambas expressões se tem
	\begin{align*}
		\left( \frac{\partial |A|}{\partial x_1} \right)^2 &= \frac{4 h_{11}^2 \left( \frac{\partial h_{11}}{\partial x_1} \right)^2 + 4 h_{12}^2 \left( \frac{\partial h_{12}}{\partial x_1} \right)^2 + 8 h_{11} h_{12} \frac{\partial h_{11}}{\partial x_1} \frac{\partial h_{12}}{\partial x_1}}{|A|^2}\\
		\left( \frac{\partial |A|}{\partial x_2} \right)^2 &= \frac{4 h_{11}^2 \left( \frac{\partial h_{11}}{\partial x_2} \right)^2 + 4 h_{12}^2 \left( \frac{\partial h_{12}}{\partial x_2} \right)^2 + 8 h_{11} h_{12} \frac{\partial h_{11}}{\partial x_2} \frac{\partial h_{12}}{\partial x_2}}{|A|^2}
	\end{align*}
	
	Usando \ref{codazzi_eq} e somando os termos correspondentes se tem
	\begin{equation*}
		| \nabla |A| |^2 = 2 \left( \frac{\partial h_{12}}{\partial x_1} \right)^2 + 2 \left( \frac{\partial h_{12}}{\partial x_2} \right)
	\end{equation*}
	
	Portanto, $ | \nabla A |^2 = 2 | \nabla |A| |^2 $. Usando o resultado em \ref{edp_sff} se tem
	\begin{equation*}
		\Delta_\Sigma \Psi - \frac{|\nabla \Psi|^2}{\Psi} + (|A|^2 - 2) \Psi = 0
	\end{equation*}
\end{demonstracao}


\begin{teorema}[Lawson]\label{teo_lawson}
	Se $\Sigma$ é uma superfície mínima em $S^3$ de curvatura intrínseca constante $K$, então ou $K=1$ e $\Sigma$ é totalmente geodésica, ou $K=0$ e $\Sigma$ é um pedaço aberto de toro do Clifford.
\end{teorema}

\begin{demonstracao}
	\cite{Lawson1969}, Corolário 3.
\end{demonstracao}

\begin{teorema}[Brendle]
	Seja $F: M \rightarrow S^3$ uma superfície mínima mergulhada em $S^3$ de gênero 1. Então $F$ é congruente a o toro de Clifford.
\end{teorema}

%\begin{demonstracao}
Definamos
\begin{equation*}
	\aleph = \sup_{\substack{x,y \in \Sigma\\ x \neq y}} \frac{| \langle  \nu(x), F(y) \rangle |}{\Psi(x) (1 - \langle F(x), F(y) \rangle)}
\end{equation*}

Supor que $\aleph \leq 1$. Então
\begin{equation*}
	\Psi(x) (1 - \langle F(x), F(y) \rangle) + \langle \nu(x), F(y) \rangle \geq 0
\end{equation*}

para todo $ x,y \in \Sigma $. Identifiquemos ao ponto $F(x)$ com o ponto $x$, i.e., $F(x) = x$, para todo $x \in \Sigma$. Seja $p \in \Sigma$ arbitrário e $\{ e_1, e_2 \}$ a base de autovetores ortonormais de $A$ que gera $T_{p} \Sigma$ tais que
\begin{equation*}
	h(e_1,e_1) = \Psi(p), \quad h(e_1,e_2)=0 \quad \text{and} \quad h(e_2,e_2) = -\Psi(p)
\end{equation*}

Observar que $\Psi(p) = |\lambda|$, onde os autovalores de $A$ são $\lambda$ e $-\lambda$.

Seja $\gamma(t)$ uma geodésica em $\Sigma$ tal que $\gamma(0)=p$ e $\gamma'(0)=e_1$. Definamos a função $f: \mathbb{R} \rightarrow \mathbb{R}$ por
\begin{equation*}
	f(t) = \Psi(p) (1 - \langle p, \gamma(t) \rangle) + \langle \nu(p), \gamma(t) \rangle \geq 0
\end{equation*}

Calculando a derivada de primeiro ordem temos
\begin{equation*}
	f'(t) = -\langle \Psi(p) p - \nu(p), \gamma'(t) \rangle + \langle \nu(p), \gamma'(t) \rangle
\end{equation*}

Observar que $ \gamma $ é uma geodésica, então $ \gamma'(t) $ é  o transporte paralelo de $ \gamma'(0) $. Logo se tem que $ \langle \nu(p), \gamma'(t) \rangle = \langle \nu(p), \gamma'(0) \rangle = 0 $
\begin{equation*}
	f'(t) = -\langle \Psi(p) p - \nu(p), \gamma'(t) \rangle
\end{equation*}

Calculando a segunda derivada temos
\begin{equation*}
	f''(t) = -\langle \Psi(p) p - \nu(p), \gamma''(t) \rangle
\end{equation*}

Observar que $ \langle \nu(\gamma(t)), \gamma'(t) \rangle = 0 $. Derivando se tem
\begin{equation*}
	\langle D \nu(\gamma(t)) \gamma'(t), \gamma'(t) \rangle + \langle \nu(\gamma(t)), \gamma''(t) \rangle = 0
\end{equation*}

Como $\gamma''(t)$ não tem componentes em $T_{\gamma(t)} \Sigma$ porque é geodésica, então
\begin{equation*}
	-\gamma''(t) = \gamma(t) + h(\gamma'(t), \gamma'(t)) \nu(p)
\end{equation*}

Portanto
\begin{equation*}
	f''(t) = \langle \Psi(p) p - \nu(p), \gamma(t) + h(\gamma'(t), \gamma'(t)) \nu(p) \rangle
\end{equation*}

Derivando outra vez se tem
\begin{multline}\label{3ra_der_f}
	f'''(t) = \langle \Psi(p)p - \nu(p), \gamma'(t) \rangle + h(\gamma'(t), \gamma'(t)) \langle \Psi(p)p - \nu(p), D_{\gamma'(t)} \nu(\gamma(t)) \rangle\\
	+ \left( D_{\gamma'(t)}^{\Sigma} h \right) (\gamma'(t), \gamma'(t)) \langle \Psi(p)p - \nu(p), \nu(\gamma(t)) \rangle
\end{multline}

Observar que $f(0)=0$. Também $f'(0)=0$ porque $\Psi(p)p - \nu(p)$ é perpendicular a $T_p \Sigma$. E finalmente, $f''(0)=0$ porque $f''(0) = \Psi(p) - h(e_1,e_1)=0$.

Vamos demonstrar que $f'''(0)=0$. Usando a expansão de Taylor em $f$ se tem
\begin{equation*}
	f(t) = f(0) + f'(0)t + f''(0)t^2 + f'''(0)t^3 + r_3(t)
\end{equation*}  
onde $\lim_{t \rightarrow 0} \frac{r_3(t)}{t_3}=0$. Como $f(t) \geq 0$ e $f(0)=f'(0)=f''(0)=0$ se tem
\begin{equation*}
	0 \leq f'''(0) t^3 + r_3(t)
\end{equation*}

Dividendo por $t^3$ e fazendo o limite quando $t \rightarrow 0$ temos que $0 \leq f'''(0)$. Por outro lado, usando a mesma ideia usando $-t$ em lugar de  $t$ temos que $f'''(0) \leq 0$. Portanto, $f'''(0)=0$.

Voltando  a \ref{3ra_der_f}, avaliando em $t=0$, se tem que $(D_{e_1}^{\Sigma} h) (e_1,e_1) = 0$ porque $e_1, D_{e_1} \nu(p) \in T_p \Sigma$ e $\Psi(p)p - \nu(p)$ e $\nu(p)$ são paralelos.

Para continuar a demonstração deste primeiro caso, podemos considerar a base $\{ e_2, e_1, -\nu \}$ em lugar de $\{ e_1, e_2, \nu \}$ para mudar a orientação definida da superfície e, fazendo o mesmo procedimento, se tem que $(D_{e_2}^{\Sigma} h) (e_2,e_2)=0$.

Pela equação de Codazzi se tem
\begin{align*}
	(D_{e_2}^{\Sigma} h) (e_1,e_1) &= (D_{e_1}^{\Sigma} h) (e_2,e_1) = 0\\
	(D_{e_1}^{\Sigma} h) (e_2,e_2) &= (D_{e_2}^{\Sigma} h) (e_1,e_2) = 0
\end{align*}

Portanto $\nabla h = 0$ e com isso se tem que $h$ é constante. Seja $K$ a curvatura intrínseca de $\Sigma$. Como $h$ é constante, então $K$ também é constante. Logo, pelo teorema de Lawson (\ref{teo_lawson}) ou $K=0$, ou $K=1$. Mas $\Sigma$ não tem ponto umbílicos por \ref{nao_existem_pontos_umbilicos}, portanto $K=0$ e $\Sigma$ é um pedaço aberto do toro de Clifford. Mas $\Sigma$ é compacto, portanto $\Sigma$ é todo o toro de Clifford.

Supor agora que $\aleph > 1$. Escolhendo adequadamente entre $\nu$ e $-\nu$ temos
\begin{equation*}
	\aleph = \sup_{\substack{x,y \in \Sigma\\ x \neq y}} \frac{- \langle \nu(x), F(y) \rangle}{\Psi(x)(1 - \langle F(x), F(y) \rangle)}.
\end{equation*}

Seja $\Phi(x) = \aleph \Psi(x)$ em a definição de $Z(x,y)$, então
\begin{equation*}
	Z(x,y) = \aleph \Psi(x)(1 - \langle F(x), F(y) \rangle) + \langle \nu(x), F(y) \rangle.
\end{equation*}

Pela equação anterior observa-se que $Z(x,y) \geq 0$.

Definamos o conjunto
\begin{equation*}
	\Omega = \{ \xbarra \in \Sigma: \exists \ybarra \in \Sigma \setminus \{\xbarra\} \text{ tal que } Z(\xbarra,\ybarra) = 0 \}
\end{equation*}

O conjunto $\Omega$ é não vazio.

Usando a proposição \ref{edp_principal}, e a definição de $\Phi(\xbarra)$, a equação \eqref{suma_2das_derivadas} fica
\begin{multline}\label{suma_2da_der_aleph}
\sum_i \frac{\partial^2 Z}{\partial x_i^2}(\xbarra,\ybarra) + 2 \sum_i \frac{\partial^2 Z}{\partial x_i \partial y_i}(\xbarra,\ybarra) + \sum_i \frac{\partial^2 Z}{\partial y_i^2}(\xbarra,\ybarra) =\\
- \frac{\aleph^2 - 1}{\aleph} \frac{\Psi(\xbarra)}{1 - \langle F(\xbarra), F(\ybarra) \rangle} \sum_i \left\langle \frac{\partial F}{\partial x_i}(\xbarra), F(\ybarra) \right\rangle
\end{multline}
para qualquer par de pontos $\xbarra \neq \ybarra$. Como no ponto $(\xbarra,\ybarra)$ atinge-se um mínimo, então
\begin{equation*}
	0 = Z(\xbarra,\ybarra) = \frac{\partial Z}{\partial x_i}(\xbarra,\ybarra) = \frac{\partial Z}{\partial y_i}(\xbarra,\ybarra)
\end{equation*}

\begin{proposicao}
	Tem-se que $\nabla \Psi(\xbarra) = 0$ para todo $\xbarra \in \Omega.$
\end{proposicao}

\begin{demonstracao}
	Seja $\xbarra \in \Omega$, então existe $\ybarra \in \Omega \setminus \{x\}$ tal que $Z(\xbarra,\ybarra) = 0$. Tem-se que em $(\xbarra,\ybarra)$ atinge-se um mínimo local, portanto o Hessiano é definido não negativo. O Hessiano está definido por
	\begin{equation*}
		H(\xbarra,\ybarra) = \left[ \begin{matrix}
		\frac{\partial^2 Z}{\partial x_1^2}(\xbarra,\ybarra) & \frac{\partial^2 Z}{\partial x_2 \partial x_1}(\xbarra,\ybarra) & \frac{\partial^2 Z}{\partial y_1 \partial x_1}(\xbarra,\ybarra) & \frac{\partial^2 Z}{\partial y_2 \partial x_1}(\xbarra,\ybarra)\\ 
		\frac{\partial^2 Z}{\partial x_1 \partial x_2}(\xbarra,\ybarra) & \frac{\partial^2 Z}{\partial x_2^2}(\xbarra,\ybarra) & \frac{\partial^2 Z}{\partial y_1 \partial x_2}(\xbarra,\ybarra) & \frac{\partial^2 Z}{\partial y_2 \partial x_2}(\xbarra,\ybarra)\\
		 \frac{\partial^2 Z}{\partial x_1 \partial y_1}(\xbarra,\ybarra) & \frac{\partial^2 Z}{\partial x_2 \partial y_1}(\xbarra,\ybarra) & \frac{\partial^2 Z}{\partial y_1^2}(\xbarra,\ybarra) & \frac{\partial^2 Z}{\partial y_2 \partial y_1}(\xbarra,\ybarra)\\
		  \frac{\partial^2 Z}{\partial x_1 \partial y_2}(\xbarra,\ybarra) & \frac{\partial^2 Z}{\partial x_2 \partial y_2}(\xbarra,\ybarra) & \frac{\partial^2 Z}{\partial y_1 \partial y_2}(\xbarra,\ybarra) & \frac{\partial^2 Z}{\partial y_2^2}(\xbarra,\ybarra)
		\end{matrix} \right]
	\end{equation*}
	
	Como o Hessiano é definido não negativo. então para todo $v \in \realnumbers^4$ tem-se
	\begin{equation*}
		v^t H(\xbarra,\ybarra) v \geq 0
	\end{equation*}
	
	Em particular, para $v_1 = [1,0,1,0]^t$ e para $v_2 = [0,1,0,1]^t$ tem-se
	\begin{equation*}
		0 \leq v_1^t H(\xbarra,\ybarra) v_1 + v_2^t H(\xbarra,\ybarra) v_ 2 = \sum_i \frac{\partial^2 Z}{\partial x_i^2}(\xbarra,\ybarra) + 2 \sum_i \frac{\partial^2 Z}{\partial x_i \partial y_i}(\xbarra,\ybarra) + \sum_i \frac{\partial^2 Z}{\partial y_i^2}(\xbarra,\ybarra)
	\end{equation*}
	
	Por outro lado, como $\aleph > 1$, então
	\begin{equation*}
		- \frac{\aleph^2 - 1}{\aleph} \frac{\Psi(\xbarra)}{1 - \langle F(\xbarra), F(\ybarra) \rangle} \sum_i \left\langle \frac{\partial F}{\partial x_i}(\xbarra), F(\ybarra) \right\rangle \leq 0
	\end{equation*}
	
	Pela equação \eqref{suma_2da_der_aleph} tem-se
	\begin{equation*}
		\left\langle \frac{\partial F}{\partial x_i}(\xbarra), F(\ybarra) \right\rangle = 0
	\end{equation*}
	
	A equação \eqref{diff_Z_x} utilizando coordenadas geodésicas pode-se expressar como
	\begin{multline*}
		0 = \frac{\partial Z}{\partial x_i}(\xbarra,\ybarra) = \aleph \frac{\partial \Psi}{\partial x_i}(\xbarra)(1 - \langle F(\xbarra),F(\ybarra) \rangle)  - \aleph \Psi(\xbarra) \left\langle \frac{\partial F}{\partial x_i}(\xbarra), F(\ybarra) \right\rangle\\
		+ \sum_i \lambda_i \left\langle \frac{\partial F}{\partial x_i}(\xbarra),F(\ybarra) \right\rangle
	\end{multline*}
	
	Como $\left\langle \frac{\partial F}{\partial x_i}(\xbarra),F(\ybarra) \right\rangle = 0$, então a equação anterior expressa-se
	\begin{equation*}
		0 = \frac{\partial Z}{\partial x_i}(\xbarra,\ybarra) = \aleph \frac{\partial \Psi}{\partial x_i}(\xbarra)(1 - \langle F(\xbarra),F(\ybarra) \rangle)
	\end{equation*}
	
	Portanto, $\nabla \Psi(\xbarra) = 0$ para todo $\xbarra \in \Omega$.
\end{demonstracao}

\begin{lema}\label{bony}
	Seja $U$ um subconjunto aberto de uma variedade Riemanniana $M$, e seja $X_1, \ldots, X_m$ campos vetoriais diferenciáveis em $U$. Assumir que $\varphi: U \rightarrow \realnumbers$ é uma função não negativa diferenciável tal que
	\begin{equation*}
		\sum_{j=1}^{m} (D^2 \varphi)(X_j,X_j) \leq -L \inf_{|\xi| \leq 1} (D^2 \varphi)(\xi,\xi) + L |d \varphi| + L \varphi,
	\end{equation*}
	onde $L$ é uma constante positiva. Seja $\Omega = \{ x \in U: \varphi(x) = 0 \}$ o conjunto de zeros da função $\varphi$. Alem disso, supor que $\gamma: [0,1] \rightarrow U$ é um caminho diferenciável tal que $\gamma(0) \in \Omega$ e $\gamma'(s) = \sum_{j=1}^{m} f_j(s) X_j(\gamma(s))$ para funções diferenciáveis $f_1, \ldots, f_m: [0,1] \rightarrow \realnumbers$ adequadas. Então $\gamma(s) \in \Omega$ para todo $s \in [0,1]$.
\end{lema}

\begin{demonstracao}
	\cite{Brendle2010}, Corolário 9.7.
\end{demonstracao}

\begin{proposicao}
	$\Omega$ é aberto.
\end{proposicao}

\begin{demonstracao}
	Nas proposições \ref{2_diff_Z_x}, \ref{diff_Z_x_y} e \ref{2_diff_Z_y} estamos assumindo que $Z(\xbarra,\ybarra) = 0$. Vamos agora repetir o cálculos sem considerar essa hipóteses. Ao final tem-se
	\begin{multline*}
	\sum_i \frac{\partial^2 Z}{\partial x_i^2}(\xbarra,\ybarra) + 2 \sum_i \frac{\partial^2 Z}{\partial x_i \partial y_i}(\xbarra,\ybarra) + \sum_i \frac{\partial^2 Z}{\partial y_i^2}(\xbarra,\ybarra) =\\
	- \frac{\aleph^2 - 1}{\aleph} \frac{\Psi(\xbarra)}{1 - \langle F(\xbarra), F(\ybarra) \rangle} \sum_i \left\langle \frac{\partial F}{\partial x_i}(\xbarra), F(\ybarra) \right\rangle + \overline{\Lambda}(\xbarra,\ybarra) \left( Z(\xbarra,\ybarra) + \sum_i \left| \frac{\partial Z}{\partial x_i}(\xbarra,\ybarra) \right| + \sum_i \left| \frac{\partial Z}{\partial y_i}(\xbarra,\ybarra) \right| \right)
	\end{multline*}
	onde $\overline{\Lambda}(\xbarra,\ybarra)$ é uma função continua no conjunto $\{ (x,y) \in \Sigma \times \Sigma: x \neq y \}$, a qual pode ser não limitada perto da diagonal.
	

\end{demonstracao}