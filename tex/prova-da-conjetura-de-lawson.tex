\cite{Brendle2013}
\cite{Brendle2013a}

\begin{proposicao}
	Um superfície mínima imersa em $S^3$ de gênero 1 não tem pontos umbílicos, i.e., a segunda forma fundamental é não nula em cada ponto da superfície
\end{proposicao}

\begin{proposicao}
	Supor que $F: M \rightarrow S^3$ é um toro mínimo mergulhado em $S^3$. Então a norma da segunda forma fundamental satisfaz a equação em derivadas parciais
	\begin{equation*}
		\Delta_M (|A|) - \frac{| \nabla |A| |^2}{|A|} + (|A|^2 - 2) |A| = 0.
	\end{equation*}
	onde $A$ é a segunda forma fundamental.
\end{proposicao}

\begin{definicao}
	A função $\Psi: M \rightarrow \realnumbers$ está definida por
	\begin{equation*}
		\Psi(x) = \frac{1}{\sqrt{2}} |A(x)|.
	\end{equation*}
	onde $A$ é a segunda forma fundamental.
\end{definicao}

\begin{definicao}
	A função $Z: \Sigma \times \Sigma \rightarrow \realnumbers$ é definida por
	\begin{equation*}
		Z(x,y) =  \Phi(x) (1 - \langle F(x), F(y) \rangle) + \langle \nu(x), F(y) \rangle.
	\end{equation*}
	onde $F: \Sigma \rightarrow S^3$ é uma imersão mínima de gênero 1 em $S^3$, e $\nu(x) \in T_{F(x)} S^3$ é um campo vectorial normal unitário.
\end{definicao}

\begin{observacao}
	 A partir de agora vamos supor que para $p \in \Sigma$ existe $q \in \Sigma$ tal que $Z(p,q) = \frac{\partial Z}{\partial x_i}(p,q) = 0$ para $i = 1,2$. Ista suposição é feita para ser usada nos cálculos posteriores.
\end{observacao}

\begin{proposicao}\label{primeira_derivada_x_Z}
	As primeiras derivadas da função $Z$ com respeito a $x_1$ e $x_2$ estão dadas por:
	\begin{equation}\label{diff_Z_x}
	\frac{\partial Z}{\partial x_i} (p, q) =  \frac{\partial \Phi}{\partial x_i}(p) (1 - \langle F(p, q) \rangle) -  \Phi(p) \left\langle \frac{\partial F}{\partial x_i}(p), F(q) \right\rangle + h_i^k(p) \left\langle \frac{\partial F}{\partial x_k}(p), F(q) \right\rangle
	\end{equation}
\end{proposicao}


\begin{demonstracao}
	Derivando com respeito a $x_1$ e $x_2$:
	\begin{equation*}
		\frac{\partial Z}{\partial x_i} (p, q) =  \frac{\partial \Phi}{\partial x_i}(p) (1 - \langle F(p, q) \rangle) -  \Phi(p) \left\langle \frac{\partial F}{\partial x_i}(p), F(q) \right\rangle + \left\langle \frac{\partial \nu}{\partial x_i}(p), F(q) \right\rangle
	\end{equation*}
	
	Em quanto ao termo $ \left\langle \frac{\partial \nu}{\partial x_i}(p), F(q) \right\rangle $, olhar que $\langle \nu(x), \nu(x) \rangle=1, \forall x \in \Sigma$. Então, quando se derivar se tem:
	\begin{equation*}
		\left\langle \frac{\partial \nu}{\partial x_i}(x), \nu(x) \right\rangle = 0, \forall x \in \Sigma.
	\end{equation*}
	
	Isto quer dizer que $\frac{\partial \nu}{\partial x_i}(p) \in T_{p} \Sigma$. Olhar que estamos identificando $p$ com $F(p)$. Nós sabemos que $\{ \frac{\partial F}{\partial x_i}(p) \}$ gera $T_{p} \Sigma$, então, usando a notação de Einstein, podemos expressar $\frac{\partial \nu}{\partial x_i}(p)$ como combinação linear dos vetores anteriores.
	\begin{equation}\label{der_nu_x}
		\frac{\partial \nu}{\partial x_i} (p) = h_i^k \frac{\partial F}{\partial x_k}
	\end{equation}
	
	Lembrar que a notação de Einstein está expressando uma suma com respeito ao índice $k$ que toma os valores 1 e 2.
	
	Portanto ao final temos a expressão requerida.
\end{demonstracao}

\begin{proposicao}
	As primeiras derivadas da função $Z$ com respeito a $y_1$ e $y_2$ estão dadas por:
	\begin{equation*}
		\frac{\partial Z}{\partial y_i} (p, q) = - \Phi(p) \left\langle F(p), \frac{\partial F}{\partial y_i} (q) \right\rangle + \left\langle \nu(p), \frac{\partial F}{\partial y_i}(q) \right\rangle
	\end{equation*}
\end{proposicao}

\begin{demonstracao}
	Cálculo simples.
\end{demonstracao}

\begin{proposicao}
	O Laplaciano de $Z$ com respeito a $x$ satisfaz a desigualdade
	\begin{multline*}
		\sum_{i=1}^{2} \frac{\partial^2 Z}{\partial x_i^2} (p,q) = \left( \Delta_{\Sigma} \Phi(p) - \frac{| \nabla \Phi(p) |^2}{\Phi(p)} + (|A(p)|^2 - 2) \Phi(p) \right) (1 - \left\langle F(p), F(q) \right\rangle) + 2 \Phi(p)\\
		 - \frac{2 \Phi(p)^2 - |A(p)|^2}{2 \Phi(p) (1 - \langle F(p), F(q) \rangle)} \sum_i \left\langle \frac{\partial F}{\partial x_i}(p), F(q) \right\rangle^2
	\end{multline*}
\end{proposicao}

\begin{demonstracao}
	Vamos derivar com respeito a $x_i$ à proposição \ref{primeira_derivada_x_Z} e depois vamos somar com respeito ao índice $i$.
	\begin{multline}\label{Z_seg_dev_x}
		\sum_i \frac{\partial^2 Z}{\partial x_i^2}(x,y) = \sum_i \frac{\partial^2 \Phi}{\partial x_i^2}(x)(1 - \langle F(x), F(y) \rangle) -2  \sum_i \frac{\partial \Phi}{\partial x_i} \left\langle \frac{\partial F}{\partial x_i}(x), F(y) \right\rangle\\
		 -  \Phi(x) \left\langle \sum_i \frac{\partial^2 F}{\partial x_i^2}(x), F(y) \right\rangle + \sum_i \frac{\partial h_i^k}{\partial x_i}(x) \left\langle \frac{\partial F}{\partial x_k}, F(y) \right\rangle + \sum_i h_i^k(x) \left\langle \frac{\partial^2 F}{\partial x_i \partial x_k}, F(y) \right\rangle
	\end{multline}
	
	Lembrar que o índice $k$ representa uma somatória. Vamos analisar o termo $\sum_i \frac{\partial^2 F}{\partial x_i^2}(x)$. Pelo teorema \ref{propriedades_sup_min_S3} se tem
	\begin{equation*}
		\sum_i \frac{\partial^2 F}{\partial x_i^2}(x) + 2 F(x) = 0.
	\end{equation*}
	
	Então $\sum_i \frac{\partial^2 F}{\partial x_i^2}(x) = -2 F(x) $.
	
	Agora vamos analisar o termo $\sum_i \frac{\partial h_i^k}{\partial x_i}(x)$. Pela equação de Codazzi se tem
	\begin{align*}
		-\frac{\partial h_1^1}{\partial x_1} = \frac{\partial h_2^2}{\partial x_1} = \frac{\partial h_1^2}{\partial x_2} = \frac{\partial h_2^1}{\partial x_2}\\
		-\frac{\partial h_2^2}{\partial x_2} = \frac{\partial h_1^1}{\partial x_2} = \frac{\partial h_2^1}{\partial x_1} = \frac{\partial h_1^2}{\partial x_1}.
	\end{align*}
	
	Olhar que 
	\begin{equation*}
		A(x) = \left[\begin{matrix}
		h_1^1(x) & h_1^2(x)\\
		h_2^1(x) & h_2^2(x)
		\end{matrix}\right],
	\end{equation*}
	
	a superfície é minima e o fato as matrizes semelhantes tem a mesma traça, então temos $h_1^1(x) + h_2^2 = 0$. Portanto $\sum_i \frac{\partial h_i^k}{\partial x_i} = 0$.
	
	Analisando o termo $\sum_i h_i^k(x) \left\langle \frac{\partial^2 F}{\partial x_i \partial x_k}, F(y) \right\rangle$, derivamos \ref{der_nu_x} com respeito a $x_i$ e somamos com o índice $i$
	\begin{equation*}
		\sum_i \frac{\partial^2 \nu}{\partial x_i^2}(x) = \sum_i \frac{\partial h_i^k}{\partial x_i}(x) \frac{\partial F}{\partial x_k}(x) + \sum_i h_i^k \frac{\partial^2 F}{\partial x_i \partial x_k}(x)
	\end{equation*}
	
	Lembrando $\sum_i \frac{\partial h_i^k}{x_i}(x) = 0$, e expandindo a somatória  temos:
	\begin{equation*}
		\sum_i \frac{\partial^2 \nu}{\partial x_i^2}(x) =  h_1^1 \frac{\partial^2 F}{\partial x_1^2}(x) + h_1^2 \frac{\partial^2 F}{\partial x_1 \partial x_2}(x) + h_2^1 \frac{\partial^2 F}{\partial x_2 \partial x_1}(x) + h_2^2 \frac{\partial^2 F}{\partial x_2^2}(x)
	\end{equation*}
	
	Fazendo produto interno com $\frac{\partial F}{\partial x_1}(x)$:
	\begin{multline*}
		\left\langle \sum_i \frac{\partial^2 \nu}{\partial x_i^2}(x), \frac{\partial F}{\partial x_1}(x) \right\rangle =\\
		\left\langle h_1^1 \frac{\partial^2 F}{\partial x_1^2}(x) + h_1^2 \frac{\partial^2 F}{\partial x_1 \partial x_2}(x) + h_2^1 \frac{\partial^2 F}{\partial x_2 \partial x_1}(x) + h_2^2 \frac{\partial^2 F}{\partial x_2^2}(x), \frac{\partial F}{\partial x_1}(x) \right\rangle.
	\end{multline*}
	
	Temos que $\left\langle \frac{\partial F}{\partial x_1}, \frac{\partial F}{\partial x_1} = 1 \right\rangle$. Derivando com respeito a $x_1$ se tem $\left\langle \frac{\partial^2 F}{\partial x_1^2}, \frac{\partial F}{\partial x_1} \right\rangle = 0$. Igualmente derivando com respeito a $x_2$ e lembrando que as derivadas mistas comutam temos que $\left\langle \frac{\partial^2 F}{\partial x_1 \partial x_2}(x), \frac{\partial F}{\partial x_1} \right\rangle = \left\langle \frac{\partial^2 F}{\partial x_2 \partial x_1}(x), \frac{\partial F}{\partial x_1} \right\rangle = 0$. Analisando $\left\langle \frac{\partial^2 F}{\partial x_2^2}, \frac{\partial F}{\partial x_1} \right\rangle$, lembrando que $\left\langle \frac{\partial F}{\partial x_1}, \frac{\partial F}{\partial x_2} \right\rangle = 0$ derivamos com respeito a $x_2$:
	\begin{equation*}
		\left\langle \frac{\partial^2 F}{\partial x_2^2}, \frac{\partial F}{\partial x_1} \right\rangle + \left\langle \frac{\partial F}{\partial x_2}, \frac{\partial^2 F}{\partial x_2 \partial x_1} \right\rangle = 0
	\end{equation*}
	
	Sabemos que $\left\langle \frac{\partial F}{\partial x_2}, \frac{\partial^2 F}{\partial x_2 \partial x_1} \right\rangle = 0$. Portanto, $\left\langle \frac{\partial^2 F}{\partial x_2^2}, \frac{\partial F}{\partial x_1} \right\rangle = 0$.
	
	Logo
	\begin{equation*}
		\left\langle \sum_i \frac{\partial^2 \nu}{\partial x_i^2}(x), \frac{\partial F}{\partial x_1}(x) \right\rangle = 0.
	\end{equation*}
	
	De maneira similar:
	\begin{equation*}
		\left\langle \sum_i \frac{\partial^2 \nu}{\partial x_i^2}(x), \frac{\partial F}{\partial x_2}(x) \right\rangle = 0.
	\end{equation*}
	
	Isto quer dizer que $\sum_i \frac{\partial^2 \nu}{\partial x_i^2}(p)$ não tem componentes no plano tangente $T_p \Sigma$ dentro de $T_p S^3$.
	
	Por outro lado, lembrado que $\left\langle \frac{\partial \nu}{\partial x_i}(x), \nu(x) \right\rangle = 0$, derivamos com respeito a $x_1$:
	\begin{equation*}
		\left\langle \frac{\partial^2 \nu}{\partial x_i^2}(x), \nu(x) \right\rangle + \left\langle \frac{\partial \nu}{\partial x_i}(x), \frac{\partial \nu}{\partial x_i}(x) \right\rangle = 0
	\end{equation*}
	
	Como $\frac{\partial \nu}{\partial x_i}(x) = h_i^k(x) \frac{\partial F}{\partial x_k}$, então
	\begin{equation*}
		\left\langle \frac{\partial^2 \nu}{\partial x_i^2}(x), \nu(x) \right\rangle + \left\langle h_i^k(x) \frac{\partial F}{\partial x_k}, h_i^j(x) \frac{\partial F}{\partial x_j}  \right\rangle = 0.
	\end{equation*}
	
	Como $\left\langle \frac{\partial F}{\partial x_k}, \frac{\partial F}{\partial x_j} \right\rangle = \delta_{kj}$, então
	\begin{equation*}
		\left\langle \frac{\partial^2 \nu}{\partial x_i^2}(x), \nu(x) \right\rangle + (h_i^k)^2 = 0.
	\end{equation*}
	
	Logo, somando com respeito a $i$
	\begin{equation*}
		\left\langle \sum_i \frac{\partial^2 \nu}{\partial x_i^2}(x), \nu(x) \right\rangle + \sum_i (h_i^k)^2 = 0.
	\end{equation*}
	
	Olhar que $\sum_i (h_i^k(x))^2$ é $| A(x) |^2$ pela norma de Frobenius. Portanto, lembrando que $\sum_i \frac{\partial^2 \nu}{\partial x_i^2}(x)$ só tem componentes na direção $\nu$, temos
	\begin{equation*}
		\sum_i \frac{\partial^2 \nu}{\partial x_i^2}(p) = - | A(p) |^2 \nu(p)
	\end{equation*}
	
	Portanto, temos:
	\begin{equation*}
		\sum_i h_i^k \frac{\partial^2 F}{\partial x_i \partial x_k}(p) = \sum_i \frac{\partial^2 \nu}{\partial x_i^2}(p) = - | A(p) |^2 \nu(p)
	\end{equation*}
	
	Ao final a equação \ref{Z_seg_dev_x} se escreve:
	\begin{multline}\label{lap_Z_x}
		\sum_i \frac{\partial^2 Z}{\partial x_i^2}(p,q) =  \sum_i \frac{\partial^2 \Phi}{\partial x_i^2}(p)(1 - \langle F(p), F(q) \rangle) -2  \sum_i \frac{\partial \Phi}{\partial x_i} \left\langle \frac{\partial F}{\partial x_i}(p), F(q) \right\rangle\\
		+ 2  \Phi(p) \left\langle F(p), F(q) \right\rangle - | A(p) |^2 \left\langle \nu(p), F(q) \right\rangle
	\end{multline}
	
	Multiplicando por $|A(p)|^2$ a $Z(p,q)$
	\begin{equation*}
		0 = |A(p)|^2 Z(p,q) = |A(p)|^2 \Phi(p) (1 - \langle F(p), F(q) \rangle) + |A(p)|^2 \langle \nu(p), F(q) \rangle
	\end{equation*}
	
	temos $-|A(p)|^2 \langle \nu(p), F(q) \rangle = |A(p)|^2 \Phi(p) (1 - \langle F(p), F(q) \rangle)$. Usando isto em \ref{lap_Z_x}:
	\begin{multline*}
	\sum_i \frac{\partial^2 Z}{\partial x_i^2}(p,q) =  \sum_i \frac{\partial^2 \Phi}{\partial x_i^2}(p)(1 - \langle F(p), F(q) \rangle) -2  \sum_i \frac{\partial \Phi}{\partial x_i} \left\langle \frac{\partial F}{\partial x_i}(p), F(q) \right\rangle\\
	+ 2  \Phi(p) \left\langle F(p), F(q) \right\rangle + |A(p)|^2 \Phi(p) (1 - \langle F(p), F(q) \rangle)
	\end{multline*}
	
	Reordenando, se tem
	\begin{multline*}
		\sum_i \frac{\partial^2 Z}{\partial x_i^2}(p,q) = (\Delta_{\Sigma} \Phi(p) + (|A(p)|^2 - 2)\Phi(p))(1 - \langle F(p), F(q) \rangle) + 2 \Phi(p)\\
		- 2 \sum_i \frac{\partial \Phi}{\partial x_i}(p) \left\langle \frac{\partial F}{\partial x_i}(p), F(q) \right\rangle
	\end{multline*}
	
	Somando e restando $\frac{|\nabla \Phi(p)|^2}{\Phi(p)} (1 - \langle F(p), F(q) \rangle) + \frac{\Phi(p)}{1 - \langle F(p),F(q) \rangle} \sum_i \left\langle \frac{\partial F}{\partial x_i} (p), F(q) \right\rangle^2$ temos:
	\begin{multline*}
		\sum_i \frac{\partial^2 Z}{\partial x_i^2}(p,q) = \left(\Delta_{\Sigma} \Phi(p) - \frac{|\nabla \Phi(p)|^2}{\Phi(p)} + (|A(p)|^2 - 2)\Phi(p)\right)(1 - \langle F(p), F(q) \rangle)\\
		+ 2 \Phi(p) + \frac{|\nabla \Phi(p)|^2}{\Phi(p)} (1 - \langle F(p), F(q) \rangle) - 2 \sum_i \frac{\partial \Phi}{\partial x_i}(p) \left\langle \frac{\partial F}{\partial x_i}(p), F(q) \right\rangle\\
		+ \frac{\Phi(p)}{1 - \langle F(p),F(q) \rangle} \sum_i \left\langle \frac{\partial F}{\partial x_i} (p), F(q) \right\rangle^2 - \frac{\Phi(p)}{1 - \langle F(p),F(q) \rangle} \sum_i \left\langle \frac{\partial F}{\partial x_i} (p), F(q) \right\rangle^2
	\end{multline*}
	
	Fatorizando adequadamente
	\begin{multline*}
		\sum_i \frac{\partial^2 Z}{\partial x_i^2}(p,q) = \left(\Delta_{\Sigma} \Phi(p) - \frac{|\nabla \Phi(p)|^2}{\Phi(p)} + (|A(p)|^2 - 2)\Phi(p)\right)(1 - \langle F(p), F(q) \rangle)+ 2 \Phi(p)	\\
		+ \frac{1}{\Phi(p)(1 - \langle F(p), F(q) \rangle)}	m(p,q) - \frac{\Phi(p)}{1 - \langle F(p),F(q) \rangle} \sum_i \left\langle \frac{\partial F}{\partial x_i} (p), F(q) \right\rangle^2
	\end{multline*}
	
	onde
	\begin{multline*}
		m(p,q) = |\nabla \Phi(p)|^2 (1 - \langle F(p), F(q) \rangle)^2 \\
		- 2 \sum_i \frac{\partial \Phi}{\partial x_i}(p) (1 - \langle F(p), F(q) \rangle) \Phi(p) \left\langle \frac{\partial F}{\partial x_i}(p), F(q) \right\rangle + \Phi(p)^2 \sum_i \left\langle \frac{\partial F}{\partial x_i} (p), F(q) \right\rangle^2
	\end{multline*}
	
	Podemos olhar que
	\begin{equation*}
		m(p,q) = \sum_i \left(  \frac{\partial \Phi}{\partial x_i}(p)(1 - \langle F(p),F(q) \rangle) - \Phi(p) \left\langle \frac{\partial F}{\partial x_i}(p), F(q) \right\rangle \right)^2
	\end{equation*}
	
	Pela equação \ref{diff_Z_x} e lembrando que $\frac{\partial Z}{\partial x_i}(p,q)=0$ temos
	\begin{equation*}
		m(p,q) = \sum_i \left( h_i^k(p) \left\langle \frac{\partial F}{x_k}(p), F(q) \right\rangle \right)^2
	\end{equation*}
	
	Expandindo o termo quadrático
	\begin{equation*}
		l
	\end{equation*}
\end{demonstracao}

\begin{lema}
	O Laplaciano de $Z$ com respeito a $y$ satisfaz a desigualdade
	\begin{equation*}
		\sum_{i=1}^{2} \frac{\partial^2 Z}{\partial y_i^2} (p,q) \leq 2  \Phi(p) + 2 |Z(p,q)|.
	\end{equation*}
\end{lema}

\begin{lema}
	\begin{equation*}
		\begin{split}
		\sum_{i=1}^{2} \frac{\partial^2 Z_{\alpha}}{\partial x_i \partial y_i} (p,q) &\leq -2 \alpha \Phi(p) + \Lambda_4 (|F(p) - F(q)|)\\
		& \left( |Z_{\alpha}(p,q)| + \sum_{i=1}^{2} \left| \frac{\partial Z_{\alpha}}{\partial x_i} (p,q) \right| + \sum_{i=1}^{2} \left| \frac{\partial Z_{\alpha}}{\partial y_i} (p,q) \right| \right),
		\end{split}		
	\end{equation*}
	onde $\Lambda_4: (0,\infty) \rightarrow (0,\infty)$ é uma função continua.
\end{lema}

\begin{proposicao}
	\begin{multline*}
		\sum_{i=1}^{2} \frac{\partial^2 Z_{\alpha}}{\partial x_i^2} (p,q) + 2 \sum_{i=1}^{2} \frac{\partial^2 Z_{\alpha}}{\partial x_i \partial y_i} (p,q) + \sum_{i=1}^{2} \frac{\partial^2 Z_{\alpha}}{\partial y_i^2} (p,q) \leq \\
		- \frac{\alpha^2 - 1}{\alpha} \frac{\Phi(p)}{1 - \langle F(p), F(q) \rangle} \sum_{i=1}^{2} \left\langle \frac{\partial F}{\partial x_i} (p), F(q) \right\rangle^2 \\
		+ \Lambda_5(|F(p) - F(q)|) \left( |Z_{\alpha}(p,q)| + \sum_{i=1}^{2} \left| \frac{\partial Z_{\alpha}}{\partial x_i} (p,q) \right| + \sum_{i=1}^{2} \left| \frac{\partial Z_{\alpha}}{\partial y_i} (p,q) \right| \right),
	\end{multline*}
	onde $\Lambda_5: (0,\infty) \rightarrow (0,\infty)$ é uma função continua.
\end{proposicao}

\begin{teorema}[Brendle]
	Seja $F: M \rightarrow S^3$ uma superfície mínima mergulhada em $S^3$ de gênero 1. Então $F$ é congruente a o toro de Clifford.
\end{teorema}