\cite{Brendle2013}
\cite{Brendle2013a}

\begin{proposicao}
	Um superfície mínima imersa em $S^3$ de gênero 1 não tem pontos umbílicos, i.e., a segunda forma fundamental é não nula em cada ponto da superfície
\end{proposicao}

\begin{proposicao}
	Supor que $F: M \rightarrow S^3$ é um toro mínimo mergulhado em $S^3$. Então a norma da segunda forma fundamental satisfaz a equação em derivadas parciais
	\begin{equation*}
		\Delta_M (|A|) - \frac{| \nabla |A| |^2}{|A|} + (|A|^2 - 2) |A| = 0.
	\end{equation*}
	onde $A$ é a segunda forma fundamental.
\end{proposicao}

\begin{definicao}
	A função $\Psi: M \rightarrow \realnumbers$ está definida por
	\begin{equation*}
		\Psi(x) = \frac{1}{\sqrt{2}} |A(x)|.
	\end{equation*}
	onde $A$ é a segunda forma fundamental.
\end{definicao}

\begin{definicao}
	Dado um numero $\alpha \geq 1$, a função $Z_{\alpha}: M \times M \rightarrow \realnumbers$ é definida por
	\begin{equation*}
		Z_{\alpha}(x,y) = \alpha \Psi(x) (1 - \langle F(x), F(y) \rangle) + \langle \nu(x), F(y) \rangle.
	\end{equation*}
	onde $F: M \rightarrow S^3$ é uma imersão mínima de gênero 1 em $S^3$, e $\nu(x) \in T_{F(x)} S^3$ é um campo vectorial normal unitário
\end{definicao}

\begin{lema}
	O Laplaciano de $Z_{\alpha}$ com respeito a $x$ satisfaz a desigualdade
	\begin{equation*}
		\begin{split}
		\sum_{i=1}^{2} \frac{\partial^2 Z_{\alpha}}{\partial x_i^2} (p,q) & \leq 2 \alpha \Psi(p) - \frac{\alpha^2 - 1}{\alpha} \frac{\Psi(p)}{1 - \langle F(p), F(q) \rangle} \sum_{i=1}^{2} \left\langle \frac{\partial F}{\partial x_i} (p), F(q) \right\rangle^2\\
		& + \Lambda_1 (| F(p) - F(q) |) \left( \sum_{i=1}^{2} \left| \frac{\partial Z_{\alpha}}{\partial x_i} (p,q) \right| \right),
		\end{split}		
	\end{equation*}
	onde $\Lambda_1: (0,\infty) \rightarrow (0,\infty)$ é uma função continua
\end{lema}

\begin{lema}
	O Laplaciano de $Z_{\alpha}$ com respeito a $y$ satisfaz a desigualdade
	\begin{equation*}
		\sum_{i=1}^{2} \frac{\partial^2 Z_{\alpha}}{\partial y_i^2} (p,q) \leq 2 \alpha \Psi(p) + 2 |Z_{\alpha}(p,q)|.
	\end{equation*}
\end{lema}

\begin{lema}
	\begin{equation*}
		\begin{split}
		\sum_{i=1}^{2} \frac{\partial^2 Z_{\alpha}}{\partial x_i \partial y_i} (p,q) &\leq -2 \alpha \Psi(p) + \Lambda_4 (|F(p) - F(q)|)\\
		& \left( |Z_{\alpha}(p,q)| + \sum_{i=1}^{2} \left| \frac{\partial Z_{\alpha}}{\partial x_i} (p,q) \right| + \sum_{i=1}^{2} \left| \frac{\partial Z_{\alpha}}{\partial y_i} (p,q) \right| \right),
		\end{split}		
	\end{equation*}
	onde $\Lambda_4: (0,\infty) \rightarrow (0,\infty)$ é uma função continua.
\end{lema}

\begin{proposicao}
	\begin{multline*}
		\sum_{i=1}^{2} \frac{\partial^2 Z_{\alpha}}{\partial x_i^2} (p,q) + 2 \sum_{i=1}^{2} \frac{\partial^2 Z_{\alpha}}{\partial x_i \partial y_i} (p,q) + \sum_{i=1}^{2} \frac{\partial^2 Z_{\alpha}}{\partial y_i^2} (p,q) \leq \\
		- \frac{\alpha^2 - 1}{\alpha} \frac{\Psi(p)}{1 - \langle F(p), F(q) \rangle} \sum_{i=1}^{2} \left\langle \frac{\partial F}{\partial x_i} (p), F(q) \right\rangle^2 \\
		+ \Lambda_5(|F(p) - F(q)|) \left( |Z_{\alpha}(p,q)| + \sum_{i=1}^{2} \left| \frac{\partial Z_{\alpha}}{\partial x_i} (p,q) \right| + \sum_{i=1}^{2} \left| \frac{\partial Z_{\alpha}}{\partial y_i} (p,q) \right| \right),
	\end{multline*}
	onde $\Lambda_5: (0,\infty) \rightarrow (0,\infty)$ é uma função continua.
\end{proposicao}

\begin{teorema}[Brendle]
	Seja $F: M \rightarrow S^3$ uma superfície mínima mergulhada em $S^3$ de gênero 1. Então $F$ é congruente a o toro de Clifford.
\end{teorema}