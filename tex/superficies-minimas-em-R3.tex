\section{Superfícies Mínimas}

\begin{definicao}
	Uma superfície regular $M$ em $\realnumbers^3$ é chamada \emph{superfície mínimas} se $H(p)=0$ para qualquer $p \in M$.
\end{definicao}

\begin{observacao}
	Se $H \equiv 0$, então $K_1 + K_2 \equiv 0$. Logo $K_1 = -K_2$
\end{observacao}

\begin{exemplo}
	Um plano em $\realnumbers^3$ é trivialmente mínima, pois $K_1=K_2=0$.
\end{exemplo}

A motivação histórica do estudo das superfícies mínimas foi dada por Lagrange o ano 1760 como seguinte problema:

Dado uma curva fechada $\gamma$ em $\realnumbers^3$, sem autointerseções, determinar a superfície de área mínima, e que tem $\gamma$ como fronteira.

Seja $M$ uma superfície regular orientada em $\realnumbers^3$, e considere uma função $f \in \smoothfunction{M}$.

\begin{definicao}
	Uma \emph{variação normal} de $M$, relativa à função $f$, é uma família de superfícies $M_t$, com $t \in (-\epsilon,\epsilon)$, dadas por:
	\begin{equation*}
		p_t = p + t f(p) N(p),
	\end{equation*}
	
	onde $N$ é o campo unitário normal a $M$, na orientação de $M$.
	
\end{definicao}

Para $\epsilon > 0$ suficientemente pequeno, cada conjunto $M_t$ também e uma superfície regular chamada uma \emph{superfície de variação}.

Note que para $t=0$, $M_0=M$. Se $f \equiv 1$, $M_t$ é uma superfície \emph{paralela} a $M$ a uma distancia $t$.

**gráfico**

Dados uma variação normal $M_t$ de $M$ relativa a uma função suave $f: M \rightarrow \realnumbers$, com $t \in (-\epsilon,\epsilon)$, e $D \subset M$ um domínio limitado, considere
\begin{equation*}
	D_t = \{ p_t \in M_t: p \in D \}
\end{equation*}

para cada $t \in (-\epsilon,\epsilon)$, $D_t$ é um domínio correspondente em $M_t$. Definimos em cada $t$
\begin{equation*}
	A(t) = \text{Area}(D_t)
\end{equation*}

\begin{teorema}
	\begin{equation*}
		A'(0) = -2 \int_D Hf dA
	\end{equation*}
\end{teorema}

A expressão acima chama-se a \emph{formula da primeira variação da área}.

\begin{proof}
	contenidos...
\end{proof}

\begin{proposicao}\label{caracteristica_das_superficies_minimas}
	Uma superfície $M$ em $\realnumbers^3$ é mínima se e somente se $A'(0) = 0$.
\end{proposicao}

\begin{demonstracao}
	contenidos...
\end{demonstracao}

\begin{observacao}
	Suponha que exista uma solução $M$ para o problema de Lagrange, e considere uma variação normal $M_t$ de $M$, com $t \in (-\epsilon,\epsilon)$, dada por uma função suave $f: M \rightarrow \realnumbers$ tal que $f_{|\partial M} = 0$. Como a área de $M$ é mínima temos, em particular, que
	\begin{equation*}
		A(t) \geq A(0)
	\end{equation*}
	
	para qualquer $t \in (-\epsilon,\epsilon)$. Portanto, $A'(0)=0$, para toda variação normal $M_t$ de $M$ com $f_{|\partial M}=0$.
	
	Isso mostra, em virtude da proposição \ref{caracteristica_das_superficies_minimas}, que as superfícies de área minima são superfícies minimas no sentido da nossa definição. A reciproca é falsa!
\end{observacao}

\begin{proposicao}
	Não existe superfície minima compacta em $\realnumbers^3$.
\end{proposicao}

\begin{demonstracao}
	Se $M$ é minima, então
	\begin{equation*}
		H = \frac{1}{2} (k_1 + k_2) = 0.
	\end{equation*}
	
	Logo $k_1 = -k_2$ e se tem que $K = k_1 k_2 \leq 0$.
	
	Se $M$ for compacta, existe $p \in M$ tal que $K(p) > 0$.
\end{demonstracao}

Dada suma superfície regular $M$ em $\realnumbers^3$, considere uma carta local isoterma $(U,\varphi)$ para $M$, i.e., 
\begin{equation*}
	E = G = \lambda^2 \text{ e } F=0,
\end{equation*}

onde $\lambda: U \rightarrow \realnumbers$ é uma função diferenciável com $\lambda > 0$. Note que, nas coordenadas isotermas $\varphi \sim (x,y)$, a curvatura media se expressa como
\begin{align*}
	H &= \frac{eG - 2fF + gE}{2(EG - F^2)}\\
	&= \frac{e + g}{2 \lambda^2}
\end{align*}

\begin{definicao}
	Dado uma função diferenciável $f: U \subset \realnumbers^2 \rightarrow \realnumbers$, o \emph{Laplaciano} de $f$, denotado por $\Delta f$, é definido por
	\begin{equation*}
		\Delta f = \frac{\partial^2 f}{\partial x^2} + \frac{\partial^2 f}{\partial y^2}.
	\end{equation*}
	
	Dizemos que $f$ é \emph{harmônica} se $\Delta f = 0$.
	
	Se $(U, \varphi)$ é uma carta local para $M$ como $\varphi = (\varphi_1, \varphi_2, \varphi_3)$, definimos
	\begin{equation*}
		\Delta \varphi = (\Delta \varphi_1, \Delta \varphi_2, \Delta \varphi_3).
	\end{equation*}
\end{definicao}

\begin{proposicao}
	Se $(U, \varphi)$ é uma carta local isoterma em $M$, então 
	\begin{equation*}
		\Delta \varphi = 2 \lambda^2 H N
	\end{equation*}
\end{proposicao}

\begin{demonstracao}
	Como $\varphi$ é isoterma, com $\varphi \sim (u, v)$, temos
	\begin{gather*}
		\innerproduct{\varphi_u}{\varphi_v} = \lambda^2 = \innerproduct{\varphi_v}{\varphi_u} \\
		\text{ e } \innerproduct{\varphi_u}{\varphi_v} = 0
	\end{gather*}
	
	Derivando, obtemos:
	\begin{align*}
		\innerproduct{\varphi_{uu}}{\varphi_u} &= \innerproduct{\varphi_{vu}}{\varphi_v}\\
		&= - \innerproduct{\varphi_u}{\varphi_{vv}}
	\end{align*}
	
	Disso decorre que
	\begin{equation}\label{eq1}
		\innerproduct{\varphi_{uu} + \varphi_{vv}}{\varphi_u} = 0
	\end{equation}
	
	Analogamente, obtemos:
	\begin{equation}\label{eq2}
		\innerproduct{\varphi_{uu} + \varphi_{vv}}{\varphi_v} = 0
	\end{equation}
	
	De \ref{eq1} e \ref{eq2} concluímos que $\varphi_{uu} + \varphi_{vv}$ é paralela a $N$. Alem disso, como
	\begin{equation}
		H = \frac{e+g}{2 \lambda^2}
	\end{equation}
	
	obtemos:
	\begin{align*}
		2 \lambda^2 H &= e + g = \innerproduct{\varphi_{uu}}{N} + \innerproduct{\varphi_{vv}}{N}\\
		&= \innerproduct{\varphi_{uu} + \varphi_{vv}}{N}.
	\end{align*}
	
	Isso mostra que
	\begin{equation}
		\Delta \varphi = 2 \lambda^2 H N
	\end{equation}
\end{demonstracao}

\begin{corolario}
	Uma superfície $M$ em $\realnumbers^3$ é minima se e somente se toda carta local isoterma é harmônica.
\end{corolario}

\begin{exemplo}
	O \emph{catenoide} é a superfície em $\realnumbers^3$ gerada pela rotação da catenária 
	\begin{equation*}
		y = a \cosh \left( \frac{z}{a} \right)
	\end{equation*}
	
	em torno ao eixo $z$.
	
	(Gráfico)
	
	Assim, o catenoide pode ser parametrizado por
	\begin{equation*}
		\varphi(u,v) = \left( a \cosh v \cos u, a \cosh v \sin u, av \right)
	\end{equation*}
	
	onde $u \in (0, 2 \pi)$ e $v \in \realnumbers$. Para tal $\varphi$, obtemos
	\begin{gather*}
		E = G = a^2 \cosh^2 v,\\
		F = 0,\\
		\varphi_{uu} + \varphi_{vv} = 0.
	\end{gather*}
	
	Portanto o catenoide e uma superfície minima.
\end{exemplo}

\begin{exemplo}
	Considere uma hélice dada por
	\begin{equation*}
		\alpha(u) = \left( \cos u, \sin u, au \right).
	\end{equation*}
	
	Por cada ponto da hélice, trace uma reta paralela ao plano $XY$ que intercepta o eixo $Z$.
	
	(Gráfico)
	
	A superfície gerada por tais retas é o \emph{helicoide} e pode ser parametrizada por
	\begin{equation*}
		\varphi(u,v) = \left( v \cos u, v \sin u, au \right)
	\end{equation*}
	
	com $u \in (0, 2 \pi)$ e $v \in \realnumbers$. Temos
	\begin{gather*}
		E = G = a^2 \cosh^2 v\\
		F = 0\\
		\varphi_{uu} + \varphi_{vv} = 0.
	\end{gather*}
	
	Portanto o helicoide é superfície minima.
\end{exemplo}

\begin{teorema}
	Alem do plano,
	\begin{enumerate}
		\item O catenoide é a única superfície rotacional minima.
		\item O helicoide é a única superfície regrada minima.
	\end{enumerate}
\end{teorema}

\begin{exemplo}
	Dado uma função diferenciável $f: U \rightarrow \realnumbers$, definida num aberto $U \subset \realnumbers^2$, considere o gráfico $\text{Gr}(f)$ de $f$, parametrizado por
	\begin{equation*}
		\varphi(x,y) = (x,y,f(x,y)), (x,y) \in U.
	\end{equation*}
	
	Temos
	\begin{align*}
		\varphi_x &= (1,0,f_x)\\
		\varphi_y &= (0,1,f_y).
	\end{align*}
	
	Assim
	\begin{align*}
		E &= \innerproduct{\varphi_x}{\varphi_x} = 1 + f_x^2\\
		F &= \innerproduct{\varphi_x}{\varphi_y} = f_x f_y\\
		G &= \innerproduct{\varphi_y}{\varphi_y} = 1 + f_y^2.
	\end{align*}
	
	Un campo $n$, normal a $\text{Gr}(f)$, é dado por
	\begin{align*}
		n = \varphi_x \times \varphi_y &= \det \left[ \begin{matrix}
		i & j & k\\
		1 & 0 & f_x\\
		0 & 1 & f_y
		\end{matrix} \right]\\
		&= (-f_x, -f_y, 1).
	\end{align*}
	
	Normalizando, temos
	\begin{equation*}
		N = \frac{n}{\norm{n}} = \frac{1}{\sqrt{1 + f_x^2 + f_y^2}}(-f_x, -f_y, 1).
	\end{equation*}
	
	Como
	\begin{align*}
		\varphi_{xx} &= (0, 0, f_{xx})\\
		\varphi_{xy} &= (0, 0, f_{xy})\\
		\varphi_{yy} &= (0, 0, f_{yy})
	\end{align*}
	
	obtemos
	\begin{align*}
		e &= \innerproduct{\varphi_{xx}}{N} = \frac{f_{xx}}{\sqrt{1 + f_x^2 + f_y^2}}\\
		f &= \innerproduct{\varphi_{xy}}{N} = \frac{f_{xy}}{\sqrt{1 + f_x^2 + f_y^2}}\\
		g &= \innerproduct{\varphi_{yy}}{N} = \frac{f_{yy}}{\sqrt{1 + f_x^2 + f_y^2}}.
	\end{align*}
	
	Assim, como
	\begin{equation*}
		H = \frac{eG - 2fF + gE}{2(EG - F^2)}
	\end{equation*}
	
	segue que se $H \equiv 0$, temos
	\begin{equation}\label{edp_superficies_minimas}
		(1 + f_y^2) f_{xx}  - 2 f_x f_y f_{xy} + (1+f_x^2) f_{yy} = 0
	\end{equation}
	
	que é uma EDP de 2da ordem.
	
	Um exemplo trivial da equação \ref{edp_superficies_minimas} é a função linear
	\begin{equation*}
		f(x,y) = ax + by + c,
	\end{equation*}
	
	como $a, b, c \in \realnumbers$.
\end{exemplo}

\begin{exemplo}[Superfície de Scherk]
	Suponha que
	\begin{equation*}
		f(x,y) = g(x) + h(y).
	\end{equation*}
	
	Neste caso, a equação \ref{edp_superficies_minimas} pode ser escrita como
	\begin{equation*}
		(1 + (h')^2(y)) g''(x) + (1 + (g')^2(x)) h''(y) = 0,
	\end{equation*}
	
	ou seja
	\begin{equation*}
		\frac{g''(x)}{1 + (g')^2(x)} + \frac{h''(y)}{1 + (h')^2(y)} = 0.
	\end{equation*}
	
	Isso implica
	\begin{equation*}
		\frac{g''(x)}{1 + (g')^2(x)} = - \frac{h''(y)}{1 + (h')^2(y)} = \text{constante}.
	\end{equation*}
	
	Integrando, obtemos (a menos de constantes) que
	\begin{align*}
		g(x) &= \ln (\cos x)\\
		h(y) &= -\ln (\cos y).
	\end{align*}
	
	A menos de dilatações e translações uma parte da superfície pode ser representada pelo gráfico da função
	\begin{equation*}
		\ln \left( \frac{\cos x}{\cos y} \right), 0 < x,y < \frac{\pi}{2}
	\end{equation*}
\end{exemplo}

\section{A representação de Weierstrass}

Considere o plano complexo $\complexnumbers$ identificado com $\realnumbers^2$
\begin{equation*}
(x,y) \in \realnumbers^2 \mapsto x + iy \in \complexnumbers.
\end{equation*}

Uma função complexa $f: U \subset \complexnumbers \rightarrow \complexnumbers$ pode ser escrita na forma
\begin{equation*}
f(u,v) = f_1(u,v) + i f_2(u,v)
\end{equation*}

onde $f_1, f_2: U \rightarrow \realnumbers$ são funções reais, denotadas por
\begin{align*}
f_1 &= \Re(f)\\
f_2 &= \Im(f)
\end{align*}

tal que $\Re(f)$ é parte real da função $f$ e $\Im(f)$ é a parte imaginaria da função $f$.

\begin{definicao}
	Uma função $f: U \subset \complexnumbers \rightarrow \complexnumbers$, definida no aberto $U$, é dita \emph{holomorfa} se $f_1, f_2$ possuem derivadas parciais continuas e satisfazem as equações de Cauchy-Riemann
	\begin{align*}
	\partialdifffrac{f_1}{u} &= \partialdifffrac{f_2}{v}\\
	\partialdifffrac{f_1}{v} &= - \partialdifffrac{f_2}{u}
	\end{align*}
\end{definicao}

\begin{definicao}
	Uma carta local $(U, \varphi)$ em $M$ é dita \emph{mínima} se $H(p) = 0, \forall p \in \varphi(U)$.
\end{definicao}

\begin{corolario}\label{equiv_isoterma_harmonica}
	Seja $(U, \varphi)$ uma carta local isoterma de uma superfície $M \subset \realnumbers^3$. Então $(U, \varphi)$ é mínima se e somente se $\varphi$ é harmônica, i.e., $\varphi_{uu} + \varphi_{vv} = 0$.
\end{corolario}

Dadas uma superfície $M \subset \realnumbers^3$ e uma carta local $(U, \varphi)$ em $M$, com
\begin{equation*}
\varphi(u,v) = (x_1(u,v), x_2(u,v), x_3(u,v)),
\end{equation*}

considere as funções complexas $f_j: U \subset \complexnumbers \rightarrow \complexnumbers, 1 \leq j \leq 3,$ dadas por
\begin{equation}\label{carta_isoterma_cauchy-riemann}
f_j = \partialdifffrac{x_j}{u} - i \partialdifffrac{x_j}{v}, 1 \leq j \leq 3
\end{equation}

\begin{lema}
	Seja $(U, \varphi)$ uma carta local isoterma em $M$. Então, $\varphi$ é mínima se e somente se cada $f_j$, definida em \ref{carta_isoterma_cauchy-riemann}, é holomorfa.
\end{lema}

\begin{demonstracao}
	Pelo corolário \ref{equiv_isoterma_harmonica}, temos que $\varphi$ é mínima se e somente se $\varphi$ é harmônica, i.e., $\varphi_{uu} + \varphi_{vv} = 0$. Isso significa que
	\begin{equation*}
	\npartialdifffrac{x_j}{u}{2} + \npartialdifffrac{x_j}{v}{2} = 0, 1 \leq j \leq 3.
	\end{equation*}
	
	Queremos provar que
	\begin{align*}
	\pdiff{u} \Re(f_j) &= \pdiff{v} \Im(f_j),\\
	\pdiff{v} \Re(f_J) &= - \pdiff{u} \Im(f_j)
	\end{align*}
	
	Assim
	\begin{multline*}
	\pdiff{u} \Re(f_J) = \pdiff{u} \partialdifffrac{x_j}{u} = \npartialdifffrac{x_j}{u}{2} = - \npartialdifffrac{x_j}{v}{2} = \pdiff{v} \left( - \partialdifffrac{x_j}{v} \right) = \pdiff{v} \Im(f_j)
	\end{multline*}
	
	Isso prova a primeira equação de Cauchy-Riemann. Por outro lado, como a superfície é regular, vale
	\begin{equation*}
	\varphi_{uv} = \varphi_{vu},
	\end{equation*}
	
	ou seja
	\begin{equation*}
	\frac{\partial^2 x_j}{\partial u \partial v} = \frac{\partial^2 x_j}{\partial v \partial u}.
	\end{equation*}
	
	Assim
	\begin{align*}
	\pdiff{v} \Re(f_j) &= \pdiff{v} \partialdifffrac{x_j}{u} = \pdiff{u} \partialdifffrac{x_j}{v}\\
	&= \pdiff{u} \left( - \Im(f_j) \right)\\
	&= - \pdiff{u} \Im(f_j),
	\end{align*}
	
	que é a segunda equação de Cauchy-Riemann.
\end{demonstracao}

\begin{lema}\label{lema_fj_2}
	Sejam $M \subset \realnumbers^3$ uma superfície mínima e $(U, \varphi)$ uma carta local isoterma. Então, as funções holomorfas $f_j$, definidas em \ref{carta_isoterma_cauchy-riemann}, satisfazem
	\begin{gather}\label{sum_fj_2}
	f_1^2 + f_2^2 + f_3^2 = 0\\ \label{sum_norm_fj_2}
	|f_1|^2 + |f_2|^2 + |f_3|^2 \neq 0
	\end{gather}
	
	Reciprocamente, sejam $f_1, f_2, f_3$ funções holomorfas, definidas num aberto simplesmente conexo $U \subset \complexnumbers$, satisfazendo \ref{sum_fj_2} e \ref{sum_norm_fj_2}. Então, tais funções dão origem a uma carta local isoterma mínima $(U, \varphi)$.
\end{lema}

\begin{demonstracao}
	Seja $(U, \varphi)$ a carta local isoterma em $M$. Então,
	\begin{align*}
	f_1^2 + f_2^2 + f_3^2 &= \sum_{j=1}^{3} \left[ \left( \partialdifffrac{x_j}{u} \right)^2 - \left( \partialdifffrac{x_j}{v} \right)^2 - 2i \partialdifffrac{x_j}{u} \partialdifffrac{x_j}{v} \right]\\
	&= E - G - 2iF = 0,
	\end{align*}
	
	pois $E=G$ e $F=0$. A equação \ref{sum_norm_fj_2} segue da regularidade de $\varphi$, pois $\varphi_u \neq 0$ e $\varphi_v \neq 0$.
	
	Reciprocamente, defina
	\begin{equation}\label{carta_isoterma_eq_integral}
	x_j(u,v) = \int_{\xi_0}^{\xi} f_j(z) dz, 1 \leq j \leq 3
	\end{equation}
	
	com $\xi = (u,v) \in U$, para algum $\xi_0 \in U$ fixado. Note que cada $x_j$ está bem definida pois $U$ é simplesmente conexo e $f_j$ é holomorfa, o que nos dá uma função holomorfa definida em $U$, para o qual podemos aplicar as equações de Cauchy-Riemann, obtendo:
	\begin{align*}
	\frac{d}{d \xi} \int_{\xi_0}^{\xi} f_j &= \frac{d}{d \xi} \left[ \Re \int_{\xi_0}^{\xi} f_j + i \Im \int_{\xi_0}^{\xi} f_j \right]\\
	&= \pdiff{u} \Re \int_{\xi_0}^{\xi} f_j + i \pdiff{u} \Im \int_{\xi_0}^{\xi} f_j\\
	&= \pdiff{u} \Re \int_{\xi_0}^{\xi} f_j - i \pdiff{v} \Re \int_{\xi_0}^{\xi} f_j
	\end{align*}
	
	de modo que a equação \ref{carta_isoterma_cauchy-riemann} é válida. Considere a aplicação $\varphi: U \rightarrow \realnumbers^3$, cujas funções coordenadas
	\begin{equation*}
	\varphi = (x_1,x_2,x_3)
	\end{equation*}
	
	são dadas como em \ref{carta_isoterma_eq_integral}. De \ref{sum_fj_2} e \ref{sum_norm_fj_2} segue que $(U,\varphi)$ é uma carta local isoterma. Além disso, as funções $f_j$ serem holomorfas implicam que as funções coordenadas $x_j$ são harmônicas, logo, pelo corolário, $\varphi$ é mínima.
\end{demonstracao}

\begin{observacao}
	As funções $x_j$ definidas em \ref{carta_isoterma_eq_integral} estão definidas a menos de uma constante aditiva, de modo que a superfície está definida a menos de uma translação. Assim, o estudo local de superfícies mínimas em $\realnumbers^3$ reduz-se a resolver as equações \ref{sum_fj_2} e \ref*{sum_norm_fj_2} para uma terna de funções holomorfas.
\end{observacao}

\begin{teorema}
	Sejam $f: U \subset \complexnumbers \rightarrow \complexnumbers$ uma função holomorfa e $g: U \rightarrow \complexnumbers$ uma função meromorfa tais que $fg^2$ seja holomorfa. Assuma que se $\xi \in U$ é um polo de ordem $n$ para $g$ então $\xi$ é um zero para $f$ de ordem $2n$, e que estes sejam os únicos zeros de $f$.
	
	Então, a aplicação
	\begin{equation}\label{carta_minima_duas_funcoes}
	\varphi(z) = \frac{1}{2} f(z) \left( (1-g(z)^2), i (1+g(z)^2), 2g(z) \right)
	\end{equation}
	
	satisfaz as condições do Lema \ref{lema_fj_2}. Além disso, para toda tal $\varphi$, existem funções holomorfa $f$ e meromorfa $g$ tais que vale \ref{carta_minima_duas_funcoes}.
\end{teorema}

\begin{demonstracao}
	Se $\varphi$ satisfaz \ref{carta_minima_duas_funcoes}, temos
	\begin{align*}
	f_1^2 + f_2^2 + f_3^2 &= \frac{1}{4} f(z)^2 (1 - g(z)^2)^2 - \frac{1}{4} f(z)^2 (1 + g(z)^2)^2 + f(z)^2 g(z)^2\\
	&= 0
	\end{align*}
	
	Afirmamos que $\varphi(z) \neq 0, \forall z \in U$. De fato, a hipótese sobre os zeros de $f$ e os polos de $g$ implicam que $f(z) g(z)^2 \neq 0$. Assim, para qualquer $z$ fixado, a primeira e a segunda coordenada de $\varphi$ não podem ser ambas nulas.
	
	Assim, podemos assumir que $\varphi$ é holomorfa satisfazendo
	\begin{equation*}
	\varphi_1^2 + \varphi_2^2 + \varphi_3^2 \not\equiv 0,
	\end{equation*}
	
	$\varphi$ nunca é zero, e considere
	\begin{align*}
	f(z) &= \varphi_1(z) - i \varphi_2(z)\\
	g(z) &= \frac{\varphi_3(z)}{\varphi_1(z) - i \varphi_2(z)}
	\end{align*}
	
	$f$ é uma função holomorfa e $g$ é o quociente de funções holomorfas. Se o denominador de $g$ é identicamente nulo, façamos
	\begin{equation*}
	g(z) = \frac{\varphi_3(z)}{\varphi_1(z) + i \varphi_2(z)}
	\end{equation*}
	
	e proceder de forma similar.
	
	Assim, sendo o denominador de $g$ não nulo, tem-se que $g$ é meromorfa. Assim, a relação
	\begin{equation*}
	\varphi_1^2 + \varphi_2^2 + \varphi_3^2 = 0
	\end{equation*}
	
	implica
	\begin{equation*}
	(\varphi_1 + i \varphi_2)(\varphi_1 - i \varphi_2) = -\varphi_3^2
	\end{equation*}
	
	que, em termos de $f$ e $g$ torna-se
	\begin{align*}
	\varphi_1 + i \varphi_2 &= \frac{-\varphi_3^2}{\varphi_1 - i \varphi_2}\\
	&= \frac{-\varphi_3^2}{(\varphi_1 - i \varphi_2)^2} (\varphi_1 - i \varphi_2)\\
	&= -fg^2
	\end{align*}
	
	Esta última equação, juntamente com as condições sobre $f$ e $g$, nos dão $\varphi$ como em \ref{carta_minima_duas_funcoes}.
\end{demonstracao}

\begin{definicao}
	Sejam $U \subset \complexnumbers$ um aberto simplesmente conexo e $\gamma \subset U$ uma curva de um ponto fixado $z_0 \in U$ a um ponto arbitrário $z \in U$, $z = u + iv$.
	
	Sejam $f,g$ como no teorema anterior. Então,
	\begin{equation*}
	\varphi(u,v) = (x_1(u,v), x_2(u,v), x_3(u,v)),
	\end{equation*}
	
	onde
	\begin{align*}
	x_1 &= \Re \int_{\gamma} \frac{1}{2} f(z) (1 - g(z)^2) dz\\
	x_2 &= \Re \int_{\gamma} \frac{1}{2} f(z) (1 + g(z)^2) dz\\
	x_3 &= \Re \int_{\gamma} f(z) g(z) dz
	\end{align*}
	
	é uma carta local mínima, chamada \emph{a representação de Weierstrass} da teoria local de superfícies mínimas.
\end{definicao}

\begin{exemplo}[Catenoide]
	O catenoide pode ser representado pelas funções holomorfas $f, g: \complexnumbers \rightarrow \complexnumbers$ dadas por
	\begin{align*}
	f(z) &= e^{-z},\\
	g(z) &= e^z.
	\end{align*}
	
	Substituindo tais funções na formula da representação de Weierstrass e integrando de $z_0 = 0$ a um ponto arbitrário $z = u + iv$, obtemos
	\begin{align*}
	\varphi(u,v) &= x_0 + \Re \int_{z_0}^{z} \frac{f(\xi)}{2} (1 - g(\xi)^2, i (1 + g(\xi)^2), 2 g(\xi)) d\xi\\
	&= x_0 + \Re \int_{z_0}^{z} \frac{e^{-\xi}}{2} (1 - e^{2\xi}, i (1 + e^{2\xi}), 2e^{\xi}) d\xi\\
	&= \Re \int_{0}^{z} \frac{1}{2} (e^{-\xi} - e^{\xi}, i (e^{-\xi} + e^{\xi}), 1) d\xi\\
	&= \Re \left[ \frac{1}{2} \left(-e^{-z} - e^z, -\frac{1}{2i} (-e^{-z} + e^z), z \right) \right] \\
	&= \Re \left( -\cosh z, i \sinh z, z \right) \\
	&= \left( -\cosh u \cos v, -\cosh u \sin v, u \right)
	\end{align*} 
\end{exemplo}

\begin{exemplo}[Superfície de Enneper]
	A superfície de Enneper pode ser representada pelas funções holomorfas $f,g: \complexnumbers \rightarrow \complexnumbers$ dadas por
	\begin{align*}
	f(z) &= 1, \\
	g(z) &= z.
	\end{align*}
	
	Assim, a representação de Weierstrass torna-se
	\begin{align*}
	\varphi(u,v) &= \Re \left( \frac{1}{2} \int_{0}^{z} \left( 1 - \xi^2, i (1 + \xi^2), 2\xi \right) \right) d\xi \\
	&= \frac{1}{2} \Re \left( z - \frac{z^3}{3}, iz + \frac{iz^3}{3}, z^2 \right) \\
	&= \frac{1}{2} \left( u - \frac{u^3}{3} + uv^2, -v + \frac{v^3}{3} - u^2v, u^2 - v^2 \right)
	\end{align*}
	
	gráfico
\end{exemplo}

\begin{exemplo}[Superfície de Scherk]
	A superfície de Scherk, definida pela equação
	\begin{equation*}
	e^z = \frac{\cos y}{\cos x},
	\end{equation*}
	
	pode ser representada pelas funções holomorfas $f: \complexnumbers \setminus \{\pm 1, \pm i \} \rightarrow \complexnumbers$ e $g: \complexnumbers \rightarrow \complexnumbers$ dadas por
	\begin{align*}
	f(z) &= \frac{2}{1 - z^4}, \\
	g(z) &= z.
	\end{align*}
	
	Note que
	\begin{align*}
	f (1 - g^2) &= \frac{2}{1 + z^2} = \frac{i}{z + i} - \frac{i}{z - i}, \\
	i f (1 + g^2) &= \frac{2i}{1 - z^2} = \frac{i}{z + 1} - \frac{i}{z - 1}, \\
	2fg &= \frac{4z}{1 - z^4} = \frac{2z}{z^2 + 1} - \frac{2z}{z^2 - 1}.
	\end{align*}
	
	Assim, substituindo na representação de Weierstrass e integrando, obtemos
	\begin{equation*}
	\varphi(z) = \left( -\arg \frac{z + i}{z - i}, -\arg \frac{z + i}{z - i}, \log \left\| \frac{z^2 + 1}{z^2 - 1} \right\| \right).
	\end{equation*}
	
	Usando as identidades
	\begin{align*}
	\frac{z + i}{z - i} &= \frac{|z|^2 - 1}{|z^2 - i|^2} + i \frac{z + \overline{z}}{|z - i|^2}, \\
	\frac{z + 1}{z - 1} &= \frac{|z|^2 - 1}{|z - 1|^2} + i \frac{\overline{z} - z}{|z - 1|^2},
	\end{align*}
	
	podemos encontrar as expressões para $\cos x$ e $\cos y$. Temos
	\begin{align*}
	\cos x &= \cos \left( -\arg \frac{z + i}{z - i} \right) \\
	&= \cos \left( \arg \frac{z + i}{z - i} \right) \\
	&= \cos \left( \arg \left( \frac{|z - i|}{|z + i|} \frac{z + i}{z - i} \right) \right) \\
	&= \Re \left( \frac{|z - i|}{|z + i|} \frac{z + i}{z - i} \right) \\
	&= \frac{|z - i|}{|z + i|} \Re \left( \frac{z + i}{z - i} \right) \\
	&= \frac{|z - i|}{|z + i|} \frac{|z|^2 - 1}{|z - i|^2} = \frac{|z|^2 - 1}{|z^2 + 1|}.
	\end{align*}
	
	Analogamente, temos
	\begin{align*}
	\cos y &= \cos \left( -\arg \frac{z + 1}{z - 1} \right) \\
	&= \frac{|z - 1|}{|z + 1|} \frac{|z|^2 - i}{|z - 1|^2} \\
	&= \frac{|z|^2 - i}{|z^2 - 1|}
	\end{align*}
	
	Isso implica que
	\begin{equation*}
	\frac{\cos y}{\cos x} = \frac{z^2 + 1}{|z^2 - 1|} = e^z
	\end{equation*}
	
	Vejamos uma aplicação da representação de Weierstrass. Dado uma superfície mínima $M \subset \realnumbers^3$, seja $(U, \varphi)$ uma carta local isoterma.
	
	Isso significa que
	\begin{align*}
	E = G &= \lambda^2, \\
	F &= 0
	\end{align*}
	
	onde
	\begin{align*}
	\lambda^2 &= \frac{1}{2} \sum_{j=1}^{3} |f_j|^2 \\
	&= \frac{1}{4} |f|^2 |1 + g|^2 + \frac{1}{4} |f|^2 |1 + g|^2 + |fg|^2\\
	&= \left( \frac{|f| (| + |g|^2)}{2} \right)^2
	\end{align*} 
	
	Além disso, temos
	\begin{align*}
	\varphi_u \times \varphi_v &= \left( \Im (f_2 \overline{f}_3), \Im (f_3 \overline{f}_1), \Im (f_1 \overline{f}_2) \right) \\
	&= \frac{|f|^2 (1 + |g|^2)}{4} \left( 2 \Re(g), 2 \Im(g), |g|^2 - | \right), \\
	\| \varphi_u \times \varphi_v \| &= \sqrt{EG - F^2} = \lambda^2,
	\end{align*}
	
	de modo que
	\begin{equation*}
	N = \left( \frac{2 \Re(g)}{|g|^2 + 1}, \frac{2 \Im(g)}{|g|^2 + 1}, \frac{|g|^2 - 1}{|g|^2 + 1} \right)
	\end{equation*}
	
	Lembremos que a projeção estereográfica
	\begin{equation*}
	\pi: S^2 \setminus \{ (0,0,1) \} \rightarrow \complexnumbers
	\end{equation*}
	
	é a aplicação
	\begin{equation*}
	\pi(x_1, x_2, x_3) = \frac{x_1 + ix_2}{1 - x_3},
	\end{equation*}
	
	e uma inversa é dada por
	\begin{equation*}
	\pi^{-1}(z) = \left( \frac{2 \Re(z)}{|z|^2 + 1}, \frac{2 \Im(z)}{|z|^2 + 1}, \frac{|z|^2 - 1}{|z|^2 + 1} \right)
	\end{equation*}
	
	Portanto, temos que
	\begin{equation*}
	N = \pi^{-1} \circ g
	\end{equation*}
	
	Podemos resumir isso no seguinte resultado.
\end{exemplo}

\begin{proposicao}
	Sejam $M \subset \realnumbers^3$ uma superfície mínima e $(U, \varphi)$ uma carta local isoterma. Então, um dos campos unitários $N$, normal a $M$ é a inversa da projeção estereográfica da função $g$ dada pela representação de Weierstrass.
\end{proposicao}

\begin{corolario}
	Seja $M \subset \realnumbers^3$ uma superfície mínima definida no plano todo. Então, ou $M$ é um plano ou a imagem da aplicação de Gauss omite pelo menos dois pontos.
\end{corolario}

\begin{demonstracao}
	Se $M$ não está contida num plano, podemos construir a função $g$ que é meromorfa no plano todo $\complexnumbers$. Pelo teorema de Picard, ela atinge todos seus valores com, pelo menos, duas excepções, ou $g$ é constante. A equação (referencia) mostra que o mesmo se aplica a $N$ e, no último caso, $M$ está contida em um plano.
\end{demonstracao}

\begin{teorema}[Existência local de parâmetros isotermos]
	Seja $M \subset \realnumbers^3$ uma superfície mínima. Então, todo $p \in M$ pertence a uma vizinhança coordenada isoterma.
\end{teorema}

\begin{demonstracao}
	Seja $U \subset M$ uma vizinhança coordenada de $p$ que é o gráfico de uma função diferenciável que, podemos assumir, ser da forma $z = h(x,y), (x,y) \in U$.
	
	Lembrando que a equação para gráficos mínimos e
	\begin{equation*}
	(1 + h_y^2) h_{xx} - 2 h_x h_y h_{xy} + (1 + h_x^2) h_{yy} = 0,
	\end{equation*}
	
	obtemos a equação
	\begin{equation*}
	\pdiff{x} \frac{1 + h_y^2}{W} = \pdiff{y} \frac{h_x h_y}{W}
	\end{equation*}
	
	em $U$, onde $W = \sqrt{1 + h_x^2 + h_y^2}$. Escolhendo $U$ simplesmente conexo, isso implica que existe uma função diferenciável $\phi: U \rightarrow \realnumbers$, com
	\begin{align*}
	\partialdifffrac{\phi}{x} &= \frac{h_x h_y}{W} \\
	\partialdifffrac{\phi}{y} &= \frac{1 + h_y^2}{W}
	\end{align*}
	
	Introduza novas coordenadas
	\begin{align*}
	\overline{x} &= x \\
	\overline{y} &= \phi(x,y)
	\end{align*}
	
	Um calculo simples mostra que
	\begin{align*}
	\partialdifffrac{x}{\overline{x}} &= 1, \\
	\partialdifffrac{x}{\overline{y}} &= 0, \\
	\partialdifffrac{y}{\overline{x}} &= -\frac{h_x h_y}{1 + h_y^2}, \\
	\partialdifffrac{y}{\overline{y}} &= \frac{W}{1 + h_y^2},
	\end{align*}
	
	e os coeficientes da Segunda Forma Fundamental, em relação a $(\overline{x}, \overline{y})$ são
	\begin{align*}
	E = G &= \frac{W^2}{1 + h_y^2} \\
	F &= 0
	\end{align*}
	
	como queríamos.
\end{demonstracao}