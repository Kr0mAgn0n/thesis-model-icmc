Neste capítulo apresentaremos a demonstração da conjetura de
Lawson obtida por Simon Brendle \cite{Brendle2013a}.


\section{Introdu\c c\~ao}

Em 1970, Blaine Lawson \cite{Lawson1970a} conjecturou que o toro de 
Clifford é a única superfície mínima, compacta e mergulhada em $\Sp^3$,
com genus $1$, e foi provada somente quatro décadas depois por Simon
Brendle \cite{Brendle2013a} em 2013. A hipótese de ser mergulhada é
fundamental. De fato, em \cite{Lawson1970} Lawson construiu uma 
família (infinita) de imersões mínimas de toros em $\Sp^3$.

A prova da conjetura de Lawson em \cite{Brendle2013a} envolve uma 
aplicação do princípio do máximo para uma função que depende de 
duas vari\'aveis. Esta técnica foi inicialmente desenvolvida por Huisken
\cite{Huisken1998} no estudo do fluxo de comprimento de curvas para 
curvas mergulhadas no plano e, posteriormente, por Andrews 
\cite{Andrews2012}.

Antes de apresentarmos os argumentos usados em \cite{Brendle2013a},
na prova da conjectura de Lawson, faremos algumas considera\c c\~oes
iniciais. Seguiremos aqui o artigo \cite{Andrews2012}, onde o autor
apresenta a no\c c\~ao de n\~ao-colapsante para hipersuperf\'icies
Euclidianas mergulhadas.

Considere uma superf\'icie $\Sigma$ em $\R^3$, cuja curvatura
m\'edia $H$ \'e positiva em todos os pontos, e limitando uma 
regi\~ao aberta $\Omega$ em $\R^3$.

\begin{definicao}
A superf\'icie $\Sigma$ \'e dita ser {\em $\delta$-n\~ao-colapsante}
se, para todo ponto $x\in\Sigma$, existe uma bola aberta $B$ de
raio $\delta/H(x)$, contida em $\Omega$, com $x\in\partial B$. 
\end{definicao}

Dado uma superf\'icie $F:\Sigma\to\R^3$, defina uma fun\c c\~ao
$Z:\Sigma\times\Sigma\to\R$ pondo
\[
Z(x,y) = \frac{H(x)}{2}\|F(y)-F(x)\|^2 + \delta\langle F(y)-F(x), \nu(x)\rangle,
\]
onde $\nu$ \'e um campo unit\'ario, normal a $\Sigma$.

\begin{proposicao}[\cite{Andrews2012}]\label{prop:Int}
A superf\'icie $\Sigma$ \'e $\delta$-n\~ao-colapsante se, e somente
se, $Z(x,y)\geq0$, para quaisquer $x,y\in\Sigma$.
\end{proposicao}
\begin{demonstracao}
Sem perda de generalidade, escolha o campo normal $\nu$ que
aponta para fora de $\Sigma$. Assim, uma bola em $\Omega$ de
raio $\delta/H(x)$, com $F(x)$ sendo um ponto da fronteira, deve
ter centro no ponto
\[
p(x) = F(x) - (\delta/H(x))\nu(x).
\]
A afirma\c c\~ao de que esta bola est\'a contida em $\Omega$
\'e equivalente ao fato de que nenhum ponto de $\Sigma$ tem
dist\^ancia menor do que $\delta/H(x)$ ao ponto $p$. Ou seja,
\[
0\leq\|F(y)-p(x)\|^2-\left(\frac{\delta}{H(x)}\right)^2 = 
2\cdot\frac{Z(x,y)}{H(x)},
\]
para quaisquer $x,y\in\Sigma$. Como $H>0$ em todos os pontos
de $\Sigma$, isso \'e equivalente ao fato de que $Z$ seja
n\~ao-negativa em todos os pontos. A rec\'iprova \'e imediata.
\end{demonstracao}

Considere agora uma superf\'icie m\'inima e mergulhada 
$F:\Sigma\to\Sp^3$ e $ \Phi: \Sigma\to\R$ uma função positiva. Defina
uma função $Z:\Sigma\times\Sigma\to\R$ pondo
\begin{equation}\label{def_de_Z}
Z(x,y) = \Phi(x) (1 - \langle F(x),F(y) \rangle ) + \langle \nu(x), F(y) \rangle,
\end{equation}
onde $\nu$ \'e um campo unitário, normal a $\Sigma$. Observando que
uma bola geod\'esica na esfera $\Sp^3$ \'e simplesmente a interseção
de uma bola de $\R^4$ com $\Sp^3$ podemos provar, de forma
an\'aloga \`a Proposi\c c\~ao \ref{prop:Int}, que a fun\c c\~ao $Z(x,y)$,
definida em \eqref{def_de_Z}, \'e n\~ao-negativa se, e somente se,
a superf\'icie $\Sigma$ \'e n\~ao-colapsante.




\section{Alguns resultados t\'ecnicos}

Nesta se\c c\~ao apresentaremos alguns resultados preliminares
que ser\~ao usados na demonstra\c c\~ao da conjectura de
Lawson. 

\vspace{.2cm}

Dados uma superfície mínima e mergulhada $F:\Sigma\to\Sp^3$ 
e uma função positiva $\Phi: \Sigma\to\R$, considere a função 
$Z:\Sigma\times\Sigma\to\R$ dada por
\begin{eqnarray}\label{eq:funcaoZ}
Z(x,y) = \Phi(x) (1 - \langle F(x), F(y) \rangle) + \langle \nu(x), F(y) \rangle,
\end{eqnarray}
onde $\nu$ \'e um campo unitário, normal a $\Sigma$. Considere dois 
pontos $\xb, \yb \in \Sigma$, com $\xb \neq \yb$, tais que $Z(\xb, \yb) = 0$ e $dZ(\xb, \yb) = 0$. Sejam $(x_1,x_2), (y_1,y_2)$ sistemas de coordenadas geodésicas em torno dos pontos $\xb, \yb$, respectivamente. No ponto $(\xb,\yb)$, temos:
\begin{eqnarray}\label{diff_Z_x}
\begin{aligned}
0 = \frac{\partial Z}{\partial x_i} (\xb, \yb) = &
\frac{\partial \Phi}{\partial x_i}(\xb) (1 - \langle F(\xb, F(\yb) \rangle) -  
\Phi(\xb) \left\langle \frac{\partial F}{\partial x_i}(\xb), F(\yb) \right\rangle \\ & + \sum_{k=1}^{2} h_{ik}(\xb) \left\langle \frac{\partial F}{\partial x_k}(\xb), F(\yb) \right\rangle
\end{aligned}
\end{eqnarray}
e
\begin{equation}\label{diff_Z_y}
0 = \frac{\partial Z}{\partial y_i} (\xb, \yb) = - \Phi(\xb) \left\langle F(\xb), \frac{\partial F}{\partial y_i} (\yb) \right\rangle + \left\langle \nu(\xb), \frac{\partial F}{\partial y_i}(\yb) \right\rangle
\end{equation}
onde $h_{ij}(\xb)$ denota a $(ij)$-ésima coordenada da matriz da segunda forma fundamental de $F$ no ponto $\xb$, i.e.,
\begin{eqnarray}\label{eq:partialNu}
\frac{\partial\nu}{\partial x_i}(\xb) = 
\sum_{k=1}^2h_{ik}\frac{\partial F}{\partial x_k}(\xb).
\end{eqnarray}

\vspace{.2cm}

Sem perda de generalidade, podemos supor que a segunda forma
fundamental de $F$ est\'a diagonalizada no ponto $\xb$, i.e.,
\[
h_{11}(\xb)=\lambda_1, \quad h_{12}(\xb) = 0 \quad\mbox{e}\quad
h_{22}(\xb) = \lambda_2.
\]
Denotemos por $w_i$ a reflex\~ao do vetor
\[
\frac{\partial F}{\partial x_i}(\xb)
\]
em rela\c c\~ao ao plano ortogonal ao vetor $F(\xb) - F(\yb)$, i.e.,
\begin{eqnarray}\label{eq:reflexao}
w_i = \frac{\partial F}{\partial x_i}(\xb) - 
2 \left\langle \frac{\partial F}{\partial x_i}(\xb), \frac{F(\xb) - 
F(\yb)}{|F(\xb) - F(\yb)|} \right\rangle \frac{F(\xb) - 
F(\yb)}{|F(\xb) - F(\yb)|}.
\end{eqnarray}
Escolhendo um sistema de coordenadas apropriado $(y_1,y_2)$,
podemos supor que
\begin{equation*}
\left\langle w_1, \frac{\partial F}{\partial y_1}(\yb) \right\rangle \geq 0, 
\quad 
\left\langle w_1, \frac{\partial F}{\partial y_2}(\yb) \right\rangle = 0 
\quad \text{e} \quad 
\left\langle w_2, \frac{\partial F}{\partial y_2}(\yb) \right\rangle \geq 0.
\end{equation*}

\begin{lema}
Os vetores $F(\yb)$ e $ \Phi(\xb) F(\xb) - \nu(\xb) $ são linearmente independentes.
\end{lema}
\begin{demonstracao}
Usando a identidade
\[
\langle \Phi(\xb) F(\xb) - \nu(\xb), F(\yb) \rangle = 
\Phi(\xb) - Z(\xb,\yb) = \Phi(\xb),
\]
obtemos
\begin{eqnarray*}
| \Phi(\xb) F(\xb) - \nu(\xb) |^2 | F(\yb) |^2 &-& 
\langle \Phi(\xb) F(\xb) - \nu(\xb), F(\yb) \rangle^2 \\
&=& | \Phi(\xb) F(\xb) - \nu(\xb) |^2 - \Phi(\xb)^2 = 1 \neq 0.
\end{eqnarray*}
Disso decorre que a desigualdade de Cauchy-Schwarz \'e estrita,
logo os vetores $F(\yb)$ e $\Phi(\xb)F(\xb)-\nu(\xb)$
s\~ao linearmente independentes.
\end{demonstracao}

Em rela\c c\~ao \`a reflex\~ao dada em \eqref{eq:reflexao}, obtemos o
seguinte:

\begin{lema} \label{lem:w_1 w_2}
Valem as seguintes igualdades:
\begin{equation*}
w_1 = \frac{\partial F}{\partial y_1}(\yb) 
\quad \text{e} \quad 
w_2 = \frac{\partial F}{\partial y_2}(\yb).
\end{equation*}
\end{lema}
\begin{demonstracao}
Usando a express\~ao de $w_i$, dada em \eqref{eq:reflexao}, obtemos:
\begin{eqnarray*}
\langle w_i, F(\yb) \rangle &=& \left\langle \frac{\partial F}{\partial x_i}(\xb), F(\yb) \right\rangle + 2 \left\langle \frac{\partial F}{\partial x_i}, F(\yb) \right\rangle \frac{\langle F(\xb) - F(\yb),F(\yb) \rangle}{| F(\xb) - F(\yb) |^2} \\
&=& \left\langle \frac{\partial F}{\partial x_i}(\xb), F(\yb) \right\rangle 
+ 2 \left\langle \frac{\partial F}{\partial x_i}, F(\yb) \right\rangle \frac{\langle F(\xb),F(\yb) \rangle - 1}{2 - 2 \langle F(\xb), F(\yb) \rangle} \\
&=& 0
\end{eqnarray*}
e
\begin{eqnarray*}
\langle w_i, \Phi(\xb) F(\xb) - \nu(\xb) \rangle &=&
2 \left\langle \frac{\partial F}{\partial x_i}(\xb), F(\yb) \right\rangle \frac{\langle F(\xb) - F(\yb), \Phi(\xb) F(\xb) - \nu(\xb) \rangle}{| F(\xb) - F(\yb) |^2} \\
& = & 2 \left\langle \frac{\partial F}{\partial x_i}(\xb), F(\yb) \right\rangle \frac{Z(\xb,\yb)}{| F(\xb) - F(\yb) |^2} \\
& = & 0.
\end{eqnarray*}
Por outro lado, os vetores
\begin{equation*}
\frac{\partial F}{\partial y_1}(\yb) \quad \text{e} \quad \frac{\partial F}
{\partial y_2}(\yb)
\end{equation*}
satisfazem
\begin{equation*}
\left\langle \frac{\partial F}{\partial y_i}(\yb), F(\yb)\right\rangle = 0
\end{equation*}
e
\begin{equation*}
\left\langle \frac{\partial F}{\partial y_i}(\yb), \Phi(\xb) F(\xb) - 
\nu(\xb) \right\rangle = -\frac{\partial Z}{\partial y_i}(\xb,\yb) 
= 0.
\end{equation*}
Como os vetores $F(\yb)$ e $ \Phi(\xb) F(\xb) - \nu(\xb)$
são linearmente independentes, conclu\'imos que o plano gerado por
\begin{equation*}
\frac{\partial F}{\partial y_1}(\yb) \quad \text{e} \quad 
\frac{\partial F}{\partial y_2}(\yb)
\end{equation*}
é o mesmo plano gerado por $w_1$ e $w_2$. Al\'em disso, os vetores
$w_1$ e $w_2$ s\~ao ortonormais. Como
\begin{equation*}
\left\langle w_1, \frac{\partial F}{\partial y_2}(\yb) \right\rangle = 0,
\end{equation*}
conclu\'imos que
\begin{equation*}
w_1 = \pm \frac{\partial F}{\partial y_1}(\yb) 
\quad \text{e} \quad 
w_2 = \pm \frac{\partial F}{\partial y_2}(\yb).
\end{equation*}
Finalmente, como
\begin{equation*}
\left\langle w_1, \frac{\partial F}{\partial y_1}(\yb) \right\rangle \geq 0 
\quad \text{e} \quad 
\left\langle w_2, \frac{\partial F}{\partial y_2}(\yb) \right\rangle \geq 0
\end{equation*}	
obtemos que
\begin{equation*}
w_1 = \frac{\partial F}{\partial y_1}(\yb) \quad \text{e} \quad 
w_2 = \frac{\partial F}{\partial y_2}(\yb),
\end{equation*}
como quer\'iamos.
\end{demonstracao}

No resultado seguinte iremos considerar as derivadas de segunda
ordem da fun\c c\~ao $Z$ no ponto $(\xb,\yb)$.

\begin{proposicao}
A derivada segunda da fun\c c\~ao $Z$, dada em \eqref{eq:funcaoZ},
satisfaz:
\begin{eqnarray*}\label{2_diff_Z_x}
\sum_{i=1}^{2} \frac{\partial^2 Z}{\partial x_i^2} (\xb,\yb) &=& 
\left( \Delta_{\Sigma} \Phi(\xb) - \frac{| \nabla \Phi(\xb) |^2}{\Phi(\xb)} 
+ (|A(\xb)|^2 - 2) \Phi(\xb) \right) (1 - \left\langle F(\xb), F(\yb) \right\rangle) \\ 
&&+ 2 \Phi(\xb) - \frac{2 \Phi(\xb)^2 - |A(\xb)|^2}{2 \Phi(\xb)
 (1 - \langle F(\xb), F(\yb) \rangle)} \sum_{i=1}^2 \left\langle 
 \frac{\partial F}{\partial x_i}(\xb), F(\yb) \right\rangle^2.
\end{eqnarray*}
\end{proposicao}
\begin{demonstracao}
Derivando a equação \eqref{diff_Z_x} em rela\c c\~ao a $x_i$,
e somando, obtemos:
\begin{eqnarray}\label{Z_seg_dev_x}
\begin{aligned}
\sum_{i=1}^2\frac{\partial^2 Z}{\partial x_i^2}(\xb,\yb) = &
\sum_{i=1}^2\frac{\partial^2 \Phi}{\partial x_i^2}(\xb)
(1 - \langle F(\xb), F(\yb) \rangle) -
2\sum_{i=1}^2\frac{\partial \Phi}{\partial x_i}(\xb) 
\left\langle \frac{\partial F}{\partial x_i}(\xb), F(\yb) \right\rangle \\
& - \sum_{i=1}^2\Phi(\xb) \left\langle\frac{\partial^2 F}{\partial x_i^2}(\xb),F(\yb)
\right\rangle + \sum_{i,k=1}^2\frac{\partial h_{ik}}{\partial x_i}(\xb)
\left\langle \frac{\partial F}{\partial x_k}(\xb), F(\yb)\right\rangle \\ 
&+ \sum_{i,k=1}^2h_{ik}(\xb)\left\langle
\frac{\partial^2 F}{\partial x_i \partial x_k}(\xb), F(\yb)\right\rangle.
\end{aligned}
\end{eqnarray}
Iremos, inicialmente, reescrever a equa\c c\~ao \eqref{Z_seg_dev_x}.
Note que, da equa\c c\~ao de Codazzi \eqref{eq-codazzi}, tem-se
\begin{eqnarray}\label{eq:Prop4.5-1}
\frac{\partial h_{11}}{\partial x_2}(\xb) = \frac{\partial h_{21}}{\partial x_1}(\xb)
\quad\mbox{e}\quad
\frac{\partial h_{12}}{\partial x_2}(\xb) = \frac{\partial h_{22}}{\partial x_1}(\xb),
\end{eqnarray}
e do fato de $\Sigma$ ser m\'inima, obtemos
\begin{eqnarray}\label{eq:Prop4.5-2}
\sum_{i=1}^2\frac{\partial h_{ii}}{\partial x_k}(\xb) = 0.
\end{eqnarray}
De \eqref{eq:Prop4.5-1} e \eqref{eq:Prop4.5-2}, obtemos
\begin{eqnarray}\label{eq:Prop4.5-3}
\sum_{i=1}^2\frac{\partial h_{ik}}{\partial x_i}(\xb) = 0.
\end{eqnarray}
Analisemos o termo 
$\displaystyle\sum_{i=1}^2 \frac{\partial^2 F}{\partial x_i^2}(\xb)$. 
Pelo teorema \ref{propriedades_sup_min_S3}, tem-se
\begin{equation*}
\sum_{i=1}^2 \frac{\partial^2 F_j}{\partial x_i^2}(\xb) + 2F_j(\xb) = 0,
\end{equation*}
para $j=1,\ldots,4$, logo,
\[
\sum_{i=1}^2 \frac{\partial^2 F}{\partial x_i^2}(\xb) = -2 F(\xb).
\]
Analisemos agora o termo 
\[
\sum_{i=1}^2 h_{ik}(\xb) \left\langle \frac{\partial^2 F}
{\partial x_i \partial x_k}(\xb), F(\yb) \right\rangle.
\]
Derivamos \eqref{eq:partialNu} em rela\c c\~ao a $x_i$, somando
e usando \eqref{eq:Prop4.5-3}, obtemos:
\begin{eqnarray}\label{eq:Prop4.5-4}
\sum_{i=1}^2 \frac{\partial^2 \nu}{\partial x_i^2}(\xb) = 
\sum_{i,k=1}^2 h_{ik}(\xb)\frac{\partial^2 F}{\partial x_i \partial x_k}(\xb).
\end{eqnarray}
Note que, como
\[
\left\langle \sum_{i=1}^2 \frac{\partial^2 \nu}{\partial x_i^2}(\xb), 
\frac{\partial F}{\partial x_j}(\xb) \right\rangle = 0,
\]
para $j=1,2$, segue que o vetor
$\displaystyle\sum_{i=1}^2 \frac{\partial^2 \nu}{\partial x_i^2}(\xb)$ 
não tem componentes no plano tangente 
$T_{\xb}\Sigma\subset T_{\xb}\Sp^3$. 
Por outro lado, derivando
\[
\left\langle \frac{\partial \nu}{\partial x_i}(\xb), \nu(\xb) \right\rangle = 0,
\]
em rela\c c\~ao a $x_i$, obtemos:
\begin{equation}\label{eq:Prop4.5-5}
\left\langle \frac{\partial^2 \nu}{\partial x_i^2}(\xb), \nu(\xb)\right\rangle 
+ \left\langle \frac{\partial \nu}{\partial x_i}(\xb), 
\frac{\partial \nu}{\partial x_i}(\xb) \right\rangle = 0.
\end{equation}	
Substituindo \eqref{eq:partialNu} em \eqref{eq:Prop4.5-5}, notando que
$\left\langle \frac{\partial F}{\partial x_k}(\xb), 
\frac{\partial F}{\partial x_j}(\xb) \right\rangle = \delta_{kj}$ e somando,
a equa\c c\~ao \eqref{eq:Prop4.5-5} torna-se
\begin{equation*}
\left\langle \sum_{i=1}^2 \frac{\partial^2 \nu}{\partial x_i^2}(\xb), \nu(\xb)
\right\rangle + \sum_{i=1}^2 (h_{ik}(\xb))^2 = 0.
\end{equation*}	
Como $\displaystyle\sum_{i=1}^2 (h_{ik}(\xb))^2 = |A(\xb)|^2$ e
lembrando que 
$\displaystyle\sum_{i=1}^2 \frac{\partial^2 \nu}{\partial x_i^2}(\xb)$ 
só tem componente na direção $\nu$, obtemos
\begin{eqnarray} \label{eq:Prop4.5-6}	
\sum_{i=1}^2 \frac{\partial^2 \nu}{\partial x_i^2}(\xb) = - | A(\xb) |^2 \nu(\xb).
\end{eqnarray}
Usando \eqref{eq:Prop4.5-4} e \eqref{eq:Prop4.5-6}, temos que
\begin{eqnarray} \label{eq:Prop4.5-7}
\sum_{i=1}^2 h_{ik}(\xb) \frac{\partial^2 F}{\partial x_i \partial x_k}(\xb) 
= - | A(\xb) |^2 \nu(\xb).
\end{eqnarray}	
Assim, usando \eqref{eq:Prop4.5-3} e \eqref{eq:Prop4.5-7}, podemos
escrever a equa\c c\~ao \eqref{Z_seg_dev_x} como:
\begin{eqnarray}
\begin{aligned} \label{lap_Z_x}
\sum_{i=1}^2 \frac{\partial^2 Z}{\partial x_i^2}(\xb,\yb) & =   
\sum_{i=1}^2 \frac{\partial^2 \Phi}{\partial x_i^2}(\xb)
(1 - \langle F(\xb), F(\yb) \rangle)    
+ 2  \Phi(\xb) \left\langle F(\xb), F(\yb) \right\rangle \\
& -2 \sum_{i=1}^2 \frac{\partial \Phi}{\partial x_i}(\xb) 
\left\langle \frac{\partial F}{\partial x_i}(\xb), F(\yb) \right\rangle
- | A(\xb) |^2 \left\langle \nu(\xb), F(\yb) \right\rangle.
\end{aligned}
\end{eqnarray}
Como
\[
\left\langle \nu(\xb), F(\yb) \right\rangle = Z(\xb,\yb)
-\phi(\xb)(1-\langle F(\xb),F(\yb)\rangle)
\]
podemos reescrever \eqref{lap_Z_x} como sendo
\begin{eqnarray*}
\begin{aligned}
\sum_{i=1}^2 \frac{\partial^2 Z}{\partial x_i^2}(\xb,\yb) = &
\left(\Delta_{\Sigma} \Phi(\xb) + 
(|A(\xb)|^2 - 2)\Phi(\xb)\right)(1 - \langle F(\xb), F(\yb) \rangle) \\
&+ 2 \Phi(\xb) \left\langle F(\xb), F(\yb) \right\rangle
- 2 \sum_{i=1}^2 \frac{\partial \Phi}{\partial x_i}(\xb) 
\left\langle \frac{\partial F}{\partial x_i}(\xb), F(\yb) \right\rangle \\
&- |A(\xb)|^2Z(\xb,\yb).
\end{aligned}
\end{eqnarray*}	
Somando e subtraindo
\[
\frac{|\nabla \Phi(\xb)|^2}{\Phi(\xb)} (1 - \langle F(\xb), F(\yb) \rangle) + \frac{\Phi(\xb)}{1 - \langle F(\xb),F(\yb) \rangle} \sum_{i=1}^2 \left\langle \frac{\partial F}{\partial x_i} (\xb), F(\yb) \right\rangle^2,
\]
obtemos:
\begin{eqnarray*}
\begin{aligned}
\sum_{i=1}^2 \frac{\partial^2 Z}{\partial x_i^2}(\xb,\yb) &=
\left(\Delta_{\Sigma} \Phi(\xb) - \frac{|\nabla \Phi(\xb)|^2}{\Phi(\xb)} + (|A(\xb)|^2 - 2)\Phi(\xb)\right)(1 - \langle F(\xb), F(\yb) \rangle) \\
&+ 2 \Phi(\xb) + \frac{|\nabla \Phi(\xb)|^2}{\Phi(\xb)} (1 - \langle F(\xb), F(\yb) \rangle) - 2 \sum_{i=1}^2 \frac{\partial \Phi}{\partial x_i}(\xb) \left\langle \frac{\partial F}{\partial x_i}(\xb), F(\yb) \right\rangle \\
&+ \frac{\Phi(\xb)}{1 - \langle F(\xb),F(\yb) \rangle} \sum_{i=1}^2 \left\langle \frac{\partial F}{\partial x_i} (\xb), F(\yb) \right\rangle^2 \\
& - \frac{\Phi(\xb)}{1 - \langle F(\xb),F(\yb) \rangle} \sum_{i=1}^2
\left\langle \frac{\partial F}{\partial x_i} (\xb), F(\yb) \right\rangle^2.
\end{aligned}
\end{eqnarray*}
Fatorando a express\~ao acima adequadamente, podemos escrever
\begin{eqnarray*}
\begin{aligned}
\sum_{i=1}^2 \frac{\partial^2 Z}{\partial x_i^2}(\xb,\yb) &= \left(\Delta_{\Sigma} \Phi(\xb) - \frac{|\nabla \Phi(\xb)|^2}{\Phi(\xb)} + (|A(\xb)|^2 - 2)\Phi(\xb)\right)(1 - \langle F(\xb), F(\yb) \rangle)+ 2 \Phi(\xb)	\\
& - \frac{\Phi(\xb)}{1 - \langle F(\xb),F(\yb) \rangle} \sum_{i=1}^2 \left\langle \frac{\partial F}{\partial x_i} (\xb), F(\yb) \right\rangle^2 \\
& + \frac{1}{\Phi(\xb)(1 - \langle F(\xb), F(\yb) \rangle)}m(\xb,\yb),
\end{aligned}
\end{eqnarray*}	
onde
\begin{eqnarray*}
\begin{aligned}
m(\xb,\yb) &= - 2 \sum_{i=1}^2 \frac{\partial \Phi}{\partial x_i}(\xb) 
(1 - \langle F(\xb), F(\yb) \rangle) \Phi(\xb) \left\langle \frac{\partial F}
{\partial x_i}(\xb), F(\yb) \right\rangle \\
&+ |\nabla \Phi(\xb)|^2 (1 - \langle F(\xb), F(\yb) \rangle)^2 
 + \Phi(\xb)^2 \sum_{i=1}^2 \left\langle \frac{\partial F}
 {\partial x_i} (\xb), F(\yb) \right\rangle^2.
\end{aligned}
\end{eqnarray*}	
Observe que
\[
m(\xb,\yb) = \sum_{i=1}^2 \left(  \frac{\partial \Phi}{\partial x_i}(\xb)
(1 - \langle F(\xb),F(\yb) \rangle) - \Phi(\xb) \left\langle 
\frac{\partial F}{\partial x_i}(\xb), F(\yb) \right\rangle \right)^2.
\]
Pela equação \eqref{diff_Z_x}, e lembrando que 
$\frac{\partial Z}{\partial x_i}(\xb,\yb)=0$, podemos escrever
\begin{eqnarray} \label{eq:Prop4.5-8}
m(\xb,\yb) = \sum_{i=1}^2 \left( h_{ik}(\xb) 
\left\langle \frac{\partial F}{\partial x_k}(\xb), F(\yb) \right\rangle \right)^2.
\end{eqnarray}	
Expandindo o termo quadr\'atico em \eqref{eq:Prop4.5-8}, obtemos:
\begin{eqnarray*}
\begin{aligned}
(h_{i1}(\xb))^2 \left\langle \frac{\partial F}{\partial x_1}(\xb), F(\yb)\right\rangle^2 
&+ 
(h_{i2}(\xb))^2 \left\langle \frac{\partial F}{\partial x_2}(\xb),F(\yb)\right\rangle^2 \\
&+ 
2 h_{i1}(\xb) h_{i2}(\xb) \left\langle \frac{\partial F}
{\partial x_1}(\xb), F(\yb) \right\rangle \left\langle 
\frac{\partial F}{\partial x_2}(\xb), F(\yb) \right\rangle.
\end{aligned}
\end{eqnarray*}	
Como o determinante de $A$ \'e $-\lambda^2$, tem-se:
\begin{equation*}
h_{11}(\xb) h_{22}(\xb) - h_{12}(\xb) h_{21}(\xb) = h_{11}(\xb) h_{22}(\xb) - (h_{12}(\xb))^2 = -\lambda^2,
\end{equation*}	
pois $h_{12}(\xb) = h_{21}(\xb)$. Al\'em disso, como 
$h_{11}(\xb) + h_{22}(\xb) = 0$, tem-se
\[
\sum_{i=1}^2 (h_{i1}(\xb))^2 = \lambda^2 \quad\mbox{e}\quad
\sum_{i=1}^2 (h_{i2}(\xb))^2 = \lambda^2.
\]
Em rela\c c\~ao ao termo $\displaystyle\sum_{i=1}^2 h_{i1}(\xb) h_{i2}(\xb)$,
tem-se
\begin{equation*}
h_{11}(\xb) h_{12}(\xb) + h_{21}(\xb) h_{22}(\xb) = h_{12}(\xb) (h_{11}(\xb) + h_{22}(\xb)) = 0.
\end{equation*}	
Portanto, podemos reescrever \eqref{eq:Prop4.5-8} como sendo
\begin{equation*}
m(\xb,\yb) = \sum_{i=1}^2 \lambda^2 \left\langle 
\frac{\partial F}{\partial x_i}(\xb), F(\yb) \right\rangle^2.
\end{equation*}	
Lembrando agora que $\lambda^2 = \frac{1}{2} |A(\xb)|$, obtemos
a equação desejada.
\end{demonstracao}


\begin{proposicao}
Em rela\c c\~ao \`as derivadas mistas, vale a seguinte rela\c c\~ao:
\begin{equation}\label{diff_Z_x_y}
\frac{\partial^2 Z}{\partial x_i \partial y_i}(\xb,\yb) = \lambda_i - \Phi(\xb).
\end{equation}
\end{proposicao}
\begin{demonstracao}
Derivando em rela\c c\~ao  a $x_i$ a equação \eqref{diff_Z_y}, obtemos:
\begin{eqnarray*}
\begin{aligned}
\frac{\partial^2 Z}{\partial x_i \partial y_i}(\xb,\yb) =  & 
-\frac{\partial \Phi}{\partial x_i}(\xb) \left\langle F(\xb), 
\frac{\partial F}{\partial y_i}(\yb) \right\rangle - \Phi(\xb)
\left\langle \frac{\partial F}{\partial x_i}(\xb), 
\frac{\partial F}{\partial y_i}(\yb) \right\rangle \\
& + \left\langle \frac{\partial \nu}{\partial x_i}(\xb), 
\frac{\partial F}{\partial y_i}(\yb) \right\rangle.
\end{aligned}
\end{eqnarray*}	
Lembrando que 
\[
\frac{\partial \nu}{\partial x_i}(\xb) = 
\lambda_i(\xb) \frac{\partial F}{\partial x_i}(\xb),
\]
temos
\begin{eqnarray} \label{eq:Prop2.3_1}
\frac{\partial^2 Z}{\partial x_i \partial y_i}(\xb,\yb) = -\frac{\partial \Phi}{\partial x_i}(\xb) \left\langle F(\xb), \frac{\partial F}{\partial y_i}(\yb) \right\rangle + (\lambda_i(\xb) - \Phi(\xb)) \left\langle \frac{\partial F}{\partial x_i}(\xb), \frac{\partial F}{\partial y_i}(\yb) \right\rangle.
\end{eqnarray}	
Da equa\c c\~ao \eqref{diff_Z_x}, tem-se:
\begin{eqnarray} \label{eq:Prop2.3_2}
-\frac{\partial \Phi}{\partial x_i}(\xb) = \frac{1}
{1 - \langle F(\xb),F(\yb) \rangle} (\lambda_i(\xb) - \Phi(\xb)) 
\left\langle \frac{\partial F}{\partial x_i}(\xb), F(\yb) \right\rangle.
\end{eqnarray}
Substituindo \eqref{eq:Prop2.3_2} em \eqref{eq:Prop2.3_1}, 
obtemos:
\begin{eqnarray*}
\begin{aligned}
\frac{\partial^2 Z}{\partial x_i \partial y_i}(\xb,\yb) =&  \frac{1}{1 - \langle F(\xb),F(\yb) \rangle} (\lambda_i(\xb) - \Phi(\xb)) \left\langle \frac{\partial F}{\partial x_i}(\xb), F(\yb) \right\rangle  \left\langle F(\xb), \frac{\partial F}{\partial y_i}(\yb) \right\rangle \\
&+ 
(\lambda_i(\xb) - \Phi(\xb)) \left\langle \frac{\partial F}{\partial x_i}(\xb), \frac{\partial F}{\partial y_i}(\yb) \right\rangle.
\end{aligned}
\end{eqnarray*}	
Como 
\[
|F(\xb) - F(\yb)|^2 = \langle F(\xb) - F(\yb), F(\xb) - F(\yb) \rangle = 
2 - 2 \langle F(\xb), F(\yb) \rangle,
\]
a equa\c c\~ao anterior torna-se
\begin{eqnarray} \label{eq:Prop2.3_3}
\begin{aligned}
\frac{\partial^2 Z}{\partial x_i \partial y_i}(\xb,\yb) = & 
-2 (\lambda_i(\xb) - \Phi(\xb)) \left\langle \frac{\partial F}{\partial x_i}(\xb), 
\frac{-F(\yb)}{|F(\xb) - F(\yb)|} \right\rangle \left\langle \frac{F(\xb)}{|F(\xb) - 
F(\yb)|}, \frac{\partial F}{\partial y_i}(\yb) \right\rangle \\
& + 
(\lambda_i - \Phi(\xb)) \left\langle \frac{\partial F}{\partial x_i}(\xb), 
\frac{\partial F}{\partial y_i}(\yb) \right\rangle.
\end{aligned}
\end{eqnarray}	
Al\'em disso, como
\[
\left\langle \frac{\partial F}{\partial x_i}(\xb), F(\xb) \right\rangle = 
\left\langle \frac{\partial F}{\partial y_i}(\yb), F(\yb) \right\rangle = 0,
\]
a equação \eqref{eq:Prop2.3_3} se escreve como sendo
\begin{eqnarray*}
%\begin{aligned}
\frac{\partial^2 Z}{\partial x_i \partial y_i}(\xb,\yb) & = &
-2 (\lambda_i(\xb) - \Phi(\xb)) \left\langle \frac{\partial F}{\partial x_i}(\xb), \frac{F(\xb) -F(\yb)}{|F(\xb) - F(\yb)|} \right\rangle \left\langle 
\frac{F(\xb) - F(\yb)}{|F(\xb) - F(\yb)|}, \frac{\partial F}{\partial y_i}(\yb) 
\right\rangle \\
&& + 
(\lambda_i - \Phi(\xb)) \left\langle \frac{\partial F}{\partial x_i}(\xb), 
\frac{\partial F}{\partial y_i}(\yb) \right\rangle \\
&= &
(\lambda_i - \Phi(\xb)) \left\langle w_i,\frac{\partial F}{\partial y_i}(\yb)
\right\rangle \\
&= & \lambda_i - \Phi(\xb),
%\end{aligned}
\end{eqnarray*}	
como quer\'iamos.
\end{demonstracao}


\begin{proposicao}
\begin{equation}\label{2_diff_Z_y}
\sum_{i=1}^2 \frac{\partial^2 Z}{\partial y_i^2}(\xb,\yb) = 2 \Phi(\xb).
\end{equation}
\end{proposicao}
\begin{demonstracao}
Derivando a equa\c c\~ao \eqref{diff_Z_y} em rela\c c\~ao a $y_i$,
e somando, obtemos:
\begin{eqnarray} \label{eq:Prop4.7-1}
\sum_{i=1}^2 \frac{\partial^2 Z}{\partial y_i^2}(\xb,\yb) = - \Phi(\xb) \left\langle F(\xb), \sum_{i=1}^2 \frac{\partial^2 F}{\partial y_i^2}(\yb) \right\rangle + \left\langle \nu(\xb), \sum_{i=1}^2 \frac{\partial^2 F}{\partial y_i^2}(\yb) \right\rangle.
\end{eqnarray}	
Como
\[
\sum_{i=1}^2 \frac{\partial^2 F}{\partial y_i}(\yb) = -2 F(\yb)
\]
e
\[
0 = Z(\xb,\yb) = \Phi(\xb)(1 - \langle F(\xb), F(\yb) \rangle) + 
\langle \nu(\xb), F(\yb) \rangle
\] 
a express\~ao em \eqref{eq:Prop4.7-1} torna-se
\begin{eqnarray*}
\sum_{i=1}^2 \frac{\partial^2 Z}{\partial y_i^2}(\xb,\yb) &=&
2 \Phi(\xb) \langle F(\xb), F(\yb) \rangle - 2 \langle \nu(\xb), F(\yb) 
\rangle \\
&=&
2 \Phi(\xb),
\end{eqnarray*}
como quer\'iamos.
\end{demonstracao}


\begin{proposicao} \label{prop:somaDerSeg}
Vale a seguinte igualdade:
\begin{eqnarray} \label{suma_2das_derivadas}
\begin{aligned}
\sum_{i=1}^2 \frac{\partial^2 Z}{\partial x_i^2}(\xb,\yb) + & 
2 \sum_{i=1}^2 \frac{\partial^2 Z}{\partial x_i \partial y_i}(\xb,\yb) + \sum_{i=1}^2 \frac{\partial^2 Z}{\partial y_i^2}(\xb,\yb) \\
&= 
- \frac{2 \Phi(\xb)^2- |A(\xb)|^2}{2 \Phi(\xb)
(1 - \langle F(\xb),F(\yb) \rangle)} \sum_{i=1}^2
\left\langle \frac{\partial F}{\partial x_i}(\xb), F(\yb) \right\rangle^2 \\
&+
\left( \Delta_{\Sigma} \Phi(\xb) - \frac{| \nabla \Phi(\xb) |^2}
{\Phi(\xb)} + ( | A(\xb) |^2 - 2 ) \Phi(\xb) \right)
\left(1 - \langle F(\xb),F(\yb) \rangle\right).
\end{aligned}
\end{eqnarray}
\end{proposicao}
\begin{demonstracao}
A equa\c c\~ao \eqref{suma_2das_derivadas} segue das equa\c c\~oes \eqref{2_diff_Z_x}, \eqref{diff_Z_x_y} e \eqref{2_diff_Z_y}.
\end{demonstracao}



\section{Prova do Teorema \ref{teo:Lawson}}

Usando os resultados preliminares apresentados nas se\c c\~oes
anteriores, apresentaremos nesta se\c c\~ao a prova do teorema
principal dessa disserta\c c\~ao. 

\vspace{.2cm}

Inicialmente, iremos determinar uma identidade tipo-Simons para
a fun\c c\~ao
\begin{eqnarray}\label{eq:PsiA(x)}
\Psi(x)=\frac{1}{\sqrt2}|A(x)|,
\end{eqnarray}
onde $A$ denota o operador de forma da superf\'icie $F:\Sigma\to\Sp^3$.

\begin{proposicao}\label{edp_principal}
Seja $F:\Sigma\to\Sp^3$ um toro m\'inimo e mergulhado em $\Sp^3$.
Então, a função $\Psi$ dada em \eqref{eq:PsiA(x)} \'e estritamente 
positiva e satisfaz a EDP
\[
\Delta_\Sigma \Psi - \frac{|\nabla \Psi|^2}{\Psi} + (|A|^2 - 2) \Psi = 0.
\]
\end{proposicao}
\begin{demonstracao}
Segue do trabalho de Lawson que um toro m\'inimo na esfera $\Sp^3$
n\~ao tem pontos umb\'ilicos (cf. \cite[Proposition 1.5]{Lawson1970}).
Disso decorre que a função $|A|$ \'e estritamente positiva. Usando a
identidade de Simons \cite[Theorem 5.3.1]{Simons1968}, obtemos
\begin{eqnarray} \label{eq:prop43-1}
\Delta_{\Sigma} h_{ik} + (|A|^2 - 2)h_{ik} = 0,
\end{eqnarray}	
visto que $(h_{ik})$ \'e a matriz que representa $A$. Multiplicando a
equa\c c\~ao \eqref{eq:prop43-1} por $2 h_{ik}$, tem-se
\begin{eqnarray} \label{eq:prop43-2}
2 \Delta_{\Sigma} h_{ik} h_{ik} + 2 (|A|^2 - 2)h_{ik}^2 = 0.
\end{eqnarray}	
Somando e subtraindo a quantidade
\[
2 \sum_{j=1}^2 \left( \frac{\partial h_{ik}}{\partial x_j} \right)^2
\]
na equa\c c\~ao \eqref{eq:prop43-2}, obtemos
\begin{eqnarray} \label{eq:prop43-3}
2 \sum_{j=1}^2 \frac{\partial^2 h_{ik}}{\partial x_j^2} h_{ik} + 2 \sum_{j=1}^2 \left( \frac{\partial h_{ik}}{\partial x_j} \right)^2 - 2 \sum_{j=1}^2 \left( \frac{\partial h_{ik}}{\partial x_j} \right)^2 + 2 (|A|^2 - 2)h_{ik}^2 = 0.
\end{eqnarray}	
Como 
\[
2 \sum_{j=1}^2 \frac{\partial^2 h_{ik}}{\partial x_j^2} h_{ik} + 2 \sum_{j=1}^2 \left( \frac{\partial h_{ik}}{\partial x_j} \right)^2 = 2 \sum_{j=1}^2 \frac{\partial }{\partial x_j} \left( \frac{\partial h_{ik}}{\partial x_j} h_{ik} \right) = \sum_{j=1}^2 \frac{\partial^2 h_{ik}^2}{\partial x_j^2},
\]
podemos reescrever a equa\c c\~ao \eqref{eq:prop43-3} como sendo
\begin{eqnarray} \label{eq:prop43-4}
\sum_{j=1}^2 \frac{\partial^2 h_{ik}^2}{\partial x_j^2} - 2 \sum_{j=1}^2 \left( \frac{\partial h_{ik}}{\partial x_j} \right)^2 + 2 (|A|^2 - 2)h_{ik}^2 = 0.
\end{eqnarray}
Na equa\c c\~ao \eqref{eq:prop43-4}, somando em rela\c c\~ao a $i$
e $k$, obtemos:
\begin{equation} \label{eq:prop43-5}
\Delta_{\Sigma} (|A|^2) - 2 | \nabla A |^2 + 2 (|A|^2 - 2) |A|^2 = 0.
\end{equation}
Note agora que
\begin{eqnarray}\label{eq:prop43-6}
\begin{aligned}
\Delta_{\Sigma} (|A|^2) &= \sum_{j=1}^2 \frac{\partial^2 |A|^2}{\partial x_j^2}
= \sum_{j=1}^2 \frac{\partial}{\partial x_j} 
\left( \frac{\partial |A|^2}{\partial x_j} \right) =
2\sum_{j=1}^2\frac{\partial}{\partial x_j}\left(|A|
\frac{\partial|A|}{\partial x_j}\right) \\
&= 
2\sum_{j=1}^2\left(\frac{\partial|A|}{\partial x_j}\frac{\partial|A|}{\partial x_j}
+|A|\frac{\partial^2 |A|}{\partial x_j^2} \right) =
2|\nabla|A||^2+2|A|\sum_{j=1}^2\frac{\partial^2 |A|}{\partial x_j^2} \\
&= 2|\nabla|A||^2+2|A|\Delta_\Sigma|A|.
\end{aligned}
\end{eqnarray}	
Assim, substituindo \eqref{eq:prop43-6} em \eqref{eq:prop43-5},
obtemos
\begin{equation}\label{edp_sff}
\Delta_{\Sigma} (|A|) + \frac{|\nabla |A||^2}{|A|} - \frac{|\nabla A|^2}{|A|} + 
(|A|^2 - 2) |A| = 0.
\end{equation}	
Provemos agora que
\begin{eqnarray} \label{eq:prop43-14}
|\nabla A|^2 = 2|\nabla|A||^2.
\end{eqnarray}
De fato, elevando ao quadrado a equa\c c\~ao \eqref{eq:Prop4.5-3}, 
obtemos:
\begin{eqnarray} \label{eq:prop43-7}
\begin{aligned}
\left( \frac{\partial h_{11}}{\partial x_1} \right)^2 + \left( \frac{\partial h_{21}}{\partial x_2} \right)^2  &= - 2 \frac{\partial h_{11}}{\partial x_1} \frac{\partial h_{21}}{\partial x_2},  \\
\left( \frac{\partial h_{12}}{\partial x_1} \right)^2 + \left( \frac{\partial h_{22}}{\partial x_2} \right)^2 &= - 2 \frac{\partial h_{12}}{\partial x_1} \frac{\partial h_{22}}{\partial x_2}.
\end{aligned}
\end{eqnarray}	
Cada uma das equa\c c\~oes em \eqref{eq:prop43-7} corresponde
a $k=1$ e $k=2$, respectivamente. Disso decorre que
\begin{eqnarray} \label{eq:prop43-8}
\begin{aligned}
\left( \frac{\partial h_{11}}{\partial x_1} \right)^2 + \left( \frac{\partial h_{21}}{\partial x_2} \right)^2  &=  2  \left( \frac{\partial h_{21}}{\partial x_2} \right)^2,  \\
\left( \frac{\partial h_{12}}{\partial x_1} \right)^2 + \left( \frac{\partial h_{22}}{\partial x_2} \right)^2 &=  2 \left( \frac{\partial h_{12}}{\partial x_1} \right)^2, \\
\left( \frac{\partial h_{11}}{\partial x_2} \right)^2 + \left( \frac{\partial h_{21}}{\partial x_1} \right)^2  &= 2 \left( \frac{\partial h_{21}}{\partial x_1} \right)^2, \\
\left( \frac{\partial h_{12}}{\partial x_2} \right)^2 + \left( \frac{\partial h_{22}}{\partial x_1} \right)^2 &=  2 \left( \frac{\partial h_{12}}{\partial x_2} \right)^2  .
\end{aligned}
\end{eqnarray}	
Somando as equações em \eqref{eq:prop43-8}, e lembrando que
$h_{12} = h_{21}$, obtemos:
\begin{equation} \label{eq:prop43-13}
| \nabla A |^2 = 4 \left( \frac{\partial h_{12}}{\partial x_1} \right)^2 + 4\left( \frac{\partial h_{12}}{\partial x_2} \right)^2.
\end{equation}	
Note que 
\begin{eqnarray} \label{eq:prop43-15}
|A| = \sqrt{2 h_{11}^2 + 2 h_{12}^2}
\end{eqnarray}
e
\begin{eqnarray} \label{eq:prop43-11}
| \nabla |A| |^2 = \left( \frac{\partial |A|}{\partial x_1} \right)^2 +
\left( \frac{\partial |A|}{\partial x_2} \right)^2.
\end{eqnarray}
Derivando $|A|$, dada em \eqref{eq:prop43-15}, em rela\c c\~ao a 
$x_1$ e $x_2$ obtemos, respectivamente, que
\begin{eqnarray} \label{eq:prop43-9}
\frac{\partial |A|}{\partial x_1} = \frac{2 h_{11} \frac{\partial h_{11}}{\partial x_1} + 2 h_{12} \frac{\partial h_{12}}{\partial x_1}}{|A|}
\quad\mbox{e}\quad
\frac{\partial |A|}{\partial x_2} = \frac{2 h_{11} \frac{\partial h_{11}}{\partial x_2} + 2 h_{12} \frac{\partial h_{12}}{\partial x_2}}{|A|}.
\end{eqnarray}	
Elevando ao quadrado as duas equa\c c\~oes em \eqref{eq:prop43-9},
obtemos
\begin{eqnarray} \label{eq:prop43-10}
\begin{aligned}
\left( \frac{\partial |A|}{\partial x_1} \right)^2 &= \frac{4 h_{11}^2 \left( \frac{\partial h_{11}}{\partial x_1} \right)^2 + 4 h_{12}^2 \left( \frac{\partial h_{12}}{\partial x_1} \right)^2 + 8 h_{11} h_{12} \frac{\partial h_{11}}{\partial x_1} \frac{\partial h_{12}}{\partial x_1}}{|A|^2} \\
\left( \frac{\partial |A|}{\partial x_2} \right)^2 &= \frac{4 h_{11}^2 \left( \frac{\partial h_{11}}{\partial x_2} \right)^2 + 4 h_{12}^2 \left( \frac{\partial h_{12}}{\partial x_2} \right)^2 + 8 h_{11} h_{12} \frac{\partial h_{11}}{\partial x_2} \frac{\partial h_{12}}{\partial x_2}}{|A|^2}
\end{aligned}
\end{eqnarray}	
Somando as duas equa\c c\~oes em \eqref{eq:prop43-10}, e usando
\eqref{eq:prop43-11} e \eqref{eq:prop43-9}, obtemos
\begin{eqnarray} \label{eq:prop43-12}
| \nabla |A| |^2 = 2 \left( \frac{\partial h_{12}}{\partial x_1} \right)^2 + 2 \left( \frac{\partial h_{12}}{\partial x_2} \right)^2.
\end{eqnarray}	
Das equa\c c\~oes \eqref{eq:prop43-13} e \eqref{eq:prop43-12}
obtem-se \eqref{eq:prop43-14}. Finalmente, substituindo a
equa\c c\~ao \eqref{eq:prop43-14} em \eqref{edp_sff}, obtemos
\begin{equation*}
\Delta_\Sigma \Psi - \frac{|\nabla \Psi|^2}{\Psi} + (|A|^2 - 2) \Psi = 0,
\end{equation*}
finalizando a demonstra\c c\~ao.
\end{demonstracao}


\begin{proposicao} \label{prop:torus1}
Seja $F:\Sigma\to\Sp^3$ um toro m\'inimo e mergulhado na esfera
$\Sp^3$. Se 
\begin{equation} \label{eq:Prop432-1}
\sup_{\substack{x,y \in \Sigma\\ x \neq y}} \frac{| \langle  \nu(x), F(y) \rangle |}
{\Psi(x) (1 - \langle F(x), F(y) \rangle)} \leq 1,
\end{equation}
ent\~ao $F$ \'e congruente ao toro de Clifford.
\end{proposicao}
\begin{demonstracao}
Segue da hip\'otese \eqref{eq:Prop432-1} que
\begin{equation} \label{eq:Prop432-2}
\Psi(x) (1 - \langle F(x), F(y) \rangle) + \langle \nu(x), F(y) \rangle \geq 0,
\end{equation}
para quaisquer $x,y\in\Sigma $. Por uma quest\~ao de simplicidade,
identificaremos a superf\'icie $\Sigma$ com sua imagem atrav\'es 
do mergulho $F$, i.e., $F(x) = x$, para todo $x \in \Sigma$. 
Fixado um ponto arbitr\'ario $\xb\in\Sigma$, podemos encontrar uma
base ortonormal $\{e_1,e_2 \}$ de $T_{\xb}\Sigma$ tal que
\begin{equation*}
h(e_1,e_1) = \Psi(\xb), \quad h(e_1,e_2)=0 \quad \text{e} \quad 
h(e_2,e_2) = -\Psi(\xb),
\end{equation*}
onde $h$ denota a segunda forma fundamental de $F$ no ponto $\xb$.
Dado uma geod\'esica $\gamma$ na superf\'icie $\Sigma$, com 
$\gamma(0)=p$ e $\gamma'(0)=e_1$, definimos uma função $f:\R\to\R$
pondo
\begin{equation*}
f(t) = Z(\xb,\gamma(t)) = 
\Psi(\xb) (1 - \langle \xb, \gamma(t) \rangle) + \langle \nu(\xb), 
\gamma(t) \rangle.
\end{equation*}
Note que, em virtude de \eqref{eq:Prop432-2}, tem-se $f(t)\geq0$,
para todo $t\in\R$.
Calculando a derivada de primeira ordem de $f$, obtemos:
\[
f'(t) = -\langle \Psi(\xb) \xb - \nu(\xb), \gamma'(t) \rangle + 
\langle \nu(\xb), \gamma'(t) \rangle
\]
Como $\gamma$ \'e geodésica, $\gamma'(t)$ \'e o transporte paralelo
de $\gamma'(0)$, logo
\[
\langle\nu(\xb),\gamma'(t)\rangle = \langle\nu(\xb),
\gamma'(0)\rangle = 0.
\]
Assim, $f'(t)$ pode ser escrito como
\begin{equation*}
f'(t) = -\langle \Psi(\xb) \xb - \nu(\xb), \gamma'(t) \rangle.
\end{equation*}
Calculando a derivada segunda de $f$, obtemos:
\begin{equation*}
f''(t) = -\langle \Psi(\xb) \xb - \nu(\xb), \gamma''(t) \rangle.
\end{equation*}
Derivando a igualdade
\[
\langle \nu(\gamma(t)), \gamma'(t) \rangle = 0,
\]
tem-se
\[
\langle D \nu(\gamma(t)) \gamma'(t), \gamma'(t) \rangle + 
\langle \nu(\gamma(t)), \gamma''(t) \rangle = 0,
\]
onde $D$ denota a derivada usual em $\Sp^3$. Como $\gamma''(t)$ 
não tem componentes em $T_{\gamma(t)} \Sigma$, tem-se
\begin{equation*}
-\gamma''(t) = \gamma(t) + h(\gamma'(t), \gamma'(t)) \nu(\xb).
\end{equation*}
Assim, $f''(t)$ se expressa como
\begin{equation*}
f''(t) = \langle \Psi(\xb) \xb - \nu(\xb), \gamma(t) + h(\gamma'(t), 
\gamma'(t)) \nu(\xb) \rangle.
\end{equation*}
Calculando a derivada de ordem $3$ de $f$, obtemos:
\begin{eqnarray}\label{3ra_der_f}
\begin{aligned}
f'''(t) =& \langle \Psi(\xb)\xb - \nu(\xb), \gamma'(t) \rangle + h(\gamma'(t), 
\gamma'(t)) \langle \Psi(\xb)\xb - \nu(\xb), D_{\gamma'(t)} \nu(\gamma(t)) 
\rangle \\
&+ \left( D_{\gamma'(t)}^{\Sigma} h \right) (\gamma'(t), \gamma'(t)) 
\langle \Psi(\xb)\xb - \nu(\xb), \nu(\gamma(t)) \rangle.
\end{aligned}
\end{eqnarray}
Observe que $f(0)=0$. Temos também $f'(0)=0$, pois 
$\Psi(\xb)\xb-\nu(\xb)$ \'e perpendicular a $T_{\xb}\Sigma$. 
Al\'em disso, temos $f''(0)=0$ pois
\[
f''(0) = \Psi(\xb) - h(e_1,e_1)=0.
\]
Provemos agora que $f'''(0)=0$. Usando a expansão de Taylor em $f$,
tem-se:
\begin{equation*}
f(t) = f(0) + f'(0)t + f''(0)t^2 + f'''(0)t^3 + r_3(t),
\end{equation*}  
onde
\[
\lim_{t \rightarrow 0} \frac{r_3(t)}{t_3}=0.
\]
Como $f(t) \geq 0$ e $f(0)=f'(0)=f''(0)=0$ tem-se
\begin{equation} \label{eq:taylor1}
0 \leq f'''(0) t^3 + r_3(t).
\end{equation}
Dividindo por $t^3$ em \eqref{eq:taylor1}, e fazendo $t\to0$, obtemos
\[
0 \leq f'''(0).
\]
Por outro lado, usando $-t$ ao inv\'es de $t$ e aplicando a mesma ideia,
obtemos
\[
f'''(0) \leq 0,
\]
logo $f'''(0)=0$. Fazendo agora $t=0$ na equa\c c\~ao \eqref{3ra_der_f},
tem-se 
\begin{equation} \label{eq:taylor2}
(D_{e_1}^{\Sigma} h) (e_1,e_1) = 0,
\end{equation}
pois $e_1, D_{e_1} \nu(\xb) \in T_p \Sigma$ e $\Psi(\xb)p - \nu(\xb)$ e 
$\nu(\xb)$ são paralelos. Trocando o referencial $\{ e_1, e_2, \nu \}$
por $\{ e_2, e_1, -\nu \}$, e argumentando de maneira an\'aloga, obtemos
\begin{equation} \label{eq:taylor3}
(D_{e_2}^{\Sigma} h) (e_2,e_2)=0.
\end{equation}
Como $h_{11}+h_{22}=0$, segue de \eqref{eq:taylor2} e \eqref{eq:taylor3}
que
\[
(D_{e_2}^{\Sigma} h) (e_1,e_1) = (D_{e_1}^{\Sigma} h) (e_2,e_1) = 0
\]
e
\[
(D_{e_1}^{\Sigma} h) (e_2,e_2) = (D_{e_2}^{\Sigma} h) (e_1,e_2) = 0,
\]
implicando que 
\[
(D_{e_1}^{\Sigma} h) (e_2,e_2) = 0
\]
e
\[
(D_{e_2}^{\Sigma} h) (e_1,e_1) = 0.
\]
Aplicando as equa\c c\~oes de Codazzi nas identidades acima, 
obtemos que $\nabla h = 0$. Disso decorre, em particular, que a
curvatura intrínseca $K$ de $\Sigma$ é constante e, assim, a 
m\'etrica induzida em $\Sigma$ por $F$ \'e flat. Logo, pelo trabalho
de Lawson (cf. \cite[Corollary 3]{Lawson1969}), tem-se que $K\equiv0$ 
ou $K\equiv1$. Como $\Sigma$ não tem pontos umbílicos, segue que
$K\equiv0$ e, portanto, $\Sigma$ é um subconjunto aberto do toro de 
Clifford. A compacidade de $\Sigma$ implica que $F$ é congruente
ao toro de Clifford.
\end{demonstracao}

Iremos a partir de agora finalizar a prova do Teorema 
\ref{teo:Lawson}. Dado um toro m\'inimo e mergulhado 
$F:\Sigma\to\Sp^3$, considere
a express\~ao
\begin{equation} \label{eq:aleph-1}
\aleph = \sup_{\substack{\xb,\yb\in\Sigma\\ \xb\neq\yb}} 
\frac{|\langle\nu(\xb),F(\yb)\rangle|}{\Psi(\xb)(1-\langle F(\xb),F(\yb)\rangle)},
\end{equation}
onde $\Psi$ \'e dada em \eqref{eq:PsiA(x)}. Se $\aleph\leq1$, segue
da Proposi\c c\~ao \ref{prop:torus1} que $F$ \'e congruente ao toro de
Clifford. Assim, resta considerar o caso em que $\aleph > 1$. Trocando
a dire\c c\~ao normal $\nu$ por $-\nu$, caso necess\'ario, podemos
escrever
\begin{equation} \label{eq:aleph-2}
\aleph = \sup_{\substack{\xb,\yb\in\Sigma\\ \xb\neq\yb}} 
\frac{-\langle\nu(\xb),F(\yb)\rangle}{\Psi(\xb)(1-\langle F(\xb),F(\yb)\rangle)}.
\end{equation}
Assim, em virtude de \eqref{eq:aleph-2}, a fun\c c\~ao $Z$, dada por
\begin{equation*}
Z(\xb,\yb) = \aleph\Psi(\xb)(1-\langle F(\xb),F(\yb)\rangle)+
\langle\nu(\xb),F(\yb) \rangle,
\end{equation*}
satisfaz
\begin{equation} \label{eq:aleph-3}
Z(\xb,\yb)\geq0,
\end{equation}
para quaisquer $\xb,\yb\in\Sigma$. Considere agora o conjunto
\begin{equation} \label{eq:ConjOmega}
\Omega = \{\xb\in\Sigma:\mbox{existe} \ \yb\in\Sigma\setminus\{\xb\}
\text{ tal que } Z(\xb,\yb) = 0 \}
\end{equation}


\begin{lema}
O conjunto $\Omega$, dado em \eqref{eq:ConjOmega}, \'e n\~ao-vazio.
\end{lema}

\begin{demonstracao}
	Sejam $x,y \in \Sigma$. Como
	$\Psi(x) (1 - \innerproduct{F(x)}{F(y)})$ é contínua e
	$\Sigma$ é compacto então existe $M > 0$ tal que
	\begin{equation*}
	\Psi(x) (1 - \innerproduct{F(x)}{F(y)}) \leq M.
	\end{equation*}
	Pela definição de $\aleph$, tem-se
	que para qualquer $n \in \N$ existem $x_n,y_n \in \Sigma$, onde $x_n \neq y_n$, tal que
	\begin{equation*}
	\frac{- \innerproduct{\nu(x_n)}{F(y_n)}}{\Psi(x_n) (1 - \innerproduct{F(x_n)}{F(y_n)})} > \aleph - \frac{1}{nM}.
	\end{equation*}
	Desenvolvendo a desigualdade anterior, podemos escrevê-la como:
	\begin{equation}\label{eq:omega-nao-e-vazio-eq-a}
	\frac{1}{n} \geq \frac{\Psi(x_n) (1 - \innerproduct{F(x_n)}{F(y_n)})}{nM} > Z(x_n,y_n) \geq 0.
	\end{equation}
	Como tais $(x_n)$ e $(y_n)$ são sequências limitadas, então existem subsequências convergentes $(x_{n_k})$ e $(y_{n_k})$ tais que
	$x_{n_k} \rightarrow \xb$ e
	$y_{n_k} \rightarrow \yb$,
	onde $\xb, \yb \in \Sigma$.
	Usando as subsequências já mencionadas em \eqref{eq:omega-nao-e-vazio-eq-a} e tomando limite, tem-se que
	\begin{equation*}
	Z(\xb,\yb) = 0.
	\end{equation*}
	Se $\xb = \yb$, então
	\begin{equation*}
	Z(\xb,\yb) = Z(\xb, \xb) = \innerproduct{\nu(\xb)}{F(\xb)} = 0.
	\end{equation*}
	Isso implica que 
	$F(\xb) \in T_{\xb} \Sigma$,
	então $F(\xb)$ pode-se expressar como combinação linear dos $\frac{\partial F}{\partial x_i}(\xb)$, logo
	\begin{equation}\label{eq:omega-nao-e-vazio-eq-b}
	F(\xb) = a^i \frac{\partial F}{\partial x_i}(\xb).
	\end{equation}
	Lembrando que
	$\innerproduct{F(\xb)}{\frac{\partial F}{\partial x_j}(\xb)} = 0$
	e tomando o produto interno em \eqref{eq:omega-nao-e-vazio-eq-b}, obtêm-se
	\begin{equation*}
	0 = \innerproduct{F(\xb)}{\frac{\partial F}{\partial x_j}(\xb)} = a^i \delta_{ij}.
	\end{equation*}
	Então
	$a^i = 0$,
	o que implica que
	$F(\xb) = 0$, o que é uma contradição.
	Portanto
	$\xb \neq \yb$ e
	$\xb \in \Omega$.
\end{demonstracao}

O pr\'oximo passo agora \'e provar que o conjunto $\Omega$,
dado em \eqref{eq:ConjOmega}, \'e aberto. Para isso, faremos
uso de dois resultados. O primeiro deles \'e o {\em princ\'ipio
do m\'aximo estrito de Bony} para equa\c c\~oes el\'ipticas 
n\~ao-degeneradas (cf. \cite[Corollary 9.7]{Brendle2010}).

\begin{teorema} \label{teo:bony}
Dados um aberto $U\subset\R^n$ e campos vetoriais 
$X_1,\ldots,X_m\in\mathfrak{X}(U)$, considere uma fun\c c\~ao
diferenci\'avel n\~ao-negativa $\varphi:U\to\R$ satisfazendo
\begin{equation}\label{eq:bony}
\sum_{j=1}^{m} (D^2 \varphi)(X_j,X_j) \leq -L \inf_{|\xi| \leq 1} 
(D^2 \varphi)(\xi,\xi) + L |d \varphi| + L \varphi,
\end{equation}
onde $L$ é uma constante positiva. Seja $F=\{x\in U:\varphi(x)=0\}$
o conjunto dos zeros de $\varphi$ e suponha que $\gamma:[0,1]\to U$
seja uma curva diferenciável tal que $\gamma(0)\in F$ e 
$\gamma'(s) = \displaystyle\sum_{j=1}^{m} f_j(s) X_j(\gamma(s))$, para certas
funções diferenciáveis $f_1,\ldots,f_m:[0,1]\to\R$. Então 
$\gamma(s)\in F$, para todo $s\in[0,1]$.
Este teorema também é válido para uma variedade Riemanniana em vez de $\R^n$.
\end{teorema}

O seguinte resultado \'e uma estimativa obtida fazendo-se uma
adapta\c c\~ao na demonstra\c c\~ao da Proposi\c c\~ao
\ref{prop:somaDerSeg}.

Considere dois pontos $\xb,\yb\in\Sigma$, com $\xb\neq\yb$. Seja
$(x_1,x_2)$ um sistema de coordenadas normais geod\'esicas em
torno de $\xb$ tal que
\[
h_{11}(\xb)=\lambda_1, \quad \lambda_{12}(\xb)=0
\quad\mbox{e}\quad h_{22}(\xb)=\lambda_2. 
\]
Al\'em disso, seja $(y_1,y_2)$ um sistema de coordenadas normais 
geod\'esicas em torno de $\yb$ tal que
\begin{equation*}
\left\langle w_1, \frac{\partial F}{\partial y_1}(\yb) \right\rangle \geq 0, 
\quad 
\left\langle w_1, \frac{\partial F}{\partial y_2}(\yb) \right\rangle = 0 
\quad \text{e} \quad 
\left\langle w_2, \frac{\partial F}{\partial y_2}(\yb) \right\rangle \geq 0,
\end{equation*}
onde $w_1$ e $w_2$ s\~ao dados em \eqref{eq:reflexao}. Adaptando
a demonstra\c c\~ao do Lema \ref{lem:w_1 w_2}, podemos concluir
que
\begin{equation}\label{eq:Gamma-ineq}
	\sum_{i=1}^2\left| w_i-\frac{\partial F}{\partial y_i}(\yb)\right| \leq
	\Lambda(\xb,\yb)\left(Z(\xb,\yb)+\sum_{i=1}^2
	\left|\frac{\partial Z}{\partial y_i}(\xb,\yb)\right|\right),
\end{equation}
onde $\Lambda$ \'e uma fun\c c\~ao cont\'inua sobre o conjunto
\[
\{(\xb,\yb)\in\Sigma\times\Sigma:\xb\neq\yb\}
\]
que, eventualmente, deixa de ser limitada numa vizinhan\c ca
da diagonal de $\Sigma\times\Sigma$. 

\begin{lema}
Dados dois pontos $\xb,\yb\in\Sigma$, com $\xb\neq\yb$, tem-se:
\begin{equation} \label{eq:lema2Omegaopen}
\begin{aligned}
\sum_{i=1}^2 \frac{\partial^2 Z}{\partial x_i^2}(\xb,\yb) \ + \ & 
2 \sum_{i=1}^2 \frac{\partial^2 Z}{\partial x_i \partial y_i}(\xb,\yb) \ + \ 
\sum_{i=1}^2 \frac{\partial^2 Z}{\partial y_i^2}(\xb,\yb)  \\
& \leq
- \frac{\aleph^2 - 1}{\aleph} \frac{\Psi(\xb)}{1 - \langle F(\xb), F(\yb) \rangle} \sum_{i=1}^2 \left\langle \frac{\partial F}{\partial x_i}(\xb), F(\yb) \right\rangle^2 \\
& + 
\tilde{\Lambda}(\xb,\yb) \left( Z(\xb,\yb) + \sum_{i=1}^2 \left| \frac{\partial Z}{\partial x_i}(\xb,\yb) \right| + \sum_{i=1}^2 \left| \frac{\partial Z}{\partial y_i}(\xb,\yb) \right| \right)
\end{aligned}
\end{equation}
onde $\tilde{\Lambda}(\xb,\yb)$ \'e uma função cont\'inua no conjunto
$\{(\xb,\yb)\in\Sigma\times\Sigma:\xb\neq\yb \}$, a qual pode ser não
limitada numa vizinhan\c ca da diagonal de $\Sigma\times\Sigma$.
\end{lema}
\begin{demonstracao}
Em virtude das Proposi\c c\~oes \ref{prop:somaDerSeg} e 
\ref{edp_principal}, podemos escrever
\begin{equation*}
\begin{aligned}
&\sum_{i=1}^2\frac{\partial^2 Z}{\partial x_i^2}(\xb,\yb) \ + \ 
2 \sum_{i=1}^2 \frac{\partial^2 Z}{\partial x_i \partial y_i}(\xb,\yb) \ + \ 
\sum_{i=1}^2 \frac{\partial^2 Z}{\partial y_i^2}(\xb,\yb)  \\
& =
- \frac{\aleph^2 - 1}{\aleph} \frac{\Psi(\xb)}{1 - \langle F(\xb), F(\yb) \rangle} \sum_{i=1}^2 \left\langle \frac{\partial F}{\partial x_i}(\xb), F(\yb) \right\rangle^2 
+ 4\aleph\Psi(\xb) - (|A(\xb)|^2+2)Z(\xb,\yb) \\ 
& 
+2\sum_{i=1}^2(\lambda_i-\aleph\Psi(\xb))\left\langle w_i,
\frac{\partial F}{\partial y_i}(\yb)\right\rangle - 
\frac{2}{1 - \langle F(\xb), F(\yb) \rangle}\sum_{i=1}^2
\frac{\partial Z}{\partial x_i}(\xb,\yb)\left\langle F(\xb),
\frac{\partial F}{\partial y_i}(\yb)\right\rangle \\
& +
\frac{1}{\aleph\Psi(\xb)(1 - \langle F(\xb), F(\yb) \rangle)}\sum_{i=1}^2
\left[\left(\frac{\partial Z}{\partial x_i}(\xb,\yb)\right)^2
-2\lambda_i\left\langle\frac{\partial F}{\partial x_i}(\xb),F(\yb)\right\rangle
\frac{\partial Z}{\partial x_i}(\xb,\yb)\right].
\end{aligned}
\end{equation*}
Como 
\[
Z(\xb,\yb)\geq0 \quad\mbox{e}\quad
|F(\xb)| = \left|\frac{\partial F}{\partial y_i}(\yb) \right|=1,
\]
podemos escrever
\begin{equation*}
\begin{aligned}
&\sum_{i=1}^2\frac{\partial^2 Z}{\partial x_i^2}(\xb,\yb) \ + \ 
2 \sum_{i=1}^2 \frac{\partial^2 Z}{\partial x_i \partial y_i}(\xb,\yb) \ + \ 
\sum_{i=1}^2 \frac{\partial^2 Z}{\partial y_i^2}(\xb,\yb)  \\
& =
- \frac{\aleph^2 - 1}{\aleph} \frac{\Psi(\xb)}{1 - \langle F(\xb), F(\yb) \rangle} \sum_{i=1}^2 \left\langle \frac{\partial F}{\partial x_i}(\xb), F(\yb) \right\rangle^2
+ 4\aleph\Psi(\xb) \\ 
& 
+2\sum_{i=1}^2(\lambda_i-\aleph\Psi(\xb))\left\langle w_i,
\frac{\partial F}{\partial y_i}(\yb)\right\rangle +
\frac{2}{1 - \langle F(\xb), F(\yb) \rangle}\sum_{i=1}^2
\left|\frac{\partial Z}{\partial x_i}(\xb,\yb)\right| \\
& +
\frac{1}{\aleph\Psi(\xb)(1 - \langle F(\xb), F(\yb) \rangle)}\sum_{i=1}^2
\left[\left(\frac{\partial Z}{\partial x_i}(\xb,\yb)\right)^2
-2\lambda_i\left\langle\frac{\partial F}{\partial x_i}(\xb),F(\yb)\right\rangle
\frac{\partial Z}{\partial x_i}(\xb,\yb)\right].
\end{aligned}
\end{equation*}
Estimemos, inicialmente, o termo
\[
\frac{\partial Z}{\partial x_i}(\xb,\yb)\left[
\frac{\partial Z}{\partial x_i}(\xb,\yb)
-2\lambda_i\left\langle\frac{\partial F}{\partial x_i}(\xb),F(\yb)\right\rangle
\right].
\]
Como
\[
\frac{\partial Z}{\partial x_i}(\xb,\yb) = \aleph\frac{\partial\Psi}{\partial x_i}(\xb)
(1 - \langle F(\xb), F(\yb) \rangle) +
(\lambda_i-\aleph\Psi(\xb))\left\langle\frac{\partial F}{\partial x_i}(\xb),F(\yb)
\right\rangle,
\]
temos
\begin{equation*}
\begin{aligned}
&
\frac{\partial Z}{\partial x_i}(\xb,\yb)\left[
\frac{\partial Z}{\partial x_i}(\xb,\yb)
-2\lambda_i\left\langle\frac{\partial F}{\partial x_i}(\xb),F(\yb)\right\rangle
\right] \\
&= 
\frac{\partial Z}{\partial x_i}(\xb,\yb)\left[
\aleph\frac{\partial\Psi}{\partial x_i}(\xb)
(1 - \langle F(\xb), F(\yb) \rangle) +
(\lambda_i-\aleph\Psi(\xb))\left\langle\frac{\partial F}{\partial x_i}(\xb),F(\yb)
\right\rangle\right] \\
&\leq
\left|\frac{\partial Z}{\partial x_i}(\xb,\yb)\right|\aleph M
(1 - \langle F(\xb), F(\yb) \rangle) +
\left|\frac{\partial Z}{\partial x_i}(\xb,\yb)\right| 2\aleph\Psi(\xb),
\end{aligned}
\end{equation*}
onde
\[
M = \sup_{\substack{\xb\in\Sigma\\ i=1,2}}
\left|\frac{\partial\Psi}{\partial x_i}(\xb)\right|
\]
e notando que $\lambda_i\leq\aleph\Psi(\xb)$, com $i=1,2$. Assim,
\begin{equation*}
\begin{aligned}
&\sum_{i=1}^2\frac{\partial^2 Z}{\partial x_i^2}(\xb,\yb) \ + \ 
2 \sum_{i=1}^2 \frac{\partial^2 Z}{\partial x_i \partial y_i}(\xb,\yb) \ + \ 
\sum_{i=1}^2 \frac{\partial^2 Z}{\partial y_i^2}(\xb,\yb)  \\
& \leq
- \frac{\aleph^2 - 1}{\aleph} \frac{\Psi(\xb)}{1 - \langle F(\xb), F(\yb) \rangle} \sum_{i=1}^2 \left\langle \frac{\partial F}{\partial x_i}(\xb), F(\yb) \right\rangle^2
+ 4\aleph\Psi(\xb) \\ 
& +
2\sum_{i=1}^2(\lambda_i-\aleph\Psi(\xb))\left\langle w_i,
\frac{\partial F}{\partial y_i}(\yb)\right\rangle +
\left(\frac{4}{1 - \langle F(\xb), F(\yb) \rangle} + \frac{M}{\Psi(\xb)}\right)
\sum_{i=1}^2\left|\frac{\partial Z}{\partial x_i}(\xb,\yb)\right|.
\end{aligned}
\end{equation*}
Estimemos agora o termo
\[
2\sum_{i=1}^2(\lambda_i-\aleph\Psi(\xb))\left\langle w_i,
\frac{\partial F}{\partial y_i}(\yb)\right\rangle.
\]
Desenvolvendo, obtemos:
\begin{equation*}
\begin{aligned}
\left| w_i-\frac{\partial F}{\partial y_i}(\yb)\right|^2 &=
|w_i|^2 + \left|\frac{\partial F}{\partial y_i}(\yb)\right|^2 -
2\left\langle w_i,\frac{\partial F}{\partial y_i}(\yb)\right\rangle \\
&=
2 - 2\left\langle w_i,\frac{\partial F}{\partial y_i}(\yb)\right\rangle.
\end{aligned}
\end{equation*}
Assim, usando a desigualdade \eqref{eq:Gamma-ineq}, obtemos:
\begin{equation*}
\begin{aligned}
2&\sum_{i=1}^2(\lambda_i-\aleph\Psi(\xb))\left\langle w_i,
\frac{\partial F}{\partial y_i}(\yb)\right\rangle \\
&=
\sum_{i=1}^2\lambda_i\left(2-\left| w_i-\frac{\partial F}{\partial y_i}(\yb)
\right|^2\right) +
\aleph\Psi(\xb)\sum_{i=1}^2\left(
\left| w_i-\frac{\partial F}{\partial y_i}(\yb)\right|^2-2\right) \\
&\leq 
\Psi(\xb)\sum_{i=1}^2\left| w_i-\frac{\partial F}{\partial y_i}(\yb)\right|^2 +
\aleph\Psi(\xb)\sum_{i=1}^2
\left| w_i-\frac{\partial F}{\partial y_i}(\yb)\right|^2 
-4\aleph\Psi(\xb) \\
&\leq
2\Psi(\xb)(\aleph+1)\sum_{i=1}^2
\left| w_i-\frac{\partial F}{\partial y_i}(\yb)\right| -4\aleph\Psi(\xb) \\
& \leq
2\Psi(\xb)(\aleph+1)\Lambda(\xb,\yb)\left(Z(\xb,\yb)
+\sum_{i=1}^2\left|\frac{\partial Z}{\partial x_i}(\xb,\yb)\right|\right)
-4\aleph\Psi(\xb).
\end{aligned}
\end{equation*}
Portanto,
\begin{equation*}
\begin{aligned}
&\sum_{i=1}^2\frac{\partial^2 Z}{\partial x_i^2}(\xb,\yb) \ + \ 
2 \sum_{i=1}^2 \frac{\partial^2 Z}{\partial x_i \partial y_i}(\xb,\yb) \ + \ 
\sum_{i=1}^2 \frac{\partial^2 Z}{\partial y_i^2}(\xb,\yb)  \\
& \leq
- \frac{\aleph^2 - 1}{\aleph} \frac{\Psi(\xb)}{1 - \langle F(\xb), F(\yb) \rangle} \sum_{i=1}^2 \left\langle \frac{\partial F}{\partial x_i}(\xb), F(\yb) \right\rangle^2 \\
&+
2\Psi(\xb)(\aleph+1)\Lambda(\xb,\yb)\left(Z(\xb,\yb)
+\sum_{i=1}^2\left|\frac{\partial Z}{\partial x_i}(\xb,\yb)\right|\right) \\
&+\left(\frac{4}{1 - \langle F(\xb), F(\yb) \rangle} + \frac{M}{\Psi(\xb)}\right)
\sum_{i=1}^2\left|\frac{\partial Z}{\partial x_i}(\xb,\yb)\right|,
\end{aligned}
\end{equation*}
e, assim, conclu\'imos que
\begin{equation*}
\begin{aligned}
&\sum_{i=1}^2\frac{\partial^2 Z}{\partial x_i^2}(\xb,\yb) \ + \ 
2 \sum_{i=1}^2 \frac{\partial^2 Z}{\partial x_i \partial y_i}(\xb,\yb) \ + \ 
\sum_{i=1}^2 \frac{\partial^2 Z}{\partial y_i^2}(\xb,\yb)  \\
& \leq
- \frac{\aleph^2 - 1}{\aleph} \frac{\Psi(\xb)}{1 - \langle F(\xb), F(\yb) \rangle} \sum_{i=1}^2 \left\langle \frac{\partial F}{\partial x_i}(\xb), F(\yb) \right\rangle^2 \\
&+
\left(2\Psi(\xb)(\aleph+1)\Lambda(\xb,\yb) +
\frac{4}{1 - \langle F(\xb), F(\yb) \rangle} + \frac{M}{\Psi(\xb)}\right) \\
&\cdot\left( Z(\xb,\yb) + \sum_{i=1}^2 \left| \frac{\partial Z}{\partial x_i}(\xb,\yb) \right| + \sum_{i=1}^2 \left| \frac{\partial Z}{\partial y_i}(\xb,\yb) \right| \right).
\end{aligned}
\end{equation*}
Basta considerar ent\~ao
\[
\tilde\Lambda(\xb,\yb) =
\left(2\Psi(\xb)(\aleph+1)\Lambda(\xb,\yb) +
\frac{4}{1 - \langle F(\xb), F(\yb) \rangle} + \frac{M}{\Psi(\xb)}\right),
\]
o que prova \eqref{eq:lema2Omegaopen}.
\end{demonstracao}


\begin{lema}
O conjunto $\Omega$, dado em \eqref{eq:ConjOmega}, \'e aberto.
\end{lema}

\begin{demonstracao}
Dado $x \in \Omega$,	definimos o conjunto
\[ Y_x = \{ y \in \Sigma: Z(x,y)=0 \}. \]
Com a notação do Teorema \ref{teo:bony},  identifiquemos a função $\varphi$ com $Z$, e o conjunto $F$ estará definido por
\[F = \bigcup_{x \in \Omega} \{x\} \times Y_x.  \]
Em cada ponto $(x,y) \in F$, $Z$ atinge o seu mínimo. Logo, a derivada segunda da função $Z$ naqueles pontos serão operadores bilineares definidos positivos. Pela continuidade da derivada segunda, para cada ponto $(x,y) \in F$ existe uma vizinhança $V_{(x,y)}$ tal que a derivada segunda avaliada nos pontos da vizinhança segue sendo um operador bilinear definido positivo. 
Considere o aberto
\[ U = \bigcup_{(x,y) \in F} V_{(x,y)} \]
e dois campos vetoriais $X_1$ e $X_2$ definidos por:
\[ X_1 = \left[ \begin{matrix}
1\\
0\\
1\\
0
\end{matrix} \right] \quad \text{e} \quad X_2 = \left[ \begin{matrix}
0\\
1\\
0\\
1
\end{matrix} \right]. \]
Calculando o valor da expressão em \eqref{eq:bony}, obtemos:
\[ \sum_{i=1}^{2} D^2 Z (X_i,X_i) = \sum_{i=1}^{2} \frac{\partial^2 Z}{\partial x_i^2} + 2 \sum_{i=1}^{2} \frac{\partial^2 Z}{\partial x_i \partial y_i} + \sum_{i=1}^{2} \frac{\partial^2 Z}{\partial y_i^2}. \] 
No aberto $U$, a derivada segunda de $Z$ é definida positiva, i.e., $0 < D^2 Z(\xi,\xi)$, para qualquer campo vetorial $\xi$ em $U$. Logo, obtemos que
\[ 0 \leq \inf_{|\xi| \leq 1} D^2 Z(\xi,\xi) \leq D^2 Z (0,0) = 0. \]
Portanto $\inf_{|\xi| \leq 1} D^2 Z(\xi,\xi) = 0$.
Como estamos no caso $\aleph > 1$, tem-se
\[ - \frac{\aleph^2 -1}{\aleph} \frac{\Psi(x)}{1 - \innerproduct{F(x)}{F(y)}} \sum_{i=1}^{2} \innerproduct{\frac{\partial F}{\partial x_i}(x)}{F(y)}^2 \leq 0. \]
Podemos construir $U$ de tal forma que esteja contido no complemento da vizinhança da diagonal onde a função $\overline{\Lambda}$ é ilimitada, logo $\overline{\Lambda}$ é limitada em $U$.
Estamos, assim, nas condições do Teorema \ref{teo:bony} e, portanto, podemos chegar à conclusão que para uma geodésica $\gamma: [0,1] \rightarrow U$ tal que $\gamma(0) \in U$ e $\gamma'(s) = f_1(s) X_1(\gamma(s)) + f_2(s) X_2(\gamma(s))$ para quaisquer funções $f_1,f_2: [0,1] \rightarrow \R$ tem-se que $\gamma(s) \in F$.
Aplicando a projeção sobre a primeira variável no conjunto $F$ obtemos o conjunto $\Omega$ que, pela escolha de $X_1$ e $X_2$, conclui-se que é aberto porque é uma bola geodésica. 
\end{demonstracao}


Considere agora dois pontos $\xb,\yb\in\Sigma$, com $\xb\neq\yb$,
tais que
\[
Z(\xb,\yb) = \frac{\partial Z}{\partial x_i}(\xb,\yb) 
= \frac{\partial Z}{\partial y_i}(\xb,\yb) = 0.
\]
Usando a Proposi\c c\~ao \ref{edp_principal}, e equação
\eqref{suma_2das_derivadas} pode ser escrita como
\begin{equation} \label{suma_2da_der_aleph}
\begin{aligned}
\sum_{i=1}^2 \frac{\partial^2 Z}{\partial x_i^2}(\xb,\yb) + & 
2 \sum_{i=1}^2 \frac{\partial^2 Z}{\partial x_i \partial y_i}(\xb,\yb) + 
\sum_{i=1}^2 \frac{\partial^2 Z}{\partial y_i^2}(\xb,\yb)  \\
& = - 
\frac{\aleph^2 - 1}{\aleph} \frac{\Psi(\xb)}{1 - \langle F(\xb), 
F(\yb) \rangle} \sum_{i=1}^2 \left\langle \frac{\partial F}
{\partial x_i}(\xb), F(\yb) \right\rangle^2.
\end{aligned}
\end{equation}



\begin{proposicao} \label{gradiente_nulo}
Para todo ponto $\xb\in\Omega$, tem-se
\[
\nabla \Psi(\xb)=0.
\] 
\end{proposicao}
\begin{demonstracao}
Fixemos um ponto arbitr\'ario $\xb\in\Omega$. Pela defini\c c\~ao
de $\Omega$, existe um ponto $\yb\in\Sigma\setminus\{x\}$ tal que 
$Z(\xb,\yb)=0$. Como a fun\c c\~ao $Z$ atinge seu m\'inimo
global no ponto $(\xb,\yb)$, a igualdade em 
\eqref{suma_2da_der_aleph} torna-se
\begin{equation*} 
\begin{aligned}
0 & \leq
\sum_{i=1}^2 \frac{\partial^2 Z}{\partial x_i^2}(\xb,\yb) + 
2 \sum_{i=1}^2 \frac{\partial^2 Z}{\partial x_i \partial y_i}(\xb,\yb) + 
\sum_{i=1}^2 \frac{\partial^2 Z}{\partial y_i^2}(\xb,\yb)  \\
& = - 
\frac{\aleph^2 - 1}{\aleph} \frac{\Psi(\xb)}{1 - \langle F(\xb), 
F(\yb) \rangle} \sum_{i=1}^2 \left\langle \frac{\partial F}
{\partial x_i}(\xb), F(\yb) \right\rangle^2 \leq 0.
\end{aligned}
\end{equation*}
Como estamos supondo $\aleph>1$, conclu\'imos que
\[
\left\langle \frac{\partial F}{\partial x_i}(\xb), F(\yb) \right\rangle = 0,
\]
para cada $i=1,2$. Usando agora a equação \eqref{diff_Z_x},
conclu\'imos que 
\[
0 = \frac{\partial Z}{\partial x_i}(\xb,\yb) = 
\aleph \frac{\partial \Psi}{\partial x_i}(\xb)
(1 - \langle F(\xb),F(\yb) \rangle),
\]
provando que $\nabla\Psi(\xb)=0$, para todo $\xb\in\Omega$, como
quer\'iamos.
\end{demonstracao}

Finalizemos agora a prova do Teorema \ref{teo:Lawson}. Como
$\Omega$ \'e aberto, segue da Proposi\c c\~ao \ref{gradiente_nulo}
que
\[
\Delta_\Sigma\Psi(\xb)=0,
\]
para todo ponto $\xb\in\Omega$. Assim, a Proposi\c c\~ao 
\ref{edp_principal} implica que $\Psi(\xb)=1$, para todo
$\xb\in\Omega$. Usando o teorema de extens\~ao \'unica
para solu\c c\~oes de equa\c c\~oes diferencias parciais
el\'ipticas (cf., por exemplo, \cite{Aronszajn1957}), conclu\'imos
que $\Psi(\xb)=1$, para todo ponto $\xb\in\Sigma$. Como
consequ\^encia, a curvatura Gaussiana de $\Sigma$ \'e
identicamente nula. Como anteriormente, segue do trabalho
de Lawson \cite{Lawson1969} que $F$ \'e congruente ao
toro de Clifford, e isso finaliza a demonstra\c c\~ao do
Teorema \ref{teo:Lawson}.