%\cite{Brendle2013}
%\cite{Brendle2013a}

%\begin{proposicao}\label{nao_tem_pontos_umbilicos}
%	Um superfície mínima imersa em $S^3$ de gênero 1 não tem pontos umbílicos, i.e., a segunda forma fundamental é não nula em cada ponto da superfície
%\end{proposicao}

%\begin{proposicao}
%	Supor que $F: M \rightarrow S^3$ é um toro mínimo mergulhado em $S^3$. Então a norma da segunda forma fundamental satisfaz a equação em derivadas parciais
%	\begin{equation*}
%		\Delta_M (|A|) - \frac{| \nabla |A| |^2}{|A|} + (|A|^2 - 2) |A| = 0.
%	\end{equation*}
%	onde $A$ é a segunda forma fundamental.
%\end{proposicao}

%Neste capítulo apresentaremos a demonstração da conjetura de Lawson obtida por Simon Brendle \cite{Brendle2013a}.
%
%\section{Introdução}
%
%Em 1970, Blaine Lawson \cite{Lawson1970a} conjeturou que o toro de Clifford é a única superfície mínima, compacta e mergulhada em $S^3$, com genus 1, e foi provada somente quatro décadas depois por Simon Brendle \cite{Brendle2013a} em 2013. A hipótese de ser mergulhada é fundamental. De fato, em \cite{Lawson1970} Lawson construiu uma família (infinita) de imersões mínimas de toros em $S^3$.
%
%A prova da conjetura de Lawson em \cite{Brendle2013a} envolve uma aplicação do princípio do máximo a uma função que depende de uma par de pontos. Esta técnica foi inicialmente desenvolvida por Huisken \cite{Huisken1998} no estudo de fluxo de comprimento de curvas para curvas mergulhadas no plano e, posteriormente, por Andrews \cite{Andrews2012}.
%
%Descreveremos a seguir os argumentos usados em \cite{Brendle2013a}. Seguindo as ideias e notações de \cite{Andrews2012}, considere uma superfície mínima e mergulhada $ F: \Sigma \rightarrow S^3 $ e $ \Phi: \Sigma \rightarrow \realnumbers $  uma função positiva. Definia uma função $ Z: \Sigma \times \Sigma \rightarrow \realnumbers $ pondo
%\begin{equation}\label{def_de_Z}
%	Z(x,y) = \Phi(x) (1 - \langle F(x),F(y) \rangle ) + \langle \nu(x), F(y) \rangle,
%\end{equation}
%onde $ \nu $ é um campo unitário , normal a $ \Sigma $.
%
%A função $Z$ tem a seguinte interpretação geométrica:
%
%\begin{proposicao}
%	A função $ Z(x,y) $, definida em \ref{def_de_Z}, é não-negativa se, e somente se, para todo ponto $ x \in \Sigma $, existe uma bola geodésica $B$ contida em $\Omega$, com curvatura média do bordo igual a $ \Phi(x) $ e $ F(x) \in \partial B $.
%\end{proposicao}
%
%\begin{demonstracao}
%	Observe, inicialmente, que uma bola geodésica em $S^3$ é simplesmente a interseção de uma bola de $\realnumbers^4$ com $S^3$. Em particular, a bola em $S^3$ contida em $\Omega$, com curvatura média do bordo igual a $\Phi$ e que é tangente a $ F(\Sigma) $ no ponto $F(x)$ é $B=B(p)$, onde 
%	\begin{equation*}
%		p = F(x) - \Phi^{-1}(x) \nu(x),
%	\end{equation*}
%	e $\nu$ é o campo unitário, normal a $F(\Sigma)$ no ponto $F(x) \in S^3$, e que aponta para fora de $\Omega$.
%	
%	Afirmar que uma bola geodésica de $S^3$ está inteiramente contida em $\Omega$ é equivalente a afirmar que, para cada ponto $y \in \Sigma$, tem-se
%	\begin{equation*}
%		| F(y) - p | \geq \Phi^{-2}(x),
%	\end{equation*}
%	ou seja,
%	\begin{equation*}
%		| F(y) - ( F(x) - \Phi^{-1}(x) \nu(x) ) |^2 - \Phi^2(x) \geq 0.
%	\end{equation*}
%	
%	Desenvolvendo o lado esquerdo da desigualdade acima e multiplicando por $ \Phi(x)/2 $, obtemos:
%	\begin{equation*}
%		\frac{\Phi(x)}{2} | F(y) - F(x) |^2 + \langle F(y) - F(x), \nu(x) \rangle \geq 0.
%	\end{equation*}
%	
%	Como $F(x), F(y) \in S^3$, tem-se
%	\begin{equation*}
%		|F(x)|^2 = |F(y)|^2 = 1 \quad \text{e} \quad \langle F(x), \nu(x) \rangle = 0.
%	\end{equation*}
%	
%	Asim,
%	\begin{equation*}
%		\Phi(x) (1 - \langle F(x), F(y) \rangle) + \langle \nu(x), F(y) \rangle \geq 0,
%	\end{equation*} 
%	como queríamos.
%\end{demonstracao}


Neste capítulo apresentaremos a demonstração da conjetura de
Lawson obtida por Simon Brendle \cite{Brendle2013a}.


\section{Introdu\c c\~ao}

Em 1970, Blaine Lawson \cite{Lawson1970a} conjeturou que o toro de 
Clifford é a única superfície mínima, compacta e mergulhada em $\Sp^3$,
com genus $1$, e foi provada somente quatro décadas depois por Simon
Brendle \cite{Brendle2013a} em 2013. A hipótese de ser mergulhada é
fundamental. De fato, em \cite{Lawson1970} Lawson construiu uma 
família (infinita) de imersões mínimas de toros em $\Sp^3$.

A prova da conjetura de Lawson em \cite{Brendle2013a} envolve uma 
aplicação do princípio do máximo para uma função que depende de 
duas vari\'aveis. Esta técnica foi inicialmente desenvolvida por Huisken
\cite{Huisken1998} no estudo do fluxo de comprimento de curvas para 
curvas mergulhadas no plano e, posteriormente, por Andrews 
\cite{Andrews2012}.

Antes de apresentarmos os argumentos usados em \cite{Brendle2013a},
na prova da conjectura de Lawson, faremos algumas consira\c c\~oes
iniciais. Seguiremos aqui o artigo \cite{Andrews2012}, onde o autor
apresenta a no\c c\~ao de n\~ao-colapsante para hipersuperf\'icies
Euclidianas mergulhadas.

Considere uma superf\'icie $\Sigma$ em $\R^3$, cuja curvatura
m\'edia $H$ \'e positiva em todos os pontos, e limitando uma 
regi\~ao aberta $\Omega$ em $\R^3$.

\begin{definicao}
	A superf\'icie $\Sigma$ \'e dita ser {\em $\delta$-n\~ao-colapsante}
	se, para todo ponto $x\in\Sigma$, existe uma bola aberta $B$ de
	raio $\delta/H(x)$, contida em $\Omega$, com $x\in\partial B$. 
\end{definicao}

Dado uma superf\'icie $F:\Sigma\to\R^3$, defina uma fun\c c\~ao
$Z:\Sigma\times\Sigma\to\R$ pondo
\[
Z(x,y) = \frac{H(x)}{2}\|F(y)-F(x)\|^2 + \delta\langle F(y)-F(x), \nu(x)\rangle,
\]
onde $\nu$ \'e um campo unit\'ario, normal a $\Sigma$.

\begin{proposicao}[\cite{Andrews2012}]\label{prop:Int}
	A superf\'icie $\Sigma$ \'e $\delta$-n\~ao-colapsante se, e somente
	se, $Z(x,y)\geq0$, para quaisquer $x,y\in\Sigma$.
\end{proposicao}
\begin{proof}
	Sem perda de generalidade, escolha o campo normal $\nu$ que
	aponta para fora de $\Sigma$. Assim, uma bola em $\Omega$ de
	raio $\delta/H(x)$, com $F(x)$ sendo um ponto da fronteira, deve
	ter centro no ponto
	\[
	p(x) = F(x) - (\delta/H(x))\nu(x).
	\]
	A afirma\c c\~ao de que esta bola est\'a contida em $\Omega$
	\'e equivalente ao fato de que nenhum ponto de $\Sigma$ tem
	dist\^ancia menor do que $\delta/H(x)$ ao ponto $p$. Ou seja,
	\[
	0\leq\|F(y)-p(x)\|^2-\left(\frac{\delta}{H(x)}\right)^2 = 
	2\cdot\frac{Z(x,y)}{H(x)},
	\]
	para quaisquer $x,y\in\Sigma$. Como $H>0$ em todos os pontos
	de $\Sigma$, isso \'e equivalente ao fato de que $Z$ seja
	n\~ao-negativa em todos os pontos. A rec\'iprova \'e imediata.
\end{proof}

Considere agora uma superf\'icie m\'inima e mergulhada 
$F:\Sigma\to\Sp^3$ e $ \Phi: \Sigma\to\R$ uma função positiva. Defina
uma função $Z:\Sigma\times\Sigma\to\R$ pondo
\begin{equation}\label{def_de_Z}
Z(x,y) = \Phi(x) (1 - \langle F(x),F(y) \rangle ) + \langle \nu(x), F(y) \rangle,
\end{equation}
onde $\nu$ \'e um campo unitário, normal a $\Sigma$. Observando que
uma bola geod\'esica na esfera $\Sp^3$ \'e simplesmente a interseção
de uma bola de $\R^4$ com $\Sp^3$ podemos provar, de forma
an\'aloga \`a Proposi\c c\~ao \ref{prop:Int}, que a fun\c c\~ao $Z(x,y)$,
definida em \eqref{def_de_Z}, \'e n\~ao-negativa se, e somente se,
a superf\'icie $\Sigma$ \'e n\~ao-colapsante.



\section{Alguns resultados técnicos}

Nesta se\c c\~ao apresentaremos alguns resultados preliminares
que ser\~ao usados na demonstra\c c\~ao da conjectura de
Lawson. 

\vspace{.2cm}

Dados uma superfície mínima e mergulhada $F:\Sigma\to\Sp^3$ 
e uma função positiva $\Phi: \Sigma\to\R$, considere a função 
$Z:\Sigma\times\Sigma\to\R$ dada por
\begin{eqnarray}\label{eq:funcaoZ}
Z(x,y) = \Phi(x) (1 - \langle F(x), F(y) \rangle) + \langle \nu(x), F(y) \rangle,
\end{eqnarray}
onde $\nu$ \'e um campo unitário, normal a $\Sigma$. Considere dois 
pontos $\xbarra, \ybarra \in \Sigma$, com $\xbarra \neq \ybarra$, tais que $Z(\xbarra, \ybarra) = 0$ e $dZ(\xbarra, \ybarra) = 0$. Sejam $(x_1,x_2), (y_1,y_2)$ sistemas de coordenadas geodésicas em torno dos pontos $\xbarra, \ybarra$, respectivamente. No ponto $(\xbarra,\ybarra)$, temos:
\begin{eqnarray}\label{diff_Z_x}
\begin{aligned}
0 = \frac{\partial Z}{\partial x_i} (\xbarra, \ybarra) = &
\frac{\partial \Phi}{\partial x_i}(\xbarra) (1 - \langle F(\xbarra, F(\ybarra) \rangle) -  
\Phi(\xbarra) \left\langle \frac{\partial F}{\partial x_i}(\xbarra), F(\ybarra) \right\rangle \\ & + \sum_{k=1}^{2} h_{ik}(\xbarra) \left\langle \frac{\partial F}{\partial x_k}(\xbarra), F(\ybarra) \right\rangle
\end{aligned}
\end{eqnarray}
e
\begin{equation}\label{diff_Z_y}
0 = \frac{\partial Z}{\partial y_i} (\xbarra, \ybarra) = - \Phi(\xbarra) \left\langle F(\xbarra), \frac{\partial F}{\partial y_i} (\ybarra) \right\rangle + \left\langle \nu(\xbarra), \frac{\partial F}{\partial y_i}(\ybarra) \right\rangle
\end{equation}
onde $h_{ij}(\xbarra)$ denota a $(ij)$-ésima coordenada da matriz da segunda forma fundamental de $F$ no ponto $\xbarra$, i.e.,
\begin{eqnarray}\label{eq:partialNu}
\frac{\partial\nu}{\partial x_i}(\xb) = 
\sum_{k=1}^2h_{ik}\frac{\partial F}{\partial x_k}(\xb).
\end{eqnarray}

\vspace{.2cm}

Sem perda de generalidade, podemos supor que a segunda forma
fundamental de $F$ est\'a diagonalizada no ponto $\xbarra$, i.e.,
\[
h_{11}(\xbarra)=\lambda_1, \quad h_{12}(\xbarra) = 0 \quad\mbox{e}\quad
h_{22}(\xbarra) = \lambda_2.
\]
Denotemos por $w_i$ a reflex\~ao do vetor
\[
\frac{\partial F}{\partial x_i}(\xbarra)
\]
em rela\c c\~ao ao plano ortogonal ao vetor $F(\xbarra) - F(\ybarra)$, i.e.,
\begin{eqnarray}\label{eq:reflexao}
w_i = \frac{\partial F}{\partial x_i}(\xbarra) - 
2 \left\langle \frac{\partial F}{\partial x_i}(\xbarra), \frac{F(\xbarra) - 
	F(\ybarra)}{|F(\xbarra) - F(\ybarra)|} \right\rangle \frac{F(\xbarra) - 
	F(\ybarra)}{|F(\xbarra) - F(\ybarra)|}.
\end{eqnarray}
Escolhendo um sistema de coordenadas apropriado $(y_1,y_2)$,
podemos supor que
\begin{equation*}
\left\langle w_1, \frac{\partial F}{\partial y_1}(\ybarra) \right\rangle \geq 0, 
\quad 
\left\langle w_1, \frac{\partial F}{\partial y_2}(\ybarra) \right\rangle = 0 
\quad \text{e} \quad 
\left\langle w_2, \frac{\partial F}{\partial y_2}(\ybarra) \right\rangle \geq 0.
\end{equation*}

\begin{lema}
	Os vetores $F(\ybarra)$ e $ \Phi(\xbarra) F(\xbarra) - \nu(\xbarra) $ são linearmente independentes.
\end{lema}
\begin{demonstracao}
	Usando a identidade
	\[
	\langle \Phi(\xbarra) F(\xbarra) - \nu(\xbarra), F(\ybarra) \rangle = 
	\Phi(\xbarra) - Z(\xbarra,\ybarra) = \Phi(\xbarra),
	\]
	obtemos
	\begin{eqnarray*}
		| \Phi(\xbarra) F(\xbarra) - \nu(\xbarra) |^2 | F(\ybarra) |^2 &-& 
		\langle \Phi(\xbarra) F(\xbarra) - \nu(\xbarra), F(\ybarra) \rangle^2 \\
		&=& | \Phi(\xbarra) F(\xbarra) - \nu(\xbarra) |^2 - \Phi(\xbarra)^2 = 1 \neq 0.
	\end{eqnarray*}
	Disso decorre que a desigualdade de Cauchy-Schwarz \'e estrita,
	logo os vetores $F(\ybarra)$ e $\Phi(\xbarra)F(\xbarra)-\nu(\xbarra)$
	s\~ao linearmente independentes.
\end{demonstracao}

Em rela\c c\~ao \`a reflex\~ao dada em \eqref{eq:reflexao}, obtemos o
seguinte:

\begin{lema}
	Valem as seguintes igualdades:
	\begin{equation*}
	w_1 = \frac{\partial F}{\partial y_1}(\ybarra) 
	\quad \text{e} \quad 
	w_2 = \frac{\partial F}{\partial y_2}(\ybarra).
	\end{equation*}
\end{lema}
\begin{demonstracao}
	Usando a express\~ao de $w_i$, dada em \eqref{eq:reflexao}, obtemos:
	\begin{eqnarray*}
		\langle w_i, F(\ybarra) \rangle &=& \left\langle \frac{\partial F}{\partial x_i}(\xbarra), F(\ybarra) \right\rangle + 2 \left\langle \frac{\partial F}{\partial x_i}, F(\ybarra) \right\rangle \frac{\langle F(\xbarra) - F(\ybarra),F(\ybarra) \rangle}{| F(\xbarra) - F(\ybarra) |^2} \\
		&=& \left\langle \frac{\partial F}{\partial x_i}(\xbarra), F(\ybarra) \right\rangle 
		+ 2 \left\langle \frac{\partial F}{\partial x_i}, F(\ybarra) \right\rangle \frac{\langle F(\xbarra),F(\ybarra) \rangle - 1}{2 - 2 \langle F(\xbarra), F(\ybarra) \rangle} \\
		&=& 0
	\end{eqnarray*}
	e
	\begin{eqnarray*}
		\langle w_i, \Phi(\xbarra) F(\xbarra) - \nu(\xbarra) \rangle &=&
		2 \left\langle \frac{\partial F}{\partial x_i}(\xbarra), F(\ybarra) \right\rangle \frac{\langle F(\xbarra) - F(\ybarra), \Phi(\xbarra) F(\xbarra) - \nu(\xbarra) \rangle}{| F(\xbarra) - F(\ybarra) |^2} \\
		& = & 2 \left\langle \frac{\partial F}{\partial x_i}(\xbarra), F(\ybarra) \right\rangle \frac{Z(\xbarra,\ybarra)}{| F(\xbarra) - F(\ybarra) |^2} \\
		& = & 0.
	\end{eqnarray*}
	Por outro lado, os vetores
	\begin{equation*}
	\frac{\partial F}{\partial y_1}(\ybarra) \quad \text{e} \quad \frac{\partial F}
	{\partial y_2}(\ybarra)
	\end{equation*}
	satisfazem
	\begin{equation*}
	\left\langle \frac{\partial F}{\partial y_i}(\ybarra), F(\ybarra)\right\rangle = 0
	\end{equation*}
	e
	\begin{equation*}
	\left\langle \frac{\partial F}{\partial y_i}(\ybarra), \Phi(\xbarra) F(\xbarra) - 
	\nu(\xbarra) \right\rangle = -\frac{\partial Z}{\partial y_i}(\xbarra,\ybarra) 
	= 0
	\end{equation*}
	Como os vetores $F(\ybarra)$ e $ \Phi(\xbarra) F(\xbarra) - \nu(\xbarra)$
	são linearmente independentes, conclu\'imos que o plano gerado por
	\begin{equation*}
	\frac{\partial F}{\partial y_1}(\ybarra) \quad \text{e} \quad 
	\frac{\partial F}{\partial y_2}(\ybarra)
	\end{equation*}
	é o mesmo plano gerado por $w_1$ e $w_2$. Al\'em disso, os vetores
	$w_1$ e $w_2$ s\~ao ortonormais. Como
	\begin{equation*}
	\left\langle w_1, \frac{\partial F}{\partial y_2}(\ybarra) \right\rangle = 0,
	\end{equation*}
	conclu\'imos que
	\begin{equation*}
	w_1 = \pm \frac{\partial F}{\partial y_1}(\ybarra) 
	\quad \text{e} \quad 
	w_2 = \pm \frac{\partial F}{\partial y_2}(\ybarra)
	\end{equation*}
	Finalmente, como
	\begin{equation*}
	\left\langle w_1, \frac{\partial F}{\partial y_1}(\ybarra) \right\rangle \geq 0 
	\quad \text{e} \quad 
	\left\langle w_2, \frac{\partial F}{\partial y_2}(\ybarra) \right\rangle \geq 0
	\end{equation*}	
	obtemos que
	\begin{equation*}
	w_1 = \frac{\partial F}{\partial y_1}(\ybarra) \quad \text{e} \quad 
	w_2 = \frac{\partial F}{\partial y_2}(\ybarra),
	\end{equation*}
	como quer\'iamos.
\end{demonstracao}

No resultado seguinte iremos considerar as derivadas de segunda
ordem da fun\c c\~ao $Z$ no ponto $(\xb,\yb)$.

\begin{proposicao}
	A derivada segunda da fun\c c\~ao $Z$, dada em \eqref{eq:funcaoZ},
	satisfaz:
	\begin{eqnarray*}\label{2_diff_Z_x}
		\sum_{i=1}^{2} \frac{\partial^2 Z}{\partial x_i^2} (\xb,\yb) &=& 
		\left( \Delta_{\Sigma} \Phi(\xb) - \frac{| \nabla \Phi(\xb) |^2}{\Phi(\xb)} 
		+ (|A(\xb)|^2 - 2) \Phi(\xb) \right) (1 - \left\langle F(\xb), F(\yb) \right\rangle) \\ 
		&&+ 2 \Phi(\xb) - \frac{2 \Phi(\xb)^2 - |A(\xb)|^2}{2 \Phi(\xb)
			(1 - \langle F(\xb), F(\yb) \rangle)} \sum_{i=1}^2 \left\langle 
		\frac{\partial F}{\partial x_i}(\xb), F(\yb) \right\rangle^2.
	\end{eqnarray*}
\end{proposicao}
\begin{demonstracao}
	Derivando a equação \eqref{diff_Z_x} em rela\c c\~ao a $x_i$,
	e somando, obtemos:
	\begin{eqnarray}\label{Z_seg_dev_x}
	\begin{aligned}
	\sum_{i=1}^2\frac{\partial^2 Z}{\partial x_i^2}(\xb,\yb) = &
	\sum_{i=1}^2\frac{\partial^2 \Phi}{\partial x_i^2}(\xb)
	(1 - \langle F(\xb), F(\yb) \rangle) -
	2\sum_{i=1}^2\frac{\partial \Phi}{\partial x_i}(\xb) 
	\left\langle \frac{\partial F}{\partial x_i}(\xb), F(\yb) \right\rangle \\
	& - \sum_{i=1}^2\Phi(\xb) \left\langle\frac{\partial^2 F}{\partial x_i^2}(\xb),F(\yb)
	\right\rangle + \sum_{i,k=1}^2\frac{\partial h_{ik}}{\partial x_i}(\xb)
	\left\langle \frac{\partial F}{\partial x_k}(\xb), F(\yb)\right\rangle \\ 
	&+ \sum_{i,k=1}^2h_{ik}(\xb)\left\langle
	\frac{\partial^2 F}{\partial x_i \partial x_k}(\xb), F(\yb)\right\rangle.
	\end{aligned}
	\end{eqnarray}
	Iremos, inicialmente, reescrever a equa\c c\~ao \eqref{Z_seg_dev_x}.
	Note que, da equa\c c\~ao de Codazzi \eqref{eq:Codazzi}, tem-se
	\begin{eqnarray}\label{eq:Prop4.5-1}
	\frac{\partial h_{11}}{\partial x_2}(\xb) = \frac{\partial h_{21}}{\partial x_1}(\xb)
	\quad\mbox{e}\quad
	\frac{\partial h_{12}}{\partial x_2}(\xb) = \frac{\partial h_{22}}{\partial x_1}(\xb),
	\end{eqnarray}
	e do fato de $\Sigma$ ser m\'inima, obtemos
	\begin{eqnarray}\label{eq:Prop4.5-2}
	\sum_{i=1}^2\frac{\partial h_{ii}}{\partial x_k}(\xb) = 0.
	\end{eqnarray}
	De \eqref{eq:Prop4.5-1} e \eqref{eq:Prop4.5-2}, obtemos
	\begin{eqnarray}\label{eq:Prop4.5-3}
	\sum_{i=1}^2\frac{\partial h_{ik}}{\partial x_i}(\xb) = 0.
	\end{eqnarray}
	Analisemos o termo 
	$\displaystyle\sum_{i=1}^2 \frac{\partial^2 F}{\partial x_i^2}(\xb)$. 
	Pelo teorema \ref{propriedades_sup_min_S3}, tem-se
	\begin{equation*}
	\sum_{i=1}^2 \frac{\partial^2 F_j}{\partial x_i^2}(\xb) + 2F_j(\xb) = 0,
	\end{equation*}
	para $j=1,\ldots,4$, logo,
	\[
	\sum_{i=1}^2 \frac{\partial^2 F}{\partial x_i^2}(\xb) = -2 F(\xb).
	\]
	Analisemos agora o termo 
	\[
	\sum_{i=1}^2 h_{ik}(\xb) \left\langle \frac{\partial^2 F}
	{\partial x_i \partial x_k}(\xb), F(\yb) \right\rangle.
	\]
	Derivamos \eqref{eq:partialNu} em rela\c c\~ao a $x_i$, somando
	e usando \eqref{eq:Prop4.5-3}, obtemos:
	\begin{eqnarray}\label{eq:Prop4.5-4}
	\sum_{i=1}^2 \frac{\partial^2 \nu}{\partial x_i^2}(\xb) = 
	\sum_{i,k=1}^2 h_{ik}(\xb)\frac{\partial^2 F}{\partial x_i \partial x_k}(\xb).
	\end{eqnarray}
	Note que, como
	\[
	\left\langle \sum_{i=1}^2 \frac{\partial^2 \nu}{\partial x_i^2}(\xb), 
	\frac{\partial F}{\partial x_j}(\xb) \right\rangle = 0,
	\]
	para $j=1,2$, segue que o vetor
	$\displaystyle\sum_{i=1}^2 \frac{\partial^2 \nu}{\partial x_i^2}(\xb)$ 
	não tem componentes no plano tangente 
	$T_{\xb}\Sigma\subset T_{\xb}\Sp^3$. 
	Por outro lado, derivando
	\[
	\left\langle \frac{\partial \nu}{\partial x_i}(\xb), \nu(\xb) \right\rangle = 0,
	\]
	em rela\c c\~ao a $x_i$, obtemos:
	\begin{equation}\label{eq:Prop4.5-5}
	\left\langle \frac{\partial^2 \nu}{\partial x_i^2}(\xb), \nu(\xb)\right\rangle 
	+ \left\langle \frac{\partial \nu}{\partial x_i}(\xb), 
	\frac{\partial \nu}{\partial x_i}(\xb) \right\rangle = 0.
	\end{equation}	
	Substituindo \eqref{eq:partialNu} em \eqref{eq:Prop4.5-5}, notando que
	$\left\langle \frac{\partial F}{\partial x_k}(\xb), 
	\frac{\partial F}{\partial x_j}(\xb) \right\rangle = \delta_{kj}$ e somando,
	a equa\c c\~ao \eqref{eq:Prop4.5-5} torna-se
	\begin{equation*}
	\left\langle \sum_{i=1}^2 \frac{\partial^2 \nu}{\partial x_i^2}(\xb), \nu(\xb)
	\right\rangle + \sum_{i=1}^2 (h_{ik}(\xb))^2 = 0.
	\end{equation*}	
	Como $\displaystyle\sum_{i=1}^2 (h_{ik}(\xb))^2 = |A(\xb)|^2$ e
	lembrando que 
	$\displaystyle\sum_{i=1}^2 \frac{\partial^2 \nu}{\partial x_i^2}(\xb)$ 
	só tem componente na direção $\nu$, obtemos
	\begin{eqnarray} \label{eq:Prop4.5-6}	
	\sum_{i=1}^2 \frac{\partial^2 \nu}{\partial x_i^2}(\xb) = - | A(\xb) |^2 \nu(\xb)
	\end{eqnarray}
	Usando \eqref{eq:Prop4.5-4} e \eqref{eq:Prop4.5-6}, temos que
	\begin{eqnarray} \label{eq:Prop4.5-7}
	\sum_{i=1}^2 h_{ik}(\xb) \frac{\partial^2 F}{\partial x_i \partial x_k}(\xb) 
	= - | A(\xb) |^2 \nu(\xb).
	\end{eqnarray}	
	Assim, usando \eqref{eq:Prop4.5-3} e \eqref{eq:Prop4.5-7}, podemos
	escrever a equa\c c\~ao \eqref{Z_seg_dev_x} como:
	\begin{eqnarray}
	\begin{aligned} \label{lap_Z_x}
	\sum_{i=1}^2 \frac{\partial^2 Z}{\partial x_i^2}(\xb,\yb) & =   
	\sum_{i=1}^2 \frac{\partial^2 \Phi}{\partial x_i^2}(\xb)
	(1 - \langle F(\xb), F(\yb) \rangle)    
	+ 2  \Phi(\xb) \left\langle F(\xb), F(\yb) \right\rangle \\
	& -2 \sum_{i=1}^2 \frac{\partial \Phi}{\partial x_i}(\xb) 
	\left\langle \frac{\partial F}{\partial x_i}(\xb), F(\yb) \right\rangle
	- | A(\xb) |^2 \left\langle \nu(\xb), F(\yb) \right\rangle.
	\end{aligned}
	\end{eqnarray}
	Como
	\[
	\left\langle \nu(\xb), F(\yb) \right\rangle = Z(\xb,\yb)
	-\phi(\xb)(1-\langle F(\xb),F(\yb)\rangle)
	\]
	podemos reescrever \eqref{lap_Z_x} como sendo
	\begin{eqnarray*}
		\begin{aligned}
			\sum_{i=1}^2 \frac{\partial^2 Z}{\partial x_i^2}(\xb,\yb) = &
			\left(\Delta_{\Sigma} \Phi(\xb) + 
			(|A(\xb)|^2 - 2)\Phi(\xb)\right)(1 - \langle F(\xb), F(\yb) \rangle) \\
			&+ 2 \Phi(\xb) \left\langle F(\xb), F(\yb) \right\rangle
			- 2 \sum_{i=1}^2 \frac{\partial \Phi}{\partial x_i}(\xb) 
			\left\langle \frac{\partial F}{\partial x_i}(\xb), F(\yb) \right\rangle \\
			&- |A(\xb)|^2Z(\xb,\yb).
		\end{aligned}
	\end{eqnarray*}	
	Somando e subtraindo
	\[
	\frac{|\nabla \Phi(\xb)|^2}{\Phi(\xb)} (1 - \langle F(\xb), F(\yb) \rangle) + \frac{\Phi(\xb)}{1 - \langle F(\xb),F(\yb) \rangle} \sum_{i=1}^2 \left\langle \frac{\partial F}{\partial x_i} (\xb), F(\yb) \right\rangle^2,
	\]
	obtemos:
	\begin{eqnarray*}
		\begin{aligned}
			\sum_{i=1}^2 \frac{\partial^2 Z}{\partial x_i^2}(\xb,\yb) &=
			\left(\Delta_{\Sigma} \Phi(\xb) - \frac{|\nabla \Phi(\xb)|^2}{\Phi(\xb)} + (|A(\xb)|^2 - 2)\Phi(\xb)\right)(1 - \langle F(\xb), F(\yb) \rangle) \\
			&+ 2 \Phi(\xb) + \frac{|\nabla \Phi(\xb)|^2}{\Phi(\xb)} (1 - \langle F(\xb), F(\yb) \rangle) - 2 \sum_{i=1}^2 \frac{\partial \Phi}{\partial x_i}(\xb) \left\langle \frac{\partial F}{\partial x_i}(\xb), F(\yb) \right\rangle \\
			&+ \frac{\Phi(\xb)}{1 - \langle F(\xb),F(\yb) \rangle} \sum_{i=1}^2 \left\langle \frac{\partial F}{\partial x_i} (\xb), F(\yb) \right\rangle^2 \\
			& - \frac{\Phi(\xb)}{1 - \langle F(\xb),F(\yb) \rangle} \sum_{i=1}^2
			\left\langle \frac{\partial F}{\partial x_i} (\xb), F(\yb) \right\rangle^2
		\end{aligned}
	\end{eqnarray*}
	Fatorando a express\~ao acima adequadamente, podemos escrever
	\begin{eqnarray*}
		\begin{aligned}
			\sum_{i=1}^2 \frac{\partial^2 Z}{\partial x_i^2}(\xb,\yb) &= \left(\Delta_{\Sigma} \Phi(\xb) - \frac{|\nabla \Phi(\xb)|^2}{\Phi(\xb)} + (|A(\xb)|^2 - 2)\Phi(\xb)\right)(1 - \langle F(\xb), F(\yb) \rangle)+ 2 \Phi(\xb)	\\
			& - \frac{\Phi(\xb)}{1 - \langle F(\xb),F(\yb) \rangle} \sum_{i=1}^2 \left\langle \frac{\partial F}{\partial x_i} (\xb), F(\yb) \right\rangle^2 \\
			& + \frac{1}{\Phi(\xb)(1 - \langle F(\xb), F(\yb) \rangle)}m(\xb,\yb),
		\end{aligned}
	\end{eqnarray*}	
	onde
	\begin{eqnarray*}
		\begin{aligned}
			m(\xb,\yb) &= - 2 \sum_{i=1}^2 \frac{\partial \Phi}{\partial x_i}(\xb) 
			(1 - \langle F(\xb), F(\yb) \rangle) \Phi(\xb) \left\langle \frac{\partial F}
			{\partial x_i}(\xb), F(\yb) \right\rangle \\
			&+ |\nabla \Phi(\xb)|^2 (1 - \langle F(\xb), F(\yb) \rangle)^2 
			+ \Phi(\xb)^2 \sum_{i=1}^2 \left\langle \frac{\partial F}
			{\partial x_i} (\xb), F(\yb) \right\rangle^2
		\end{aligned}
	\end{eqnarray*}	
	Observe que
	\[
	m(\xb,\yb) = \sum_{i=1}^2 \left(  \frac{\partial \Phi}{\partial x_i}(\xb)
	(1 - \langle F(\xb),F(\yb) \rangle) - \Phi(\xb) \left\langle 
	\frac{\partial F}{\partial x_i}(\xb), F(\yb) \right\rangle \right)^2.
	\]
	Pela equação \eqref{diff_Z_x}, e lembrando que 
	$\frac{\partial Z}{\partial x_i}(\xb,\yb)=0$, podemos escrever
	\begin{eqnarray} \label{eq:Prop4.5-8}
	m(\xb,\yb) = \sum_{i=1}^2 \left( h_{ik}(\xb) 
	\left\langle \frac{\partial F}{\partial x_k}(\xb), F(\yb) \right\rangle \right)^2
	\end{eqnarray}	
	Expandindo o termo quadr\'atico em \eqref{eq:Prop4.5-8}, obtemos:
	\begin{eqnarray*}
		\begin{aligned}
			(h_{i1}(\xb))^2 \left\langle \frac{\partial F}{\partial x_1}(\xb), F(\yb)\right\rangle^2 
			&+ 
			(h_{i2}(\xb))^2 \left\langle \frac{\partial F}{\partial x_2}(\xb),F(\yb)\right\rangle^2 \\
			&+ 
			2 h_{i1}(\xb) h_{i2}(\xb) \left\langle \frac{\partial F}
			{\partial x_1}(\xb), F(\yb) \right\rangle \left\langle 
			\frac{\partial F}{\partial x_2}(\xb), F(\yb) \right\rangle
		\end{aligned}
	\end{eqnarray*}	
	Como o determinante de $A$ \'e $-\lambda^2$, tem-se:
	\begin{equation*}
	h_{11}(\xb) h_{22}(\xb) - h_{12}(\xb) h_{21}(\xb) = h_{11}(\xb) h_{22}(\xb) - (h_{12}(\xb))^2 = -\lambda^2,
	\end{equation*}	
	pois $h_{12}(\xb) = h_{21}(\xb)$. Al\'em disso, como 
	$h_{11}(\xb) + h_{22}(\xb) = 0$, tem-se
	\[
	\sum_{i=1}^2 (h_{i1}(\xb))^2 = \lambda^2 \quad\mbox{e}\quad
	\sum_{i=1}^2 (h_{i2}(\xb))^2 = \lambda^2.
	\]
	Em rela\c c\~ao ao termo $\displaystyle\sum_{i=1}^2 h_{i1}(\xb) h_{i2}(\xb)$,
	tem-se
	\begin{equation*}
	h_{11}(\xb) h_{12}(\xb) + h_{21}(\xb) h_{22}(\xb) = h_{12}(\xb) (h_{11}(\xb) + h_{22}(\xb)) = 0.
	\end{equation*}	
	Portanto, podemos reescrever \eqref{eq:Prop4.5-8} como sendo
	\begin{equation*}
	m(\xb,\yb) = \sum_{i=1}^2 \lambda^2 \left\langle 
	\frac{\partial F}{\partial x_i}(\xb), F(\yb) \right\rangle^2.
	\end{equation*}	
	Lembrando agora que $\lambda^2 = \frac{1}{2} |A(\xb)|$, obtemos
	a equação desejada.
\end{demonstracao}


\begin{proposicao}
	Em rela\c c\~ao \`as derivadas mistas, vale a seguinte rela\c c\~ao:
	\begin{equation}\label{diff_Z_x_y}
	\frac{\partial^2 Z}{\partial x_i \partial y_i}(\xb,\yb) = \lambda_i - \Phi(\xb)
	\end{equation}
\end{proposicao}
\begin{demonstracao}
	Derivando em rela\c c\~ao  a $x_i$ a equação \eqref{diff_Z_y}, obtemos:
	\begin{eqnarray*}
		\begin{aligned}
			\frac{\partial^2 Z}{\partial x_i \partial y_i}(\xb,\yb) =  & 
			-\frac{\partial \Phi}{\partial x_i}(\xb) \left\langle F(\xb), 
			\frac{\partial F}{\partial y_i}(\yb) \right\rangle - \Phi(\xb)
			\left\langle \frac{\partial F}{\partial x_i}(\xb), 
			\frac{\partial F}{\partial y_i}(\yb) \right\rangle \\
			& + \left\langle \frac{\partial \nu}{\partial x_i}(\xb), 
			\frac{\partial F}{\partial y_i}(\yb) \right\rangle.
		\end{aligned}
	\end{eqnarray*}	
	Lembrando que 
	\[
	\frac{\partial \nu}{\partial x_i}(\xb) = 
	\lambda_i(\xb) \frac{\partial F}{\partial x_i}(\xb),
	\]
	temos
	\begin{eqnarray} \label{eq:Prop2.3_1}
	\frac{\partial^2 Z}{\partial x_i \partial y_i}(\xb,\yb) = -\frac{\partial \Phi}{\partial x_i}(\xb) \left\langle F(\xb), \frac{\partial F}{\partial y_i}(\yb) \right\rangle + (\lambda_i(\xb) - \Phi(\xb)) \left\langle \frac{\partial F}{\partial x_i}(\xb), \frac{\partial F}{\partial y_i}(\yb) \right\rangle.
	\end{eqnarray}	
	Da equa\c c\~ao \eqref{diff_Z_x}, tem-se:
	\begin{eqnarray} \label{eq:Prop2.3_2}
	-\frac{\partial \Phi}{\partial x_i}(\xb) = \frac{1}
	{1 - \langle F(\xb),F(\yb) \rangle} (\lambda_i(\xb) - \Phi(\xb)) 
	\left\langle \frac{\partial F}{\partial x_i}(\xb), F(\yb) \right\rangle.
	\end{eqnarray}
	Substituindo \eqref{eq:Prop2.3_2} em \eqref{eq:Prop2.3_1}, 
	obtemos:
	\begin{eqnarray*}
		\begin{aligned}
			\frac{\partial^2 Z}{\partial x_i \partial y_i}(\xb,\yb) =&  \frac{1}{1 - \langle F(\xb),F(\yb) \rangle} (\lambda_i(\xb) - \Phi(\xb)) \left\langle \frac{\partial F}{\partial x_i}(\xb), F(\yb) \right\rangle  \left\langle F(\xb), \frac{\partial F}{\partial y_i}(\yb) \right\rangle \\
			&+ 
			(\lambda_i(\xb) - \Phi(\xb)) \left\langle \frac{\partial F}{\partial x_i}(\xb), \frac{\partial F}{\partial y_i}(\yb) \right\rangle
		\end{aligned}
	\end{eqnarray*}	
	Como 
	\[
	|F(\xb) - F(\yb)|^2 = \langle F(\xb) - F(\yb), F(\xb) - F(\yb) \rangle = 
	2 - 2 \langle F(\xb), F(\yb) \rangle,
	\]
	a equa\c c\~ao anterior torna-se
	\begin{eqnarray} \label{eq:Prop2.3_3}
	\begin{aligned}
	\frac{\partial^2 Z}{\partial x_i \partial y_i}(\xb,\yb) = & 
	-2 (\lambda_i(\xb) - \Phi(\xb)) \left\langle \frac{\partial F}{\partial x_i}(\xb), 
	\frac{-F(\yb)}{|F(\xb) - F(\yb)|} \right\rangle \left\langle \frac{F(\xb)}{|F(\xb) - 
		F(\yb)|}, \frac{\partial F}{\partial y_i}(\yb) \right\rangle \\
	& + 
	(\lambda_i - \Phi(\xb)) \left\langle \frac{\partial F}{\partial x_i}(\xb), 
	\frac{\partial F}{\partial y_i}(\yb) \right\rangle.
	\end{aligned}
	\end{eqnarray}	
	Al\'em disso, como
	\[
	\left\langle \frac{\partial F}{\partial x_i}(\xb), F(\xb) \right\rangle = 
	\left\langle \frac{\partial F}{\partial y_i}(\yb), F(\yb) \right\rangle = 0,
	\]
	a equação \eqref{eq:Prop2.3_3} se escreve como sendo
	\begin{eqnarray*}
		%\begin{aligned}
		\frac{\partial^2 Z}{\partial x_i \partial y_i}(\xb,\yb) & = &
		-2 (\lambda_i(\xb) - \Phi(\xb)) \left\langle \frac{\partial F}{\partial x_i}(\xb), \frac{F(\xb) -F(\yb)}{|F(\xb) - F(\yb)|} \right\rangle \left\langle 
		\frac{F(\xb) - F(\yb)}{|F(\xb) - F(\yb)|}, \frac{\partial F}{\partial y_i}(\yb) 
		\right\rangle \\
		&& + 
		(\lambda_i - \Phi(\xb)) \left\langle \frac{\partial F}{\partial x_i}(\xb), 
		\frac{\partial F}{\partial y_i}(\yb) \right\rangle \\
		&= &
		(\lambda_i - \Phi(\xb)) \left\langle w_i,\frac{\partial F}{\partial y_i}(\yb)
		\right\rangle \\
		&= & \lambda_i - \Phi(\xb),
		%\end{aligned}
	\end{eqnarray*}	
	como quer\'iamos.
\end{demonstracao}


\begin{proposicao}
	\begin{equation}\label{2_diff_Z_y}
	\sum_{i=1}^2 \frac{\partial^2 Z}{\partial y_i^2}(\xb,\yb) = 2 \Phi(\xb).
	\end{equation}
\end{proposicao}
\begin{demonstracao}
	Derivando a equa\c c\~ao \eqref{diff_Z_y} em rela\c c\~ao a $y_i$,
	e somando, obtemos:
	\begin{eqnarray} \label{eq:Prop4.7-1}
	\sum_{i=1}^2 \frac{\partial^2 Z}{\partial y_i^2}(\xb,\yb) = - \Phi(\xb) \left\langle F(\xb), \sum_{i=1}^2 \frac{\partial^2 F}{\partial y_i^2}(\yb) \right\rangle + \left\langle \nu(\xb), \sum_{i=1}^2 \frac{\partial^2 F}{\partial y_i^2}(\yb) \right\rangle.
	\end{eqnarray}	
	Como
	\[
	\sum_{i=1}^2 \frac{\partial^2 F}{\partial y_i}(\yb) = -2 F(\yb)
	\]
	e
	\[
	0 = Z(\xb,\yb) = \Phi(\xb)(1 - \langle F(\xb), F(\yb) \rangle) + 
	\langle \nu(\xb), F(\yb) \rangle
	\] 
	a express\~ao em \eqref{eq:Prop4.7-1} torna-se
	\begin{eqnarray*}
		\sum_{i=1}^2 \frac{\partial^2 Z}{\partial y_i^2}(\xb,\yb) &=&
		2 \Phi(\xb) \langle F(\xb), F(\yb) \rangle - 2 \langle \nu(\xb), F(\yb) 
		\rangle \\
		&=&
		2 \Phi(\xb),
	\end{eqnarray*}
	como quer\'iamos.
\end{demonstracao}


\begin{proposicao}
	Vale a seguinte igualdade:
	\begin{eqnarray} \label{suma_2das_derivadas}
	\begin{aligned}
	\sum_{i=1}^2 \frac{\partial^2 Z}{\partial x_i^2}(\xbarra,\ybarra) + & 
	2 \sum_{i=1}^2 \frac{\partial^2 Z}{\partial x_i \partial y_i}(\xbarra,\ybarra) + \sum_{i=1}^2 \frac{\partial^2 Z}{\partial y_i^2}(\xbarra,\ybarra) \\
	&= 
	- \frac{2 \Phi(\xbarra)^2- |A(\xbarra)|^2}{2 \Phi(\xbarra)
		(1 - \langle F(\xbarra),F(\ybarra) \rangle)} \sum_{i=1}^2
	\left\langle \frac{\partial F}{\partial x_i}(\xbarra), F(\ybarra) \right\rangle \\
	&+
	\left( \Delta_{\Sigma} \Phi(\xbarra) - \frac{| \nabla \Phi(\xbarra) |^2}
	{\Phi(\xbarra)} + ( | A(\xbarra) |^2 - 2 ) \Phi(\xbarra) \right)
	\left(1 - \langle F(\xbarra,\ybarra) \rangle\right)
	\end{aligned}
	\end{eqnarray}
\end{proposicao}
\begin{demonstracao}
	A equa\c c\~ao \eqref{suma_2das_derivadas} segue por somando-se
	as equa\c c\~oes \eqref{2_diff_Z_x}, \eqref{diff_Z_x_y} e \eqref{2_diff_Z_y}.
\end{demonstracao}

\begin{teorema}\label{Simon's_identity}
	Seja $\Sigma$ uma superfície mínima imersa em $S^3$. Então
	\begin{equation*}
		\Delta_{\Sigma} A = 2 A - |A|^2 A
	\end{equation*} 
\end{teorema}

\begin{demonstracao}
	Olhar \cite{Simons1968} Teorema 5.3.1.
\end{demonstracao}

\begin{teorema}\label{nao_existem_pontos_umbilicos}
	Uma superfície minima imersa em $S^3$ de gênero 1 não tem pontos umbílicos, então, a segunda forma fundamental não é zero em nenhum ponto da superfície.
\end{teorema}

\begin{demonstracao}
	Olhar \cite{Brendle2013} Proposição 3.3.
\end{demonstracao}

\begin{definicao}
	A função $\Psi: M \rightarrow \realnumbers$ está definida por
	\begin{equation*}
	\Psi(x) = \frac{1}{\sqrt{2}} |A(x)|.
	\end{equation*}
	onde $A$ é a segunda forma fundamental.
\end{definicao}

\begin{proposicao}\label{edp_principal}
	Supor que $F: \Sigma \rightarrow S^3$ é um toro mínimo mergulhado em $S^3$. Então a função $\Psi = \frac{|A|}{\sqrt{2}}$ é estritamente positiva e satisfaz a E.D.P
	\begin{equation*}
		\Delta_\Sigma \Psi - \frac{|\nabla \Psi|^2}{\Psi} + (|A|^2 - 2) \Psi = 0
	\end{equation*}
\end{proposicao}

\begin{demonstracao}
	A função $|A|$ é estritamente positiva pelo teorema \ref{nao_existem_pontos_umbilicos} . Usando o  teorema \ref{Simon's_identity} sabendo que $A = [h_{ik}]$, a equação pode se escrever
	\begin{equation*}
		\Delta_{\Sigma} h_{ik} + (|A|^2 - 2)h_{ik} = 0
	\end{equation*}	
	Multiplicando por $2 h_{ik}$
	\begin{equation*}
		2 \Delta_{\Sigma} h_{ik} h_{ik} + 2 (|A|^2 - 2)h_{ik}^2 = 0
	\end{equation*}	
	Somando e restando $ 2 \sum_{j=1}^2 \left( \frac{\partial h_{ik}}{\partial x_j} \right)^2 $
	\begin{equation*}
		2 \sum_{j=1}^2 \frac{\partial^2 h_{ik}}{\partial x_j^2} h_{ik} + 2 \sum_{j=1}^2 \left( \frac{\partial h_{ik}}{\partial x_j} \right)^2 - 2 \sum_{j=1}^2 \left( \frac{\partial h_{ik}}{\partial x_j} \right)^2 + 2 (|A|^2 - 2)h_{ik}^2 = 0
	\end{equation*}	
	Como $ 2 \sum_{j=1}^2 \frac{\partial^2 h_{ik}}{\partial x_j^2} h_{ik} + 2 \sum_{j=1}^2 \left( \frac{\partial h_{ik}}{\partial x_j} \right)^2 = 2 \sum_{j=1}^2 \frac{\partial }{\partial x_j} \left( \frac{\partial h_{ik}}{\partial x_j} h_{ik} \right) = \sum_{j=1}^2 \frac{\partial^2 h_{ik}^2}{\partial x_j^2} $, então
	\begin{equation*}
		\sum_{j=1}^2 \frac{\partial^2 h_{ik}^2}{\partial x_j^2} - 2 \sum_{j=1}^2 \left( \frac{\partial h_{ik}}{\partial x_j} \right)^2 + 2 (|A|^2 - 2)h_{ik}^2 = 0
	\end{equation*}	
	Somando com respeito aos índices $i$ e $k$
	\begin{equation*}
	\Delta_{\Sigma} (|A|^2) - 2 | \nabla A |^2 + 2 (|A|^2 - 2) |A|^2 = 0
	\end{equation*}	
	Olhar que $ \Delta_{\Sigma} (|A|^2) = \sum_{j=1}^2 \frac{\partial^2 |A|^2}{\partial x_j^2} = \sum_{j=1}^2 \frac{\partial}{\partial x_j} \left( \frac{\partial |A|^2}{\partial x_j} \right) = 2 \sum_{j=1}^2 \frac{\partial^2 |A|}{\partial x_j} + 2 \sum_{j=1}^2 \left( \frac{\partial |A|}{\partial x_j} \right)^2 $. Portanto
	\begin{equation}\label{edp_sff}
		\Delta_{\Sigma} (|A|) + \frac{|\nabla |A||^2}{|A|} - \frac{|\nabla A|^2}{|A|} + (|A|^2 - 2) |A| = 0
	\end{equation}	
	Temos que $ | \nabla A |^2 = \sum_{j=1}^2 \left( \frac{\partial h_{ik}}{\partial x_j} \right)^2 $ onde se está somando sobre os índices $i$ e $k$. Pela equação \eqref{codazzi_eq} temos
	\begin{align*}
		\frac{\partial h_{11}}{\partial x_1} + \frac{\partial h_{21}}{\partial x_2} &= 0\\
		\frac{\partial h_{12}}{\partial x_1} + \frac{\partial h_{22}}{\partial x_2} &= 0
	\end{align*}	
	Elevando ao quadrado temos
	\begin{align*}
		\left( \frac{\partial h_{11}}{\partial x_1} \right)^2 + \left( \frac{\partial h_{21}}{\partial x_2} \right)^2  &= - 2 \frac{\partial h_{11}}{\partial x_1} \frac{\partial h_{21}}{\partial x_2}  \\
		\left( \frac{\partial h_{12}}{\partial x_1} \right)^2 + \left( \frac{\partial h_{22}}{\partial x_2} \right)^2 &= - 2 \frac{\partial h_{12}}{\partial x_1} \frac{\partial h_{22}}{\partial x_2}
	\end{align*}	
	Usando a equação \eqref{codazzi_eq} outra vez temos
	\begin{align*}
		\left( \frac{\partial h_{11}}{\partial x_1} \right)^2 + \left( \frac{\partial h_{21}}{\partial x_2} \right)^2  &=  2  \left( \frac{\partial h_{21}}{\partial x_2} \right)^2  \\
		\left( \frac{\partial h_{12}}{\partial x_1} \right)^2 + \left( \frac{\partial h_{22}}{\partial x_2} \right)^2 &=  2 \left( \frac{\partial h_{12}}{\partial x_1} \right)^2\\
		\left( \frac{\partial h_{11}}{\partial x_2} \right)^2 + \left( \frac{\partial h_{21}}{\partial x_1} \right)^2  &= 2 \left( \frac{\partial h_{21}}{\partial x_1} \right)^2\\
		\left( \frac{\partial h_{12}}{\partial x_2} \right)^2 + \left( \frac{\partial h_{22}}{\partial x_1} \right)^2 &=  2 \left( \frac{\partial h_{12}}{\partial x_2} \right)^2  
	\end{align*}	
	Somando e lembrando que $h_{12} = h_{21}$ temos
	\begin{equation*}
		| \nabla A |^2 = 4 \left( \frac{\partial h_{12}}{\partial x_1} \right) + 4 \left( \frac{\partial h_{12}}{\partial x_2} \right)
	\end{equation*}	
	Observar que $ |A| = \sqrt{2 h_{11}^2 + 2 h_{12}^2} $ e que $ | \nabla |A| |^2 = \left( \frac{\partial |A|}{\partial x_1} \right)^2 + \left( \frac{\partial |A|}{\partial x_2} \right)^2 $. Derivando $ |A| $ com respeito a $x_1$ temos
	\begin{equation*}
		\frac{\partial |A|}{\partial x_1} = \frac{2 h_{11} \frac{\partial h_{11}}{\partial x_1} + 2 h_{12} \frac{\partial h_{12}}{\partial x_1}}{|A|}
	\end{equation*}	
	De forma similar, derivando $ |A| $ com respeito a $ x_2 $ temos
	\begin{equation*}
		\frac{\partial |A|}{\partial x_2} = \frac{2 h_{11} \frac{\partial h_{11}}{\partial x_2} + 2 h_{12} \frac{\partial h_{12}}{\partial x_2}}{|A|}
	\end{equation*}	
	Obtendo o quadrado de ambas expressões se tem
	\begin{align*}
		\left( \frac{\partial |A|}{\partial x_1} \right)^2 &= \frac{4 h_{11}^2 \left( \frac{\partial h_{11}}{\partial x_1} \right)^2 + 4 h_{12}^2 \left( \frac{\partial h_{12}}{\partial x_1} \right)^2 + 8 h_{11} h_{12} \frac{\partial h_{11}}{\partial x_1} \frac{\partial h_{12}}{\partial x_1}}{|A|^2}\\
		\left( \frac{\partial |A|}{\partial x_2} \right)^2 &= \frac{4 h_{11}^2 \left( \frac{\partial h_{11}}{\partial x_2} \right)^2 + 4 h_{12}^2 \left( \frac{\partial h_{12}}{\partial x_2} \right)^2 + 8 h_{11} h_{12} \frac{\partial h_{11}}{\partial x_2} \frac{\partial h_{12}}{\partial x_2}}{|A|^2}
	\end{align*}	
	Usando \eqref{codazzi_eq} e somando os termos correspondentes se tem
	\begin{equation*}
		| \nabla |A| |^2 = 2 \left( \frac{\partial h_{12}}{\partial x_1} \right)^2 + 2 \left( \frac{\partial h_{12}}{\partial x_2} \right)
	\end{equation*}	
	Portanto, $ | \nabla A |^2 = 2 | \nabla |A| |^2 $. Usando o resultado em \ref{edp_sff} se tem
	\begin{equation*}
		\Delta_\Sigma \Psi - \frac{|\nabla \Psi|^2}{\Psi} + (|A|^2 - 2) \Psi = 0
	\end{equation*}
\end{demonstracao}


\begin{teorema}[Lawson]\label{teo_lawson}
	Se $\Sigma$ é uma superfície mínima em $S^3$ de curvatura intrínseca constante $K$, então ou $K=1$ e $\Sigma$ é totalmente geodésica, ou $K=0$ e $\Sigma$ é um pedaço aberto de toro do Clifford.
\end{teorema}

\begin{demonstracao}
	\cite{Lawson1969}, Corolário 3.
\end{demonstracao}

\begin{teorema}[Brendle]
	Seja $F: M \rightarrow S^3$ uma superfície mínima mergulhada em $S^3$ de gênero 1. Então $F$ é congruente a o toro de Clifford.
\end{teorema}


Definamos
\begin{equation*}
	\aleph = \sup_{\substack{x,y \in \Sigma\\ x \neq y}} \frac{| \langle  \nu(x), F(y) \rangle |}{\Psi(x) (1 - \langle F(x), F(y) \rangle)}
\end{equation*}
Supor que $\aleph \leq 1$. Então
\begin{equation*}
	\Psi(x) (1 - \langle F(x), F(y) \rangle) + \langle \nu(x), F(y) \rangle \geq 0
\end{equation*}
para todo $ x,y \in \Sigma $. Identifiquemos ao ponto $F(x)$ com o ponto $x$, i.e., $F(x) = x$, para todo $x \in \Sigma$. Seja $p \in \Sigma$ arbitrário e $\{ e_1, e_2 \}$ a base de autovetores ortonormais de $A$ que gera $T_{p} \Sigma$ tais que
\begin{equation*}
	h(e_1,e_1) = \Psi(\xb), \quad h(e_1,e_2)=0 \quad \text{and} \quad h(e_2,e_2) = -\Psi(\xb)
\end{equation*}
Observar que $\Psi(\xb) = |\lambda|$, onde os autovalores de $A$ são $\lambda$ e $-\lambda$.
Seja $\gamma(t)$ uma geodésica em $\Sigma$ tal que $\gamma(0)=p$ e $\gamma'(0)=e_1$. Definamos a função $f: \mathbb{R} \rightarrow \mathbb{R}$ por
\begin{equation*}
	f(t) = \Psi(\xb) (1 - \langle p, \gamma(t) \rangle) + \langle \nu(\xb), \gamma(t) \rangle \geq 0
\end{equation*}
Calculando a derivada de primeiro ordem temos
\begin{equation*}
	f'(t) = -\langle \Psi(\xb) \xb - \nu(\xb), \gamma'(t) \rangle + \langle \nu(\xb), \gamma'(t) \rangle
\end{equation*}
Observar que $ \gamma $ é uma geodésica, então $ \gamma'(t) $ é  o transporte paralelo de $ \gamma'(0) $. Logo se tem que $ \langle \nu(\xb), \gamma'(t) \rangle = \langle \nu(\xb), \gamma'(0) \rangle = 0 $
\begin{equation*}
	f'(t) = -\langle \Psi(\xb) \xb - \nu(\xb), \gamma'(t) \rangle
\end{equation*}
Calculando a segunda derivada temos
\begin{equation*}
	f''(t) = -\langle \Psi(\xb) \xb - \nu(\xb), \gamma''(t) \rangle
\end{equation*}
Observar que $ \langle \nu(\gamma(t)), \gamma'(t) \rangle = 0 $. Derivando se tem
\begin{equation*}	\langle D \nu(\gamma(t)) \gamma'(t), \gamma'(t) \rangle + \langle \nu(\gamma(t)), \gamma''(t) \rangle = 0
\end{equation*}
Como $\gamma''(t)$ não tem componentes em $T_{\gamma(t)} \Sigma$ porque é geodésica, então
\begin{equation*}
	-\gamma''(t) = \gamma(t) + h(\gamma'(t), \gamma'(t)) \nu(\xb)
\end{equation*}
Portanto
\begin{equation*}
	f''(t) = \langle \Psi(\xb) \xb - \nu(\xb), \gamma(t) + h(\gamma'(t), \gamma'(t)) \nu(\xb) \rangle
\end{equation*}
Derivando outra vez se tem
\begin{multline}\label{3ra_der_f}
	f'''(t) = \langle \Psi(\xb)p - \nu(\xb), \gamma'(t) \rangle + h(\gamma'(t), \gamma'(t)) \langle \Psi(\xb)p - \nu(\xb), D_{\gamma'(t)} \nu(\gamma(t)) \rangle\\
	+ \left( D_{\gamma'(t)}^{\Sigma} h \right) (\gamma'(t), \gamma'(t)) \langle \Psi(\xb)p - \nu(\xb), \nu(\gamma(t)) \rangle
\end{multline}
Observar que $f(0)=0$. Também $f'(0)=0$ porque $\Psi(\xb)p - \nu(\xb)$ é perpendicular a $T_p \Sigma$. E finalmente, $f''(0)=0$ porque $f''(0) = \Psi(\xb) - h(e_1,e_1)=0$.
Vamos demonstrar que $f'''(0)=0$. Usando a expansão de Taylor em $f$ se tem
\begin{equation*}
	f(t) = f(0) + f'(0)t + f''(0)t^2 + f'''(0)t^3 + r_3(t)
\end{equation*}  
onde $\lim_{t \rightarrow 0} \frac{r_3(t)}{t_3}=0$. Como $f(t) \geq 0$ e $f(0)=f'(0)=f''(0)=0$ se tem
\begin{equation*}
	0 \leq f'''(0) t^3 + r_3(t)
\end{equation*}
Dividendo por $t^3$ e fazendo o limite quando $t \rightarrow 0$ temos que $0 \leq f'''(0)$. Por outro lado, usando a mesma ideia usando $-t$ em lugar de  $t$ temos que $f'''(0) \leq 0$. Portanto, $f'''(0)=0$.
Voltando  a \eqref{3ra_der_f}, avaliando em $t=0$, se tem que $(D_{e_1}^{\Sigma} h) (e_1,e_1) = 0$ porque $e_1, D_{e_1} \nu(\xb) \in T_p \Sigma$ e $\Psi(\xb)p - \nu(\xb)$ e $\nu(\xb)$ são paralelos.
Para continuar a demonstração deste primeiro caso, podemos considerar a base $\{ e_2, e_1, -\nu \}$ em lugar de $\{ e_1, e_2, \nu \}$ para mudar a orientação definida da superfície e, fazendo o mesmo procedimento, se tem que $(D_{e_2}^{\Sigma} h) (e_2,e_2)=0$.
Pela equação de Codazzi se tem
\begin{align*}
	(D_{e_2}^{\Sigma} h) (e_1,e_1) &= (D_{e_1}^{\Sigma} h) (e_2,e_1) = 0\\
	(D_{e_1}^{\Sigma} h) (e_2,e_2) &= (D_{e_2}^{\Sigma} h) (e_1,e_2) = 0
\end{align*}
Portanto $\nabla h = 0$ e com isso se tem que $h$ é constante. Seja $K$ a curvatura intrínseca de $\Sigma$. Como $h$ é constante, então $K$ também é constante. Logo, pelo teorema de Lawson (\ref{teo_lawson}) ou $K=0$, ou $K=1$. Mas $\Sigma$ não tem ponto umbílicos por \ref{nao_existem_pontos_umbilicos}, portanto $K=0$ e $\Sigma$ é um pedaço aberto do toro de Clifford. Mas $\Sigma$ é compacto, portanto $\Sigma$ é todo o toro de Clifford.

Supor agora que $\aleph > 1$. Escolhendo adequadamente entre $\nu$ e $-\nu$ temos
\begin{equation*}
	\aleph = \sup_{\substack{x,y \in \Sigma\\ x \neq y}} \frac{- \langle \nu(x), F(y) \rangle}{\Psi(x)(1 - \langle F(x), F(y) \rangle)}.
\end{equation*}
Seja $\Phi(x) = \aleph \Psi(x)$ em a definição de $Z(x,y)$, então
\begin{equation*}
	Z(x,y) = \aleph \Psi(x)(1 - \langle F(x), F(y) \rangle) + \langle \nu(x), F(y) \rangle.
\end{equation*}
Pela equação anterior observa-se que $Z(x,y) \geq 0$.
Definamos o conjunto
\begin{equation*}
	\Omega = \{ \xbarra \in \Sigma: \exists \ybarra \in \Sigma \setminus \{\xbarra\} \text{ tal que } Z(\xbarra,\ybarra) = 0 \}
\end{equation*}
O conjunto $\Omega$ é não vazio.
Usando a proposição \eqref{edp_principal}, e a definição de $\Phi(\xbarra)$, a equação \eqref{suma_2das_derivadas} fica
\begin{multline}\label{suma_2da_der_aleph}
\sum_{i=1}^2 \frac{\partial^2 Z}{\partial x_i^2}(\xbarra,\ybarra) + 2 \sum_{i=1}^2 \frac{\partial^2 Z}{\partial x_i \partial y_i}(\xbarra,\ybarra) + \sum_{i=1}^2 \frac{\partial^2 Z}{\partial y_i^2}(\xbarra,\ybarra) =\\
- \frac{\aleph^2 - 1}{\aleph} \frac{\Psi(\xbarra)}{1 - \langle F(\xbarra), F(\ybarra) \rangle} \sum_{i=1}^2 \left\langle \frac{\partial F}{\partial x_i}(\xbarra), F(\ybarra) \right\rangle
\end{multline}
para qualquer par de pontos $\xbarra \neq \ybarra$. Como no ponto $(\xbarra,\ybarra)$ atinge-se um mínimo, então
\begin{equation*}
	0 = Z(\xbarra,\ybarra) = \frac{\partial Z}{\partial x_i}(\xbarra,\ybarra) = \frac{\partial Z}{\partial y_i}(\xbarra,\ybarra)
\end{equation*}

\begin{proposicao}\label{gradiente_nulo}
	Tem-se que $\nabla \Psi(\xbarra) = 0$ para todo $\xbarra \in \Omega.$
\end{proposicao}

\begin{demonstracao}
	Seja $\xbarra \in \Omega$, então existe $\ybarra \in \Omega \setminus \{x\}$ tal que $Z(\xbarra,\ybarra) = 0$. Tem-se que em $(\xbarra,\ybarra)$ atinge-se um mínimo local, portanto o Hessiano é definido não negativo. O Hessiano está definido por
	\begin{equation*}
		H(\xbarra,\ybarra) = \left[ \begin{matrix}
		\frac{\partial^2 Z}{\partial x_1^2}(\xbarra,\ybarra) & \frac{\partial^2 Z}{\partial x_2 \partial x_1}(\xbarra,\ybarra) & \frac{\partial^2 Z}{\partial y_1 \partial x_1}(\xbarra,\ybarra) & \frac{\partial^2 Z}{\partial y_2 \partial x_1}(\xbarra,\ybarra)\\ 
		\frac{\partial^2 Z}{\partial x_1 \partial x_2}(\xbarra,\ybarra) & \frac{\partial^2 Z}{\partial x_2^2}(\xbarra,\ybarra) & \frac{\partial^2 Z}{\partial y_1 \partial x_2}(\xbarra,\ybarra) & \frac{\partial^2 Z}{\partial y_2 \partial x_2}(\xbarra,\ybarra)\\
		 \frac{\partial^2 Z}{\partial x_1 \partial y_1}(\xbarra,\ybarra) & \frac{\partial^2 Z}{\partial x_2 \partial y_1}(\xbarra,\ybarra) & \frac{\partial^2 Z}{\partial y_1^2}(\xbarra,\ybarra) & \frac{\partial^2 Z}{\partial y_2 \partial y_1}(\xbarra,\ybarra)\\
		  \frac{\partial^2 Z}{\partial x_1 \partial y_2}(\xbarra,\ybarra) & \frac{\partial^2 Z}{\partial x_2 \partial y_2}(\xbarra,\ybarra) & \frac{\partial^2 Z}{\partial y_1 \partial y_2}(\xbarra,\ybarra) & \frac{\partial^2 Z}{\partial y_2^2}(\xbarra,\ybarra)
		\end{matrix} \right]
	\end{equation*}	
	Como o Hessiano é definido não negativo. então para todo $v \in \realnumbers^4$ tem-se
	\begin{equation*}
		v^t H(\xbarra,\ybarra) v \geq 0
	\end{equation*}	
	Em particular, para $v_1 = [1,0,1,0]^t$ e para $v_2 = [0,1,0,1]^t$ tem-se
	\begin{equation*}
		0 \leq v_1^t H(\xbarra,\ybarra) v_1 + v_2^t H(\xbarra,\ybarra) v_ 2 = \sum_{i=1}^2 \frac{\partial^2 Z}{\partial x_i^2}(\xbarra,\ybarra) + 2 \sum_{i=1}^2 \frac{\partial^2 Z}{\partial x_i \partial y_i}(\xbarra,\ybarra) + \sum_{i=1}^2 \frac{\partial^2 Z}{\partial y_i^2}(\xbarra,\ybarra)
	\end{equation*}	
	Por outro lado, como $\aleph > 1$, então
	\begin{equation*}
		- \frac{\aleph^2 - 1}{\aleph} \frac{\Psi(\xbarra)}{1 - \langle F(\xbarra), F(\ybarra) \rangle} \sum_{i=1}^2 \left\langle \frac{\partial F}{\partial x_i}(\xbarra), F(\ybarra) \right\rangle \leq 0
	\end{equation*}	
	Pela equação \eqref{suma_2da_der_aleph} tem-se
	\begin{equation*}
		\left\langle \frac{\partial F}{\partial x_i}(\xbarra), F(\ybarra) \right\rangle = 0
	\end{equation*}	
	A equação \eqref{diff_Z_x} utilizando coordenadas geodésicas pode-se expressar como
	\begin{multline*}
		0 = \frac{\partial Z}{\partial x_i}(\xbarra,\ybarra) = \aleph \frac{\partial \Psi}{\partial x_i}(\xbarra)(1 - \langle F(\xbarra),F(\ybarra) \rangle)  - \aleph \Psi(\xbarra) \left\langle \frac{\partial F}{\partial x_i}(\xbarra), F(\ybarra) \right\rangle\\
		+ \sum_{i=1}^2 \lambda_i \left\langle \frac{\partial F}{\partial x_i}(\xbarra),F(\ybarra) \right\rangle
	\end{multline*}	
	Como $\left\langle \frac{\partial F}{\partial x_i}(\xbarra),F(\ybarra) \right\rangle = 0$, então a equação anterior expressa-se
	\begin{equation*}
		0 = \frac{\partial Z}{\partial x_i}(\xbarra,\ybarra) = \aleph \frac{\partial \Psi}{\partial x_i}(\xbarra)(1 - \langle F(\xbarra),F(\ybarra) \rangle)
	\end{equation*}	
	Portanto, $\nabla \Psi(\xbarra) = 0$ para todo $\xbarra \in \Omega$.
\end{demonstracao}

\begin{lema}[Principio do máximo estrito de Bony]\label{bony}
	Seja $U$ um subconjunto aberto de uma variedade Riemanniana $M$, e seja $X_1, \ldots, X_m$ campos vetoriais diferenciáveis em $U$. Assumir que $\varphi: U \rightarrow \realnumbers$ é uma função não negativa diferenciável tal que
	\begin{equation*}
		\sum_{j=1}^{m} (D^2 \varphi)(X_j,X_j) \leq -L \inf_{|\xi| \leq 1} (D^2 \varphi)(\xi,\xi) + L |d \varphi| + L \varphi,
	\end{equation*}
	onde $L$ é uma constante positiva. Seja $\Omega = \{ x \in U: \varphi(x) = 0 \}$ o conjunto de zeros da função $\varphi$. Alem disso, supor que $\gamma: [0,1] \rightarrow U$ é um caminho diferenciável tal que $\gamma(0) \in \Omega$ e $\gamma'(s) = \sum_{j=1}^{m} f_j(s) X_j(\gamma(s))$ para funções diferenciáveis $f_1, \ldots, f_m: [0,1] \rightarrow \realnumbers$ adequadas. Então $\gamma(s) \in \Omega$ para todo $s \in [0,1]$.
\end{lema}

\begin{demonstracao}
	\cite{Brendle2010}, Corolário 9.7.
\end{demonstracao}

\begin{proposicao}
	$\Omega$ é aberto.
\end{proposicao}

\begin{demonstracao}
	Nas proposições \eqref{2_diff_Z_x}, \eqref{diff_Z_x_y} e \eqref{2_diff_Z_y} estamos assumindo que $Z(\xbarra,\ybarra) = 0$. Vamos agora repetir o cálculos sem considerar essa hipóteses. Ao final tem-se
	\begin{multline*}
	\sum_{i=1}^2 \frac{\partial^2 Z}{\partial x_i^2}(\xbarra,\ybarra) + 2 \sum_{i=1}^2 \frac{\partial^2 Z}{\partial x_i \partial y_i}(\xbarra,\ybarra) + \sum_{i=1}^2 \frac{\partial^2 Z}{\partial y_i^2}(\xbarra,\ybarra) =\\
	- \frac{\aleph^2 - 1}{\aleph} \frac{\Psi(\xbarra)}{1 - \langle F(\xbarra), F(\ybarra) \rangle} \sum_{i=1}^2 \left\langle \frac{\partial F}{\partial x_i}(\xbarra), F(\ybarra) \right\rangle + \overline{\Lambda}(\xbarra,\ybarra) \left( Z(\xbarra,\ybarra) + \sum_{i=1}^2 \left| \frac{\partial Z}{\partial x_i}(\xbarra,\ybarra) \right| + \sum_{i=1}^2 \left| \frac{\partial Z}{\partial y_i}(\xbarra,\ybarra) \right| \right)
	\end{multline*}
	onde $\overline{\Lambda}(\xbarra,\ybarra)$ é uma função continua no conjunto $\{ (x,y) \in \Sigma \times \Sigma: x \neq y \}$, a qual pode ser não limitada perto da diagonal.	
	Pelo principio do máximo estrito de Bony \ref{bony} tem-se que $\Omega$ é aberto.
\end{demonstracao}

Voltando à proposição \ref{gradiente_nulo} e usando o fato que $\Omega$ é aberto, então $\Psi(\xbarra)= |\lambda_i|$ é constante. Pelo teorema de Lawson \ref{teo_lawson}, da fórmula da curvatura de Gauss
\begin{equation*}
	K = 1 + \lambda_1 \lambda_2
\end{equation*}
tem-se $\lambda_1 = \lambda_2 = 0$ ou, $\lambda_1 = 1$ e $\lambda_2=-1$. Mas a superfície não tem ponto umbílicos, então cumpre-se o segundo caso. Portanto, $\Psi(\xbarra)=1$ para todo $\xbarra \in \Omega$.	
Pelo Teorema de continuação única estândar de equações diferencias parciais \cite{Aronszajn1957} tem-se que $\Psi(\xbarra) = 1$ para todo $\xbarra \in \Sigma$. Portanto a curvatura Gaussiana de $\Sigma$ anula-se. Então $\Sigma$ é congruente com o toro de Clifford.